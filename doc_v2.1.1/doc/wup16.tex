%\magnification=1200
%\input wupstuff.tex
\newtoks\footline \footline={\hss\tenrm 16.\folio\hss}
%
\pageno=1
%
\top
\vskip 1.5 true in
\centerline{Section 16: {\bf Velocities}}
\blankline
\blankline
\centerline{\bf ***}
\blankline
\blankline
PANDORA uses several velocity tables, for various purposes. A velocity
\break table may be specified in the input either as a tabulated
function of depth (like other input tables), or by means of
parameters appearing in the equation used \break to generate velocity tables
as functions of depth.

This section 1) describes the various velocities, by exhibiting instances
of their use in the program, 2) describes input procedures and post-read defaults,
and 3) lists additional information about how PANDORA uses these velocities.
The descriptions given here are rather sketchy (full information can be
found in E. H. Avrett's program specifications and explanatory memoranda);
the descriptions and equations appearing in this section are only meant to
be illustrative, not exhaustive.

All the velocities treated in this section are arrays of length {\bf N}.
\ej
\centerline{{\bf 1) Descriptions of Velocities}}
\blankline
\blankline
\underbar{{\it a) Broadening velocities} {\bf V} {\it and} {\bf VR}}
\blankline
PANDORA uses two ``broadening velocities'' to compute doppler widths, depending
on the option VSWICH. When VSWICH is off, then one velocity is used: {\bf V}$_i$
(in km/s), called the ``broadening'' velocity or ``microturbulent'' velocity.
When VSWICH is on, then two velocities are used: {\bf V}$_i$ (in km/s), called
the ``tangential'' broadening velocity and {\bf VR}$_i$ (in km/s), called the
``radial'' broadening velocity.

The doppler width ${DW}$ is computed from:
$$ {DW}_{im} = \alpha \sqrt{ \beta \times {\bf TE}_i + {\cal V}^2_{im}} \quad , $$
(where $\alpha$ and $\beta$ are terms specified elsewhere). When VSWICH is off:
$$ {\cal V}^2_{im} = {\bf V}^2_i \quad , $$
but when VSWICH is on:
$$ {\cal V}^2_{im} = {\bf VR}^2_i {MU}^2_m + {\bf V}^2_i (1 - {MU}^2_m) \quad , $$
where $MU_m$ is cosine of the angle between the direction $m$ and the outward 
normal.
% \ej
\blankline
\blankline
\underbar{{\it b) Expansion velocities} {\bf VXS} {\it and} {\bf VX}}
\blankline
Two ``expansion velocities'' may be specified: {\bf VXS}$_i$ (in km/s,
positive outwards), called the ``Source Function'' expansion velocity,
and various tables of {\bf VX}$^m_i$, $1 \leq m \leq {\bf NVX}$ (in km/s,
positive outwards), called ``additional'' expansion velocities. 

{\bf VXS} describes differential mass motion in the atmosphere and is used
together with option EXPAND. ``Expansion'' affects the line source function
calculation in fundamental ways: a whole-profile solution must be done (see
also Section 18); and WN-matrices ({\it i.e.,} $\Lambda$-operators) must be 
computed with the general ray tracing method, which uses numerical (explicit)
angle integration, in place of the other methods,which use analytic angle
integration (and require much less computing). ({\it Note} that, for study
and testing purposes, ${\bf VXS}_i = 0$ is allowed when option EXPAND is on;
this forces PANDORA to use the general ray tracing method even in a stationary
atmosphere.)
\ej
The effect of velocity on the calculated line profile can be studied
approximately but economically by computing the line source function statically
and then calculating not only the corresponding static line profile but also
one or more additional profiles using ``additional'' expansion velocities {\bf VX}.
\break
({\bf Important:} tables of {\bf VX} are used {\it only} for the line profile
calculations, {\it not} for source function calculations; however, their use
does cause all line source function calculations to be whole-profile
solutions; see Section 18.)

Line absorption profiles are given by the Voigt function
$$ {\phi}_{ik}^{jm} = {{a_i} \over {{\pi}^{3/2}}} \int_{-\infty}^{+\infty}
{{\exp (-x^2) dx} \over {a_i^2 + (U_{ik}^{jm} -x)^2}} \quad , $$
where $a_i = {DP}_i/{DW}_i$ (the doppler width ${DW}_i$ is given in section (1a)
above), and ${DP}_i$ is the damping parameter. The damping parameter for
transition $(u,\ell)$ is computed from:
$$ \eqalign{ {DP}^{u,\ell}_i & = {\bf CRD}^{u,\ell} + {\bf CVW}^{u,\ell}
    \left( {{\bf HN}^1_i \over 10^{16}} \right) 
    \left( {{\bf TE}_i \over 5000} \right)^{0.3} +
    {\bf CSK}^{u,\ell} \left( {{\bf NE}_i \over 10^{12}} \right)^{\bf PW} \cr
    & + {\bf CRS}^{u,\ell} \left( {{\bf ND}^1_i \over 10^{16}} \right) +
    {HDP}^{u,\ell}_i \quad , \cr } $$
where ${HDP}$ is the ion broadening term (upper levels of Hydrogen only).

The parameter $U$ is computed from
$$ U_{ik}^{jm} = { { ({DL}_k + {DV}_i^m ) - {\bf DDL}^j \, {\bf FDDL}_i }
    \over {DW}_i } \, , $$
where ${DL}_k$ is wavelength offset (``delta Lambda'') computed from ${\bf XI}_k$,
${DV}_i^m$ is the wavelength shift due to velocity, and ${\bf DDL}^j$ is the
offset of the $j$th component of a blended line (most transitions are not
``blended lines'' and have ${\bf LDL} = 1$ and ${\bf DDL}^1 = 0$.)

The wavelength shift due to velocity is given by
$$ {DV}_i^m = {VP}_i^m \, ( \lambda^{u,\ell} / c ) \times 10^5 \, , $$
where ${VP}^m$ is the velocity projected along direction $m$, $\lambda^{u,\ell}$
is the ``core'' wavelength of transition $(u,\ell)$, and c is the light speed
(in cm/s). The projected velocity is given by
$$ {VP}_i^m = {MU}^m \times ({\cal V}_i - {\cal V}^*) \, , $$
where ${MU}^m$ is the cosine of the angle between direction $m$ and the outward
normal, ${\cal V}^* = 0$ in the restframe but ${\cal V}^* = {\cal V}_I$ in
the frame comoving with point $I$, and ${\cal V} = {\bf VXS}$ or
${\cal V} = {\bf VX}$, as the case may be.
\ej
PANDORA provides two types of {\bf VX}: ``general {\bf VX}'' and
``shock {\bf VX}.'' This distinction (described below) arises from the
different generating formulas used to generate tables of ${\bf VX}_i$ when
such tables were not explicitly given in the input.
\blankline
\blankline
\underbar{{\it c) Flow-brodadening velocities, $VXFB$, for profiles}}
\blankline
When option FLWBROAD is on, a special set of velocities is generated and 
used to compute a set of line profiles; these profiles are averaged to yield a
``flow-broadened'' profile. These flow velocities, $VXFB$, enter the profile
calculations just like the ``additional'' expansion velocities, {\bf VX},
discussed above.
% \ej
\blankline
\blankline
\underbar{{\it d) Sobolev velocity} {\bf VSB}}
\blankline
The ``Sobolev'' velocity {\bf VSB}$_i$ (in km/s, positive outwards) is needed
when the moving escape
probability ({\it i.e.} Sobolev) solution is selected for one or more
transitions. The Sobolev formula for the net radiative bracket is
$$ {RHO}^{\rm Sobolev}_i = G_i \left( 1 - { J_i \over S_i } \right) \quad , $$
$$ G_i = \int^1_0 {f_i(x)dx} \quad , $$
$$ f_i(x) = { {1 - \exp [-H_i(x)] } \over {H_i(x)} } \quad , $$
$$ H_i(x) = { {FXI}_i \over { x^2 {\bf VSB}^{\prime}_i + 
( 1 - x^2 ) {RV}_i } } \quad , $$
where {\bf VSB}$^{\prime}_i$ is the derivative of {\bf VSB} with respect
to {\bf Z}. In the plane-parallel case: ${RV}_i = 0$; but in the
spherical case:
$$ {RV}_i = { {\bf VSB}_i \over { ( {\bf Z}_N - {\bf Z}_i ) + {\bf R1N} } } \quad . $$
\blankline
\blankline
\underbar{{\it e) Mass motion velocity} {\bf VM}}
\blankline
The ``mass motion'' velocity {\bf VM}$_i$ (in km/s, positive inwards) is needed 
when the option VELGRAD is on and is further explained in section (2b) below.
The mass motion velocity can also be introduced into the
hydrostatic equilibrium calculations, depending on the option HSEV. This is
illustrated next, in the description of the use of {\bf VT}.
\ej 
% \blankline
% \blankline
\underbar{{\it f) Turbulent pressure velocity} {\bf VT}}
\blankline
{\bf VT} is the ``turbulent pressure'' velocity (in km/s), (note that when
option VTV is on then, if no values of {\bf VT}$_i$ were specified in the input,
{\bf VT}$_i$ will be set equal to {\bf V}$_i$). {\bf VT} is used in the
hydrostatic equilibrium calculations in the equation for the total pressure.

The hydrostatic equilibrium equation is
$$ dp/dx = g\rho \, , $$
where p is the total pressure, x is the distance from the outer boundary (in cm),
and the density $\rho$ is given by
$$ \rho_i = m_{\rm H} \, {\bf NH}_i \, \left( 1 + { m_{\rm He} \over m_{\rm H} } 
           { {NHE}_i \over {\bf NH}_i }\right) \, , $$
where ${NHE}_i$ is the Helium density. The total pressure is given by
$$ p_i = ( 1 + \gamma ) P^{\rm gas}_i + P^{\rm turb}_i + P^{\rm exp}_i \quad . $$
$\gamma$ is the input constant {\bf RMAGP} that can be used
to simulate the effect of magnetic pressure; $P^{\rm gas}_i$ is the gas pressure
$$ P^{\rm gas}_i = ({NHA}_i + {NHE}_i + {\bf NE}_i - {NH2}_i ) \, k \, {\bf TE}_i $$
where ${NHA}_i$ is the atomic Hydrogen density and ${NH2}_i$ is the molecular
Hydrogen density; $P^{\rm turb}_i$ is the turbulent pressure
$$ P^{\rm turb}_i = { 1 / 2 } {\rho}_i ({\bf VT}_i \times 10^5)^2 \quad ; $$
and $P^{\rm exp}_i$ is the expansion pressure
$$ P^{\rm exp}_i = {\rho}_i ({\bf VM}_i \times 10^5)^2 $$
{\it only} when the option HSEV is on; when the option HSEV is off, then 
$P^{\rm exp}_i = 0$ (note that option HSEV is on by default). Here {\bf VM}
is the mass motion velocity, further explained in section (2b) below.
\ej
\centerline{{\bf  2) Establishing and checking velocity values}}
\blankline
After all the input has been read, the values of {\bf V}, {\bf VR}, and
{\bf VT}, are accepted as they stand (but note that option VTV is relevant). 
\blankline
{\bf a)} Values of {\bf VSB}, {\bf VXS}, and ``general'' {\bf VX}$^n$ may
either be input directly as simple arrays, or by means of their corresponding
generating parameters \break {\bf CVSB}, {\bf CVXS}, and {\bf CVX}$^n$,
from which the input tables are computed. \break ``General'' {\bf VX} are
computed when ${\bf ISSV}^1 = 0$; for ``shock'' {\bf VX}, see below. \break
(Values of $VXFB$ cannot be input as simple arrays but can only be obtained
with generating parameters; see below.)

Such values of velocity
${\cal V}_i$ are computed from the corresponding generating parameter
$\cal C$ as follows:
$$ {\cal V}_i = { { {\cal C} \times 10^{10} \times {TERM}_i } \over 
                  {\bf NH}_i } \times { {FMV}_i \over {FFRS}^2_i } \quad $$
(see E. H. Avrett's program specifications ``Use of the continuity equation
for the fluid velocity,'' dated 8/9/88, and ``Depth-dependent helium
abundance,'' dated 12/31/90, and the further modifications dated 1/25/93 
and 4/26/95). Here 
$$ {TERM}_i = { {(1 + \mu \times a^*)} \over {(1 + \mu \times a_i)} } , $$
where $\mu = m_{\rm He} / m_{\rm H}$, $a_i = {ABD}_{\rm He}
\times {\bf RHEAB}_i$, and $a^*$ is a reference value of $a_i$ determined as
specified in [12/31/90].

The multiplier ${FMV}_i$ is used to set ${\cal V}_i = 0$ deep in the
atmosphere and is obtained as follows:
$$ {FMV}_i = 0, \qquad t_i < {\bf FMVLIM} , $$
or
$$ {FMV}_i = t_i, \qquad t_i \geq {\bf FMVLIM} , $$
where
$$ t_i = {1 \over 2} [ 1 - \tanh (x_i) ] , $$
and
$$ x_i = ({\bf Z}_i - {\bf CVZ}) / {\bf CDZ}, \qquad {\bf CDZ} \neq 0 , $$
or
$$ x_i = 1, \qquad {\bf CDZ} = 0 . $$
{\bf CDZ} (default = 1 km) and {\bf CVZ} (default = the smallest value of
${\bf Z}_i$ for which ${\bf NH}_i > {10}^{11}$), and {\bf FMVLIM} (default
 $ = 10^{-4}$), are input parameters.
\ej
The term ${FFRS}_i$ is required when spherical geometry has been selected. \break
Thus, when the option SPHERE is off, ${FFRS}_i = 1$; but when the option \break
SPHERE is on,
$$ {FFRS}_i = { { ({\bf R1N} + {\bf Z}_{\bf N}) - {\bf Z}_i } \over
                { ({\bf R1N} + {\bf Z}_{\bf N}) - Z^* } } \, , $$
where $Z^* = {\bf Z}_1$ when ${\bf NH}_1 > 10^{10}$; $Z^* = {\bf Z}_{\bf N}$
when ${\bf NH}_{\bf N} < 10^{10}$; but otherwise is obtained by interpolation
in the {\bf Z} and {\bf NH} tables so that ${\bf NH}(Z^*) = 10^{10}$.
\blankline
\blankline
\underbar{Shock velocity}
\blankline
When input parameter ${\bf ISSV}^1 > 0$, then tables of ``shock'' 
velocity values ${\bf VX}^m_i$,
$1 \leq m \le {\bf NVX}$, are generated from other input parameters as follows:
\break $ {\bf VX}^m_i = {\bf SCVB}$ when $i > {\bf ISSV}^m$, but
$$ {\bf VX}^m_i = - {\bf SCVA} \exp [({\bf Z}_i - {\bf Z}_{{\bf ISSV}^m})
                    / {\bf SCVS} ] $$
for $i \leq {\bf ISSV}^m$. Here {\bf ISSV} is the shock velocity depth index,
{\bf SCVA} is the velocity amplitude, and {\bf SCVS} is the velocity scale height.
\blankline
\blankline
\underbar{Flow-broadening velocities, $VXFB$}
\blankline
$VXFB$s consist of two groups: the $2 \times {\bf NFB}$ isotropic velocities
generated with {\bf CVXM}, and the 2 mass-conserving flow velocities
generated with {\bf CVXF}.
The generating procedure discussed above does not produce proper values of the
isotropic velocities, which are intended to represent ubiquitous ``random'' flows
in the Sun's upper chromosphere and lower transition region. So these isotropic
velocities are edited, after computation, to insure that they are nowhere smaller
than values obtained by interpolation from the input table {\bf FNH}
(a function of {\bf HNDF}).

These $VXFB$ effectively constitute the {\bf VX}-set for this run (with {\bf NVX}
determined by the program); when option FLWBROAD is on no other types of
``additional'' expansion velocities are allowed (and the run will be aborted if
an input value of ${\bf NVX} > 0$ is provided).
\ej
% \blankline
{\bf b)} Values of {\bf VM} may either be input directly as a simple array, or by
means of the generating parameters {\bf CFH} or {\bf CFHE}. These are used
to compute {\bf VM} by a procedure that is intimately related to how {\bf VM}
is used in the diffusion calculations when the option VELGRAD is on, as follows.

In a Hydrogen run:
$$ {\bf VM}_i = 10^{-5} \left( { {\bf CFH} \over {\bf NH}_i } 
   - {RS}_i \times {\bf VBMB}_i \right) \times
   { {FMV}_i \over {FFRS}^2_i } \, , $$
while in a Helium run:
$$ {\bf VM}_i = 10^{-5} \left( { {\bf CFHE} \over {NHE}_i } 
   + {RH}_i \times {\bf VBMB}_i \right) \times
   { {FMV}_i \over {FFRS}^2_i } \, ; $$
where
$$ {RS}_i = \mu \times {HEND}_i / ( {\bf NH}_i + \mu \times {HEND}_i ) \, , $$
while
$$ {RH}_i = {\bf NH}_i / ( {\bf NH}_i + \mu \times {HEND}_i ) \, ; $$
where again $\mu = m_{\rm He} / m_{\rm H}$ and ${NHE}_i$, ${FMV}_i$ and ${FFRS}_i$ 
are as in section (2a), above.
\ej
{\bf c)} When values of velocity tables are established by these procedures, then
the following three remarks apply:

\noindent {\bf 1)} whenever the input value of a generating
parameter ${\cal C} \neq 0$, then the corresponding $\cal V$ will be computed
from it, regardless of whether or not an input table of values was
provided;

\noindent {\bf 2)} whenever one velocity table is set equal to another 
(see just below), then the corresponding generating parameters are set
equal as well; and

\noindent {\bf 3)} whenever ${NH}_i$ is recomputed during a run, those
velocity tables whose corresponding generating parameters are $\neq 0$
are recomputed as well.
\blankline
\blankline
\blankline
{\bf d)} After these procedures have been applied as needed during post-read
input processing, the tables of {\bf VM}, 
{\bf VXS} and {\bf VSB} are checked for reasonableness and consistency,
as follows:

First, the values {\bf VM} are checked. They may $= 0$, but they must
be $\neq 0$ if VELGRAD is.

Next, the values of {\bf VXS} are set up. When there are input values of
{\bf VXS}, they are accepted and named ${VXI}$ ({\it i.e.} the `` original
input'' values of {\bf VXS}).
The values of ${VXI}$ must $= 0$ when EXPAND is off; they
may be $\geq 0$ when EXPAND is on. [{\it NOTE: the values of {\bf VXS} that
are used in the line source function calculations (and the emergent profile
calculations) are redetermined from time to time, as needed; {\rm normally},
however, {\bf VXS} is set = ${VXI}$.}]
Then, when EXPAND is on, the following occcurs. If the values of {\bf VM} 
$\neq 0$, then ${VXI}$ is set = -{\bf VM}. But if the values of
{\bf VM} $= 0$, then ${VXI}$ retains its input values as read; note
that these may $= 0$! After these rules have been enforced, 
{\bf VXS} is set = ${VXI}$.

Last, the values of {\bf VSB} are checked. If the values of {\bf VM}
$\neq 0$, then {\bf VSB} is set = -{\bf VM}. If the values of
{\bf VM} $= 0$ and the values of {\bf VXS} $\neq 0$, then {\bf VSB} is
set = {\bf VXS}. If the values of {\bf VM} and of {\bf VXS} both
$= 0$, then {\bf VSB} retains its input values as read. Note that
{\bf VSB} must $\neq 0$ if a Sobolev solution is requested!
\ej
% \blankline
% \blankline
\centerline{{\bf 3) More about how velocities are used}}
\blankline
{\bf a)} As part of every diffusion calculation ({\it i.e.} when options
AMDIFF and/or VELGRAD are on), tables ${VADD}_i$, ${VV1}_i$, and
${VV2}_i$ are computed as follows: 
$$ {VADD}_i = {VV1}_i + {VV2}_i \quad ; $$ \np
where, when AMDIFF is on in a Hydrogen run:
$$ {VV1}_i = - {VH}_i \times 10^{-5} \quad , $$ \np
when AMDIFF is on in a Helium I run:
$$ {VV1}_i = - {V1}_i \times 10^{-5} \quad , $$ \np
or, when AMDIFF is on in a Helium II run:
$$ {VV1}_i = - {V2}_i \times 10^{-5} \quad ; $$ \np
and, when VELGRAD is on:
$$ {VV2}_i = - {\bf VM}_i \quad ; $$ \np
(see E. H. Avrett's program specification ``Use of diffusion velocities
in emergent profile calculations,'' dated 10/25/89). Then, if the
option VELS is on, the current values of {\bf VXS} are replaced
by the following:
$$ {\bf VXS}_i = {VXI}_i + {VV1}_i \quad . $$
\ej
% \blankline
{\bf b)} Emergent line profile calculations are done with as many
velocity tables as possible. (These enter the calculation
by way of the ``doppler width'' ${DW}$, given above).
A ``set'' of velocity tables is assembled as follows:

The first velocity is either {\bf VXS}$_i$ or {\bf VSB}$_i$, depending
on whether or not the Sobolev solution was used for this transition.

If {VADD}$_i \neq 0$ and $\neq$ the first velocity, then {VADD} is added
to the set as the second velocity.

Thereafter, the tables ${\cal V}^n_i = {\bf VX}^n_i + {VADD}_i, \;
1 \leq n \leq {\bf NVX}$, are added to \break the set, 
provided that ${\cal V}^n_i \neq 0$ and $\neq$ any of the 
preceding velocities. Recall that these {\bf VX} are either
``additional'' expansion velocities (either of type ``general''
or of type ``shock'') or flow-broadening velocities.
%\vfill \vfill
%\vfill \vfill
%\vfill \vfill
%\vfill \vfill
%\vfill \vfill
%\vfill \vfill
%\vfill \vfill
%\vfill 
\vfill
\noindent (Section 16 -- last revised: 2005 Aug 17) \par
\message{Section 16 ends at page 16.\the\pageno}
\ej
%\end
