%\magnification=1200
5\input wupstuff.tex
\newtoks\footline \footline={\hss\tenrm 99.\folio\hss}
\top
\pageno=1
\vskip 1.5 true in
\centerline{Section 99: {\bf Tutorial --- How to Run PANDORA}}
\blankline
\blankline
\centerline{\bf ***}
\blankline
\blankline
This section describes the input files and output files for four PANDORA
demonstration runs. I hope that first-time users will find this helpful.
By way of background: when we first began work on PANDORA, we envisioned 
a `Dial-an-Atom' program; thus PANDORA's original {\it raison d'\^etre} was
the coupled line source function calculation, and the statistical equilibrium
calculations required for them, with arbitrary model ions. Over the years
the program has grown to do many more and different things, but PANDORA's
primary purpose remains the calculation of line source functions and
emergent line profiles.
{\bf Before reading this section}: be sure you have become familiar with
the main body of this handbook, what kind of information it contains,
and how it is organized.
\blankline
\blankline
Any PANDORA run requires input from four `logical data Groups':
1) atmosphere model data;
2) atom (ion) model data; 3) levels- and transition-related restart data 
({\ie} partially-converged solutions) from the preceding run (which may be
omitted for runs starting from scratch); and 4) run-specific numeric and control
data. Items from any of these `groups' may/will occur in each of the
four major structural parts of PANDORA's input: Parts B, D, F and H.
Also, all runs require a run-specific {\bf HEADING} line as the first
line of the input file. \break (PANDORA runs that use PRD for one or more
transitions may also require \break JNU restart values.)

A basic minimum of Group-1 data and Group-2 data normally {\it must} be
supplied, since there are no appropriate default values for them.
Group-3 data normally are not supplied for runs beginning from scratch
({\ie} the defaults are used); however, sometimes
it makes sense to supply output from a related run as input for a new run.
Restart runs normally use Group-3 data that were written in an
output file (a `restart' file) by the preceding run.
Group-4 data need not be supplied (especially for runs beginning
from scratch) since the automatic defaults should be applicable to a
variety of situations. As a calculation progresses through a series of restarts,
it should become clear what switch settings, option settings,
alternative numerical methods,
and numerical control parameter values are most appropriate for any
particular case.
\blankline
There are six Types of PANDORA runs: 1) population-update runs with Hydrogen;
2) population-update runs with ions other than Hydrogen; 3) regular runs;
4) no-ion runs; 5) continuum-only runs; and 6) input-only runs. 

The distinction between population-update and regular runs arises from the
following. PANDORA computes both `line' absorption and `continuum'
(more properly: `background') absorption. Many `absorbers' can be included
in the total computed `continuum' opacity, including the bound-free
absorption of H, He-I, He-II, C-I, Si-I, Al-I, Mg-I, Fe-I, Na-I, Ca-I,
O-I, and S-I.
The calculations of the contributions of these absorbers require values
of number densities and departure coefficients as functions of depth;
computed non-LTE values will be used {\it if they are provided}.
(If none are provided, then LTE default values of number densities will
be computed, and/or default values of departure coefficients will be
computed from whatever values of number density are available.)
Consider a run with, say, Si-I.
At the end of every overall iteration, new values of the number
densities and departure coefficients of the `ion-of-the-run' (in this
case, Si-I) are computed. These values {\it can} be used when the
`continuum' opacity is recomputed at the start of 
the next iteration ({\ie} PANDORA
can be instructed that new values of number densities and departure
coefficients for Si-I are available, and that therefore Si-I's contribution
to the opacity must be recomputed). For this purpose, the new data
must be copied from their `ion-of-the-run' slots into the appropriate
`population-ion-data' slots ({\ie} into the `Si-I-population-data'
slots). PANDORA will do this `updating' if the {\bf POPUP} switch is set.
(If the population-ion-data are not updated at the end of each overall
iteration, then they will retain their initial values.) 
(These population ion data
can be printed by turning on the appropriate population-ion-data
print option -- in this case, SILPRNT.) Moreover, when the
{\bf POPUP} switch is set, then the latest recomputed population-ion-data
will be written in one of the restart files, for later use. Thus, any
PANDORA run for one of the 12 ions listed above {\it may} be
(but {\it need not} be) a population-update run. A run with a valid
{\bf POPUP} statement among the input is a `population-update' run;
runs without the {\bf POPUP} switch set are `regular' runs. (In Type-4
and Type-5 runs the {\bf POPUP} switch is not used.)

(The input table {\bf RUNTOPOP} is important for population update runs.
The parameters describing the built-in population-ion-models are printed
as part of the population ion data for a particular population-update ion -- 
{\eg} with \break SILPRNT on. Note the input parameters {\bf POPION}, 
{\bf POPXLM}, and \break {\bf POPRCP}, provided for emergency use.)

Hydrogen runs are special because PANDORA does certain things only when
Hydrogen is the ion-of-the-run. These include: recomputation of {\bf Z},
of {\bf NH}, and of {\bf NE}. Such recomputed tables will be written in
restart files, and normally \break should be used in subsequent runs.
(Population-update runs also recompute {\bf NE} and write it out --
however, we generally ignore such {\bf NE} values, preferring to rely
on those produced by Hydrogen runs.)

Type-5, continuum-only, runs are about as far removed from PANDORA's
basic purpose as one can imagine. Only `continuum' source functions,
and emergent continuum emission (depending on specific option settings),
are computed. This is done at specific wavelengths only, as controlled
by the switch {\bf JSTCN}.

Type-4, no-ion, runs resemble continuum-only runs in that no line
transfer is computed. In any run with the option DOION off, all
those things, but only those things, that specifically concern an
ion-of-the-run are omitted. Thus a no-ion run does more things (such as,
hydrostatic equilibrium calculation, Dust temperature recalculation)
than a continuum-only run.

Type-6, input-only, runs are useful for checking whether the input data being 
supplied for a run are what they should be. An input-only run only reads and 
massages the raw input, and does some other input-related initializations,
and then stops -- {\ie} its output file (provided it ran to completion)
exhibits the {\bf Phase 0} output (see Section 11), and the input JNU
values (if there were any). After this output has been inspected and found
satisfactory, the statement \break ``DO ( JSTIN ) '' can be removed from the
input, and the run resubmitted as a normal run. 
\blankline
\blankline
\centerline{Tutorial Runs}
\blankline
DEMO1 is a type-1 run from scratch, using a single input file;
DEMO2 is identical to DEMO1 except for using multiple input files; DEMO3 is
a restart, with changes, of DEMO2; and DEMO4 is a type-3 run from scratch.

In this section I refer to the various files by the file ``extension
codes'' used with {\tt schema} (see Section 7). The complete file names
of these demonstration runs are constructed with these extensions; thus:
{\tt demo1.dat}, {\tt demo4.jnr}, etc.

({\it Please note:} PANDORA's input provisions are very flexible,
and users are free to adopt a variety of conventions for organizing
input data. I will describe what I have found convenient;
feel free to establish different procedures.)
\blankline
\blankline
\centerline{DEMO1}
\blankline
This run uses a single input file, {\tt .dat}. This file first contains the
{\bf HEADING} line (Part A); then Group-1 data, Group-2 data, and
Group-4 data in Part B (before the first {\bf GO}); then more
Group-1 data, Group-2 data, and Group-4 data in Part D (before the
second {\bf GO}); Part F and Part H are empty ({\ie} defaults 
will be used).

Group-1, atmosphere model, data comprise the statements: \underbar{\bf N},
{\bf NVH}, {\bf R1N}, {\bf CGR}, \underbar{\bf Z}, \underbar{\bf TE}, \underbar{\bf NH},
\underbar{\bf NE}, and {\bf BDHM}. The underlined ones constitute the basic
minimum that {\it must} be provided; if any of these are missing, the
run will fail. (However, as a special case, when {\bf ZMASS}, {\bf TE}, and
{\bf NE} are provided then {\bf Z} and {\bf NH} need not be.)
(Note: I have to set {\bf NVH} = 0 because I want to have {\bf V} = 0.)

Group-2, atom model, data comprise the statements: \underbar{\bf NL},
{\bf NTE}, \underbar{\bf NT}, \break \underbar{\bf INPAIR}, {\bf NAME},
\underbar{\bf ELSYM}, \underbar{\bf IONSTAGE}, \underbar{\bf MASS}, {\bf PART},
\underbar{\bf ABD}, {\bf PW}, \break 
\underbar{\bf P}, {\bf CP}, {\bf TER}, {\bf CI 1}, {\bf CI 2}, {\bf CI 3},
{\bf CE 2 1}, {\bf CE 3 1}, {\bf A 3 2}, 
{\bf CE 3 2}, {\bf CRD 3 2},
{\bf CVW 3 2}, {\bf CSK 3 2}, and {\bf CRS 3 2}. Again, the underlined ones
represent the basic minimum that {\it must} be provided, or the run will fail
({\it Note:} see Section 19.)
These model atom data constitute a basic three-level Hydrogen atom,
with the (2,1) and (3,1) lines in detailed balance.
A value of {\bf PW} is provided because a {\bf CSK} statement
appears. If the temperature-dependence of {\bf CI} and {\bf CE} were not
known, then {\bf NTE} and {\bf TER} would not appear, and only single,
constant values of {\bf CI} and {\bf CE} would have to be provided.
(But note that while the foregoing is true generally, there is much more
to it for Hydrogen, which is a special case---see Section 19.)
The {\bf PART} statement is not really needed (since 0 is the default);
I like to use it to remind me that depth-dependent partition functions
will be used (the default state of PARTVAR is on).

Group-4 data comprise: {\bf DO}, {\bf OMIT}, {\bf IOMX}, {\bf POPUP}, {\bf RUNTOPOP}
and {\bf TRN}. PANDORA provides defaults for all of them -- but this run wants
the values appearing in {\tt .dat}. The {\bf TRN} statements appear because
the option USETRIN is on. The {\bf POPUP} switch is on, for the
purpose described earlier. {\bf RUNTOPOP} appears because values other than the
defaults are required in this case.

To execute this run, I submit the run-specific script {\tt demo1}, which uses
the general script {\tt schema}. The main purpose of these scripts is to
connect actual files to the files that Fortran I/O expects by default.

Between them, {\tt pandora} and {\tt schema} send a slew of messages
to standard output (the screen; these messages can be redirected to a
``log'' file in the usual manner---it is useful to do that). These messages
consist of `computation progress reports,' relevant directory listings, and
other information that is useful for the record. The last message that comes
from {\tt pandora} is ``{\tt PANDORA done}''; this is the {\it only} reliable
indication that a run completed properly.

The main output file (`printout' file) from PANDORA is the file {\tt .aaa};
here, {\tt demo1.aaa}. A few of the printout sections in this output file
always appear; the remaining printouts are controlled by option settings,
as described in Section 11.

If option LSFPRNT = off in addition to
{\bf LSFPRINT}$^{3,2} = 0$, then the printout headed ``LINE  (3/2)''
would not have appeared, leaving only the subsequent ``Plot of logs of ST
$\ldots$'' as a record of the line source function calculation.
In a converged solution the values of ``S(n)'' and ``S'' would be 
more nearly equal.

The printed values of ``Consistency CHECKs,'' at the end of the printout
\break ``RHO AND RBD,'' look pretty good.  If these tables do not have
almost all values close to 1, then the calculation has not converged and
needs to be iterated further.

The printout ``LEVEL  1 TO K'' appears because the option LYMAN is on and
{\bf KOLEV} $= 1$. The fact that the printed values of ``Old/New'' for RK-1
(on the page just before ``NE'') are rather far from unity is an
indication that this calculation has not converged.

While the sections ``WAVE SUMM 0'' and ``WAVE SUMM 1'' can be turned off
with option WAVEPRNT, I do not recommend doing that.

The various ``Iterative Ratio'' plots show significant changes from
iteration to iteration, another strong indication that the calculation
has not yet converged. NE has settled down in the outer half of the
atmosphere.

``Execution Data'' always appears (the amount of printout is controlled
by option IRUNT). Before signing off, the program version identifies itself.

The remaining files produced by this run are {\tt .msc}, {\tt .pop} and
{\tt .rst}; their contents are described in Section 8. Most of these data are
needed in order to restart this run for additional iterations -- DEMO3
is an example of such a restart.
\blankline
\blankline
\centerline{DEMO2}
\blankline
This run uses the same input parameters as DEMO1, and thus produces the same
output files --- what is different is that the input, instead of being all
contained in {\tt .dat}, is now distributed over four files: 
{\tt .mod}, {\tt .atm}, {\tt .res}, and {\tt .dat}. (These four
files correspond to the four `logical data Groups' described earlier.)

I do this because I find this a convenient way to organize my data. My
conventions are: {\tt .atm} contains the data specifying the physical parameters
of an atom or ion model; {\tt .mod} contains the data specifying the physical
parameters of a model atmosphere; {\tt .res} contains the same data as
the output file {\tt .rst}, {\ie} the restart data for the line source
function calculations, and the `Lyman' calculation (if any); and {\tt .dat}
contains the heading and
the various run-specific processing and numerological control
parameters, {\it and the} {\bf GO} {\it statements} that define the structure
of the input.

PANDORA {\it always} begins reading from {\tt .dat};
thereafter, {\bf USE} statements tell it which file to read from next.

Input reading begins at the {\bf HEADING} in {\tt demo2.dat}.
`{\tt USE ( MODEL ) }' then switches reading to {\tt .mod} ({\ie}
{\tt demo2.mod} -- recall that {\tt demo2} and {\tt schema}
set up this connection), and reading of Part B data then continues there
until `{\tt USE ( INPUT ) }' switches reading back to {\tt .dat}. 
`{\tt USE ( ATOM ) }' then immediately switches reading
to {\tt .atm} ({\ie} {\tt atom2.atm}),
and reading of Part B data then continues there until `{\tt USE ( INPUT ) }'
switches reading back to {\tt .dat}. Reading of Part B data from {\tt .dat}
then continues until the first {\bf GO}, which terminates Part B. Then reading
is again switched to {\tt .mod} and {\tt .atm} for Part D data, and then
to {\tt .res} ({\ie} {\tt demo2.res}) which, you will note, is
{\it empty}. Thereupon the last of the Part D data are read from
{\tt .dat}, until the second {\bf GO}, which terminates Part D. As with DEMO1,
Part F is empty; and so is Part H, even though reading of Part H is
switched to {\tt .mod} (the reason why I do this may become clear with
DEMO3).

When PANDORA first reads from a given file, it starts at the beginning of
the file. When it reads from a partially-read file again, it resumes right
after the last statement previously read.

The {\tt .mod} files, in general, are convenient when you wish to calculate
solutions for various ions in the same atmosphere -- all those runs can be
set up to use the same {\tt .mod} file. The {\tt .atm} files, in general,
are convenient when you wish to calculate solutions for the same ion in
different atmospheres -- all those runs can be set up to use the same 
{\tt .atm} file. Over the years we have accumulated many {\tt .atm} files
for a variety of atom and ion models ({\eg} there are many different
Hydrogen models, using different numbers of levels and transitions;
we also have various models of atoms and ions important in the Sun;
plus some odds and ends). All these are available for general use.
\blankline
\blankline
\centerline{DEMO3}
\blankline
This run shows how I go about setting up a restart of DEMO2.

DEMO3 needs the files: {\tt .dat}, {\tt .mod}, {\tt .atm}, and
{\tt .res}. (Normally, the
files for a restart of DEMO2 would have `DEMO2-names' -- these would be edited
versions of the previous DEMO2 input files. However, for tutorial
purposes, here I have given them new `DEMO3-names'.) I obtained the DEMO3
input files from the input {\it and} output files of DEMO2 as follows:

{\tt .res} is simply a copy of DEMO2's {\it output} file {\tt .rst}.
(If I were restarting \break DEMO2 `normally', without changing 
names from DEMO2 to DEMO3,
I would: \break first delete {\tt demo2.res}, and then change the name of
{\tt demo2.rst} to {\tt demo2.res}.) 

{\tt .mod} is an edited copy of DEMO2's {\tt .mod}, with the following changes:
\bull old values of {\bf NE} were removed, and replaced by those in
{\tt demo2.pop};
\bull values of {\bf NP}, {\bf HN} 1  $\ldots$  {\bf HN} 15 and 
{\bf BDH} 1  $\ldots$  {\bf BDH} 15, from {\tt demo2.pop}, were added after
the second `{\tt USE ( INPUT ) }' to Part H of {\tt .mod},
where they will be read when the third occurrence of `{\tt USE ( MODEL ) }'
in {\tt .dat} switches reading back to this {\tt .mod} file.
To alert PANDORA to the existence of these 
\break `Hydrogen-population-data', I had to add `{\tt NLH ( 15 ) }'
near the beginning, in \break Part B of {\tt .mod}.
(If DEMO2 had calculated FNRMLA,FNRMLB, then I would have used the
updated values of these tables, found in {\tt demo2.msc}, to update
any previous values in {\tt demo2.mod}.) \par

I also introduced some further changes into Part D of this atmospheric model
({\it not} for restart purposes, but because I wanted to change 
the calculation):
\bull a set of values of {\bf V} is added;
\bull the molecular Hydrogen abundance calculation is turned off ({\bf NHTSW});
\bull the He-II-lines and X-ray absorptions are turned off ({\bf NABS}).

Atomic model data normally do not need to be changed for a {\it restart} run,
and so {\tt demo2.atm} could (and normally would) have been used for this
run. However, I wanted a more elaborate atomic model for DEMO3, which I
obtained by copying {\tt demo2.atm} and adding `rates integrations' data:
the {\bf MR}, {\bf WRAT} and {\bf RRCP} statements. (Note: none for level 1
because the `Lyman' calculation is done for level 1; starting values for the
the level-1 rates {\bf RL} 1 and {\bf RK} 1 are in {\tt .res}, having been
computed by DEMO2.)

Finally, {\tt .dat} for DEMO3 is a modified version of {\tt demo2.dat}. Only
one of these modifications is strictly for {\it restart} purposes:
the {\bf HEADING} line. The other modifications
change the details of the calculation itself, and the contents
of the `printout' file:
\bull a value of {\bf LF} that differs from the default;
\bull different option settings (note that the options PHASE2 and USETRIN
now revert to their default settings, which are `on' and `off', 
respectively);
\bull the option IRUNT is changed to the `prolix' setting;
\bull a value of {\bf YLINE}$^{3,2}$ that differs from the default;
\bull the emergent profile calculation for the (3,2) line is turned on
({\bf PROF});
\bull sets of values of {\bf TRN} are omitted, since they are no longer
needed;
\bull and the other parameter changes that you see there. \par

DEMO3 produces the same output files as DEMO2, but the contents differ.
In {\tt .aaa}, you will note the following (among others):
\bull ``ATMOSPHERE'' has different values of `Electron density';
`Broadening velocity' appears;
\bull "ND, NK, + BD" now has nonzero values (from {\tt .res});
\bull ``ATOM'' has different `Line source function method' and `Show computed
line profile'; it no longer has `Radiation Temperature';
\bull the list of contributors in ``BACKGROUND'' is different;
\bull ``INPUT'' has data for ``Rates Integrations''; non-zero input values
of {\bf RKWT} 1, {\bf EP1}, and {\bf EP2} appear (from {\tt .res});
``He II Lines Opacity data'' is gone; data for ``Spectrum Calculations'' appears;
\bull ``CONSTANTS'' and ``TABLES'' are gone (why?);
\bull ``HYDROGEN'' population data now have input values, and no longer
their approximate ({\ie} LTE) default values;
\bull ``RATES'' (through ``RIJ'') appears (option RATEPRNT), ``Details of
Recombination Calculation'' is gone (why?);
\bull ``Results of TAU Calculation, $\ldots$'' appears (option TAUPRNT);
\bull ``NE'' and the printed tables in ``POPULATIONS'' are gone (option POPPRNT);
\bull ``Old/New'' for RK-1 are closer to unity, overall;
\bull ``WAVE SUMM 0'' shows that DEMO3 computed more continuum solutions
than did DEMO2 (why?);
\bull ``Iterative behavior $\ldots$'' printouts are gone (why?);
\bull ``Background Intensity'' (for {\bf LF} values of {\bf MUF}) appears
(why?, and why does ``Background Flux'' not appear?);
\bull ``Profile of the 3/2 Line'' (for {\bf L}={\bf LF} values of 
{\bf MU}) appears;
\bull ``Scratch I/O $\ldots$'' appears (option IRUNT);
\bull ``Execution Data'', and program version description, give more
details $\qquad \qquad$ \break (option IRUNT).

DEMO3's output files {\tt .msc} and {\tt .pop} contain the same amounts of
data as those of DEMO2; {\tt .rst} has a bit more (because {\bf RHOWT}$^{3,2}$
now has some non-zero values).

DEMO3 could now be restarted further, just as before. If this were the kind
of Hydrogen run that produces new values of {\bf Z} (say, because values of
{\bf TAUKIN} are specified), or of {\bf NH} (say, because of option 
HSE), or of {\bf BDHM} (option HMS), then these new sets
would also have to be edited into {\tt .mod},
to replace the old sets.
\blankline
\blankline
\centerline{DEMO4}
\blankline
This is an Mg-II run, `from scratch' like DEMO1, except that it is a `regular'
type of run and it is a PRD run.

The {\tt .atm} file for this run contains a simple Mg-II model with two levels
and one radiative transition. The {\tt .atm} file does {\it not} contain
the specifications of the PRD parameters, since this ion model does not
{\it require} the PRD treatment. Instead, I consider the PRD parameters
to be run-specific, and provide them in {\tt .dat}.
Since {\bf JNUNC} is mentioned
neither in {\tt .mod} nor in {\tt .dat}, it retains its default value 0, and
a {\tt .jnu} input file is not required.

(By the way, regarding 
{\bf GMMA}$^{2,1} = -0.999$: this value might put too severe a strain on a
`from scratch' calculation. One might instead begin with, say,
{\bf GMMA}$^{2,1} = -0.8$ and then, through a series of restarts,
gradually work up to the `desired' value $-0.999$.)

The {\tt .mod} file contains 8 levels worth of `population-ion-data' for
H, He-I, He-II, C-I, and Si-I. Values of departure coefficients are specified
for levels 1 - 5 of H only; default values (see the `printout') will be
set up for everything else. {\tt .mod} also contains five auxiliary depth
tables and illustrates their use; note that {\tt .mod} does not contain {\bf Z},
the depth table of the run ({\it cf.} Section 13). {\tt .mod} also contains
some {\bf DO} statements affecting the `printout' only; one can debate
whether these belong here, or in {\tt .dat}.

As with DEMO2, {\tt .res} is empty (why?).

The {\tt .dat} file is relatively simple; note that it contains the {\bf N} and
{\bf Z} statements, these being run-specific in this case. Of course, there
is no {\bf POPUP} statement (that makes this a `regular' run). The {\bf YLINE}
statement is superfluous (see next paragraph), but does no harm.
(By the way, a {\tt .dat} file for most runs---other than H runs---might
contain a table of {\bf RABD} or {\bf RABDL} values [see Section 17]; a
{\tt .dat} file for a He-II run with diffusion might contain tables of
{\bf PALBET} and {\bf PBETAL}.)

In the `printout' file {\tt .aaa}, note that ``ATOM'' now has `Partial
Redistribution' and several parameters for P.R.D.; moreover,
`{\tt Line source function method}' is `{\tt (not used)}'
because the `{\tt Background source function method}' specification
applies to both calculations. Output from rates integrations is
short (RATEPRNT and RATESUMM are off), consisting of graphs only (RATEGRAF is
on). The ``LINE (2,1)'' printouts here have only very abbreviated PRD calculation
results, because PRDPRNT and PRDCOPR are both off, and only \break
JNUPRNT is on. Notice that the `normal' line source function calculation
output states (between the lines of dashes) that this is a `{\tt Solution with
partial redistribution}'. (Note option PRDMETH; often it makes sense to use
PRDMETH off initially, until the solution is nearly converged, and then
use PRDMETH on.) No ``Consistency Checks'' are computed
for a 2-level ion model. 

Since {\bf LF} is greater than 1, a line flux profile was also computed from
the calculated intensity profiles (however, with only two {\bf MUF} values,
this is not much of a calculation). Since this line is computed with PRD
(${\bf SCH}^{2,1} = 1$), a whole profile, rather than a half profile, was
computed. Moreover, a PRD calculation computes the background continuum for
every point in the line, and that is why the ``Line-Free'' continuum was also
computed for every point in the line.

The other output files produced by this run are {\tt .jnr}, {\tt .msc}, and
{\tt .rst}; of course there is no {\tt .pop} output file (why?).

Restarting this run is simple. Delete {\tt .res}, and change the name of
{\tt .rst} to {\tt .res}; change the name of {\tt .jnr} to {\tt .jnu} (first
deleting the previous {\tt .jnu}, if any). Update the {\bf HEADING} in
{\tt .dat}, and make any other desired changes. Make sure that the value
of {\bf JNUNC} $> 0$ (Part D of the input, in {\tt .dat}).

Again, this run only {\it uses}, but does not {\it affect},
the atmospheric model data. (At least, not directly. In principle, values
of {\bf RABD} for a Mg-I run could be affected by these Mg-II results, and that
would change the `Mg-I-population-data' which are part of the atmospheric
model --- but DEMO4 treats Mg-I in LTE. For {\bf RABD} see Section 17.)
%\vfill \vfill
%\vfill \vfill
%\vfill \vfill
%\vfill \vfill
%\vfill \vfill
%\vfill \vfill
%\vfill \vfill
%\vfill 
\vfill
\noindent (Section 99 -- last revised: 2007 Feb 05) \par
\message{Section 99 ends at page 99.\the\pageno}
\ej
%\end
