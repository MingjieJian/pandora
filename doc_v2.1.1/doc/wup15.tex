%\magnification=1200
%\input wupstuff.tex
\newtoks\footline \footline={\hss\tenrm 15.\folio\hss}
%
\pageno=1
%
\top
\vskip 1.5 true in
\centerline{Section 15: {\bf \underbar{P}artial 
                         \underbar{R}e\underbar{D}istribution}}
\blankline
\blankline
\centerline{\bf ***}
\blankline
\blankline
PANDORA uses either ``complete redistribution'' (CRD, the default) or \break
``partial redistribution'' (PRD, selected by input switch {\bf SCH}$^{u, \ell}$)
to compute the line source function of transition $(u,\ell)$. PRD can be done
by one of two methods, depending on the option PRDMETH. When PRDMETH is on,
PANDORA uses a procedure based on the formulation of Hubeny \& Lites as
discussed in ApJ., {\bf 455}, 376. When PRDMETH is off, PANDORA uses a
simplified procedure based on the formulation of Kneer \& Heasley as discussed
in Appendix A of Vernazza, Avrett \& Loeser (1981), Astrophys.J.Suppl.
{\bf 45}, 635, and in a paper by Avrett \& Loeser (1984) appearing in
{\it Methods in Radiative Transfer}, Kalkofen, ed., pp. 341--379. Note that
PRDMETH selects the method for all PRD-transitions in a run.
\blankline
\blankline
\blankline
\centerline{\underbar{Input Parameters}}
\blankline
Set {\bf SCH}$^{u, \ell} = 1$ when PRD is to be used for transition
$(u, \ell)$ (see also Note 47 of Section 5), and set option PRDMETH for the
desired procedure. The input parameters ${\bf GMMA}^{u,\ell}$ and
{\bf IGMSW} control the calculation of $\gamma$ and are always required.
\blankline
When PRDMETH is on the other input parameters are {\bf ITPRD}, {\bf PRDCV}
and {\bf IGII}. The iterations limit {\bf ITPRD} and the convergence
criterion {\bf PRDCV} control the (JNU : S) iterations.
{\bf IGII} determines how the function RII is computed:
{\bf IGII} = 1 selects the fast approximations developed by Gouttebroze, while 
{\bf IGII} = 2 selects the more accurate calculation of Adams, Hummer \& Rybicki.
\ej
When PRDMETH is off the other input parameters are:
${\bf XC}^{u,\ell}$, ${\bf XP}^{u,\ell}$, ${\bf XR}^{u,\ell}$, ${\bf DDR}_k$
and ${\bf XDR}_k$ for $0 \leq k \leq {\bf NDR}$, {\bf XCL}, {\bf TAUCL},
{\bf LMDL2}, {\bf LMDL3}, and {\bf DRLIM}, used to compute $DR$.
Note that values of ${\bf XC}^{u,\ell}$, ${\bf XP}^{u,\ell}$,
and ${\bf XR}^{u,\ell}$ can be specified for each PRD transition;
but there is only one table of {\bf DDR} vs. {\bf XDR} values,
used by all PRD transitions of a run, and only one pair of {\bf XCL} and
{\bf TAUCL} values.
\blankline
The remainder of this section relates to Gamma($\gamma$) and to $DR$.
\blankline
\blankline
\centerline{\underbar{Gamma}}
\blankline
$\gamma_i^{u, \ell}$ is the degree of coherent scattering in the wings
of the $(u,\ell)$ line. $\gamma$ is defined in equation (73) of
Avrett \& Loeser (1984).  $\gamma$ depends on depth (index $i$) and is very
close to unity for low densities. PANDORA provides three ways of specifying
$\gamma_i^{u, \ell}$, selected by the value of input parameter
{\bf GMMA}$^{u, \ell}$.
\blankline
When {\bf GMMA}$^{u, \ell} = -1$
(the default value), then $\gamma_i^{u, \ell}$ = 
{\bf CRD}$^{u, \ell} / DP_i^{u, \ell}$, where the input parameter
{\bf CRD}$^{u, \ell}$ is the radiative damping parameter and \break
$DP_i^{u, \ell}$ is the computed total (radiative and collisional) damping
parameter. 
\blankline
When {\bf GMMA}$^{u, \ell} < 0$, but $> -1$, (say,
{\bf GMMA}$^{u, \ell} = -.99$), then the smaller of
($|${\bf GMMA}$^{u, \ell}|$, {\bf CRD}$^{u, \ell} / DP_i^{u, \ell}$)
is used as the value of $\gamma_i^{u, \ell}$.
\blankline
When {\bf GMMA}$^{u, \ell} > 0$, then 
$\gamma_i^{u, \ell}$ = {\bf GMMA}$^{u, \ell}$ ({\it i.e.} that constant
value at all depths), disregarding {\bf CRD}$^{u, \ell} / DP_i^{u, \ell}$.
\blankline
When {\bf IGMSW}$ = 1$, then the alternate formulas specified in the writeup
dated [05 Feb 18] are used for $\gamma^{2,1}$ and $\gamma^{3,1}$
(Lyman alpha and beta) in Hydrogen runs.
\blankline
\blankline
\centerline{\underbar{DR}}
\blankline
When PRDMETH is off one must specify the function $DR(x)$, which describes
the change from zero coherent scattering in the Doppler core to
partial coherent scattering in the wing. $DR(x)$ is the
function $f(x)$ defined by equation (78) of Avrett \& Loeser (1984).

PANDORA provides three options for computing values of $DR(x)$,
selected by the value of input parameter ${\bf XC}^{u,\ell}$.
\ej
% \blankline
% \blankline
1) When {\bf XC}$^{u, \ell} < 0$, then $DR(x)$ is
obtained by interpolation from the input table {\bf DDR}$_k$ vs.
{\bf XDR}$_k$, $1 \le k \le $ {\bf NDR}. The default values of this
input table are the same as Table 33 of Vernazza, Avrett \& Loeser (1981).
\blankline
\blankline
2) When {\bf XC}$^{u, \ell} > 0$, PANDORA uses
%
$$ DR(x) = \cases { 1, 
                 & $x \le x_c$, \cr
                 \exp \left[ - \left( {\displaystyle
                 {{x-x_c} \over x_c}   }   \right) ^p \right],
                 & $x > x_c$. \cr}   \eqno(1)  $$
%
({\it c.f.} equation (78) in Avrett \& Loeser (1984)). Here
$x_c$ = {\bf XC}$^{u, \ell}$ and $p$ = {\bf XP}$^{u, \ell}$.
Moreover, when $DR(x)$ is computed in this way it may not be less than
$drlim^{u,\ell}$ (see below).
\blankline
\blankline
3) When {\bf XC}$^{u,\ell} = 0$, $DR(x)$ is computed from equation (1)
but with $XXC_i$ in place of $x_c$. $XXC_i$ depends on the
depth-dependent Voigt function parameter $a_i$; on the mean optical
depth of the line
%
$$ TAUM_i = 10^5 \int_{Z_1}^{Z_i} (GTN_i + KPC_{i1}) dz \, ;$$
%
and on the input parameters {\bf XCL} (default $= 3.5$) and {\bf TAUCL}
(default $= 10^4$) as follows. For $TAUM_i \le$ {\bf TAUCL}, $XXC_i =$
{\bf XCL}. For $TAUM_i \ge 10^6$, \break $XXC_i = FXC_i$ where 
%
$$ FXC_i = \left({{4 a_i TAUM_i} \over {\pi}} \right)^{0.3333}. $$
%
For {\bf TAUCL} $< TAUM_i < 10^6$, linear interpolation is used, so that
%
$$ XXC_i = {{(6 - {\log}_{10}TAUM_i){\tenbf XCL} + 
         (\log_{10}TAUM_i - \log_{10}{\tenbf TAUCL}) FXCC_i}
         \over {6 - \log_{10}TAUCL}} $$
%
where
%
$$ FXCC_i = \left({{4 a_i 10^6} \over {\pi}} \right)^{0.3333}. $$
%
However, whenever $XXC_i$ is less than {\bf XCL}, it is replaced
by {\bf XCL}. Again, when $DR(x)$ is computed in this way it may not be
less than $drlim^{u,\ell}$ (see below).
\ej
\centerline{\underbar{DR limit}}
\blankline
The value of $DR(x)^{u,\ell}$ computed from equation (1) above is not
allowed to be less than the specified limit $drlim^{u,\ell}$ for that
transition. There are two procedures for obtaining the value of
$drlim^{u,\ell}$: a) the general case, and b) Hydrogen $(u,1)$
transitions, {\it i.e.}, the Hydrogen Lyman lines. These procedures
use the input parameters ${\bf XR}^{u,\ell}$, {\bf DRLIM}, {\bf LMDL2},
and {\bf LMDL3}.
\blankline
\noindent{\it a) \underbar{General case}}
\blankline
If ${\bf XR}^{u,\ell} \ne -1$, then
$$ drlim^{u,\ell} = {\bf XR}^{u,\ell} \, ; $$

but if ${\bf XR}^{u,\ell} = -1$, then
$$ drlim^{u,\ell} = {\bf DRLIM} \, . $$
\blankline
\noindent{\it b) \underbar{Hydrogen Lyman lines}}
\blankline
If ${\bf XR}^{u,1} \ne -1$, then
$$ drlim^{u,1} = {\bf XR}^{u,1} \, ; $$

but if ${\bf XR}^{u,1} = -1$, then
\blankline
if $u = 2$, {\it i.e.} the Lyman-$\alpha$ line, then
$$ drlim^{2,1} = {\bf LMDL2} \, , $$

if $u = 3$, {\it i.e.} the Lyman-$\beta$ line, then
$$ drlim^{3,1} = {\bf LMDL3} \, , $$

but if $4 \leq u \leq 15$, then
$$ drlim^{u,1} = { \sum_{k=2}^{u-1} A^{u,k}
                   \over 
                   { A^{u,1} + \sum_{k=2}^{u-1} A^{u,k} } } \, , $$

\noindent the branching ratio equation,
where the $A^{i,j}$ are the computed Einstein A coefficients
(see Section 19).
%\vfill \vfill
%\vfill \vfill
%\vfill \vfill
%\vfill \vfill
%\vfill \vfill
%\vfill \vfill
%\vfill \vfill
%\vfill 
\vfill
\noindent (Section 15 -- last revised: 2005 Jun 15) \par
\message{Section 15 ends at page 15.\the\pageno}
\ej
%\end
