% \documentstyle[11pt,newpasp,twoside,epsf]{article}
\documentclass[11pt,twoside]{article}
\usepackage{newpasp,epsf}
\markboth{Avrett \& Loeser}{Modeling with Pandora}

\def\blankline{\par\vskip 11 pt}
\def\hang{}

\begin{document}
\title{Solar and Stellar Atmospheric Modeling Using the
Pandora Computer Program}

\author{Eugene H. Avrett and Rudolf Loeser}

\affil{ Smithsonian Astrophysical Observatory,
Harvard-Smithsonian Center for Astrophysics,
60 Garden Street, Cambridge, MA 02138, USA \break
{\tt avrett,rloeser@cfa.harvard.edu} }

\begin{abstract}
The Pandora computer program is a general-purpose non-LTE atmospheric
modeling and spectrum synthesis code which has been used extensively to
determine models of the solar atmosphere, other stellar atmospheres, and
nebulae. The Pandora program takes into account,
for a model atmosphere which is either planar or spherical,
which is either stationary or in motion, and which may
have an external source of illumination, the time-independent optically-thick 
non-LTE transfer of line and continuum radiation for multilevel atoms and multiple 
stages of ionization, with partial frequency redistribution, fluorescence, and 
other physical processes and constraints, including momentum balance and radiative 
energy balance with mechanical heating. Pandora includes the detailed
effects of both particle diffusion and flow velocities in the equations of 
ionization equilibrium. Such effects must be taken into account whenever the 
temperature gradient is large, such as in a chromosphere-corona transition 
region.
\end{abstract}

\section{Introduction}

In this paper we first present a description of four different types of model 
atmosphere calculations: radiative equilibrium, general equilibrium, 
semi-empirical, and hydrodynamical, followed by a brief discussion of the 
important effects that should be included in model calculations.  Then
we describe the Pandora program which can be used for many
different purposes, but is restricted at present by the assumptions of
one-dimensionality and time-independence.  Pandora can be used
for the first three types of model calculations but does not include the
detailed time-dependent effects used in the hydrodynamic calculations.  
We list the publications that document the methods used in the program and 
that give the results of many different applications.  The program is 
available to new users on request (with the caveat that it is an extensive
general-purpose program that requires a significant commitment of learning
time on the part of the user; see the Pandora webpage
{\tt http://cfa-www.harvard.edu/$\sim$rloeser/pandora.html}).

\section{Types of Model Atmosphere Calculations: An Overview}

Model atmospheres can be classified as being of four general types: 
radiative equilibrium, general equilibrium, semi-empirical, and 
hydrodynamical.  The first is based on the constraint of radiative 
equilibrium for which the temperature distribution is determined so that 
the outward radiative flux is constant with depth.  Radiative equilibrium 
models assume that there are no sources of non-radiative heating, and are 
normally time-independent without mass flows.  Except in special cases, 
the temperature decreases monotonically with decreasing depths.  Examples 
are the deeper layers of stellar atmospheres (photospheres) that show no 
evidence of non-radiative heating effects.  See Kurucz (1970, 1979, 
1991, 1998).

General equilibrium models are usually time-independent but include the
effects of non-radiative energy flow due to thermal conduction, particle
diffusion, and mass flows, and can include mechanical heating specified
in some parametric way.  Examples are the thin transition regions between
neutral and ionized regions where a very steep temperature gradient 
results from strong resonance-line cooling at the lower temperatures.  
See Fontenla, Avrett, \& Loeser (2002).

Semi-empirical models use a prescribed temperature distribution which is
selected to obtain agreement between the spectrum calculated from the model
and an observed spectrum.  From such a model one can use the calculated
departures from constant radiative flux to infer the corresponding
mechanical heating distribution.  Examples are stellar
chromospheres that show emission due to an outward increase in temperature
caused by the dissipation of mechanical energy in some form.  See
Vernazza, Avrett, \& Loeser (1981) and Avrett (2002).

Hydrodynamical models simulate dynamical processes and use their properties 
to supply the mechanical heating necessary to account for an 
observed spectrum that shows emission in excess of that determined by 
radiative equilibrium.  Such hydrodynamical models must assume some initial 
conditions to get the gas motions started, but then the model relies on 
internal wave motions to produce the mechanical heating.  The aim in such 
calculations is to include all important physical processes and to match the 
given observations well enough to have confidence that the simulation is
realistic.  The time-dependent hydrodynamical models cannot be expected to lead 
to good agreement with observations if any important physical processes are not 
included or are not treated correctly.  To the extent that the hydrodynamical 
models agree with observations they tell us more about the physical 
mechanisms at work than can be learned from the semi-empirical models, since
the latter can indicate only the general properties of the temperature and 
mechanical heating distributions yielding a given spectrum.  In 
extreme cases the variations with time in hydrodynamical models can be 
very large, suggesting that time-averaged quantities in a corresponding
time-independent semi-empirical model may not properly represent the
physical conditions in the atmosphere.  Examples are the dynamical solar
atmospheric models of Carlsson \& Stein (1997, 1999).

For a radiative equilibrium model one must specify the effective temperature 
(which measures the total radiative flux), the surface gravity, and the 
chemical composition of the atmosphere, and can then calculate a model 
atmosphere and the corresponding spectrum to compare with an observed 
stellar spectrum.  In many cases, combinations of values of the effective
temperature, gravity, and composition can be found that lead to good
agreement between calculated and observed spectra.  However, if the observed
spectrum shows emission features that cannot be explained with radiative
equilibrium models, one can carry out general equilibrium calculations,
or can change the radiative equilibrium temperature distribution to a
semi-empirical distribution chosen by trial-and-error to give a best match 
between the calculated and observed spectra.  The final alternative is to 
introduce wave motions that heat the atmosphere and to calculate a 
self-consistent time-dependent hydrodynamical model in an attempt to match 
the observed spectrum.  The hydrodynamical calculations are complex, 
and the results reported to date by Carlsson \& Stein match some 
observations well but fail to match others.  See Kalkofen (2001).  


\section{Important Processes in Model Atmosphere Calculations}

\subsection{Atomic and Molecular Data}

Consider the calculation of the spectrum of atomic hydrogen.  A
good representation of the hydrogen atom in most cases is one having 15
levels, with 105 line transitions.  It is not possible to calculate the
hydrogen spectrum without including the many other atoms, ions, and
molecules that influence the lines and bound-free continua of hydrogen
and that provide the dominant opacity contributions in various parts of
the spectrum. These contributions include not only the negative hydrogen
ion and the bound-free continua of He, C, Mg, Si, Al, Ca, Na, Fe, and
other atoms, but vast numbers of line transitions from these and other 
atoms, ions, and molecules.  All of these need to be represented in 
detail with many levels and transitions, and all interact with each other.  
Fortunately, most of these interactions can be treated iteratively with
no intrinsic convergence problems.  The model calculations of all 
four types require extensive multilevel atomic and molecular systems
in order to calculate in detail the intensity of radiation as a function
of both wavelength and depth.  When LTE (see below) can be assumed, it is
sufficient to include in the model calculation only the opacity due to
all the lines and the various continua as functions of wavelength and
depth (since the ratio of emission to absorption is then given by the
Planck function). Extensive line-opacity tabulations are available
on CD-ROMs from R. Kurucz (see {\tt http://kurucz.harvard.edu}). 

\subsection{Non-LTE Effects}

Radiative equilibrium models usually are calculated assuming local
thermodynamic equilibrium (LTE), i.e., assuming that all atomic and
molecular energy levels are populated at each depth as they would be in
thermal equilibrium corresponding to the kinetic temperature at that depth.
This is a good approximation for high-density regions at large optical
depths where the radiation intensity at a given location is produced in a
small surrounding volume and hence can be represented by the Planck
function at the local temperature.  But in low-density atmospheric regions
where photons are likely to be scattered (i.e., re-emitted after an
absorption with no intervening collisional transitions) the radiation is
produced in a larger surrounding volume.  If this volume extends into or
beyond the surface layers, where there is little or no emission, 
the radiation intensity at the given location
is usually smaller than that given by the local Planck function.
Such non-LTE effects are of critical importance
in low-density atmospheric regions, such as in stellar chromospheres and
higher layers.   To account for non-LTE effects it is necessary to solve 
complicated systems of radiative transfer and statistical equilibrium 
equations.  See Vernazza, Avrett, \& Loeser (1973), Mihalas (1978), 
Anderson (1989), Avrett \& Loeser (1992), Avrett (1996), and especially 
the proceedings of the recent T\"ubingen workshop on Stellar Atmospheric 
Modeling (Hubeny, Mihalas, \& Werner 2002).  

\subsection{Partial Redistribution}

The thermal motions of atoms which absorb and re-emit photons in a given
line cause frequency redistribution within the narrow Doppler core of the
line.  In the far wings, however, frequency redistribution is limited and the
scattering is essentially coherent.  Complete frequency redistribution (CRD)
over the entire line is a useful simplifying approximation, but one that is
inappropriate for strong lines formed in low-density atmospheric regions.

The theory of partial frequency redistribution (PRD) is reasonably well 
understood and should be included for all strong lines.  See the review of 
early developments by Linsky (1985), and the more recent papers by Cooper, 
Ballagh, \& Hubeny (1989), Hubeny \& Lanz (1995), and Uitenbroek (2001).

The CRD line source function is frequency-independent while the line source
function determined from the more general PRD theory varies from core to
wing.  It is often important to include PRD interlocking between lines having 
an energy level in common, and between blended or partially overlapping
lines (e.g., see Mauas, Avrett, \& Loeser 1989), and to include the detailed 
effects of Doppler shifts due to relative gas motions.

\subsection{Particle Diffusion and Advection}

In atmospheric regions with steep temperature gradients, e.g., a
chromosphere-corona transition region, three sources of heat flow should
be considered: thermal conduction, particle diffusion, and, in the case
of mass flows, advection (understood to include the ionization effects
described below).  Thermal conduction depends in a simple way on the
temperature gradient.

In a partially ionized gas, ambipolar diffusion
of ions (diffusing toward lower temperatures) and of atoms (diffusing
toward higher temperatures) also depends on the temperature gradient, 
and is often more important than thermal conduction.  The main heat flow 
contribution made by diffusion is the ionization energy carried by ions 
that recombine to release energy at lower temperatures.

Advection refers to the effects of mass and particle flow velocities.  Flows
toward lower temperature regions also cause ions to release their ionization
energy at the lower temperatures and heat the gas, but this component 
of the heat flow does not depend on the temperature gradient. If roughly 
the same total heat flow toward lower temperatures is needed to balance 
the losses of energy by radiation, then the temperature gradient will be
reduced in order to reduce the conduction and diffusion contributions.
Conversely, given roughly the same radiative losses to be balanced,
a flow toward higher temperature regions counteracts the opposite flow of
heat carried by conduction and diffusion, so that the temperature gradient
must be larger in order to enhance the conduction and diffusion contributions.
 
The profiles of the hydrogen and helium resonance lines formed in 
the solar transition region are greatly affected by diffusion and by flows, 
and Doppler line shifts are much less important than the asymmetric changes 
in line intensity and central reversal due to the influence of flows on 
excitation and ionization. These conclusions are based on our current reference 
model of the solar atmosphere that extends from the photosphere into the
corona, and that is roughly in accord with the observed solar spectrum
from radio waves to X-rays (Fontenla, Avrett, \& Loeser 2002).  


\section{The Pandora Atmosphere Program: General Description}

Pandora deals with a time-independent one-dimensional atmospheric 
region that is either finite or semi-infinite in the plane-parallel
case, or that has spherical symmetry. The atmospheric
layers can be stationary or can be moving relative to each
other. Illumination from an external source can be
prescribed at the atmospheric boundary (front or back in
the finite case).

The basic calculation assumes a given temperature and density stratification. 
Typically a temperature distribution is prescribed and the density is
determined assuming 1) hydrostatic equilibrium (balancing gas pressure 
and gravity), or 2) pressure equilibrium in the absence of gravity (or any
other constraint). After the radiative properties of the atmosphere are 
calculated, a revised temperature distribution can be determined subject 
to energy balance constraints, using parameters that describe given 
non-radiative processes.  For example, Avrett and Loeser (1988) in a study
of quasar broad emission line regions calculated the internal structure 
and the emergent spectrum of a constant-pressure cloud of a given large 
optical thickness with given illumination incident upon one face. The 
temperature distribution in this case was determined from the constraint 
of radiative energy balance.  This is an example of the general equilibrium
type of calculation discussed above.  A new temperature distribution also
can be determined by trial-and-error to obtain agreement between the
calculated spectrum and an observed one, thus obtaining a semi-empirical
model.  

Given the temperature and density distributions, the
non-LTE energy level populations of the various atoms and
ions can be calculated. In typical problems the line and
continuum optical depths can reach very large values, but collisions
between atoms and electrons are too infrequent to establish LTE
except in the deepest layers. The statistical equilibrium equations
determine the populations at any point in the atmosphere given the
properties of the radiation at that point. According to the
radiative transfer equation, the radiation at that point 
depends on the radiative sources and hence on the
populations throughout a large surrounding volume. The set
of coupled transfer and statistical equilibrium equations
must be solved for both the populations and the radiative
intensities at all points in the atmosphere. For a single
line transition this can be called the two-level transfer
problem. The general case of an atom or ion with many
interacting line transitions can be called the multi-level
transfer problem. These cases will be discussed further in
the following sections. First, however, we consider some 
other parameters that need to be determined as part of
the overall solution.

As the temperature changes throughout the atmosphere we may
need to calculate the level populations of each successive
ionization stage of a given element, e.g., Si-I, Si-II, Si-III,
Si-IV, $\ldots$, and each of these stages may have
a large number of discrete energy levels. For simplicity,
Pandora treats, in any one computer run, the levels of a
single ionization stage together with the lowest level of
the next higher stage to determine the relative populations
of these levels. The sum of the populations of these two
stages is a certain fraction of the total element abundance.
After calculating the populations for Si-I--II, assuming all
silicon to be in these two stages, the subsequent Si-II--III
calculation will then exclude the Si-I fraction, etc.
The Si-I--II populations must then be redetermined by a
calculation that excludes the fraction in Si-III and higher stages.
Such iterations require several repeats to get consistent
results. This approach assumes that ionization and recombination
take place only between successive stages and not, e.g., between
Si-I and Si-III.

The electron number density needed in these calculations depends
on the ionization of various constituents. Consider the simple case
in which the electron density $n_{\rm e}$ is the sum of the proton
density $n_{\rm p}$ and the contribution $(Z \times n_{\rm H})$ from all
other elements, where $n_{\rm H}$ is the total hydrogen density and
$Z$, in the simplest case, is the fraction of those elements that
are once ionized, thus contributing one electron. For hydrogen we
can write $n_{\rm H} = n_{\rm HI} + n_{\rm p}$ (when molecular
hydrogen can be ignored), and
$n_{\rm HI} = b_1 n_{\rm e} n_{\rm p} \psi (T)$ where
$n_{\rm HI} = n_{\rm e} n_{\rm p} \psi (T)$ is the LTE Saha-Boltzmann
equation for the neutral hydrogen density (in the lowest level) and
$b_1$ is the departure coefficient, or correction factor, 
obtained from the detailed non-LTE calculation for hydrogen.
Eliminating $n_{\rm HI}$ and $n_{\rm p}$ gives a quadratic equation
that can be solved for $n_{\rm e}$, given $Z$, $n_{\rm H}$, $b_1$,
and $T$. The indirect dependence of $Z$ and $b_1$ on $n_{\rm e}$
is treated iteratively.

The different ions (including neutral atoms) interact
with one another not only through their contributions to $n_{\rm e}$
but also because the different ions influence and are
influenced by the common radiation field which varies with
wavelength and with location in the atmosphere. When solving
for the populations of a given ion we need to know in detail
how other ions absorb and emit radiation at the transition
frequencies of the given ion. Pandora treats this
dependence iteratively, including all important non-LTE
effects. In the case of the Sun the non-LTE populations of
H, H$^-$, He-I, He-II, C-I, Si-I, Mg-I, 
Fe-I, Al-I, Ca-I, and Na-I are
computed to determine $n_{\rm e}$ and the observed continuum, and the
influence of the very large number of lines in the spectrum
is considered when calculating photoionization rates. For
this purpose Pandora uses a sampled set of Kurucz's line
opacities (see Avrett, Machado, and Kurucz 1986).


\section{The two-level transfer problem}

Consider a line transition involving levels $1$ and $2$.
Ignoring absorption and emission by other sources and
ignoring stimulated emission for simplicity, the radiation
intensity at a given location results from the emission and
absorption at every frequency. The radiative transfer
equation for frequency $\nu$ in the line is
%  1
\begin{equation}
 {d I_{\nu} \over d z} =  { -h\nu\over  4\pi} \phi_{\nu}
(n_1B_{12}I_{\nu} - n_2A_{21}),  
\end{equation}
%
where $B_{12}$ and $A_{21}$ are the Einstein coefficients for
absorption and spontaneous emission, $z$ is geometrical
distance measured in the direction of the radiation intensity
$I_{\nu}$, $n_\ell$ represents number density of level
$\ell$, and $\phi_{\nu}$ is the normalized absorption profile
that includes the effects of Doppler broadening near line
center and other types of broadening in the wings.
Here for simplicity we have assumed that the emission
profile is also $\phi_{\nu}$ so that the
absorbed and emitted photons are uncorrelated; this is the
assumption of complete frequency redistribution (CRD). When
a line such as the hydrogen L$\alpha$ line has substantial
opacity in the wings, and when the perturbing densities are
low, then the absorption of a photon in the wings is
followed by the re-emission of a photon with a frequency
that is likely to be close to that of the absorbed photon,
rather than a frequency near line center; this is the more
general case of partial frequency redistribution (PRD).
Pandora allows any line to include PRD effects (see
Vernazza, Avrett, and Loeser 1981 and Avrett and Loeser 1984).

Using CRD as in equation (1), we introduce the monochromatic
optical depth $\tau_{\nu}$ (again in the direction of $I_{\nu}$),
and write
%  2
\begin{equation}
 {d I_{\nu} \over d \tau_{\nu}} = - I_{\nu} + S , 
\end{equation}
%
where $S$ is the line source function
%  3
\begin{equation}
 S = {n_2 \over n_1} {A_{21} \over B_{12}}  
\end{equation}
%
(without stimulated emission). In the simple case of
only radiative and collisional transitions between levels $1$
and $2$ (and no continuum), the statistical equilibrium equation is
%  4
\begin{equation}
 0 = n_1 \bigl(B_{12} {\overline J} + C_{12}\bigr) - 
n_2 \bigl(A_{21} + C_{21}\bigr) , 
\end{equation}
%
where
%  5
\begin{equation}
 {\overline J} = \int\! \phi_\nu J_\nu d\nu  
\end{equation}
%
and where the mean intensity is given by
%  6
\begin{equation}
 J_\nu = { 1 \over {4 \pi } } \int \! I_\nu d\Omega  .  
\end{equation}
%
One can solve equation (2) for $I_\nu$ in terms of $S$ and express
the result as
%  7
\begin{equation}
 I_{ik} = \sum_j \Lambda_{ijk} S_j  ,  
\end{equation}
%
where $i$ and $j$ are depth indices between 1 and $N_D$ and $k$ is the 
frequency index. Here the integral has been approximated by the sum of the
values of $S$ weighted by appropriate coefficients. These Lambda-operator
weighting coefficients depend only on the values of $\tau_{ik}$ and on the
choice of formal approximation of the variation of $S(\tau_\nu)$ between
one value of $\tau_\nu$ and the next. Several different functional
representations are available in Pandora; the choice we find works best
is simply to approximate $S$ in the interval $\tau_{i-1,k} \leq \tau \leq
\tau_{i+1,k}$ by the parabola through $S(\tau_{i-1,k})$, $S(\tau_{i,k})$,
and $S(\tau_{i+1,k})$, $i \neq 1, N_D$, and linearly in all other intervals.

Combining equations (5) -- (7) gives
%  8
\begin{equation}
 {\overline J}_i = \sum_j W_{ij}^\Lambda S_j  .  
\end{equation}
%
The statistical equilibrium equation (4) can be
written in terms of $S$ as
%  9
\begin{equation}
 S_i = {{{\overline J}_i + \epsilon B_i} \over {1 + \epsilon}}, 
\end{equation}
%
where $\epsilon = C_{21}/A_{21}$ and
$B_i = (2h{\nu^3}/c^2) \, \exp (-h{\nu}/kT_i)$
(the Planck function without stimulated emission). Combining
equations (8) and (9) gives a set of simultaneous equations
for $S$ at each depth.

This is basically the way the two-level problem should be solved
when $\epsilon$ is very small. Normally over $1/\epsilon$ iterations
would be required to obtain a solution by alternately evaluating
equations (8) and (9), i.e. by iterating between the radiative transfer
and statistical equilibrium equations. Accelerated Lambda Iteration
(ALI) techniques have been developed that allow this iterative
approach to succeed (see Rybicki \& Hummer 1991, 1992, Heinzel 1995,
and the review by Avrett 1996). For a single transition we have
found that solving the set of simultaneous equations is faster than
ALI. For multilevel cases (discussed below) however, ALI is
usually much faster because ALI can take account of the
interactions between transitions while computing the iterative
solutions for each transition.


\section{The multi-level transfer problem}

When there are many radiative transitions in the model atom,
equations (8) and (9) still can be applied
to each transition, but $\epsilon$ and $B$ now take a more
general form based on all transition pathways between the
upper and lower levels of the given transition, apart from
the direct radiative transition. Pandora is based on such a
generalization of the two-level solution. Since the $\epsilon$
and $B$ terms for a given radiative transition depend on the
solutions of other radiative transitions, one must iterate on
these terms even though the solution for each transition, given
$\epsilon$ and $B$, is exact. The advantage of this method is
that the individual solutions show the causes of calculated results
(e.g., why a given source function has a particular variation with
depth). The disadvantage is that this method generally requires
more computing than other methods developed in recent years.

As explained above, Pandora combines the radiative transfer and
the statistical equilibrium equations for a given radiative
transition into sets of simultaneous equations, one such set for
each of the $N_D$ depth points of the atmosphere.  For a model ion
with $N_L$ levels and a continuum, we have $N_L$ statistical
equilibrium equations analogous to equation (4). For
a model ion with $N_T$ transitions there are $N_T$ sets of
simultaneous equations that must be solved.
(If every radiative transition between $N_L$ levels is treated
in the ion model, then $N_T = N_L ( N_L -1)/2$.) These calculations
must be repeated iteratively because the $\epsilon$ and $B$ terms
for one radiative transition depend on the other radiative
transitions. Solving $N_T$ sets of
simultaneous equations for $N_D$ depths 
for many iterations can require much computing.
However, it is rarely necessary to solve all these sets of
simultaneous equations (i.e., to compute a ``full solution'') for
each of the $N_T$ radiative transitions of a particular model ion.
For all but the relatively strong line transitions it is sufficient
just to let Pandora iterate between equations (8) and (9), i.e.,
just to carry out ``Lambda iterations'' for the relatively weak
transitions. For other, often faster, methods in current use,
see the papers in Hubeny, Mihalas, \& Werner (2002).


\section{Non-local statistical equilibrium (non-LTE)}

We have also incorporated velocity terms in the Pandora
statistical equilibrium equations, so that equation (4)
would be written as
%  10
\begin{equation}
 -{d \over d z} (n_1 V) = n_1 \bigl( B_{12}{\overline J}
+ C_{12}\bigr) - n_2 \bigl( A_{21} + C_{21} \bigr) 
\end{equation}
%
to take account of the effect of a mass velocity $V$
(increasing in the direction of the $z$ coordinate). The
derivative term causes $n_1$ at each depth to depend on $n_1$ at
other depths, so that we use a finite difference procedure to
solve the differential equation for $n_1(z)$ assuming given
values for the other terms, including $n_2$.  Then we evaluate
%  11
\begin{equation}
 G = { 1 \over n_1} {d \over d z} \bigl( n_1 V \bigr), 
\end{equation}
%
and finally write equation (8) as
%  12
\begin{equation}
 0 = n_1 \bigl( B_{12} {\overline J} + C_{12} + G \bigr) 
- n_2 \bigl( A_{21} + C_{21} \bigr), 
\end{equation}
%
which is then solved as before.
In this way an advection term has been
introduced into the equations of statistical equilibrium. 
(The Doppler shift due to mass flow is also included in the
line absorption profile calculations.) The term on the left side
of equation (10) is more important for ionization than for
excitation (i.e., when $n_2$ represents the ion density rather
than excitation level 2).

We also include particle diffusion velocities when
there are steep temperature and ionization gradients. The
hydrogen atom and proton diffusion velocities are given by
$V_{\rm H} = [X/(1+X)]V_A$ and $V_{\rm p} = -[1/(1+X)]V_A$ 
where $X$ is the ionization fraction $n_{\rm p}/n_{\rm HI}$ and
%  13
\begin{equation}
 V_A = D_X { d \over d z} (\ln X)
+ D_T { d \over d z} (\ln T) 
\end{equation}
%
is the ambipolar diffusion velocity. The coefficients $D_X$
and $D_T$ are functions of the local number densities and the
temperature, respectively.  The diffusion velocity enters
the calculation in the same way as the mass flow velocity
except that hydrogen atoms and protons diffuse in opposite
directions. See Fontenla, Avrett, \& Loeser (1990, 1991)
for details.


\section{Implementation}

Rather than going through the calculations in a fixed way,
Pandora assumes a hands-on approach by the user. There are
many input switches for specifying 1) alternative methods for
specific calculations, 2) amounts of printout, 3) levels of
printout detail, and 4) auxiliary output files (for use by
other programs needing various Pandora-computed quantities).
The user need not specify any of these options, alternative
methods, or numerical control parameters at the start of a new
calculation because Pandora provides defaults for all of them.
There now exists a collection of atomic data input files for
model ions of general interest; in many cases more than one
version of a given ion is available, ranging from abbreviated
to detailed models.

Pandora proceeds by iterating. It 
computes a specified number of grand iterations and then stops,
having saved in various disk files all the data needed to resume
the calculation for another specified number of iterations.
Thus the evolving solution can be supervised closely and 
various control parameters and choices of method can be
adjusted for optimal progress.

The computational properties of some of the procedures
Pandora uses for various steps in the calculation have not
been studied in detail for all types of applications. What
has worked well for specific calculations in the past may
not work well when applied to new, different regimes. We
have provided many input control parameters to try to
prevent unreasonable numerical behavior.

For some steps in the calculation, e.g. the evaluation
of the $\Lambda_{ijk}$ coefficients in
equation (7), Pandora provides a menu of different
methods that have been found to be well suited for specific
situations. In each case the user can specify which method
to use. We have begun to document what we have learned about
the advantages and disadvantages of these alternatives, to
help others choose.


\section{Documentation}

There are four sources of information about the program.
1) The Pandora printout is intended to be reasonably
self-explanatory.  There are printing options that allow the
details of almost any calculation to be printed so that any step
can be studied or verified. 2) A User's Guide is available that
lists all input parameters, program options, and program outputs,
along with extensive explanations of parameters and options.
We continue to add material to this Guide. 3) The basic Pandora
documentation, begun  36 years ago in October 1966, now consists of
over 3400 pages of handwritten program notes specifying every
program change or addition. Finally, a readily accessible source
of information consists of 4) the derivations and explanations
included in the following publications:
%
%
\blankline
\noindent Avrett, E. H., \& Loeser, R. 1969, Formation of line and
continuous spectra. I. Source-function calculations, Smithsonian
Astrophys. Obs. Spec. Rept. No. 303, 99pp

\indent\hang [Complete derivation of the two-level and multi-level
statistical equilibrium and transfer equations, with integral 
operators used for the formal solution of the transfer equation.]
%
%
\blankline
\noindent Avrett, E. H. 1971, Solution of non-LTE transfer problems,
J. Quant. Spectrosc. Radiat. Transfer, 11, 511-529

\indent\hang [Basic formulation and discussion of non-LTE
computational methods.]
%
%
\blankline
\noindent Vernazza, J. E., Avrett, E. H., \& Loeser, R. 1973,
Structure of the solar chromosphere. I. Basic computations and summary
of results, ApJ 184, 605-631

\indent\hang [Derivation of the equations of statistical equilibrium and
radiative transfer for lines and the hydrogen Lyman-continuum; electron 
number density, hydrostatic equilibrium, and H$^-$ non-LTE equations.]
%
%
\blankline
\noindent Vernazza, J. E., Avrett, E. H., \& Loeser, R. 1976,
Structure of the solar chromosphere. II. The photosphere and the
temperature-minimum region, ApJS 30, 1-60

\indent\hang [Effect of line opacities throughout the spectrum based on
Kurucz data; continuum data calculations for wavelengths 0.13 -- 500 $\mu$m.]
%
%
\blankline
\noindent Vernazza, J. E., Avrett, E. H., \& Loeser, R. 1981,
Structure of the solar chromosphere. III. Models of the EUV
brightness components of the quiet Sun, ApJS 45, 635-725

\indent\hang [Derivation of the energy balance equations, L$\alpha$ partial
redistribution, supplementary levels in the statistical equilibrium
equations and in the Lyman-continuum transfer equation. Continuum
data calculations for wavelengths 40 nm -- 3 cm.]
%
%
\blankline
\noindent Avrett, E. H., and Loeser, R. 1984, Line transfer in static and
expanding spherical atmospheres, in Methods in Radiative Transfer,
ed. W. Kalkofen (Cambridge: Cambridge Univ. Press), 341-379

\indent\hang [Derivation of the line transfer equations in spherical
geometry with radial expansion and partial frequency redistribution.]
%
%
\blankline
\noindent Hartmann, L., \& Avrett, E. H. 1984, On the extended chromosphere
of $\alpha$ Orionis, ApJ 284, 238-249

\indent\hang [Application of Pandora calculations in spherical geometry 
with radial expansion.]
%
%
\blankline
\noindent Avrett, E. H. 1985, Recent thermal models of the chromosphere, in
Chromospheric Diagnostics and Modelling, ed. B. W. Lites
(Sunspot, NM: National Solar Observatory), 67-127

\indent\hang [Detailed calculations of net radiative cooling rates.]
%
%
\blankline
\noindent Avrett, E. H., Machado, M. E., \& Kurucz, R. L. 1986,
Chromospheric flare models, in The Lower Atmosphere in Solar Flares,
ed. D. F. Neidig (Sunspot, NM: National Solar Observatory), 216-281

\indent\hang [Describes the use of Kurucz's sampled line opacities in
photoionization rate calculations and spectrum synthesis.]
%
%
\blankline
\noindent Avrett, E. H., \& Loeser, R. 1987, Iterative solution of multilevel
transfer problems, in Numerical Radiative Transfer, ed. W. Kalkofen
(Cambridge: Cambridge Univ. Press), 135-161

\indent\hang [Derivation and discussion of the equivalent two-level
atom method of solving multi-level problems, with tabulated numerical
solutions.]
%
%
\blankline
\noindent Avrett, E. H., \& Loeser, R. 1988, Radiative transfer in the
broad emission-line regions of quasi-stellar objects, ApJ 331, 211-246

\indent\hang [Energy-balance non-LTE model atmosphere calculations with
incident radiation, heavy-element cooling, and X-ray absorption.]
%
%
\blankline
\noindent Mauas, P. J., Avrett, E. H., \& Loeser, R. 1988,
Mg-I as a probe of the solar chromosphere, 
ApJ 330, 1008-1021

\indent\hang [Includes a discussion of how to combine multiplet transitions
in a simple way.]
%
%
\blankline
\noindent Luttermoser, D. G., Johnson, H. R., Avrett, E. H., \& Loeser, R. 1989,
Chromospheric structure of cool carbon stars, ApJ 345, 543-553

\indent\hang [Modeling of N-type carbon stars with PRD calculations 
of the Mg-II lines.]
%
%
\blankline
\noindent Mauas, P. J., Avrett, E. H., \& Loeser, R. 1989, Computed profiles
of the C-I multiplets at $\lambda$1561 and $\lambda$1657,
ApJ 345, 1104-1113

\indent\hang [Partial frequency redistribution in multiplet lines.]
%
%
\blankline
\noindent Fontenla, J. M., Avrett, E. H., \& Loeser, R. 1990, Energy balance
in the solar transition region. I. Hydrostatic thermal models with
ambipolar diffusion, ApJ 355, 700-718

\indent\hang [Derivation of the effects of hydrogen ambipolar diffusion
on the hydrogen number density calculations and on energy transport.]
%
%
\blankline
\noindent Mauas, P. J., Avrett, E. H., \& Loeser, R. 1990, On carbon monoxide
cooling in the solar atmosphere, ApJ 357, 279-287

\indent\hang [CO opacity and net radiative cooling rate calculations.]
%
%
\blankline
\noindent Mauas, P. J., Machado, M. E., \& Avrett, E. H. 1990, The white light
flare of June 15, 1982: Models, ApJ 360, 715-726

\indent\hang [Detailed set of flare model calculations using Pandora.]
%
%
\blankline
\noindent Fontenla, J. M., Avrett, E. H., \& Loeser, R. 1991, Energy balance 
in the solar transition region. II. Effects of pressure and energy input
on hydrostatic models, ApJ 377, 712-725

\indent\hang [Results of model calculations with hydrogen ambipolar
diffusion for quiet and active regions of the solar atmosphere.]
%
%
\blankline
\noindent Chang, E. S., Avrett, E. H., Mauas, P. J., Noyes, R. W., \&
Loeser, R. 1991, Formation of the infrared emission lines of 
Mg-I in the solar atmosphere, ApJ 379, L79-L82

\indent\hang [Line transfer calculation with a 41-level atomic
model for Mg-I, including charge exchange and 
collisions with hydrogen atoms.]
%
%
\blankline
\noindent Fontenla, J. M., Avrett, E. H., \& Loeser, R. 1993, Energy balance
in the solar transition region. III. Helium emission in hydrostatic,
constant-abundance models with diffusion, ApJ 406, 319-345

\indent\hang [Calculation of solar He-I and He-II lines.]
%
%
\blankline
\noindent Avrett, E. H., Chang, E. S., \& Loeser, R. 1994, Modeling the infrared
magnesium and hydrogen lines from quiet and active solar regions, in
Infrared Solar Physics, eds. D. M. Rabin, J. T. Jefferies, \& 
C. Lindsey (Dordrecht: Kluwer), 323-339

\indent\hang [Calculation of infrared line spectra.]
%
%
\blankline
\noindent Avrett, E. H., Fontenla, J. M., \& Loeser, R. 1994, Formation of the
solar 10830 A line, in Infrared Solar Physics, eds. D. M. Rabin,
J. T. Jefferies, \& C. Lindsey (Dordrecht: Kluwer), 35-47

\indent\hang [Modeling the absorption of the infrared continuum by the
He-I line due to coronal-line illumination.]
%
%
\blankline
\noindent Avrett, E. H. 1995, Two-component modeling of the solar IR CO lines, in
Infrared Tools for Solar Astrophysics: What's Next, eds. J. Kuhn \&
M. Penn (Singapore: World Scientific), 303-311

\indent\hang [Construction of models based on spacecraft observations
of the infrared CO lines.]
%
%
\blankline
\noindent Avrett, E. H. 1996, Next generation model atmospheres, in IAU Symp. 176,
\break Stellar Surface Structure, eds. K. G. Strassmeier \& J. L. Linsky
(Dordrecht: Kluwer), 503-518

\indent\hang [Review of current stellar atmosphere modeling.]
%
%
\blankline
\noindent Avrett, E. H., Hoeflich, P., Uitenbroek, H., \& Ulmschneider, P. 1996,
Temporal Variations in solar chromospheric modeling, in Cool Stars,
Stellar Systems, and the Sun: Ninth Cambridge Workshop, eds. 
R. Pallavicini \& A. K. Dupree (San Francisco: ASP), {\bf 109}, 105

\indent\hang [Effects of time variation in the formation of CO lines.]
%
%
\blankline
\noindent Fontenla, J. M., Avrett, E. H., \& Loeser, R. 2002, Energy balance
in the solar transition region. IV. Hydrogen and helium mass flows with
diffusion, ApJ 572, 636-662

\indent\hang [Effects of flow velocities on energy-balance solutions.]


\begin{references}

%
\reference{} Anderson, L. S. 1989, ApJ, 339, 558
%
\reference{} Avrett, E. H. 1996, in Stellar Surface Structure, IAU Symp. 176, ed. 
K. G. Strassmeier \& J. L. Linsky (Dordrecht: Kluwer), 503
\reference{} {---}{---}{---}{---}. 2002, in Current Theoretical Models and Future High Resolution 
Solar Observatioins: Preparing for ATST, ed. A. A. Pevtsov \& H. Uitenbroek,
ASP Conference Series, in press
%
\reference{} Avrett, E. H., and Loeser, R. 1984, in
Methods in Radiative Transfer, ed. W. Kalkofen
(Cambridge: Cambridge Univ. Press), 341
\reference{} {---}{---}{---}{---}. 1988, ApJ, 331, 211
\reference{} {---}{---}{---}{---}. 1992, in Cool Stars, Stellar Systems, and
the Sun, ed. M. S. Giampapa \& J. A. Bookbinder 
(San Francisco: ASP), {\bf 26}, 489
%
\reference{} Avrett, E. H., Machado, M. E., and Kurucz, R. L. 1986
in The Lower Atmosphere in Solar Flares, ed. D. F. Neidig
(Sunspot, NM: National Solar Observatory), 216
%
\reference{} Carlsson, M. \& Stein, R. G. 1997, ApJ, 481, 500
\reference{} {---}{---}{---}{---}. 1999, in AIP Conf. Proc. 471, Solar Wind 9, 
ed. S. R. Habbal (New York: AIP), 23
%
\reference{} Cooper, J., Ballagh, R. J., \& Hubeny, I. 1989, ApJ, 344, 949
%
\reference{} Fontenla, J. M., Avrett, E. H., \& Loeser, R. 1990, ApJ, 355, 700.
\reference{} {---}{---}{---}{---}. 1991, ApJ, 377, 712
\reference{} {---}{---}{---}{---}. 2002, ApJ, 572, 636
%
\reference{} Heinzel, P. 1995, A\&A, 299, 563
%
\reference{} Hubeny, I., \& Lanz, T. 1995, ApJ, 439, 875
%
\reference{} Hubeny, I., Mihalas, D., \& Werner, K., eds. 2002, Stellar Atmospheric
Modeling, ASP Conference Series, in press
%
\reference{} Kalkofen, W. 2001, ApJ, 557, 376
%
\reference{} Kurucz, R. 1979, ApJS, 40, 1
\reference{} {---}{---}{---}{---}. 1970, Atlas: A computer program for calculating model stellar
atmospheres, Smithson. Astrophys. Obs. Spec. Rept. No. 309, 291 pp.
\reference{} {---}{---}{---}{---}. 1991, in 
Stellar Atmospheres: Beyond Classical Models,
ed. L. Crivellari, I. Hubeny, \& D. G. Hummer (Dordrecht: Kluwer), 440
\reference{} {---}{---}{---}{---}. 1998, in Fundamental Stellar Properties: The Interaction 
between Observation and Theory, ed T. R. Bedding, A. J. Boothe \& J. Davis, IAU Symp. 
189 (Dordrecht: Kluwer) 217
%
\reference{} Linsky, J. L. 1985, in Progress in Stellar Spectral Line Formation Theory,
ed. J. E. Beckman \& L. Crivellari (Dordrecht: Reidel), 1
%
\reference{} Mauas, P. J., Avrett, E. H., \& Loeser, R. 1989, ApJ 345, 1104
%
\reference{} Mihalas, D. 1978, Stellar Atmospheres, (San Francisco: Freeman)
%
\reference{} Rybicki, G. B., \& Hummer, D. G. 1991, A\&A, 245, 171
\reference{} {---}{---}{---}{---}. 1992, A\&A, 262, 209
%
\reference{} Uitenbroek, H. 2001, ApJ, 557, 389
%
\reference{} Vernazza, J. E., Avrett, E. H., and Loeser, R. 1973, ApJ, 184, 605
\reference{} {---}{---}{---}{---}. 1981, ApJS, 45, 635

\end{references}

\blankline
\blankline
\blankline
\blankline
\noindent This paper was prepared for a conference in Summer 2002: 
{\it Modeling of Stellar Atmospheres} (I.A.U. Symposium 210), 
ed. W. Weiss \& N. Piskunov (Dordrecht: Kluwer), in press.
\blankline
\noindent 2002 Aug 06

\end{document}

