\magnification=1200
\input wupstuff.tex
\newtoks\footline \footline={\hss\tenrm 19.\folio\hss}
%
\pageno=1
%
\top
\vskip 1.5 true in
\centerline{Section 19: {\bf Atomic Models}}
\blankline
\blankline
\centerline{\bf ***}
\blankline
\blankline
PANDORA uses the numeric parameters of a model of the ion-of-the-run
to compute the line source functions and emergent line profiles of interest.
The user must see to it that the minimum required set of atomic model
parameters is specified in the input. Some of these atomic parameters
(in the case of Hydrogen: most) can take on default values supplied
by the program.

Over the years, and with the the help of many people, we have compiled and
collected several dozen atomic model data sets. These data sets are known
as ``atomic model files'' and are available for general use.
\blankline
The purpose of this section is to provide some general guidance for setting
up an atomic model and to describe the available defaults. Please refer to
Section 5 for complete details about the various parameters mentioned here.

Note that PANDORA-supplied default parameter values are written to file
{\tt fort.20} (\ie {\tt .msc}) when option ATOMSAV is on (see Section 8).
\blankline
There now follow: 1) General Information; 2) Defaults For Ions Other Than
Hydrogen; 3) Defaults for Hydrogen; and 4) More About Collision Rates.
\ej
\centerline{\bf 1) General Information}
\blankline
An atomic model consists of a set of energy levels, a continuum level,
specifications of the properties of levels, and specifications of the
properties of transitions between levels and between levels and the continuum.
(As well, PANDORA uses many other atomic model-related parameters that
pertain to the details of various specific computational and organizational
procedures.)
\blankline
\centerline{Levels}
\blankline
The number of energy levels (`regular levels') is {\bf NL}, $({\bf NL} 
\geq 2)$. Additional higher levels can be included in the calculations in
an approximate way as `supplementary levels' by setting
${\bf NSL} > {\bf NL}$. (Normally these are not used, and the program
sets ${\bf NSL} = {\bf NL}$.)

The principal quantum numbers $n$ and $\ell$ for each level are specified by
means of the {\bf NLPAIR} statement. (Another parameter {\bf QNL}, the
``number of $n\ell$ electrons,'' must be specified for each level if default
{\bf CI} values [see below] are used.)

The value of ${\bf NU}^j, 1 \leq j \leq {\bf NSL}$, is the energy of
level $j$ in frequency units (in units of $10^{15}$ Hz);
the convention ${\bf NU}^1 = 0$ is enforced.
The value of {\bf NUK}, similarly, is the energy of the continuum.
If the higher level(s) of the ion model are `autoionizing' level(s),
then auxiliary continuum energies, ${\bf NUC}^j > {\bf NUK}$, can be
specified (the default value of ${\bf NUC}^j = {\bf NUK}$).

Other parameter values needed for each level include: {\bf P}, statistical
weight; {\bf CP}, photoionization cross-section; and values of {\bf ELSYM} and
{\bf IONSTAGE} (e.g., for Mg-II, {\bf ELSYM} = MG and {\bf IONSTAGE} = 2).
\blankline
The collisional ionization rate is ${\bf CI}^j_i \times {\bf NE}_i \times
\exp(-h \, {\bf NU}^j \times 10^{15} / k \, {\bf TE}_i)$, where {\bf CI} is the
collisional ionization coefficient. Values of {\bf CI} can also be
provided in the input. {\bf CI} can be given as functions of a specified short
temperature table ${\bf TER}_k, \; 1 \leq k \leq {\bf NTE}$. When values of
${\bf CI}^j_i$ at particular depths $i$ (\ie for particular values of
${\bf TE}_i$) are required as the calculation proceeds, they are then
obtained by interpolation from these given tables of
${\bf CI}^j({\bf TER}_k), \; 1 \leq j \leq {\bf NSL}, \; 1 \leq k \leq {\bf NTE}.$
When ${\bf NTE} = 1$, then the constant value ${\bf CI}^j$ is used for all
temperatures.

However, {\bf CI} can also be computed ``on-the-fly''
for the specific value of {\bf TE}$_i$ (and {\bf NC}$_i$ in a Hydrogen run)
at every depth $i$; see ({\bf 4}), below.
\ej
% \blankline
\centerline{Transitions between levels}
\blankline
PANDORA computes the collisonal transitions, both due to electrons and due
to hydrogen, between all levels, and from all levels to the continuum
(thus providing rates between levels via the continuum). However radiative
transitions are computed only between specified pairs of levels $(u,\ell), 
\; u > \ell$. There are {\bf NT} such pairs, which are specified by means of the
{\bf INPAIR} statement. The ``radiative'' transitions so specified, depending
on how PANDORA should treat them, can be of five different types as indicated
by the parameter {\bf KTRANS}$^{u,\ell}$: {\tt RADIATIVE, PASSIVE,
THICK, THIN,} and {\tt 2-PHOTON}. Note that {\tt RADIATIVE} transitions
can be designated as `blended lines'; this is triggered by setting
${\bf LDL}^{u,\ell} > 1$ (see also input parameter {\bf LDLMAX}).

Values of the Einstein parameter ${\bf A}^{u,\ell}$ are required for all
{\tt RADIATIVE} and {\tt PASSIVE} transitions, and may be specified for
{\tt THICK} transitions. Values of the broadening parameters 
${\bf CRD}^{u,\ell}$ (radiative), ${\bf CVW}^{u,\ell}$ (van der Waals),
${\bf CSK}^{u,\ell}$ (Stark) and ${\bf CRS}^{u,\ell}$ (resonance) may
be specified for {\tt RADIATIVE} and {\tt PASSIVE} transitions.
\blankline
The collisional de-excitation rate is ${\bf CE}^{u,\ell}_i \times {\bf NE}_i
\times ({\bf P}^u/{\bf P}^\ell)$, while the excitation rate is
${\bf CE}^{u,\ell}_i \times {\bf NE}_i \times \exp[-h \, ({\bf NU}^u-{\bf NU}^\ell)
\times 10^{15}/k \, {\bf TE}_i]$; here {\bf CE} is the excitation coefficient.
Values of {\bf CE} may also be provided in the input. {\bf CE} can be
given as functions of the short temperature table ${\bf TER}_k, \; 1 \leq k \leq
{\bf NTE}$, mentioned above. When values of
${\bf CE}^{u,\ell}_i$ at particular depths $i$ (\ie for particular values of
${\bf TE}_i$) are required as the calculation proceeds, they are then obtained
by interpolation from these given tables of ${\bf CE}^{u,\ell}({\bf TER}_k), 
\; 2 \leq u \leq {\bf NL}, \; 1 \leq \ell < u, \; 1 \leq k \leq {\bf NTE}$.
When ${\bf NTE} = 1$, then the constant value ${\bf CE}^{u,\ell}$ is used
for all temperatures.

However, {\bf CE} can also be computed
``on-the-fly'' for the specific value of {\bf TE}$_i$ (and {\bf NC}$_i$
in a Hydrogen run) at every depth $i$; see (4), below.
\blankline
\centerline{Minimum input}
\blankline
Values for at least the following atomic model parameters are required
in order to be able to do anything at all: {\bf NL}, {\bf NU}, {\bf NUK},
{\bf P}, {\bf CP}, {\bf NT}, {\bf A}, and values of {\bf CI} and {\bf CE}
for one value of {\bf TER} (for, say, the default value 4000 K).
\ej
% \blankline
\centerline{\bf 2) Defaults For Ions Other Than Hydrogen}
\blankline
a) Unspecified values of
${\bf CI}^j({\bf TER}_k), \; 1 \leq j \leq {\bf NSL}, \; 1 \leq k 
\leq {\bf NTE}$ are replaced by computed ones. Three different methods are
available; the (set of) method(s) to be used can be specified in a
CIMETHOD statement. The \break CIMETHOD statement lists methods by codename
as follows:

{\tt CLARK}: \quad Clark, Abdallah, and Mann (1991), ApJ {\bf 381}, 597;

{\tt AR}: \quad Arnaud and Rothenflug (1985), A\&A {\bf 60}, 425;

{\tt VORONOV}: \quad Voronov (1997), Atomic Data and Nuclear Tables {\bf 65}, 1. \np
{\tt AR} and {\tt VORONOV} can only be used for level $j = 1$, of some ions;
{\tt CLARK} can be used in all cases.

The default methods set is: {\tt AR} for $j = 1$ if possible,
and {\tt CLARK} otherwise.
\blankline
b) Unspecified values of {\bf KTRANS}$^{u,\ell}$, for all $(u,\ell)$ pairs
mentioned in {\bf INPAIR}, are set equal to {\tt RADIATIVE}.
\blankline
c) If {\bf CP} is not specified, then for level 1 a default value is
computed according to Verner {\it et al.} (1996), ApJ, {\bf 465}, 487
if they list data for this ion. If not, and for other levels, default
values are obtained from the hydrogenic approximation.
\blankline
d) Unspecified values of
${\bf CE}^{u,\ell}({\bf TER}_k), \; 2 \leq u \leq {\bf NL}, \;
1 \leq \ell < u, \break 1 \leq k \leq {\bf NTE}$ are replaced by computed ones.
Two different methods are available; the (set of) method(s) to be used
can be specified in a CEMETHOD statement. The CEMETHOD statement lists
methods by codename as follows:

{\tt SEATON}: \quad Seaton (1962), Proc.Phys.Soc. {\bf 79}, 1105;

{\tt VREGE}: \quad van Regemorter (1962), ApJ {\bf 136}, 906. \np
{\tt SEATON} can only be used for transitions with nonzero
{\bf A}-values, of neutral atoms; {\tt VREGE} can be
used for all cases. The {\tt VREGE} method needs {\bf A}-values.
If ${\bf A}^{u,\ell} = 0$ then theoretical values of ${\bf A}^{u,\ell}$
for such `forbidden' transitions are computed assuming that the
oscillator strength of the transition is given by the parameter
{\bf FROSCE} (whose default value = 0.01).

The default methods set is: {\tt SEATON} for neutral atoms,
and {\tt VREGE} otherwise.
\blankline
e) Unspecified values of {\bf CRD}$^{u,\ell}$, {\bf CVW}$^{u, \ell}$,
and {\bf CSK}$^{u, \ell}$ are updated with computed ones.
The calculation of {\bf CRD}-vales uses all available {\bf A}-values.
\blankline
f) If all values of {\bf RRCP} for any level are unspecified,
default sets for such levels
are computed either according to Verner {\it et al.} (see (c), above),
or using $1/{\nu^3}$.
\ej
% \blankline
\centerline{\bf 3) Defaults For Hydrogen}
\blankline
a) Unspecified values of {\bf NU}$^j$, {\bf P}$^j$ and {\bf CP}$^j, 
\; 1 \leq j \leq {\bf NSL}$, and of {\bf NUK} are replaced by
computed ones.
\blankline
b) Unspecified values of
${\bf CI}^j({\bf TER}_k), \; 1 \leq j \leq {\bf NSL}, \; 1 \leq k 
\leq {\bf NTE}$ are replaced by computed ones. Five different methods are
available; the (set of) method(s) to be used can be specified in a
CIMETHOD statement. The \break CIMETHOD statement lists methods by codename
as follows:

{\tt CLARK}: \quad Clark, Abdallah, and Mann (1991), ApJ {\bf 381}, 597;

{\tt AR}: \quad Arnaud and Rothenflug (1985), A\&A {\bf 60}, 425;

{\tt JOHNSON}: \quad Johnson (1972), ApJ {\bf 174}, 227;

{\tt VS}: \quad Vriens and Smeets (1980), Phys.Rev.A {\bf 72}, 940;

{\tt SHAH}: \quad Shah, Elliott, and Gilbody (1987), J.Phys.B {\bf 20}, 3506. \np
{\tt AR} and {\tt SHAH} can be used for level $j = 1$ only; the others
can be used in all cases.

The default methods set is: {\tt SHAH} for $j = 1$, and {\tt CLARK} otherwise.
\blankline
c) Unspecified values of
${\bf CE}^{u,\ell}({\bf TER}_k), \; 2 \leq u \leq {\bf NL}, \;
1 \leq \ell < u, \break 1 \leq k \leq {\bf NTE}$ are replaced by computed ones.
Six different methods are available; the (set of) method(s) to be used can be
specified in a CEMETHOD statement. The CEMETHOD statement lists
methods by codename as follows:

{\tt SEATON}: \quad Seaton (1962), Proc.Phys.Soc. {\bf 79}, 1105;

{\tt VREGE}: \quad van Regemorter (1962), ApJ {\bf 136}, 906;

{\tt SCHOLZ}: \quad Scholz {\it et al.} (1990), MNRAS {\bf 242}, 692;

{\tt PB}: \quad Przybilla and Butler (2004), ApJ {\bf 609}, 1181; 

{\tt AGGRWL}: \quad Aggarwal {\it et al.} (1991), J.Phys.B {\bf 24}, 1385;

{\tt VS}: \quad Vriens and Smeets (1980), Phys.Rev.A {\bf 22}, 940;

{\tt JOHNSON}: \quad Johnson (1972), ApJ {\bf 174}, 227. \np
{\tt SCHOLZ} can be used for transition (2/1) only;
{\tt PB} works for all transitions $(u,\ell), \break u \leq 7$;
{\tt AGGRWL} works for all transitions $(u,\ell), \; u \leq 5$;
the others can be used for all cases.

The default methods set is: {\tt AGGRWL} for all transitions it works for,
and {\tt JOHNSON} otherwise.
\blankline
d) Unspecified values of ${\bf A}^{u,\ell}$, for all $(u,\ell)$ pairs
mentioned in {\bf INPAIR}, are computed according to Johnson (1972), ApJ
{\bf 174}, 227 (uses 'lowering of the ionization potential when
{\bf IXNCS} = 1). All unspecified values of ${\bf KTRANS}^{u,\ell}$, for
all $(u,\ell)$ pairs mentioned in {\bf INPAIR}, are set equal to 
{\tt RADIATIVE}.
\blankline
e) Unspecified values of {\bf CRD}$^{u,\ell}$, {\bf CVW}$^{u,\ell}$,
{\bf CSK}$^{u,\ell}$, and {\bf CRS}$^{u,\ell}$ are updated with computed ones.
The calculation of {\bf CRD}-values uses all available {\bf A}-values;
the calculation of {\bf CSK}-values uses the method of Sutton (1978),
JQSRT {\bf 20}, 233.
\ej
% \blankline
\centerline{\bf 4) More About Collision Rates}
\blankline
When PANDORA computes values of {\bf CI} and/or {\bf CE},
as described in (2) and (3) above, such raw computed values are modified
before use, as follows:
$$CI^j_{default} = ACI^j + MCI^j \times CI^j_{raw}$$
$$CE^{u,\ell}_{default} = ACE^{u,\ell} + MCE^{u,\ell} \times CE^{u,\ell}_{raw}$$
(The default values of the input parameters {\bf ACI} and {\bf ACE} are all 0,
and of the input parameters {\bf MCI} and {\bf MCE} are all 1.)
\blankline
\blankline
Before any input or computed {\bf CI} and {\bf CE} values are used in
the calculations, they are all multiplied by input parameter {\bf RFAC}
(default = 1).
\blankline
\blankline
When option CEFACTS is on and any values of ${\bf PCE}^{u, \ell} > 0$, then
sets of ${\bf FCE}^{u, \ell}_i$ are computed and adjusted to prevent
$S^{u, \ell}_i \leq 0$. Such {\bf FCE} values are saved for use in restart runs.
\blankline
\blankline
If {\tt ONTHEFLY} was specified for {\bf CI} in a CIMETHOD statement, and/or
for {\bf CE} in a CEMETHOD statement, then the tabulated values in the ATOM
printout section are for information only. Actual values of {\bf CI} and/or
{\bf CE} will be computed at every depth as needed.

In a Hydrogen run computed default {\bf CI} and {\bf CE} values using
{\tt JOHNSON} or {\tt VS} also depend on {\bf NC} when the value of the input
parameter {\bf IXNCS} = 1 (the default is {\bf IXNCS} = 0, {\it i.e.} the
{\bf NC}-dependent factor is set = 1). When the {\bf NC}-dependent factor
is computed explicitly it is intended to account for the `lowering of the
ionization potential.'
\blankline
\blankline
A comparison of the various default {\bf CI} and {\bf CE} values appears
in the printout section INPUT NOTES. These sample values are computed using
{\bf TE}$_i$, where $i$ is the input parameter {\bf INCEI}.
%\vfill \vfill
%\vfill \vfill
%\vfill \vfill
%\vfill \vfill
%\vfill \vfill
%\vfill \vfill
%\vfill \vfill
%\vfill 
\vfill
\noindent (Section 19 -- last revised: 2007 Apr 18) %\par
\message{Section 19 ends at page 19.\the\pageno}
\ej
%\end
