%\magnification=1200
%\input wupstuff.tex
\newtoks\footline \footline={\hss\tenrm 13.\folio\hss}
\pageno=1
\top
\vskip 1.5 true in
\centerline{Section 13: {\bf Auxiliary Depth Tables}}
\blankline
\blankline
\centerline{\bf ***}
\blankline
\blankline
PANDORA deals with many depth-dependent quantities. When these exist in
tabular form, they are specified as functions of the set of discrete 
geometrical depth values that constitute the standard kilometer depth
table {\bf Z}$_i, 1 \leq i \leq ${\bf N}.

However, it may happen that when such a quantity is to be input to
PANDORA, it is known only as a function of a different set of depth values.
When that is the case, such a different set of depth values may be input
as an auxiliary depth table. Upon reading tables corresponding to
that auxiliary Z table, PANDORA will perform, {\it during the input
reading process}, the interpolations necessary to convert such tables
to functions of the actual {\bf Z}-scale of the run. After input reading
is finished, and before beginning the actual calculations, PANDORA
discards the auxiliary Z tables.

The remainder of this section discusses in detail how to introduce and
use auxiliary Z tables.
\blankline
Every auxiliary Z table, {\bf ZAUX}, has a unique identifier $m$,
$1 \leq  m \leq 50$. \break By definition, the value $m = 0$ refers
to the standard kilometer depth table of the run, {\bf Z}.

To introduce an auxiliary Z table, it is only necessary to specify its
length, {\bf LZA}$_m$, in an {\tt LZA} statement in Part B of the
input (see Section 4), and then the table itself, ${\bf ZAUX}^m_i, 1 \leq i
\leq {\bf LZA}_m$, in a {\tt ZAUX} statement in a subsequent Section
of the input file.
\ej
There are three ways of referring to ({\it i.e.} specifying) a depth
table when specifying input values of depth-dependent quantities:

\noindent 1) If none is referrenced explicitly, or if $m = 0$, then
the standard kilometer depth table of the run, {\bf Z}, is assumed.

\noindent 2) The auxiliary depth table index {\bf m} can be specified in the
same input statement as the corresponding input quantity itself. Input
statements of the \break Type {\bf 2*}, {\bf 3*} and {\bf 5*} (see
Section 2) have been provided for this purpose. When the value of $m$ is
specified in this manner, then it affects {\it that input statement only}.

\noindent 3) A global value of auxiliary depth table identifier $m$ can be
specified as the input parameter {\bf MAUX}. (The default value of
{\bf MAUX} = 0.) Upon encountering a {\tt MAUX} statement, PANDORA assumes
that all depth-dependent quantities that follow (except those that use input
statement Types {\bf 2*}, {\bf 3*}, or {\bf 5*}) correspond to the auxiliary
depth table whose index was specified in that statement, and will perform
the necessary interpolations mentioned above. A value of $m$ as set by
a {\tt MAUX} statement remains in effect until countermanded by a subsequent
{\tt MAUX} statement. 

The auxiliary depth table identifier value $m = 0$ is permitted in
{\tt MAUX} statements, but not in {\tt LZA} or {\tt ZAUX} statements.
\blankline
Since the interpolation from {\bf ZAUX} to {\bf Z} is done during the
reading of the input statements (as opposed to being done after all the
input has been read), it is necessary that a given {\bf ZAUX} table
must have already occurred in the input before it is first referred to,
and the {\bf Z} table must already have occurred in the input before
the first ZAUX-dependent input is read.
\blankline
\blankline
{\it Note}: It is normally possible (indeed, convenient at times)
to use several distinct, non-contiguous input statements to specify
the sequence of elements of an input array (using {\bf I}, the
explicit facility for specifying the index of the `array member
currently being read into'; see Section 2). 

{\it This is not possible with arrays that are functions of 
auxiliary Z tables.} 

For such arrays (even when PANDORA only needs to `go through the 
motions' of interpolation), all elements must be specified in 
{\it one} input statement.
\blankline
\blankline
\centerline{{\bf Note}}
\noindent Many years' experience with auxiliary Z tables
has shown that their use causes confusion; it seems best only to
use them as temporary expedients.
\blankline
%\blankline
%\vfill \vfill
%\vfill \vfill
%\vfill \vfill
%\vfill \vfill
%\vfill \vfill
%\vfill \vfill
%\vfill \vfill
%\vfill 
\vfill
\noindent (Section 13 -- last revised: 1998 Sep 17) \par
\message{Section 13 ends at page 13.\the\pageno}
\ej
%\end
