%\magnification=1200
%\input wupstuff.tex
\newtoks\footline \footline={\hss\tenrm 3.\folio\hss}
\pageno=1
\top
\vskip 1.5 true in
\centerline{Section 3: {\bf Input Files Set-Up}}
\blankline
\blankline
\centerline{\bf ***}
\blankline
\blankline
Years ago I thought it would be ideal if input statements could appear in
PANDORA's input files in any order whatever; it seemed to me that this
was almost required in the interest of consistency with self-identified
free-field input. It became clear, however, that in order to do this
I would have to use a two-pass input reading procedure instead of the
current single-pass one; since, for programming purposes, certain parameters
must already have been specified before others can be accepted. I have not
yet been persuaded that the extra complication of a two-pass procedure
is worth it. Thus input statements may not appear in whatever order a
user might like best, but must conform to the structure described here.
(Don't overlook the small number of rather detailed special requirements
noted below.)
\blankline
The set of input statements for a run is divided into nine parts, A -- I.
Their contents are sketched below. All these nine parts must appear,
and in that order. The input statements in parts B, D, F and H may appear
in any order (except as noted below). However any of
B, D, F and/or H may be empty. Further information about input
reading and processing appears in Section 4. 
\blankline
The following subsections describe: 1) overall input file structure;
2) special requirements (concerning maximum array sizes, and specific
statement order); and 3) input error handling.
\ej
\centerline{1. \underbar{Input File Structure}}
\blankline
\blankline
\underbar{Part A} must contain: {\bf HEADING}.
\blankline
\underbar{Part B} contains mainly table lengths and option settings,
as well as other control parameters.
\blankline
\underbar{Part C} must contain: {\bf GO}.
\blankline
\underbar{Part D} contains all the input parameters for the basic
calculations, except that populations data appear in part H.
\blankline
\underbar{Part E} must contain: {\bf GO}.
\blankline
\underbar{Part F} contains input parameters for the ``spectrum''
calculations.
\blankline
\underbar{Part G} must contain: {\bf GO}.
\blankline
\underbar{Part H} contains populations data and related parameters.
\blankline
\underbar{Part I} must contain: {\bf GO}.
\blankline
\blankline
\blankline
\blankline
\blankline
\blankline
\blankline
The {\bf HEADING} is a single comment line, which will be printed on the
`banner' pages at the start of the printed output, and will be inserted
at various \break places in the restart file(s) (see Section 8).
\blankline
The {\bf GO} lines must appear as shown; they indicate the ends of parts
B, D, F and H, respectively. The input statement after
{\bf GO} must begin on a new line.
\blankline
The complete list of all input parameters is given in Section 5. That
list specifies which of the parts B, D, F, or H each parameter may
appear in. The names of input parameters may be given in UPPER-, 
lower-, or MixEd-case. The complete set of input statements need not
occur in just one file, but may be distributed over several files
that are coordinated by {\bf USE} statements (see Section 5, note 44).
\ej
\centerline{2. \underbar{Special Requirements}}
\blankline
\blankline
\centerline{In Part B:}
\bull {\bf NT} must occur before {\bf INPAIR};
\bull {\bf NSL} must be less than 51;
\bull {\bf NSL} must occur before {\bf MR} and/or {\bf LR};
\bull {\bf NL} must be less than {\bf NSL};
\bull {\bf NMT} must be less than 51;
\bull {\bf NCL} must be less than 100;
\bull {\bf NAB} must be less than 100;
\bull {\bf NAB} must occur before {\bf BANDL}, {\bf BANDU}, or {\bf BANDE};
\bull {\bf NVX} must be less than 100.
\blankline
\blankline
\centerline{In Part D:}
\bull {\bf LDL}$_{i,j}^{u,\ell}$ must not exceed {\bf LDLMAX};
\bull {\bf LDL}$_{i,j}^{u,\ell}$ must occur before 
{\bf DDL}$_{i,j}^{u,\ell}$, {\bf CDL}$_{i,j}^{u,\ell}$, and the
broadening halfwidths for transition $(u,\ell)$;
\bull {\bf KST}$^{u,\ell}$ must not exceed {\bf KSTMAX};
\bull {\bf KST}$^{u,\ell}$ must occur before {\bf XISYMT}$^{u,\ell}$;
\bull {\bf KBT}$^{u,\ell}$ must not exceed {\bf KBTMAX};
\bull {\bf KBT}$^{u,\ell}$ must occur before {\bf XIBLUT}$^{u,\ell}$;
\bull {\bf KRT}$^{u,\ell}$ must not exceed {\bf KRTMAX};
\bull {\bf KRT}$^{u,\ell}$ must occur before {\bf XIREDT}$^{u,\ell}$.
\blankline
\blankline
\centerline{Also:}
\bull the total number of Composite Line Opacity wavelengths, from all
bands, must be less than 20001 (see also Section 9);
\bull {\bf MAUX} must be less than 51 (see also Section 13).
\ej
\centerline{3. \underbar{Input Errors}}
\blankline
\blankline
Several types of errors commonly occur: violations of input language
syntax; violations of prescribed structure; typos in field names and
control character names; logical inconsistencies ({\it i.e.}, 
specifying undefined transitions, giving more array members than
the stated size); diddly errors related to comments or field
separators (too few or too many blanks).

It can be frustratingly time-consuming to correct all errors in a 
newly-typed input file. Many extensive, detailed error diagnostics
have been provided to help with this. Most input error notifications
provided by PANDORA display the image of the current input line, and
then a single asterisk on the printout line immediately below that 
input line. This asterisk is positioned beneath the terminating blank
(break character) of the input field which triggered the error
notification.
\blankline
When option DELABORT is on, PANDORA attempts to ignore an erroneous
input statement (by attempting to find its closing ``)'') in order
to read all the input (perhaps encountering additonal errors and
attempting to ignore them as well) before stopping. This does not
always work well, but when it does work, it saves time.

(Option DELABORT is on by default; turning it off
[{\tt OMIT ( DELABORT ) }]
will affect the reading of the remaining input statements only.)
\blankline
Note the input parameter {\bf JSTIN} (also, equivalently, available
as an option) which stops the run after all input has been
read and most of the preprocessing leading up to the first iteration
has been done. Such an
``input-only'' run typically takes little time and provides an
efficient means of checking the input for a run starting from
scratch.
\blankline
Since most PANDORA runs are restarts of preceding runs, the input
files for a given run are usually obtained through only minor
incremental changes to the files of preceding runs. Thus, in
practice, only few input errors tend to occur once a new series of
runs (restarts) has been launched successfully.
\blankline
%\blankline
%\vfill \vfill
%\vfill \vfill
%\vfill \vfill
%\vfill \vfill
%\vfill \vfill
%\vfill \vfill
%\vfill \vfill
%\vfill 
\vfill
\noindent (Section 3 -- last revised: 2007 Apr 11) \par
\message{Section 3 ends at page 3.\the\pageno}
\ej
%\end
