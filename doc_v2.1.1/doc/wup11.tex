%\magnification=1200
%\input wupstuff.tex
\newtoks\footline \footline={\hss\tenrm 11.\folio\hss}
\pageno=1
\top
\vskip 1.5 true in
\centerline{Section 11: {\bf Printout}}
\blankline
\blankline
Initial PANDORA development occurred in a batch-oriented environment
using a remote central computing system. Output was written to a file
that was printed sometime after completion of the run. The printers
were expected to use 11$\times$14${7\over8}''$ fan-fold paper providing
about 60 lines per page and 132-136 character positions per line.
The resulting hardcopy was called ``output'' or ``printout.''

Things have changed. Now PANDORA runs on desktop workstations. (But it
is not an interactive program---things have not changed that much.)
The user still deals with output files and normally waits for a run
to finish before doing so. Complete output files are no longer routinely
printed but are stored on disk to be displayed and examined on
CRT screens. Nevertheless, in this writeup we retain the old terms
``output'' and ``printout'' to refer to PANDORA's output files and
their contents.

PANDORA has been revised to limit the lengths of lines in the output
files to no more than 128 characters.
\blankline
This Section consists of two parts. Part A has general information
about output files and their use while Part B has detailed information
about how to control their contents.
\ej
\top
\vskip 1 true in
\centerline{\bf A) Printout Files}
\blankline
The `general printout' file is connected to unit 15, and the `error printout'
file may be connected to unit 16; see Section 7.

The `general printout' file is the main output file of the run. Its normal
contents, while to some extend fixed, can generally be controlled by various
switches and options (see Part B, below). In addition, PANDORA also prints
various `debug' data and/or error messages, and these are written to the
`error printout' file. Normally the general printout file and the error
printout file are {\it both} connected to unit 15, so that normal and
error printouts appear in one file, in a single, {\it merged} sequence.
(A merged arrangement preserves the context of error messages.)

However, the {\bf OUTPUT} statement can be used to make PANDORA connect the
error printout file to unit 16. Thus the two types of output can be
{\it split}. This makes for cleaner normal output, which may be
especially desirable in those situations where various error messages are
considered inconsequential. (But note that a `split' arrangement tends to
obscure the context of error messages.)
\blankline
The general printout file can be very large---this can make it difficult to
scan it interactively with an editor. Option MAKIX has been provided
to help with this. When MAKIX is on, then a `printout index' file (connected
to unit 31; see Section 7) will be generated, as follows:

Every `major' printout section in the general printout file will be
preceded by a unique `identifier line'. These indentifier lines,
and these lines only, will also appear in the `printout index' file. Having
in hand a copy of this index, the user can decide which printout sections
to examine, and can use the unique identifier string as an unambiguous
target for a string-search. 
\blankline
\centerline{\tt LOOKAT}
\blankline
A separate program {\tt LOOKAT} helps to display the `general printout'
file on a screen, and to select portions to be printed. It works best when the
associated `printout index' file is also available (see above). {\tt LOOKAT}
is intended to be self-contained, self-explanatory, and easy to use.
\ej
\top
\vskip 1 true in
\centerline{\bf B) Printout control}
\blankline
Most sections of PANDORA's output can be enabled or suppressed at will.
The various program `options' ({\it i.e.} those mentioned in {\bf DO}
and {\bf OMIT} statements) provide most of the control facilities for
this purpose, but some other input parameters (especially those that
control various `debug dumps' and `details dumps') also can affect the
amount of printout.

There follows a list of every `section' of PANDORA's printout, with
descriptions of the conditions that determine whether or not
that section appears.
Many of these descriptions mention `subsections' of major `parent'
sections. Such subsections are {\it not printed}
if conditions are such that the parent section does not appear.
Many parent sections contain stuff in addition to the subsections
mentioned; some do not.
\blankline
Besides the printout sections listed below, there are some debugging
aid \break printouts that are under `option' control; these options are
listed in the TEST/ DEBUG part of the printout section OPTIONS.
The option MITPRNT controls whether or not even a 
minimum of information appears for all but the last iteration. Also,
some error printouts appear invariably whenever certain problems are
detected.
\blankline
{\it Printed Graphs.} PANDORA produces many graphs of selected tables.
The resolution is crude: character positions. Nevertheless, in many cases
these convenient graphs provide adequate pictures of what is poing on.
Before any graph image is actually printed, however, PANDORA determines
whether it is ``interesting'' (\ie whether a `sufficient' number of data
points fall within, rather than on, the graph borders)---boring graphs
are not printed. In such cases only a single print line, consisting of the
graph title and the word ``BORING,'' will appear.   \par
\ej
\vskip 1 in
\centerline{\underbar{The following {\bf Notes} are referred to in the
remainder of this Section:}}
\blankline
\centerline{\it Note A}
\np
This section appears for every overall iteration of a run in which any
atmospheric model tables, involved in the calculations giving rise to this
printout section, are being recomputed, {\it i.e.} if the option HSE
is on, or if the {\bf POPUP} switch has been set. However, if the option
ALL is off, then this printout section appears only once, in any
case.
\blankline
\centerline{\it Note B}
\np
If the option EVERY is off, then this printout section appears for
the last sub-iteration only.
\blankline
\centerline{\it Note C}
\np
If the option ALL is off, then this printout section appears for the
last overall \break iteration only.
\blankline
\centerline{\it Note D}
\np
If the option ALLY is off, then this printout section appears for the
last HSL-iteration only.
\ej
\parindent=0pt
\centerline{\bf Output from {\rm PHASE 0}: Input and Initialization}
\blankline
\blankline
\blankline
\underbar{PROVISIONAL INPUT} \par
appears if STANPRNT is on, and if at least one value of {\bf LZA}
$ > 0$.
\blankline
\underbar{STATUS OF OPTION} \par
appears if OPTPRNT is on. (An abbreviated version of this listing appears
if AOPTPRNT is on.)
\blankline
\underbar{INPUT NOTES} \par
appears always.
\blankline
\underbar{CO LINES} \par
appears when CO line absorption is included and CODMP is on.
\blankline
\underbar{ATMOSPHERE} \par
appears if ATMOPRNT is on. When ZPRNT is on this printout has two
sections, \underbar{ATMOSPHERE-1} and \underbar{ATMOSPHERE-2}; the
second section gives {\bf Z}, {\bf TE}, etc., to nine figures.
\blankline
\underbar{INPUT NUMBER DENSITY AND DEPARTURE COEFFICIENT} \par
appears if INNBPRNT is on.
\blankline
\underbar{ATOM} \par
appears if ATOMPRNT is on.
\blankline
\underbar{DEPTH-DEPENDENT TRANSITIONS DATA} \par
appears if INRBPRNT is on.
\ej
% \blankline
\underbar{ELEMENTS} \par
appears always.
\blankline
\underbar{BACKGROUND} \par
appears always.
\blankline
\underbar{INPUT} \par
appears if INDAPRNT is on. (An abbreviated version of this printout
appears if AINDPRNT is on.) Many diverse parameters may be printed here.
The first page is a bit of a grab bag; the remaining input values are
grouped in various subsections dealing with specific topics, shown in
the headings. Still, some input parameters could equally well have been
included in several of these groups, so, which group a given parameter
is printed with may seem somewhat arbitrary. Subsections are printed
as needed.
\blankline
\underbar{CONSTANTS} \par
appears if STANDARD is on.
\ej
% \blankline
\underbar{HYDROGEN} and \underbar{H2} \par
appear if HNPRNT is on (note input parameter {\bf MOPRNT}).
\blankline
\underbar{CARBON} \par
appears if CARPRNT is on (note input parameter {\bf MOPRNT}).
\blankline
\underbar{SILICON} \par
appears if SILPRNT is on (note input parameter {\bf MOPRNT}).
\blankline
\underbar{HELIUM} \par
appears if HELPRNT is on (note input parameter {\bf MOPRNT}).
\blankline
\underbar{HELIUM-II} \par
appears if HEL2PRNT is on (note input parameter {\bf MOPRNT}).
\blankline
\underbar{ALUMINUM} \par
appears if ALUPRNT is on (note input parameter {\bf MOPRNT}).
\blankline
\underbar{MAGNESIUM} \par
appears if MAGPRNT is on (note input parameter {\bf MOPRNT}).
\blankline
\underbar{IRON} \par
appears if FEPRNT is on (note input parameter {\bf MOPRNT}).
\blankline
\underbar{SODIUM} \par
appears if SODPRNT is on (note input parameter {\bf MOPRNT}).
\blankline
\underbar{CALCIUM} \par
appears if CALPRNT is on (note input parameter {\bf MOPRNT}).
\blankline
\underbar{OXYGEN} \par
appears if OXYPRNT is on (note input parameter {\bf MOPRNT}).
\blankline
\underbar{OXYGEN2} \par
appears if OXY2PRNT is on (note input parameter {\bf MOPRNT}).
\blankline
\underbar{OXYGEN3} \par
appears if OXY3PRNT is on (note input parameter {\bf MOPRNT}).
\blankline
\underbar{SULPHUR} \par
appears if SULPRNT is on (note input parameter {\bf MOPRNT}).
\ej
% \blankline
\underbar{DIRECTIONS} \par
appears if SPHERE is off and STANDARD is on, and if either
GRCONT or \break GRLINE is on.
\blankline
\underbar{SPHERE} \par
appears if SPHERE is on, and if either STANDARD or
SPHEGEOM is on.
\blankline
\underbar{PARTITION} \par
appears if PARTPRNT is on.
\blankline
\underbar{STIMULATED EMISSION FACTORS} \par
appears if STIMPRNT in on. BETA(TR) appears only if CSWITCH is on.
\blankline
\underbar{TABLES} \par
appears if STANDARD is on.
\blankline
\underbar{INTEGRATION} \par
appears if INTAPRNT is on. (An abbreviated version of this listing
appears if AINTPRNT is on.)
\blankline
\underbar{DOPPLER} \par
appears if DPDWPRNT is on, for every radiative transition $u,\ell$ for
which \break {\bf LSFPRINT}$^{u,\ell} = 1$ or {\bf PROF}$^{u,\ell} = 1$.
\blankline
\underbar{CHARGE EXCH.} \par
appears if CHXPRNT is on.
\ej
\underbar{STATISTICAL} \par
appears if KURPRNT is on (note input parameter {\bf LWNT}).
This is a parent section, with one subsection as follows:
\bull dump of unprocessed data: the value of {\bf KUDNT} must be $ > 0$.
\blankline
\underbar{COMPOSITE} \par
appears if KOMPRNT is on (note input parameter {\bf LWNT}).
This is a parent section, with one subsection as follows:
\bull dump of unprocessed data: the value of {\bf KODNT} must be $ > 0$.
\blankline
\underbar{AVERAGED} \par
appears if AVOPRNT is on (note input parameter {\bf LWNT}).
\blankline
\underbar{STANDARD RATES INTEGRATION WAVELENGTHS} \par
appears if both USEWTAB and WTABPRNT are on.
\blankline
\underbar{JNU} \par
appears if there is at least one PRD transition, and if {\bf JNUNC} $> 0$.
\ej

\headline={\centerline{{\bf *****}
{\it Notes A, B, C, D are explained on page 11.4 } {\bf *****}}}

\centerline{\bf Output from {\rm PHASE 1}: Iterations}
\blankline
\blankline
\blankline
\underbar{ND, NK and BD} \par
appears at the start of every overall iteration, if ALL is off and
MITPRNT is on.
\blankline
\underbar{Z-SCALE} \par
appears in a Hydrogen run if values of {\bf TAUKIN} were provided, 
so that the values of {\bf Z} had to be updated (see also Note A).
\blankline
\underbar{CONTINUUM DATA} \par
may appear in every overall iteration of a run which recomputes any
of the atmosphere model data needed for continuum calculations ($JNU$, etc.)
--- {\it i.e.} if option HSE is on, if the {\bf POPUP} switch has been
set, and/or if {\bf Z} must be updated. In other runs continuum calculations
need to be done (and this printout can appear!) in the first overall
iteration only. (If such a run has ${\bf IOMX} > 1$ and option ALL
is off, this printout will not appear!)

This is a major parent section whose subsections are related to the
various contexts in which continuum calculations arise, as follows:
\bull for line core wavelengths: LINECOPR or LBDPRNT must be on
(when \break LINECOPR is on, LBDPRNT is superfluous here)
(see also Notes A and C);
\bull for transition rates integrations: RATEPRNT and RATECOPR
must both be on and USETRIN must be off (see also Notes A and C);
\bull for standard background wavelengths: USEWTAB and STANCOPR
must both be on (see also Notes A and C);
\bull for H-minus calculations: HMS and HMSCOPR
must both be on (see also Notes A and C);
\bull for dust opacity wavelengths: DUSTYPE, DUSTEMP,
and DUSTCOPR must all be on (see also Notes A and C);
\bull for `additional wavelengths': ADDCOPR must be on;
\bull for additional photoionization: RATEPRNT and APHICOPR
must both be on (see also Notes A and C);
\bull for Composite Line Opacity data wavelengths: appears only if these
data are actually used, and if COMPCOPR is on (see also
Notes A and C);
\bull for K-shell wavelengths: KSHLCOPR must be on (see also
Notes A and C);
\bull for FDB wavelengths: FDBCOPR must be on (see also Notes
A and C);
\bull for CO-lines absorption wavelengths: COCOPR must be on (see
also Notes A and C);
\bull for every wavelength whose corresponding context OPTION is off:
output {\it will} appear if that wavelength occurs in the {\bf SCOW}
table.
\blankline
\centerline{\it More control information on next page.}
\ej
% \blankline
These Continuum Data printouts, from all contexts, are grouped as
one batch and appear in order of increasing wavelength.
\blankline
{\it The printout for any particular wavelength} consists
of several parts, each of which appears only if the appropriate option is
on, as follows:
\settabs 4 \columns
\spice
\+ & absorbers printout & OPAPRNT \cr
\+ & emitters printout  & EMIPRNT \cr
\+ & CSF printout       & CSFPRNT (note option ACSFPRNT)\cr
\+ & absorbers graph    & OPAGRAF \cr
\+ & emitters graph     & EMIGRAF \cr
\+ & CSF graph          & CSFGRAF \cr
\+ & summary graph      & CSFGRAF \cr
\blankline
\blankline
\blankline
\underbar{RATES} \par
is a major parent section; a summary explanation of the calculations,
and the printouts available from them, always appears. Substantive
printouts appear if RATEPRNT is on. This section may appear both
in the first and the last iteration of some runs.
The amount of information printed depends on options
RATEALL and RATEFULL. It has subsections as follows:
\bull for CHKI and CHIJ: COLHPRNT must be on;
\bull for upper-level charge exchange: CHEXUP and CHXPRNT must be on;
\bull for CIJ: CIJPRNT must be on;
\bull for PIJ: PIJPRNT must be on;
\bull for RIJ: RIJPRNT must be on;
\bull for additional photoionization: APHIPRNT must be on;
\bull for fast electrons: FELEPRNT must be on. \np
{\it Note} the extensive explanatory text in the regular printout.
\blankline
\underbar{RATES GRAPHS} \par
appears if RATEGRAF is on.
\ej
\underbar{AMBIPOLAR DIFFUSION} \par
appears if RATEPRNT is on and if either AMBPRNT or VLGPRNT is on. \np
{\it Note:} this printout occcurs near the end of the regular RATES printout,
if one or both of the options AMDIFF or VELGRAD are on. Options ADN1DMP,
AMDDMP, DIFFANA, and VELGDMP affect this printout.
\blankline
% \blankline
\underbar{DIELECTRONIC RECOMBINATION} \par
appears if RCOMPRNT is on.
\blankline
\underbar{H MINUS} \par
appears if HMS is on (see also Notes A and C). This is a parent
section, with one subsection as follows:
\bull for JNU: HMSJPRNT must be on.
\blankline
\underbar{DUST} \par
appears if DUSTEMP and DUSTYPE are both on (see also Notes
A and C).
\ej
\centerline{\bf*****}
\centerline{\bf***}
\centerline{\bf*}
\blankline
\centerline{Start of Line Source Function printout
{\it groups} (if any).}
\blankline
Printout can appear if there is at least one radiative transition.
At least some part of the printout group for transition
$u,\ell$ appears if {\bf LSFPRINT}$^{u,\ell} = 1$, or if option
LSFPRNT is on, or if option LSFGRAF is on.
({\it Note:} {\bf LSFPRINT}$^{u,\ell}$ and option LSFPRNT are different
beasties, but they work together.)
\blankline
\underbar{STARTING NUMBER DENSITY AND DEPARTURE COEFFICIENT} \par
appears for the first iteration if default values were computed.
\blankline
\underbar{DAMPING} \par
appears if DPDWPRNT is on (see also Notes A and C).
It will {\it not} appear
if a) HSE is on and this is not the last overall iteration, or
b) HSE is off and this is not the last overall iteration but
this is a Hydrogen run.
\blankline
\underbar{ANALYSIS} \par
appears if ANALYSIS is on (see also Notes A and C).
\blankline
\underbar{LINE (U/L)} \par
always appears; it contains a summary explanation of the various
printouts related to the line source function calculations for
radiative tramsitions.
\blankline
% \blankline
\underbar{STIM, and TAU} \par
appear if TAUPRNT is on (see also Note B). (See also option PEGTNALL.)
\blankline
\underbar{STATISTICAL EQUILIBRIUM} \par
is a major parent section which has three subsections as follows:
\bull for timing data: SETIME must be on;
\bull for PE, FE, {\it etc.}: SEPRNT must be on;
\bull for methods comparison: SECOMP or METPRNT must be on. \np
{\it Note:} option DIFFANA also affects this printout (see also Note B).
\ej
% \blankline
\underbar{PRD} \par
Printouts related to PRD calculations appear, for all
PRD transitions, if option PRDPRNT is on. The amount of output depends on
input parameters {\bf IPRDF} and {\bf IPRDD}. Tables of JNU are printed
if option JNUPRNT is on. Printouts from PRD-related background (continuum)
calculations, for wavelengths (frequencies, XI-values) selected by
{\bf IPRDF}, appear if option PRDCOPR is on (and/or for every PRD
wavelength occurring in the {\bf SCOW} table). For the purposes of PRD
these ``continuum printouts'' have been split into two separate parts,
but they have the standard format used for `CONTINUUM DATA,' above, and
use the same options for detailed control as described there.
If iterations are used with the Hubeny-Lites formulation (i.e. option
PRDMETH is on), then only the results of the last iteration are printed
unless option PRDITER is on (see also Note B).
\blankline
\underbar{LINE SOURCE FUNTION} \par
is a major parent section which has several subsections as follows:
\bull TNU-analysis is controlled by {\bf KANTNU} (q.v.); \np
\bull for Line Source Function details for transition $u,\ell$:
{\bf LSFPRINT}$^{u,\ell} = 1$ or option LSFPRNT must be on
(an abbreviated version of this printout appears if option ALSFPRNT is
on); a part of this detail printout also depends on option LSFFULL; \np
\bull for source function graph for transition $u,\ell$: 
{\bf LSFPRINT}$^{u,\ell} = 1$ or option LSFGRAF must be on;
\bull for radiative force for transition $u,\ell$:
{\bf LFLUX}$^{u,\ell} = 1$, (the full computed Line Flux Distribution
appears if LFDPRNT is on). \np
{\it Note:} See also option PESRJALL (see also Note B). \np
{\it Note:} The section `LINE SOURCE FUNCTION' (if not a PRD transition)
will appear twice in a diffusion run with DIFFANA on, concluding with a
summary graph comparing the results with and without diffusion terms.
\blankline
\centerline{ End of Line Source Function printout {\it groups} (if any).}
\blankline
\centerline{\bf*}
\centerline{\bf***}
\centerline{\bf*****}
\ej
\underbar{LOG PLOT OF TAU SCALES} \par
appears if there is at least one radiative transition and if LSCALE
is on, in the very last sub-iteration only.
\blankline
\underbar{COLLATED TAU SCALES} \par
appears if there is at least one radiative transition and if SCALE
is on, in the very last sub-iteration only.
\blankline
\underbar{RHO AND RBD} \par
is a major parent section. An explanation of the available printouts
always appears. Substantive results appear if
RHBPRNT is on, and subsections may appear as follows:
\bull for details of each radiative transition: RHBPRDT must be on;
\bull for details of BDQ calculation: BDQPRDT must be on;
\bull for complete sets of BDR, BDJ, BDS, and S*: BDPRNT must be on;
\bull for final sets of Rho and b-ratios: RHBPRSM must be on. \par
{\it Note} that the results printed by RHBPRSM are also printed by RHBPRDT;
thus it is not normally necessary to turn both on; (RHBPRSM is intended
to provide ``minimal'' printout).
\blankline
\underbar{CONSISTENCY CHECKS} \par
appears if CHKPRNT is on, and is controlled by EVERY (see also Note B).
\ej
% \blankline
\underbar{POPULATIONS} \par
is a major parent section which appears here if LYMAN is on (see also Note D).
See detailed description below.
\blankline
\underbar{CONTINUUM DATA FOR LYMAN} \par
appears if LYMAN is on, and if either LYMCOPR is on or for every
wavelength that occurs in the {\bf SCOW} table
(see also Note D). Detailed control of the printout parts for each
Lyman wavelength is as for other `CONTINUUM DATA', described above.
\blankline
\underbar{LYMAN} \par
is a major parent section which appears if LYMAN is on (see also
Note D). (An abbreviated version of this major parent section is printed
if ALYMPRNT is on.) It has three subsections as follows:
\bull for EP methods comparison: EPCOMP must be on;
\bull for RK comparison: COMPRK must be on;
\bull for PIJ: HSE must be off and this must not be the last
iteration.
\blankline
\underbar{POPULATIONS} \par
is a major parent section which appears here if HSE is on (see also Note D).
See detailed description below.
\blankline
\underbar{NE} \par
appears if HSE and NESWICH are both off and {\bf POPUP} has been
set.
\blankline
\underbar{HSE} \par
appears if HSE is on (see also Note D).
(An abbreviated version of this section is printed if AHSEPRNT is on.)
\blankline
\underbar{MODEL DATA} \par
is a major parent section which appears if HSE is on (see also Note D);
it is a continuation of the HSE section. It has subsections as follows:
\bull for absorbers at 500 nm: OPAPRNT must be on;
\bull for HYDROGEN: this always appears;
\bull recalculated Z-scale appears if values of {\bf TAUKIN} were provided,
or if values of {\bf ZMASS} were provided and ZCOMP is on;
\bull for electrons calculations results (with graphs): ELECPRNT must
be on (appears in the last iteration only).
\blankline
\underbar{GAS} \par
appears if HSE is off and {\bf POPUP} equals {\tt HYDROGEN} (see also
Note D).
\ej
\underbar{POPULATIONS} \par
is a major parent section controlled by ALL (see also Note D).
An explanation of the various controls for this section always appears.
Substative printout appears if either
POPPRNT or POPGRAF are on. There are five subsections as follows:
\bull for NE results: {\bf POPUP} must equal {\tt HYDROGEN} and 
POPPRNT must be on;
\bull for the full tables: POPPRNT must be on, (the supplementary
printout appears if NBPRNT is on), (an abbreviated version of the full
tables appears if $\qquad$ \break APOPPRNT is on);
\bull for graphs of ND: POPGRAF must be on (see Note D);
\bull for a trace of the Populations Calculations at depth \# {\bf IBNVIEW}:
\break PDETPRNT must be on;
\bull for graph of BD vs. {\bf Z}: BDGRAF must be on (the plots of TAU
vs. {\bf Z} appear only if TAUPLOT is on) (see Note D);
\bull for graph of BDIJ vs. {\bf Z}: BDGRAF must be on and there must be
at least one radiative transition (the plots of TAU vs. {\bf Z} appear only if
TAUPLOT is on) (see Note D);
\bull for graph of {\bf TE} vs. log(TAU): TEGRAF must be on and there must be
at least one radiative transition (see Note D).
\blankline
\underbar{UPDATED POPULATIONS DATA} \par
is a major parent section, controlled by ALLY (see also Note D). Its
various subsections are concerned with the various `population-update ions,'
as follows:
\bull for Hydrogen: HNPRNT must be on and {\bf POPUP} must equal
{\tt HYDROGEN};
\bull H2: {\bf POPUP} must equal {\tt HYDROGEN} and ${\bf NHTSW} > 0$;
\bull for Carbon: CARPRNT must be on and {\bf POPUP} must equal
{\tt CARBON};
\bull for Silicon: SILPRNT must be on and {\bf POPUP} must equal
{\tt SILICON};
\bull for Helium: HELPRNT must be on and {\bf POPUP} must equal
{\tt HELIUM};
\bull for Helium-II: HEL2PRNT must be on and {\bf POPUP} must equal
{\tt HELIUM2};
\bull for Aluminum: ALUPRNT must be on and {\bf POPUP} must equal
{\tt ALUMINUM};
\bull for Magnesium: MAGPRNT must be on and {\bf POPUP} must equal
{\tt MAGNESIUM};
\bull for Iron: FEPRNT must be on and {\bf POPUP} must equal
{\tt IRON};
\bull for Sodium: SODPRNT must be on and {\bf POPUP} must equal
{\tt SODIUM};
\bull for Calcium: CALPRNT must be on and {\bf POPUP} must equal
{\tt CALCIUM};
\bull for Oxygen: OXYPRNT must be on and {\bf POPUP} must equal
{\tt OXYGEN};
\bull for Oxygen-II: OXY2PRNT must be on and {\bf POPUP} must equal
{\tt OXYGEN2};
\bull for Oxygen-III: OXY3PRNT must be on and {\bf POPUP} must equal
{\tt OXYGEN3};
\bull for Sulphur: SULPRNT must be on and {\bf POPUP} must equal
{\tt SULPHUR}.
\ej
% \blankline
\underbar{K-SHELL} \par
appears if a K-shell ionization calculation has been requested in a 
Carbon run (see Note D).
\blankline
\underbar{CHARGE EXCHANGE} \par
appears if upper-level charge exchange has been requested in a Hydrogen
run and option CHXPRNT is on (see Note D).
\blankline
\underbar{RESTART VALUES OF JNU} \par
appears in a PRD run when option JNUPRNT is on.
\blankline
\underbar{DAMPING} \par
appears if PHASE2, PASSPRNT, and DPDWPRNT are
all on, and if there is at least one passive transition for which the emergent
profile calculation has been turned on.
\blankline
\underbar{ANALYSIS} \par
appears if PHASE2, PASSPRNT, and ANALYSIS are
all on, and if there is at least one passive transition for which the emergent
profile calculation has been turned on.
\blankline
\underbar{TAU} \par
appears if PHASE2, PASSPRNT, and TAUPRNT are
all on, and if there is at least one passive transition for which the emergent
profile calculation has been turned on.
\blankline
\underbar{PASSIVES} \par
appears if PHASE2 and PASSPRNT are both on.
\ej

\headline={\hfil}

% \blankline
\underbar{DAMPING} and \underbar{ANALYSIS} \par
appear if PHASE2 is on, only for those transitions $u,\ell$
whose values of \break {\bf BLCSW}$^{u,\ell}$ require this
and for which emergent profile calculations are \break requested. 
For \underbar{DAMPING}, DPDWPRNT must be on, for
\underbar{ANALYSIS}, ANALYSIS must be on.
\blankline
\underbar{TAU} \par
appears if PHASE2, TAUPRNT, LTE and LTEDATA are all
on, only for those transitions for which emergent profiles calculations
are requested.
\blankline
\underbar{LTE DATA} \par
appears if PHASE2, LTE and LTEDATA are all
on, only for those transitions for which emergent profiles calculations
are requested.
\blankline
\underbar{ANALYSIS} \par
appears if PHASE2, ANALYSIS, LTE and LTEDATA are all
on, only for those transitions for which emergent profiles calculations
are requested.
\blankline
\underbar{FREQUENCY-DEPENDENT LINE SOURCE FUNCTION} \par
is a major section, which may appear for the ({\bf MS}, {\bf NS})
transition only. It has two subsections, as follows:
\bull for printout: SLFPRNT must be on;
\bull for graph: SLFGRAF must be on.
\blankline
\underbar{SCATTERING ALBEDO ANALYSIS} \par
appears, for suitable transitions, if LINECOMP is on.
\blankline
\underbar{CONTINUUM DATA} \par
appears for all wavelengthe needed for line profile calculations
if FBDCOPR is on, or if TRUECONT and TRUECOPR are both on. Detailed
control of the printout is as for other `CONTINUUM DATA', described above.
\ej
% \blankline
\underbar{WAVE SUMM 0} \par
always appears.
\blankline
\underbar{WAVE SUMM 1} \par
appears if WAVEPRNT is on. It provides a basic summary of the background
(continuum) calculations at each wavelength.
\blankline
\underbar{WAVE SUMM 2} \par
appears if WAVEPRNT is on and {\bf IWSMD} = 1. It provides a basic
summary of the contributors to the background (continuum) at each
wavelength.
\blankline
\underbar{COMPOSITE LINE OPACITY ANALYSIS} \par
appears if COMOPAN is on. (An abbreviated version of this printout
appears if {\bf KCOAA} is set = 1.)
\blankline
\underbar{COOLING} \par
appears if CALCOOL is on. This is a parent section, with one subsection
as follows:
\bull for integrated rates: COOLINT must be on.
\blankline
\underbar{HEATING} \par
appears if CALHEAT is on. This is a parent section, with one subsection
as follows:
\bull for integrated rates: COOLINT must be on.
\blankline
\underbar{HEATING/COOLING SUMMARY} \par
appears if CALCOOL and CALHEAT are both on.
\ej
% \blankline
\underbar{FUDGERS} \par
appears as needed.
\blankline
\underbar{A-TROUBLES} \par
appears as needed.
\blankline
\underbar{CHECKS GRAPHS} \par
appears if there is at least one radiative transition and CHKGRAF is on.
\blankline
\underbar{ITERATIVE SUMMARIES} \par
is a major parent section which appears as needed and if SUMMARY is on.
An Explanation of the various controls for this section appears if
either SUMMARY or SUMTREND is on.
Its various subsections are controlled by separate options, and are
related to various calculated quantities which
are obtained by iterative improvement; they appear only in those cases
where more than one iterate has been saved so that comparison among
successive iterates is possible, as follows:
\bull for CHECKs: ITERCHK must be on;
\bull for S$^{u,\ell}$: ITERS must be on, and $u,\ell$ must be a 
radiative transition;
\bull for RHO$^{u,\ell}$: ITERRHO must be on, and $u,\ell$ must be a 
radiative transition;
\bull for CHI$^{u,\ell}$: ITERCHI must be on, and $u,\ell$ must be a 
radiative transition;
\bull for TAU$^{u,\ell}$: ITERTAU must be on, and $u,\ell$ must be a 
radiative transition;
\bull for RK: ITERRK must be on;
\bull for ND: ITERN must be on;
\bull for RHOWT$^{u,\ell}$: ITERRWT must be on, and $u,\ell$ must be a 
radiative transition;
\bull for BD: ITERB must be on;
\bull for NE: ITERNE must be on;
\bull for Z: ITERZ must be on;
\bull for TDST: ITERTD must be on. \np
Iterative Summaries appear in tabular and/or in graphical from,
depending on option SUMGRAF and input parameter {\bf ISMSW}.
\blankline
\underbar{ITERATION TREND SUMMARY} \par
may appear if there is at least one radiative transition and SUMTREND is on.
\ej
\centerline{\bf Output from {\rm PHASE 2}: Spectrum and Summaries}
\blankline
\blankline
\blankline
\centerline{\it None of these sections appear if {\rm PHASE2} 
is off.}
\blankline
\blankline
\blankline
\underbar{EMERGENT CONTINUOUS INTENSITY} \par
appears if EMERINT is on. This is a parent section, with subsections
as follows:
\bull for depths-of-formation: ORIGIN must be on (see option ORSHORT);
\bull for dI/dh: DIDHC must be on (see also input parameter {\bf ICDIT};
\bull for color temperatures: COLTEMP must be on;
\bull for average continuum intensities: AVCON muts be on. \np
{\it Note} the effect of option WAVENUMB.
\blankline
\underbar{CONTINUOUS FLUX} \par
appears if CONFLUX is on. This section includes the integrated flux
quantities, and the Rosseland means. It has one subsection as follows:
\bull for details at each wavelength: FLUXDMP must be on; (which specific
details are printed depends on the value of {\bf IFXDS}). \np
{\it Note} the effect of option WAVENUMB.
\blankline
\underbar{CONTINUUM ECLIPSE INTENSITIES} \par
appears if EMERINT and ECLIPSE are both on, only for those 
`additional wavelengths' and Composite Line Opacity wavelengths for which
this was requested explicitly, and for CO lines wavelengths if COCLIPSE
is on. Calculations for selected beams of specified widths are done
when ${\bf NZE} > 0$. \np
{\it Note} the effect of option WAVENUMB.
\blankline
\underbar{VELOCITIES} \par
appears if LIGHT is on and {\bf LPVEL} = 1, but only if there are non-zero
velocity tables worth printing. (The corresponding mass-loss rates are
printed only if {\bf LPVEL} and {\bf LPMLR} both = 1.)
\ej
\centerline{\bf*****} 
\centerline{\bf***}
\centerline{\bf*}
\blankline
\centerline{ Start of Emergent Line Profile printout {\it groups} (if any).}
\blankline
These appear, if option LIGHT is on, for every transition
$(u, \ell)$ for which the input parameter {\bf PROF}$^{u,\ell} > 0$.
\blankline
\underbar{BACKGROUND INTENSITY} \par
computed continuum intensities amd flux needed for residuals calculation
will be printed if option PROCPRNT is on.
\blankline
\underbar{GRAPH OF S, B VS. Z} \par
appears if LIGHT and INTGRAF are both on, only for those
transitions for \break which emergent profiles calculations were requested.
These graphs do not appear if graphs were printed earlier in the
LINE SOURCE FUNCTION section.
\blankline
\underbar{EMERGENT LINE PROFILES} \par
shows the various versions of the emergent intensity profiles for a
given transition, for all values of lookangle ($\mu$), velocity, and
viewing position (front-face (always), back-face (if requested); followed
by the flux profiles computed from the intensity profiles.
(A somewhat abbreviated version of this printout appears if 
APRFPRNT is on.) This parent section has subsections as follows:
\bull for profile analyses other than those printed earlier: ANALYSIS
must be on;
\bull for depths-of-formation: ORIGIN must be on ({\it note} also option
ORSHORT);
\bull for dI/dh: DIDHL must be on;
\bull for location analysis graph: LOGAS must be $> 0$;
\bull for FNRMLA, FNRMLB: CLNORM must be on in a Hydrogen run. \np
{\it Note} the effect of option WAVENUMB.
\ej
% \blankline
\underbar{GRAPH OF RESIDUAL INTENSITY PROFILE} \par
appears if LIGHT and INTGRAF are both on, only for those
transitions for \break which emergent profiles calculations were requested. \np
{\it Note} the effect of option WAVENUMB.
\blankline
\underbar{GRAPH OF ABSOLUTE INTENSITY/FLUX PROFILE} \par
appears if LIGHT and INTGRAF are both on, only for those
transitions for \break which emergent profiles calculations were requested. \np
{\it Note} the effect of option WAVENUMB.
\blankline
\underbar{ECLIPSE} \par
appears if LIGHT is on, only for those transitions $u,\ell$ for
which {\bf ECLI}$^{u,\ell} > 0$. \np
{\it Note} the effect of option WAVENUMB.
\blankline
\centerline{ End of Emergent Line Profile printout {\it groups} (if any).}
\blankline
\centerline{\bf*}
\centerline{\bf***}
\centerline{\bf*****}
\blankline
\blankline
\underbar{SPECTRUM SUMMARY} \par
appears if SPECSUM is on and if either LIGHT or EMERINT
is on.
\blankline
\underbar{CONTINUUM CONTRIBUTIONS SUMMARIES} \par
is a major parent section which appears if EMERINT is on. It has
three subsections as follows:
\bull for absorbers: OPASUM must be on;
\bull for emitters: EMISUM must be on;
\bull for TAU: TAUSUM must be on.
\ej
\centerline{\bf Performance Data, and Program Version Description}
\blankline
\blankline
\blankline
\underbar{SCRATCH I/O SUMMARY} \par
appears when {\bf IRUNT}$\; > 0$.
\blankline
\underbar{TIMING SUMMARY} \par
appears when {\bf IXSTA}$\; > 0$.
\blankline
\underbar{VERSION DESCRIPTION} \par
appears always (amount of detail depends on {\bf IRUNT}).
\blankline
%\blankline
%\vfill \vfill
%\vfill \vfill
%\vfill \vfill
%\vfill \vfill
%\vfill \vfill
%\vfill \vfill
%\vfill \vfill
%\vfill 
\vfill
\noindent (Section 11 -- last revised: 2007 Feb 05) \par
\message{Section 11 ends at page 11.\the\pageno}
\ej
%\end
