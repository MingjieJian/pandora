%\magnification=1200
%\input wupstuff.tex
\newtoks\footline \footline={\hss\tenrm 1.\folio\hss}
\pageno=1
\top
\vskip 1.5 true in
\centerline{Section 1: {\bf Input Language}}
\blankline
\blankline
\centerline{\bf ***}
\blankline
\blankline
PANDORA uses free-field self-identified input. The basic input unit is the 
vector or one-dimensioned array; single variables, on the one hand, and
multiply-dimensioned arrays, on the other, are treated as special cases of
the vector. A PANDORA input statement specifies the name of the
array, and the values to be stored in consecutive locations of it. Several
control facilities are provided, to make specifying input statements less
tedious.
\blankline
The general input statement is:  {\bf NAME} {\it j} {\bf (} {\it Q} {\bf )}  .
\blankline
Each component of that statement (including `{\bf (}' or `{\bf )}')
is a separate `field' or a set of fields.
PANDORA's input routines use {\tt NUDEAL}, whose conventions require that 
every `field' be preceded by at least one blank, and followed by at least
one blank; ({\it i.e.} blank is the required `separator' between fields).
In the example above, {\bf NAME} is the name of the input quantity (array)
whose values follow; {\it j} is none or several auxiliary indices; and
{\it Q} is the sequence of fields and control parameters that specify
the values to be stored in memory.

PANDORA's input routines maintain an internal pointer, initially set to 1
and subsequently incremented or set as specified or implied by {\it Q},
which identifies that member of the array that will be affected by the
next input specification in {\it Q}. Therefore, when in the
remainder of this section
I use phrases like `the array member currently being read into,'
I mean the array member identified by that internal pointer.

The remainder of this section deals with: 
1) field format basics; 2) {\bf (} {\it Q} {\bf)}; and 3) examples.
The various types of input statements ({\it i.e.}, combinations of
{\bf NAME} and {\it j}) are  discussed in Section 2; the many {\bf NAME}s
recognized by \break PANDORA are treated in Section 5.
\ej
\centerline{1. \underbar{Format Basics}}
\blankline
\blankline
PANDORA reads input file(s) line by line; only the first 80 characters
of each line are read. Each line is scanned, character-by-character,
from left to right, beginning with the first character, until end-of-line
(which normally occurs after the 80. character). Five types of input
fields are recognized: 1) null field, 2) excessive field, 3) {\intg} field,
4) {\flpt} field, and 5) {\alfa} field. Only {\intg}, {\flpt} and {\alfa}
fields are legal PANDORA input fields, the other two are considered
errors.
\blankline
1) Null field: the scan encountered more than 159 successive blanks,
without a non-blank character turning up.
\blankline
2) Excessive field: the scan encountered more than 60 successive
non-blank characters, without a blank ({\it i.e.} break character)
turning up.
\blankline
3) Integer field: like an integer constant in FORTRAN, containing up
to 9 significant digits. An {\intg} field may contain: digits,
and at most a single sign.
\blankline
4) Floating point field: like a real constant in FORTRAN, containing up
16 significant digits, with an exponent value not exceeding 300.
A {\flpt} field may contain: digits, at most a single sign for the 
mantissa, at most a single decimal point for the mantissa,
at most a single exponent flag, at most a single sign for the exponent.
\blankline
5) Alphanumeric field: anything that is not one of the field types
(1) -- (4), above.
\blankline
Digits: [ 0 1 2 3 4 5 6 7 8 9 ]
\blankline
Signs: [ $+ \quad -$ ]
\blankline
Decimal point: [ . ]
\blankline
Exponent flags: [ E e D d ]
\ej
\centerline{2. \underbar{Q}}
\blankline
\blankline
{\it Q} may contain the following fields: {\it n}, {\bf M}, {\bf m}, {\it m},
{\bf I}, {\bf i}, {\it i}, {\bf R}, {\bf r}, {\it r}, {\bf S}, {\bf s}, {\bf F}
and {\bf f}. Here the
{\bf boldface} fields are names (identifiers) of control parameters, and
the {\it italic} fields are numerical values, either of input
quantities, or of the control parameters whose names precede them.
\blankline
{\it n} is the quantity (an {\intg}, a {\flpt} number, or an {\alfa} quantity)
to be stored in the array member currently being read into; before
being stored in memory, the value of {\it n} (if an {\intg} or a {\flpt}
number) will be multiplied by {\it m}.
\blankline
{\bf M} or {\bf m} signal that a value of {\it m} will follow.

{\it m} is either a {\flpt} number, by which all {\flpt} {\it n's} following
{\it m} will be multiplied before being stored in memory; or is an \intg,
by which all {\intg} {\it n's} will be multiplied before being stored in memory.
The modes of {\it m} and {\it n} must agree ({\it i.e.} both
{\intg} or both \flpt). The default value of {\it m} is 1; this value will be
used if a different value of {\it m} is not specified explicitly in {\it Q}.
Successive values of {\it m} are not multiplied together; it is always
the latest value encountered that will be used.
\blankline
{\bf I} or {\bf i} signal that a value of {\it i} will follow.

{\it i} is an {\intg} which explicitly specifies the next array member to be read
into ({\it i.e.} it is used to reset the internal array pointer). Any {\it n}
not preceeded by an {\it i} will be stored in the array member specified
by the internal array pointer. If the first {\it n} in {\it Q} is not
preceeded by an {\it i}, it will be stored as the first array member.
\blankline
{\bf R} or {\bf r} signal that a value of {\it r} will follow.

{\it r} is an {\intg} specifying the value of the `repeat' counter. The repeat
\break counter specifies that the next {\it n} should be `used' {\it r} times.
For example:  {\bf R} 3 1.7  means that the next array member to be read into,
and the two succeeding ones as well, will all be set equal to 1.7; also:
{\bf I} 7 {\bf r} 3 1.0  (which is equivalent to  {\bf R} 3 {\bf i} 7 1.0 )
means that array members 7 -- 9 will all be set equal to 1.0. If the specified
value of {\it r} implies members beyond the high end of the array, then
such `excess' will be ignored. The default value of {\it r} is 1.
\ej
{\bf S} or {\bf s} signal that the next array member to be read into should be skipped
\break ({\it i.e.} no value will be specified for it, and its current contents
should be left \break unchanged). The sequence  {\bf R} {\it r} {\bf S}
means that the next {\it r} array members to be read into should be skipped.
\blankline
{\bf F} or {\bf f} signal that the {\it n} following {\bf F} should be stored in the
current array member to be read into and all succeeding ones as well
({\it i.e.} the sequence  {\bf F} {\it n}  `fills' the remainder of the array,
or indeed the entire array, with {\it n}).
\blankline
The sequence  {\bf [} {\it A} {\bf ]}  constitutes a comment or remark.
The occurrence of
opening square bracket (followed by a blank) signals the beginning of a
comment, {\it A}, which may comprise any number of fields consisting of any
characters, except closing square bracket. The occurrence of closing square
bracket (preceeded and followed by blank) signals the end of the comment.
PANDORA ignores the entire comment (including the delimiting brackets).
\blankline
The field  $ > $ (`greater than' sign) indicates end-of-line ({\it i.e.} 
the remainder of the input line on which it occurs should be skipped).
\blankline
{\it Note}: only UPPERCASE alphabetic characters were recognized originally;
the generalization to mixeD-CAse was implemented later.
\ej
\centerline{3. \underbar{Examples}}
\blankline
\blankline
Here are some sample lines of input:
\blankline
\blankline
\blankline
\blankline
{\tt
\noindent {\it 0001} \qquad N ( 34 ) \par
\noindent {\it 0002} \qquad KK ( 13 ) $>$ \par
\noindent {\it 0003} \qquad IOMX ( 2 ) $>$  STUFF AFTER `GREATER THAN' 
IS IGNORED \par
\noindent {\it 0004} \qquad [ THIS IS A REMARK ] \par
\noindent {\it 0005} \qquad XK ( I 13 5.5 )  DLU  ( 0.1 )  $>$ \par
\noindent {\it 0006} \qquad TE ( F 10000. ) \par
\noindent {\it 0007} \qquad NE ( 0. M 1.E1 1. 2. 3. 4. 5. \par
\noindent {\it 0008} \qquad R 5 6. F 7. ) \par
\noindent {\it 0009} \qquad [THIS STUFF IS NOT A COMMENT ]  \par
\noindent {\it 0010} \qquad [ THIS ALSO IS NOT A   R E M A R K]  \par
\noindent {\it 0011} \qquad A 2 1 ( 1.E8 )  [ ESTIMATE $>$ ]  \par
\noindent {\it 0012} \qquad CRD 2 1 ( 1.E-5 )  [ ESTIMATE ] $>$ \par
}
\blankline
\blankline
\blankline
\blankline
--- Lines 1, 4, 6, 7, 8, 9 and 10 are scanned in their entirety
({\it i.e.}, all 80 characters); the others up to $>$ only.
 
--- Line 3 shows an alternate form of comment.
 
--- In Line 5, the 13. value of XK is set to 5.5, and DLU is set
to 0.1; showing that there may be more than one input statement in a line.
 
--- Lines 7 and 8 show that one input statement may extend over
several lines; it is the final `)' that indicates the end of the NE statement.
The values in memory of NE will be:
0, 10, 20, 30, 40, 50, 60, 60, 60, 60, 60, 70 $\ldots$ 70.
 
--- Line 9 would be a valid comment if there were a blank after [ , Line 10
likewise if there were a blank before  ] .
 
--- Line 12 illustrates a pernicious
error: it is ignored since it is considered part of the comment begun
on Line 11. To get the intended effect, interchange $>$ and ] on Line 11.
\blankline
\centerline{{\it Note}: line numbers are not allowed in PANDORA input.}
%\blankline
%\blankline
%\blankline
%\vfill \vfill
%\vfill \vfill
%\vfill \vfill
%\vfill \vfill
%\vfill \vfill
%\vfill \vfill
%\vfill \vfill
%\vfill \vfill
\noindent (Section 1 -- last revised: 2003 Aug 01) \par
\message{Section 1 ends at page 1.\the\pageno}
\ej
%\end
