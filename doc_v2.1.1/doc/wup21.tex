%\magnification=1200
%\input wupstuff.tex
\newtoks\footline \footline={\hss\tenrm 21.\folio\hss}
%
\pageno=1
%
\top
\vskip 1.5 true in
\centerline{Section 21: {\bf The Hydrogen Lyman Lines in the Continuum Calculations}}
\blankline
\blankline
\centerline{\bf ***}
\blankline
\blankline
\blankline
\blankline
\centerline{1. \underbar{Theory}}
\blankline
\blankline
The Hydrogen Lyman lines are transitions between any upper level $U$ and the 
lower level $L = 1$. The absorption coefficient for a line transition between
levels $U$ and 1 is
%
$$ \kappa_\nu = n_1 { {h \nu_{U1} } \over { 4\pi } } {\cal B}_{1U}
                { { \phi(a,x) } \over { \Delta \nu_D } } \eqno(1) $$
%
where $n_1$ is the number density of the lower level,
%
$$ \phi(a,x) = { a \over \pi^{3/2} } \int_{-\infty}^{+\infty} 
               { { \exp^{-{z^2}} \over {a^2 + (z-x)^2} } } \, , \eqno(2) $$
%
and where $x = \Delta \nu / {\Delta \nu_D}$, $\Delta \nu = \nu -\nu_{U1}$,
and $a = \delta / {\Delta \nu_D}$. In equation (1) we have ignored stimulated
emission since it is negligible at Lyman line wavelengths. Here $\nu_{U1}$ is
the line-center frequency, and the Doppler width $\Delta \nu_D$ is given by
%
$$ \Delta \nu_D = { \nu_{U1} \over c } \sqrt { { {2kT} \over M } + V^2 } \eqno(3) $$
%
where $M$ is the mass of the atom and $V$ is the line-broadening or
microturbulent velocity.

The damping parameter is the sum
%
$$ \delta = \delta_{\rm Nat} + \delta_{\rm Res} + \delta_{\rm Stark} + \cdots  \eqno(4) $$
%
and here we consider only the natural, resonance and Stark components
%
$$ \delta_{\rm Nat} = { 1 \over {4\pi} } \sum_{k=1}^{U-1}A_{Uk}      \eqno(5) $$
%
and
%
$$ \delta_{\rm Res} = { 1 \over {4\pi} } { {3e^2} \over m }
                      \sqrt { g_L \over g_U } { f_{1U} \over \nu_{U1} }
                      n_1   \, , \eqno(6) $$
%
and
% 
$$ \delta_{\rm Stark} = 7.06 \times 10^{-6} \, {\bf PMSK} \, a^{U1}
                        { U^4 \over { U^2 -1 } }
                        \left( { NE \over 10^{12} } \right) ^{2/3} \eqno(7) $$
%
where
%
$$ a^{U1} = \cases{
                   0.642
                   & if $U = 2$
                   \cr
                   \cr
                   1.0
                   & otherwise
                   \cr
                  } \, , \eqno(8) $$
% 
and $\bf PMSK$ is an input parameter (default = 1). This expression for Stark
\break damping is from Sutton, K., 1978, J.Q.S.R.T., 20, 333.
(Including $\delta_{\rm Stark}$ is optional in PANDORA; see below.)

In equation (6) $n_1$ represents the number
density of perturbing atoms of the same ionization stage. The Einstein
coefficients in equations (1) and (5) are related by the equations
$A_{U1} = ( 2 h \nu_{U1}^3 / c^2 ) {\cal B}_{U1}$ and 
$g_U {\cal B}_{U1} = g_1{\cal B}_{1U}$, where $g_U$ and $g_1$ are the
statistical weights of the upper and lower levels. $A_{U1}$ and the
oscillator strength $f_{1U}$ are related by
%
$$ A_{U1} = { g_1 \over g_U } { { 8 \pi^2 e^2 \nu_{U1}^2 } \over {m c^3} }
            f_{1U}  \eqno(9)  $$
%
where $e$ and $m$ are the electron's charge and mass, respectively.


When $a=0$ in equation (2) the profile function reduces to
%
$$ \phi (0,x) = { 1 \over \sqrt{\pi} } \exp^{-x^2}  \, .  \eqno(10)  $$
%
For $x$ larger than 3 (typically), and $a > 0$,
%
$$ \phi (a,x) \to { a \over {\pi x^2} }  \, ,  \eqno(11) $$
%
or
%
$$ { \phi(a,x) \over {\Delta \nu_D} } \to { \delta \over { \pi (\Delta\nu)^2 } }
               \, .  \eqno(12)  $$
%
Thus in the line wings we can write equation (1) as
%
$$ \kappa_\nu = n_1 { {h \nu_{U1} } \over {4 \pi^2} } {\cal B}_{1U}
                { { \delta_{\rm Nat} + \delta_{\rm Res} + \delta_{\rm Stark}} \over
                { (\Delta \nu )^2 } }    \, .  \eqno(13)  $$
%
In PANDORA we use equation (1) to determine $\kappa_\nu$ in the line core
and equation (13) for larger $\Delta \nu$ in the line wings.

In the line core  the source function is the frequency-independent line source
function
%
$$ S_{U1} = { { 2 h \nu_{U1}^2 / c^2 } \over { { g_U \over g_1 }
            { n_1 \over n_U } -1 } }  \, .    \eqno(14)   $$
%
In the line wings, however, photons can be scattered (changing only their direction
by the encounter with the atom) with no change in frequency, rather than being
absorbed and the re-emitted with $\phi(a,x)$ frequency distribution. For those
scattered photons the source function (i.e., the ratio of emission to
absorption coefficients) is the mean intensity $J_\nu$.

The transfer equation then is written as
%
$$ \mu { {dI_\nu} \over {dz} } = \kappa_\nu^{\rm abs} ( I_\nu - S_{U1} )
                 + \kappa_\nu^{\rm sct} ( I_\nu - J_\nu )  \, .  \eqno(15) $$
%
We introduce a scattering albedo $\alpha_\nu$ and let
%
$$ \kappa_\nu^{\rm abs} = (1 - \alpha_\nu) \kappa_\nu  \, ,  \eqno(16) $$
%
and
%
$$ \kappa_\nu^{\rm sct} = \alpha_\nu \kappa_\nu  \, .  \eqno(17)  $$
%
We specify $\alpha_\nu$ as a function of $x = \Delta \nu / \Delta \nu_D$. 
(Note that $\Delta \nu_D$, a function of $T$ and $V$, varies with depth.)
We define $(1-\alpha_\nu) = DR(x)$, which is determined as described below.
$DR(x)$ must not be less than the allowed lower limit which, for
$3 \le U \le 15$ is given by the branching ratio equation
%
$$ drlim^{u,1} = { { \sum_{k=2}^{U-1} A_{Uk} } \over
                   { \sum_{k=1}^{U-1} A_{Uk} } } \, . \eqno(18) $$
%

Equation (13), which we use in the line wings ({\it i.e.}, when
$x \ge XLIM$; see below), is written as
%
$$ \kappa_\nu = \kappa_\nu^{\rm Nat} + \kappa_\nu^{\rm Res}  + 
                \kappa_\nu^{\rm Stark} \eqno(19) $$
%
where
%
$$ \kappa_\nu^{\rm Nat} = \cases{
                          n_1 { { h \nu_{U1} } \over { 4 \pi^2 } }
                         {\cal B}_{1U} { \delta_{\rm Nat}
                          \over { (\Delta \nu)^2 } }
                          & $\lambda < 142.5\ {\rm nm}$
                          \cr
                          \cr
                          0
                          & $\lambda \ge 142.5\ {\rm nm}$
                          \cr
                          }  \eqno(20) $$
%
and
%
$$ \kappa_\nu^{\rm Res} = \cases{
                          n_1 { { h \nu_{U1} } \over { 4 \pi^2 } }
                         {\cal B}_{1U} { \delta_{\rm Res}
                          \over { (\Delta \nu)^2 } } 
                          & $U > 2$
                          \cr
                          \cr
                          10^{-17} n_1^2 \pi f_{12} r_{\rm el} F(\lambda,T)
                          & $U = 2$
                          \cr
                          } \eqno(21)  $$
%
(where $r_{\rm el} = 2.818 \times 10^{-13}$ cm is the classical electron radius),
and
% 
$$ \kappa_\nu^{\rm Stark} = n_1 { {h\nu_{U1}} \over {4\pi^2} } {\cal B}_{1U}
                            { \delta_{\rm Stark} \over (\Delta\nu)^2 }
                                      \, .  \eqno(22) $$
% 

We set $\kappa_\nu^{\rm Nat} = 0$ for $\lambda \geq 142.5$ nm in equation (20)
since this is a component of the Rayleigh scattering opacity which we determine
separately according to
%
$$ \sigma_\nu^{\rm Ray} = \cases{
                          0
                          & $\lambda < 142.5\ {\rm nm}$
                          \cr
                          \cr
                          n_1 R(\lambda)
                          & $\lambda \geq 142.5\ {\rm nm}$
                          \cr
                          } \, .  \eqno(23) $$
%
The values of $R(\lambda)$ are given by: M. Gavrila, 1967, Phys.Rev., 163, 147, 
Table I.

The standard formula for $\kappa_\nu^{\rm Res}$ in equation (21) is used only
for the $U > 2$ Lyman lines. For the wings of Lyman alpha we can use values of
$F(\lambda,T)$ calculated by N. F. Allard (private communication, October 2002),
based on general unified theory, including all the transitions contributing to
Lyman alpha, taking into account the variation of the dipole moment during the
collision, averaging over velocity. (HH potentials from: T. Detmer, P. Schmelcher,
L. S. Cederbaum, 1998, J.Chem.Phys., 109, 9694; dipole moments from: I. Drira,
1999, J.Mol. \break Spectrosc., 198, 52.) $F(\lambda,T)$ has been tabulated
for two values of temperatures,
$T = 5000$ K and $T = 8000$ K, and for the wavelength range $261.3429\ {\rm nm}
\geq \lambda \geq 110.1760\ {\rm nm}$ (the actual values can be found in the
listing of subroutine {\tt reaper}; for example,
$F(140.1269, 8000) = 9.7522 \times 10^{-11}$).
We obtain any value of $F(\lambda,T)$ by linear
interpolation/extrapolation between 4000 K and 10,000 K; we use
$F(\lambda,4000)$ for $T < 4000$ K and $F(\lambda,10000)$ for $T > 10,000$ K.
We set $F(\lambda,T) = 0$ for values of $\lambda$ outside the tabulated
wavelength range.
\blankline
While we use equation (1) in the line core and equation (13) in the line wings
for $1 \le U \le 15$, for $U > 15$ only the line core eqution is used.
At wavelengths near the $U = \infty$ Rydberg limit $(911.75347 \ldots$ \AA) there
are contributions from many overlapping Lyman lines. As many of these 
contributions are computed and summed as seem significant.

For $U > {\bf NQLYM}$ linear combinations of the Lyman line
and Lyman continuum absorption coefficients and source functions can be used
to avoid a discontinuity at the Rydberg limit; when ${\bf NQLYM} = 0$, however,
such linear combinations are not used (and ${\bf NQLYM} = 0$ is the default).
\ej
% \blankline
% \blankline
% \blankline
\centerline{2. \underbar{Practice}}
\blankline
\blankline
A simulation of the H Lyman-$\alpha$ line has been used in PANDORA as one of
the contributors to the background (or ``continuum'') absorption for a long time.
(In the 1960s Yvette Cuny's work, revealing the enormous wings of this line,
had shown this to be an important opacity in the sun.) From the beginning
H Ly-$\alpha$ absorption and scattering have been treated separately. Recently
further incremental additions were made to the treatment of background H Lyman
lines simulations.

This incremental development has resulted in five different contributors to
the background absorption and emission calculations, as shown in the printout
section BACKGROUND:
\settabs 6 \columns
\blankline
\+  & 11 --- H Ly alpha Abs. && = H(2/1) Line absorption \cr
\+  & 16 --- H Ly alpha Sct. && = H(2/1) Line scattering \cr
\+  & 34 --- H Ly 3-15 Abs.  && = H(u,1), $3 \le {\rm u} \le 15$, Lines absorption \cr
\+  & 36 --- H Ly 3-15 Sct.  && = H(u,1), $3 \le {\rm u} \le 15$, Lines scattering \cr
\+  & 37 --- H Ly$>$15 Abs.  && = H(u,1), ${\rm u} > 15$, Lines absorption \cr
\blankline
\noindent(See printout section BACKGROUND for more information regarding these
background contributors.)
% \ej
\blankline
The following input parameters control these calculations: {\bf NLY};
${\bf LMXX}_j$, ${\bf LMDR}_j$, $1 \le j \le {\bf LLY}$; {\bf LMXC}, {\bf LMXP},
{\bf LMDL2}; {\bf LMDL3}; {\bf LMZ}, {\bf LMH}; and {\bf NQLYM}, {\bf IFALL},
{\bf JHLSK}, {\bf PMSK}. These
are briefly discussed below; their values for the run are listed in printput
section INPUT, under the subheading ``Hydrogen Lyman alpha opacity.''

If the additional input parameter ${\bf LYODS} > 0$, then extensive printouts
of computational details, for every {\bf LYODS}'th depth, will be printed for
every wavelength for which dumps of the background calculation details have been
turned on (input table {\bf DWAVE}; see Section 5, Note 65).
\blankline
For the lines $(u,1), 2 \le u \le 15$, PANDORA's simulation needs values of
$XLIM$ (the boundary beween core and wing) and of $DR(x)$ (which describes the
change from zero coherent scattering in the Doppler core to partial coherent
scattering in the wings).
%
$$ XLIM = \cases{ {\bf LMXC} 
                  & ${\bf LMXC} > 0$ \cr
                  \cr
                  {\bf LMXX}_1 
                  & otherwise \cr
                  } \, . $$
% 
\ej

$DR(x)$ is computed by the same procedure as that used for calculating PRD
terms in PANDORA's detailed line source function calculations. (In many
Hydrogen runs the $(2,1)$-line and the $(3,1)$-line are computed with PRD.)
This calculation of $DR(x)$ is described in subsections (3.) and (4.)
of Section 15. The input parameters {\bf LMXC}, {\bf LMXP}, {\bf LMDL2},
and {\bf LMDL3}, and the input tables ${\bf LMXX}_j$,
${\bf LMDR}_j$, $1 \le j \le {\bf LLY}$, are used for this.
(Note that ${\bf LMXC} = 0$ is not allowed.)
\blankline
When ``H Ly 3-15'' (the ``higher Lyman lines'') are simulated, the upper
limit is the value of {\bf NLY} (default = 15). ``H Ly$>$15'' (the
``highest Lyman lines'') are done only when ${\bf NLY} = 15$; in that
case, as many of the converging lines are computed as ``make a difference''
to the total absorption (or emission) resulting from many overlapping line
wings (this calculation uses the built-in upper limit 500).
\blankline
Values of the contributions to the background absorption or emission due
to the cores and wings of the Lyman-$\alpha$ and the higher Lyman lines
are computed only at wavelengths less than {\bf LMZ} \AA (default = 1682);
values of the highest Lyman lines are computed only at wavelengths less
than {\bf LMH} \AA (default = 950). The lower wavelength limit for all
these simulations is the head of the Lyman continuum (the Rydberg
wavelength, $\approx$ 911 \AA).
\blankline
The Lyman-$\alpha$ line simulation can use N. F. Allard's experimental
values of $F(\lambda,T)$, as mentioned earlier. These data are used only
when {\bf IFALL} = 1 (the default); otherwise $F(\lambda,T) = 0$ 
is used.
\blankline
Stark broadening can be included in the simulations, as mentioned earlier.
This is done only when {\bf JHLSK} = 1 (the default). {\bf PMSK} is
a parameter for computing Stark damping.
\blankline
{\bf NQLYM} controls the provision for avoiding absorption and emission
discontinuities at the Lyman continuum edge.
%\vfill \vfill
%\vfill \vfill
%\vfill \vfill
%\vfill \vfill
%\vfill \vfill
%\vfill \vfill
%\vfill \vfill
%\vfill 
\vfill
\noindent (Section 21 -- last revised: 2004 Nov 09)% \par
\message{Section 21 ends at page 21.\the\pageno}
\ej
%\end
