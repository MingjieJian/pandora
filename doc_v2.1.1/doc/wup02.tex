%\magnification=1200
%\input wupstuff.tex
\newtoks\footline \footline={\hss\tenrm 2.\folio\hss}
\pageno=1
\top
\vskip 1.5 true in
\centerline{Section 2: {\bf Input Statements}}
\blankline
\blankline
\centerline{\bf ***}
\blankline
\blankline
PANDORA's input routines distinguish among several types of input
statements, according to their format.
\blankline
\blankline
\noindent {\bf 1}) Input of single quatities:
\space
NAME \quad ( \quad {\it q} \quad ) 
\space
where {\it q} can be a single {\intg}, {\flpt} or {\alfa} variable.
\blankline
\blankline    
\noindent {\bf 1*}) Input of single quatities with one Z index:
\space
NAME \quad {\it z} \quad ( \quad {\it q} \quad )
\space
where {\it z} (\intg) is a Z index, and {\it q} can be a single
{\intg} or {\flpt} variable.
\blankline
\blankline
\noindent {\bf 2}) Input of simple arrays:
\space
NAME \quad ( \quad {\it q} \quad )
\space
where {\it q} can be an array of {\intg} elements or of {\flpt} elements.
\blankline
\blankline
\noindent {\bf 2*}) Input of simple arrays with one Z index:
\space
NAME \quad Z \quad {\it z} \quad ( \quad {\it q} \quad )
\space
where {\it z} (\intg) is a Z index, and {\it q} can be an array of {\intg}
elements or of {\flpt} elements.
\ej
\noindent {\bf 3}) Input of simple arrays with one level index or one Z index:
\space
NAME \quad {\it k} \quad ( \quad {\it q} \quad )
\space
where {\it k} (\intg) is either a level index or a Z index,
and {\it q} can be an array of {\intg}
elements or of {\flpt} elements.
\blankline
\blankline
\noindent {\bf 3*}) Input of simple arrays with one level index and one Z index:
\space
NAME \quad {\it k} \quad Z \quad {\it z} \quad ( \quad {\it q} \quad )
\space
where {\it k} (\intg) is a level index, {\it z} (\intg) is a Z index, and {\it q}
can be an array of {\intg} elements or of {\flpt} elements.
\blankline
\blankline
\noindent {\bf 4}) Input of single quantities with two level indices:
\space
NAME \quad {\it u} \quad {\it l} \quad ( \quad {\it q} \quad )
\space
where {\it u} (\intg) and {\it l} (\intg) are two indices specifying a particular
transition, ({\it u} $>$ {\it l}$\,$), and {\it q} can be a single {\intg},
{\flpt} or {\alfa} variable.
\blankline
\blankline
\noindent {\bf 5}) Input of simple arrays with two level indices:
\space
NAME \quad {\it u} \quad {\it l} \quad ( \quad {\it q} \quad )
\space
where {\it u} (\intg) and {\it l} (\intg) are two indices specifying a particular
transition, ({\it u} $>$ {\it l}$\,$), and {\it q} can be an array of {\intg}
elements or of {\flpt} elements.
\blankline
\blankline
\noindent {\bf 5*}) Input of simple arrays with two level indices and one
Z index:
\space
NAME \quad {\it u} \quad {\it l} \quad Z \quad {\it z} 
\quad ( \quad {\it q} \quad )
\space
where {\it u} (\intg) and {\it l} (\intg) are two indices specifying a particular
transition, ({\it u} $>$ {\it l}$\,$), {\it z} (\intg) is a Z index,
and {\it q} can be an array of {\intg}
elements or of {\flpt} elements.
\ej
\noindent {\bf 6}) Input of {\bf WEIGHT}:
\space
{\bf WEIGHT} \quad {\it u} \quad {\it l} \quad (
\quad {\it m} \quad {\it n} \quad {\it w} \quad )
\space
where {\it u} (\intg) and {\it l} (\intg) are two indices specifying a particular
transition, ({\it u} $>$ {\it l}$\,$), {\it m} (\intg) and {\it n} (\intg) are two
indices specifying a particular term, ({\it m} $>$ {\it n}), and {\it w}
(\flpt) is the weight itself.
\blankline
\blankline
\noindent {\bf 7}) Input of single quatities with two level indices and one
$\mu$ index:
\space
NAME \quad {\it u} \quad {\it l} \quad {\it k} \quad ( \quad {\it q} \quad )
\space
where {\it u} (\intg) and {\it l} (\intg) are two indices specifying a particular
transition, ({\it u} $>$ {\it l}$\,$), {\it k} (\intg) is an index specifying a
particular value of {\bf MU}, and {\it q} can be a single \intg, {\flpt} or
{\alfa} variable.
\blankline
\blankline
\noindent {\bf 8}) Input of simple arrays with two level indices and one
$\mu$ index:
\space
NAME \quad {\it u} \quad {\it l} \quad {\it k} \quad ( \quad {\it q} \quad )
\space
where {\it u} (\intg) and {\it l} (\intg) are two indices specifying a particular
transition, ({\it u} $>$ {\it l}$\,$), {\it k} (\intg) is an index specifying a
particular value of {\bf MU}, and {\it q} can be an array of {\intg} elements or 
of {\flpt} elements.
\blankline
\blankline
\centerline{NOTE}
\blankline
\centerline{``NAME,'' ``Z,'' and ``{\bf WEIGHT},'' can be UPPER-, lower-, or
MIxed-CasE.}
%\vfill \vfill
%\vfill \vfill
%\vfill \vfill
%\vfill \vfill
%\vfill \vfill
%\vfill \vfill
%\vfill \vfill
%\vfill 
\vfill
\noindent (Section 2 -- last revised: 1997 Apr 02) \par
\message{Section 2 ends at page 2.\the\pageno}
\ej
%\end
