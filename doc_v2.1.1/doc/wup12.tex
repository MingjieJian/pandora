%\magnification=1200
%\input wupstuff.tex
\newtoks\footline \footline={\hss\tenrm 12.\folio\hss}
\pageno=1
\top
\vskip 1.5 true in
\centerline{Section 12: {\bf Source Functions}}
\blankline
\blankline
\centerline{\bf ***}
\blankline
\blankline
PANDORA has three methods for calculating the weight matrices needed for
source function calculations:
\blankline
\bull {\bf QR} -- \np
the quadratic representation method, which requires an input parameter, Y,
the `damping parameter' (see: SAO Special Report No. 303);
\blankline
\bull {\bf RT} -- \np
the ray tracing method, with analytic angle integration (see: E. H. Avrett's
program specification `[7/1/71] New Subroutine for Weighting Coefficients');
\blankline
\bull {\bf GR} -- \np
a general ray tracing method based on an integral equation approach,
with numerical angle integration (see: E. H. Avrett's program specification
`[81 Feb 27] Expansion Velocity in the Source Function Calculations').
\ej
When the {\bf QR} method is used for the Line Source Function calculation
\break (where {\bf KS} or KF weight matrices are required) or for the 
Level-${\cal N}$-to-Continuum ({\it i.e.} Lyman) Source Function
calculation (where {\bf KK} weight matrices are required), then it can be
used in one of two flavors: `QR-direct' or `QR-mapped.'

When `QR-direct' is specified, weight matrices (as functions of TNU, 
mono-chromatic optical depth) are computed directly as needed. 

When `QR-mapped' is specified, a `standard weight matrix'
is computed once (using the standard TAU table {\bf TS},
of length {\bf M}, and the standard value of the damping parameter {\bf YPRE});
the required weight matrices 
(which are functions of {\bf KS}, KF, or {\bf KK}
different TNU tables) are obtained from the standard matrix by an interpolation
procedure. {\it Note}: the purpose of the mapping procedure is to save
time; it is less accurate than the direct method.
In cases where the {\bf QR} method is chosen, 
mapping might be appropriate for the early iterations
of a calculation, switching to direct calculation of weight matrices 
when homing in on the final solution.

The input values of the various `method control parameters' MCP are used
to specify which method to use for the various source function calculations
to be performed. A value of MCP is interpreted as follows: 
\blankline \noindent
$\qquad$ $0 \, \leq$ MCP $\leq$ +1 means: {\bf QR}-direct; \np
$\qquad$ $\phantom{0 \, \leq}$ MCP = --2 means: {\bf QR}-mapped; \np
$\qquad$ $\phantom{0 \, \leq}$ MCP = --1 means: {\bf RT}; \np
$\qquad$ $\phantom{0 \, \leq}$ MCP = --3 means: {\bf GR}.
\blankline 
\noindent {\it Note:} When MCP selects {\bf QR}-direct, then the value of
MCP is also used as the required `damping parameter' Y = MCP. \np
{\it Note}: {\bf GR} is used automatically for all source
function calculations when options SPHERE and/or EXPAND are on. \np
{\it Note}: {\bf RT} is the default method for stationary
plane-parallel cases.
\blankline
Remember the following input parameters: {\bf TMS}, which affects
{\bf RT} and {\bf GR}; {\bf TBAR}, which affects {\bf QR}-mapped only;
{\bf TLARGE}, which affects {\bf QR} and {\bf QR}-mapped;
and {\bf TSMALL}, which affects all except {\bf GR} (see Section 5).
\blankline
({\it Note} added 2006 Aug 30: The decision to use a single floating
point input parameter combining the values of MCP and Y dates from the
ancient past. I have long wanted to undo it in favor of using two separate
quantities. Today such a change would require a great deal of work because
this ancient convention so pervades PANDORA. It is now much too late for that.)
\ej
The following list recapitulates the names of the various method control
parameters (single values, or tables -- see Section 5)
used to select the weight matrix
method to be used for the various types of source function calculations:
\blankline
\bull {\bf YPRE} -- for {\bf QR}-mapped, as discussed above;
\blankline
\bull {\bf YLINE} -- for Line Source Function;
\blankline
\bull {\bf YL} -- for Level-${\cal N}$-to-Continuum (Lyman) source function;
\blankline
\bull {\bf YCONT} -- for Continuum Source Function at the 
wavelengths of the cores of radiative transitions;
\blankline
\bull {\bf YWAVE} -- for Continuum Source Function at `additional'
 wavelengths;
\blankline
\bull {\bf YRATE} -- for Continuum Source Function at the wavelengths 
specified for the rates integrations;
\blankline
\bull {\bf YHM} -- for Continuum Source Function at the wavelengths
specified for the H-minus calculation;
\blankline
\bull {\bf YLDT} -- for Continuum Source Function at the wavelengths 
specified for the Type-2 Dust Temperature calculation;
\blankline
\bull {\bf YLYM} -- for Continuum Source Function at the wavelengths
specified for the Level-${\cal N}$-to-Continuum (Lyman) transfer calculation;
\blankline
\bull {\bf YCR} -- for Continuum Source Function at the wavelengths 
at which incident coronal radiation is specified;
\blankline
\bull {\bf YKR} -- for Continuum Source Function at the wavelengths
for which additional photoionization is specified;
\blankline
\bull {\bf BANDY} -- for Continuum Source Function at Composite Line
opacity wavelengths in a particular band;
\blankline
\bull {\bf YRATS} -- for Continuum Source Function at standard rates
integrations wavelengths;
\blankline
\bull {\bf YCOL} -- for Continuum Source Function at CO-lines absorption
wavelengths.
\blankline
\blankline
\vfill \vfill
\vfill \vfill
\vfill \vfill
\vfill \vfill
\vfill \vfill
\vfill \vfill
\vfill \vfill
\vfill \vfill
\noindent (Section 12 -- last revised: 2006 Nov 02) \par
\message{Section 12 ends at page 12.\the\pageno}
\ej
%\end
