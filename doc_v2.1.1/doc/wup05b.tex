%\magnification=1200
%\input wupstuff.tex
\newtoks\footline \footline={\hss\tenrm 5.\folio\hss}
%\pageno=58
%  available note # : 141
%
\headline={\centerline{\it Notes}}
\blankline
\blankline
\centerline{\underbar{$\qquad \qquad$ {\bf *1} $\qquad \qquad$}}
\space \noindent
The default value of {\bf ABD} can be obtained from the element tables
(see Section 10), provided that {\bf ELSYM} is a recognizable chemical
element symbol.
\blankline
\blankline
\centerline{\underbar{$\qquad \qquad$ {\bf *2} $\qquad \qquad$}}
\space \noindent
{\bf INPAIR} is a list of pairs of integers $u, \ell \; (u > \ell)$,
which specify transitions between levels. These pairs may be listed in
any order, as long as the proper pair relationships are preserved.
For example, if (2,1), (5,1), (5,2), (5,3) and (5,4) are the transitions
to be specified, then the input statement might be: \np
``{\tt INPAIR ( 5 1   5 2   5 3   2 1   5 4 ) }''. \np
The first pair will be used as the default for ({\bf MS},{\bf NS}).
\blankline
\blankline
\centerline{\underbar{$\qquad \qquad$ {\bf *3} $\qquad \qquad$}}
\space \noindent
{\bf NAME} must not have more than 8 characters, and may not contain
imbedded blanks.
\blankline
\blankline
\centerline{\underbar{$\qquad \qquad$ {\bf *4} $\qquad \qquad$}}
\space \noindent
{\bf RHOPT} specifies the RHO option, and may only take on one of the following
values: ``{\tt{RHOS}}'', ``{\tt{RHOJ}}'', or ``{\tt{RHOW}}''.
\blankline
\blankline
\centerline{\underbar{$\qquad \qquad$ {\bf *5} $\qquad \qquad$}}
\space \noindent
When {\bf JSTCN} $> 0$, then PANDORA will do only background continuum
calculations and emergent continuous spectrum calculations (depending on the
specific options settings); this is called: a `continuum-only' run. \np
The value of {\bf JSTCN} controls which wavelengths will be included in the
computations of this run. \np
{\bf JSTCN} should be set equal to $KAW + 2 \times KOM + 4 \times KCO$, where 
\np
$KAW = 1$ means: use `additional wavelengths' ({\it i.e.}
{\bf WAVES}), $KAW = 0$ means: do not; \np
$KOM = 1$ means: use Composite Lines Opacity wavelengths;
$KOM = 0$ means: do not; \np
$KCO = 1$ means: use CO-lines opacity wavelengths; $KCO = 0$ means: do not.
\ej
% \blankline
% \blankline
\centerline{\underbar{$\qquad \qquad$ {\bf *6} $\qquad \qquad$}}
\space \noindent
{\bf JSTCN} and {\bf JSTIN} may not both be $> 0$ for the same run. \np
{\bf JSTCN} and {\bf NOION} may not both be $> 0$ for the same run.
\blankline
\blankline
\centerline{\underbar{$\qquad \qquad$ {\bf *7} $\qquad \qquad$}}
\space \noindent
Any of the absorbers/emitters automatically included in the 
background opacity/emission
calculations can be turned off by mentioning them in {\bf NABS} statement(s).
For this purpose, the absorbers/emitters are specified using the ``index''
by which they are identified in the left margin of the `List of potential
contributors to ``continuum'' or ``background'' absorption and
emission.' which is part of the ATMOSPHERE printout at the beginning of
the regular output from a run. For example, the following statement
will cause the H Ly alpha Abs and the H$_2^+$
opacity to be turned off: ``{\tt NABS ( 11 9 ) }''.
\blankline
\blankline
\centerline{\underbar{$\qquad \qquad$ {\bf *8} $\qquad \qquad$}}
\space \noindent
The {\bf POPUP} switch is used to tell PANDORA whether this is a run in which
the number densities computed at the end of every overall iteration must also
be copied into the appropriate `non-LTE populations' tables set aside for
certain ions. The {\bf POPUP} switch may only take on the values: \np
1) {\tt HYDROGEN}, 2) {\tt CARBON}, 3) {\tt SILICON}, 4) {\tt HELIUM},
5) {\tt HELIUM2}, 6) {\tt ALUMINUM},\np
7) {\tt MAGNESIUM}, 8) {\tt IRON}, 9) {\tt SODIUM}, 10) {\tt CALCIUM},
11) {\tt OXYGEN}, or 12) {\tt SULFUR}; \np
{\it i.e.}: ``{\tt POPUP ( CARBON ) }''.
A run that is not concerned with one of these should not have {\bf POPUP}
among its input statements. Moreover, runs with `population ions' can be
made without `population updating' simply by not specifying {\bf POPUP}. \np
In a `population update' run, the values of the integer array {\bf RUNTOPOP}
specify the correspondence between the levels of the ion-of-the-run and the
levels of the built-in population-ion-model. (The description of the built-in
population-ion-model is printed as part of the first printout of the 
corresponding `population ion' number densities and departure coefficients;
note that specific options must be turned on for this, {\it e.g.} CARPRNT.)
{\bf RUNTOPUP}$_i = j$ means that level $j$ of the ion-of-the-run
corresponds to level $i$ of the built-in population ion model. \np
If $k$ is the lowest level of the built-in population-ion-model to which
no level of the ion-of-the-run corresponds, then {\bf RUNTOPOP}$_\ell = 0, 
k < \ell \leq $ LIMDAT, is required. (Note that the default values are
{\bf RUNTOPOP}$_i = 0, 1 \leq i \leq $ LIMDAT.)
\ej
% \blankline
% \blankline
\centerline{\underbar{$\qquad \qquad$ {\bf *9} $\qquad \qquad$}}
\space \noindent
Input tables of {\bf TRN} may be specified in abbreviated forms: either \np
(a) $\qquad$
{\bf TR ( I} $k \; x_k \; x_{k+1} \; x_{k+2} \; \ldots 
\; x_m$ {\bf ) },\np
{\it i.e.} ``{\tt TR ( I 12 3600. 3700. 3750. 3800. ) }'', or \np
(b) $\qquad$
{\bf TRN} $j$ {\bf ( I} $k \; x_k \; x_{k+1} \; x_{k+2} \; \ldots 
\; x_m$ {\bf ) },\np
{\it i.e.} ``{\tt TRN 2 ( I 12 3600. 3700. 3750. 3800. ) }''\np
In both these forms, several elements of an array, (beginning, here,  with
the 12. one), are set equal to the floating point numbers given.
\np
Case (a) tells PANDORA to do the following (after {\bf GO} has been read):
\np
to set TR$_i = x_k, \; 1 \leq i \leq k$, to leave TR$_i = x_i,
\; k+1 \leq i \leq m$, as specified in the input statement, and to set
TR$_i = $ TE$_i, \; m+1 \leq i \leq $N. \np
Case (b) is equivalent to (a) when $j = 1$; otherwise, it specifies the
$j$'th TRN table. \np
The rationale for this procedure derives from the structure of a typical
TRN table, as follows: the initial elements of the table are equal to
some constant value, then there is a variation extending over several
values, until, finally, the values of TRN are equal to the values of
TE at the corresponding interior depths. Thus the abbreviated input form
specifies the index of the last of the constant values, that constant
itself, and then the set of values which differ from TE. \np
{\it Note}: 
This abbreviated input form may only be used when the TRN
table is specified with respect to the main depth table of the run;
it may not be used when the TRN table is specified with respect to a
{\bf ZAUX} (auxiliary depth) table.
\blankline
\blankline
\centerline{\underbar{$\qquad \qquad$ {\bf *10} $\qquad \qquad$}}
\space \noindent
After {\bf GO} has been read, PANDORA examines the values of 
{\bf KPC}$^{u,\ell}$ for every radiative transition $(u,\ell)$. 
If all the values of {\bf KPC}$^{u,\ell}$ are = 0 (normal default), 
then it sets 
{\bf KPC}$^{u,\ell}$ = {\bf KPCR}$^{u,\ell}_i \times $ {\bf KPC}$^{MS,NS}_i,
\; 1 \leq i \leq $ N, where ({\bf MS},{\bf NS}) is the reference transition.
After this process is finished, {\bf KPC}$^{u,\ell}_i = 0$
can only come about because {\bf KPC}$^{MS,NS}_i = 0$ (if there were
no explicit input values $\neq 0$), or, for $i,j \neq MS,NS$, because
either {\bf KPC}$^{MS,NS}_i = 0$, or because {\bf KPCR}$^{u,\ell} = 0$. \np
Those transitions for which nonzero input values of continuous opacity
were obtained by this procedure, will retain those values for the
first overall iteration. For those transitions for which no
nonzero values of continuous opacity could be obtained, PANDORA will
compute them. (Normally, PANDORA computes opacity values for
every radiative transition in every overall iteration.)
\ej
% \blankline
% \blankline
\centerline{\underbar{$\qquad \qquad$ {\bf *11} $\qquad \qquad$}}
\space \noindent
When {\bf NOION} = 1 ({\it i.e.} if the option DOION is off), 
then no ion-related calculations will be done.
{\bf JSTCN} and {\bf NOION} may not both be $> 0$ for the same run.
See also Note 84.
\blankline
\blankline
\centerline{\underbar{$\qquad \qquad$ {\bf *12} $\qquad \qquad$}}
\space \noindent
{\bf KONFORM} controls the format of the printing of detailed contributions
to the background opacity and the absorption source function ({\it i.e.}
the detailed printouts of `absorbers' and `emitters'). \np
{\bf KONFORM} = 1
means: print their absolute values, using Fortran E-format conversion; \np
{\bf KONFORM} = 2 means: print them as fractions of the total, using
Fortran \break F-format conversion.
\blankline
\blankline
\centerline{\underbar{$\qquad \qquad$ {\bf *13} $\qquad \qquad$}}
\space \noindent
$X^{s+1}_i = {(1 - W_i)} \times X^{new}_i + W_i \times X^s_i, \; 
1 \leq i \leq N$, where $X$ stands for RHO; or \np
$X^{s+1}_i = [ (X^{new}_i)^{(1 - W_i)} ] \times [ (X^s_i)^{W_i} ], \; 1 \leq i 
\leq N$, where $X$ stands for RK-KOLEV (Lyman). \np
The values of $X_i$ for iteration $s+1$ will be obtained, in the manner
shown, from the values of $X_i$ used during iteration $s$ and the $new$
values of $X_i$ calculated at the end of iteration $s$. The values
of $W_i$ (which may all be equal to some constant) are obtained from
WRHO, {\bf WR}, {\bf WRLY}, and {\bf WTW}$_j$ as described in the
writeup [7/3/72], (WRHO is continually recomputed as the calculation proceeds),
or from {\bf RHOWT}$_i$ or {\bf RKW}$_i$, respectively, as described in the
writeup [74 Oct 23]. \np
{\bf Important}: Note 13 is now obsolete; {\bf see Note 131}.
\blankline
\blankline
\centerline{\underbar{$\qquad \qquad$ {\bf *14} $\qquad \qquad$}}
\space \noindent
{\bf BLCSW}$^{u,\ell}$ tells which components of DP to use for the Line
Source Function calculation for transitions $(u, \ell)$.
Its input value should be established
according to $BLCSW = SRD + 2 \times SVW + 4 \times SSK 
+ 8 \times SRS + 16 \times SIC$. \np
Here $SRD = 1$ if radiative broadening should be used, $= 0$ if not;
$SVW$, $SSK$, $SRS$, and $SIC$ similarly control van der Waals, Stark,
resonance, and ion collision broadening, respectively. ({\it Note}: ion
collision broadening is used for Hydrogen transitions above level 5 only.)
In the ATOM printout, the values of $SRD$, $SVW$, $SSK$, $SRS$, and 
$SIC$ are shown as a string of digits ({\it i.e.} as a
binary number) in order from right to left.
\ej
% \blankline
% \blankline
\centerline{\underbar{$\qquad \qquad$ {\bf *15} $\qquad \qquad$}}
\space \noindent
$X^{s+1}_i = $ {\bf HSEC}$ \times X^{new}_i + (1 - ${\bf HSEC}$) \times X^s_i, 
\; 1 \leq i \leq N$, where X stands for either NE or NH. \np
When the option HSE is on, the $X_i$ for overall iteration $s+1$ will be
obtained, in the manner shown, from the $X_i$ used during iteration $s$
and the $new$ values of $X_i$ calculated at the end of overall iteration $s$.
\blankline
\blankline
\centerline{\underbar{$\qquad \qquad$ {\bf *16} $\qquad \qquad$}}
\space \noindent
{\bf LSFBOC}$^{u, \ell}$ and {\bf OML}$^{u, \ell}$ control whether any of the
`Line Background' \break opacities (see Section 9) are allowed as potential 
contributors to the total background (or ``continuum'') opacity at wavelength(s)
pertaining to transition $(u, \ell)$. \break If {\bf LSFBOC} = 0 then all these
`Line Background' opacities will be suppressed; \break if {\bf LSFBOC} = 1 then
the appropriate one, multiplied by {\bf OML}, will be used. \np
By default, all {\bf LSFBOC} = 0 and all {\bf OML} = 1.
\blankline
\blankline
\centerline{\underbar{$\qquad \qquad$ {\bf *17} $\qquad \qquad$}}
\space \noindent
If the value of {\bf PROF}$^{u,\ell} = 0$, then no emergent intensity nor flux
profiles will be computed for transition $(u,\ell)$. If any value of
{\bf PROF}$^{u,\ell}$ is $> 0$, then {\bf LF} $> 0$ is required. (A flux profile
will not be computed if {\bf LF} = 1; see also Note 24.) \np
Line profile intensity and flux values are computed for tables of
$\Delta \lambda$ values which are derived for frequency ({\bf XI}) values;
see Section 18, Frequency Tables, for additional information.
\blankline
\blankline
\centerline{\underbar{$\qquad \qquad$ {\bf *18} $\qquad \qquad$}}
\space \noindent
{\bf OLL}$^{u, \ell}$ multiplies the values of GTN$^{u, \ell}$.
(GTN$^{u, \ell}_i$ multiplies the line absorption profile $\phi^{u, \ell}
_i$ in the equation for the monochromatic total opacity at the various
frequency points of transition $(u, \ell)$.)
\ej
% \blankline
% \blankline
\centerline{\underbar{$\qquad \qquad$ {\bf *19} $\qquad \qquad$}}
\space \noindent
The RHO selection parameters: \np
{\bf RHOPT} -- \np
Values of RHO$^{u,\ell}_i$ are calculated three different ways:
\bull as part of the Line Source Function calculation (RHOS);
\bull from S, S*, RHOS and {\bf CWJ} (RHOJ); and
\bull by combining RHOS and RHOJ (RHOW). 
(See also the explanation printed with the RHO AND RBD section of the
normal output.) 
The value of {\bf RHOPT} tells which of these is to be
chosen as the final RHO$^{u,\ell}_i$, to be used in subsequent iterations.
\spice \noindent
{\bf NWRHO}, {\bf WMN} and {\bf WMX} -- \np
At the end of every iteration $s$, the current values of
CHECK$_i^j$ are compared to the ones from the previous iteration $s-1$,
for $1 \leq i \leq N$ and $3 \leq j \leq NL$. \break A counter $K$ is
initialized to zero.
Whenever: $CC = |$CHECK$_i^j -1| > 0.002$ and $OC = |$CHECK$_{i-1}^j
-1| > 0.002$ and $OC / CC \leq 0$, then $K$ will be increased by 1.
All values of CHECK$^j_i$ will be tested to determine $K$.
If $K > $ {\bf NWRHO} then,
if WRHO $<$ {\bf WMX}, WRHO will be increased by 0.1. 
If $K \leq $ {\bf NWRHO}, then,
if WRHO $>$ {\bf WMN}, WRHO will be decreased by 0.1. (If WRHO $<$ {\bf WMN} or
WRHO $\geq$ {\bf WMX}, then WRHO will remain unchanged.) This new value of WRHO
will then be used in the next iteration, as specified in Note 13.
\spice \noindent
{\bf CHOP}, {\bf CWR}, {\bf ILI}, {\bf NIL}, {\bf CHLIM}, and option RHOWOPT -- \np
The significance of these parameters is explained in the text accompanying
the RHO AND RBD printout (when option RHBPRNT is on). (If such a printout
happens not to be immediately at hand, it won't hurt to set up a one-iteration
run using the automatic defaults.)
\blankline
\blankline
\centerline{\underbar{$\qquad \qquad$ {\bf *20} $\qquad \qquad$}}
\space \noindent
If the option ISCRS is on ({\it i.e.} if {\bf ISCRS} = 0),
then `scratch I/O' will be done `in memory' to the
extent possible ({\it i.e.} depending on the amount of memory reserved for the
{\tt MEMOIR} routines). When memory is full, scratch I/O will overflow to the
temporary scratch disk file (logical unit 1, see Section 7).
(For some runs, scratch I/O can be accomodated entirely in memory,
and no temporary scratch disk file will be required; in other runs,
the size of the temporary scratch disk file will be reduced.) 
This `in-memory scratch I/O mode' may be particularly
beneficial when PANDORA is run on systems with real memory so large that
no hard page faults occur. \np
If the option ISCRS is off ({\it i.e.} if {\bf ISCRS} = 1),
then `in-memory scratch I/O' is not allowed; 
{\it all} scratch I/O will use the temporary scratch disk file. \np
See also Note 84.
\ej
% \blankline
% \blankline
\centerline{\underbar{$\qquad \qquad$ {\bf *21} $\qquad \qquad$}}
\space \noindent
{\bf METEP} specifies the method for computing EP1 and EP2 in the
Level-${\cal N}$-to-Continuum (Lyman) transfer calculation. \np
{\bf METEP}=0 means: NOVA-like method (writeup dated 06/04/68); \np
{\bf METEP}=1 means: COMPLEX/UPPER-like method (writeup dated 04/12/90); \np
{\bf METEP}=2 means: COMPLEX/LOWER-like method (writeup dated 06/22/76); \np
{\bf METEP}=3 means: CHAIN-like method (writeup dated 11/24/76). \np
(The values of {\bf METEP} are analogous to those of {\bf METSE};
see Note 35.)
\blankline
\blankline
\centerline{\underbar{$\qquad \qquad$ {\bf *22} $\qquad \qquad$}}
\space \noindent
For a particular energy level $j$, {\bf WRAT}$^j$ specifies a set of wavelength
values, {\bf RRCP}$^j$ a tabular function of these wavelength values, and
{\bf YRATE}$^j$ a set of method control parameters (see Section 12) for
Continuum Source Function calculations at these wavelengths. We have
{\bf WRAT}$^j_m$, {\bf RRCP}$^j_m$, and {\bf YRATE}$^j_m, \; 1 \leq m \leq $
{\bf MR}$^j+1$. Since the value of {\bf WRAT}$^j_1$ is obtained from
$({\bf NUC}^j - {\bf NU}^j)$, this value need not (actually: {\it cannot}) be
input; and since {\bf RRCP}$^j_1$ usually $= 1.0$, this value normally
need not be input. Thus, for every level for which {\bf MR}$^j > 0$,
the input must contain specifications of the values of {\bf WRAT}$^j_m$ and
{\bf RRCP}$^j_m$, \break $2 \leq m \leq $ {\bf MR}$^j+1$. On the other hand,
all values of {\bf YRATE}$^j_m$ can \break (actually: {\it must}) be input.
Thus, the input statements for {\bf WRAT} and {\bf RRCP} are different
from those for {\bf YRATE}. For example, if we have:
\space
\settabs 6 \columns
\+ & $m$ & WRAT$^2_m$ & RRCP$^2_m$ & YRATE$^2_m$ \cr
\+ & 1 & (3612.0) & (1.0) & -1.0 \cr
\+ & 2 & 3575.0 & 0.99 & 1.0 \cr
\+ & 3 & 3540.0 & 0.97 & 0.9 \cr
\space \noindent
the input statements might be:\np
``{\tt WRAT 2 ( 3575. 3540. ) }'', \np
``{\tt RRCP 2 ( 0.99, 0.97 ) }'', \np
``{\tt YRATE 2 ( -1.0 1.0 0.9 ) }''. \np
However, there are infrequent occasions when {\bf RRCP}$^j_1$ should 
not $= 1.0$. In such a case, such a value can be input by referring to
that array under another name, namely {\bf RQCP}, for which the first
value can (actually: {\it must}) be input (as with {\bf YRATE}).
Thus if, in the above example, {\bf RRCP}$^2_1 = 1.2$ is wanted,
the input statements might be: \np
``{\tt WRAT 2 ( 3575. 3540. ) }'', \np
``{\tt RQCP 2 ( 1.2 .99, .97 ) }'', \np
``{\tt YRATE 2 ( -1.0 1.0 0.9 ) }''.
\ej
% \blankline
% \blankline
\centerline{\underbar{$\qquad \qquad$ {\bf *23} $\qquad \qquad$}}
\space \noindent
{\bf OUTPUT} can take on the values: {\tt MERGE}, {\tt SPLIT}. See
Section 7, the part headed ``{\bf Output files}''.
\blankline
\blankline
\centerline{\underbar{$\qquad \qquad$ {\bf *24} $\qquad \qquad$}}
\space \noindent
The {\bf MU} table (for intensity) must be a proper subset of the 
{\bf MUF} table (for flux); thus it is required that {\bf L} $\leq$ 
{\bf LF} and that, for every $m$, {\bf MU}$_m$ = {\bf MUF}$_n$, some $n$.
Both {\bf MU} and {\bf MUF} must be in order of decreasing values.
\blankline
\blankline
\centerline{\underbar{$\qquad \qquad$ {\bf *25} $\qquad \qquad$}}
\space \noindent
{\bf CE} is used in the calculations of the collisional transition rates,
which normally depend on NE. However, if some input value of
{\bf CE}$^j$ is negative, then its absolute value will be used, and the
corresponding rate calculation for level $j$ will use NH in place of NE.
\blankline
\blankline
\centerline{\underbar{$\qquad \qquad$ {\bf *26} $\qquad \qquad$}}
\noindent
{\bf WPOP} is used for the iterative calculation of number densities,
as follows: \np
$ P^{i+1} = 10^{[ W \times \log(P^{new}) + (1 - W) \times \log(P^i) ]}$,
where $W$ stands for {\bf WPOP} and
$P$ stands for the level populations or the ionized number density
computed at the end of an overall iteration. 
When the POPUP switch is on, the values of $P^{i+1}$ for overall iteration
$i+1$ will be obtained, in the manner shown, from the $P^i$ available at the
start of iteration $i$ and the $P^{new}$ computed during iteration $i$. \np
{\bf WBD} is used similarly for departure coefficients.
\blankline
\blankline
\centerline{\underbar{$\qquad \qquad$ {\bf *27} $\qquad \qquad$}}
\space \noindent
Any values of {\bf WAVES}$_k$ may be $< 0.0$, and only $|${\bf WAVES}$_k|$
will be used, both for the `additional'
background continuum calculations and the emergent continuous spectrum
calculations. Moreover, when {\bf WAVES}$_k < 0.0$, then the Continuum
Eclipse Intensities for $|${\bf WAVES}$_k|$ will also be computed 
(provided the option ECLIPSE is on).
\ej
% \blankline
% \blankline
\centerline{\underbar{$\qquad \qquad$ {\bf *28} $\qquad \qquad$}}
\space \noindent
These input statements are not intended for general use. They provide
emergency means for changing the built-in contents of the population-ion-model
data tables. (These data are printed the first time that the values of
number density and departure coefficient for the `population update ion'
are printed.) The preferred way to change these data is for me to change the
program. \np
The statements have the following forms: \np
{\bf POPION} $ \, k \quad j \quad ( \quad V \quad )$, \np
where $k$ is the `population update ion' number (as in Note 8); $V$ is a simple
array specifying all or some of the elements of data table $j$; and $j$
designates one of the population-ion-model data tables, as follows:
$j = 1$ for XLMTHR, threshhold wavelength (\flpt), $j = 2$ for CCPLEV,
coefficient CCP (\flpt), $j = 3$ for NPTABL, LM-table length (\intg),
$j = 4$ for SCPLEV, exponent SCP (\flpt), $j = 5$ for PILEVL, statistical
weight (\flpt), $j = 6$ for LLABEL, term designation (\alfa, $\leq$ 16
characters, embedded blanks not permitted). \np
{\bf POPXLM} $\, k \quad \ell \quad ( \quad V \quad )$, \np
where $k$ and $V$ are as for {\bf POPION} above, and $\ell$ designates a
level of the \break population-ion-model. This statement provides
values for one of the LM tables. \np
{\bf POPRCP} $\, k \quad \ell \quad ( \quad V \quad )$, \np
which is like {\bf POPXLM}, but provides values for one of the
RCP tables.
\blankline
\blankline
\centerline{\underbar{$\qquad \qquad$ {\bf *29} $\qquad \qquad$}}
\space \noindent
The size limits on {\bf NAB}, {\bf NCL}, {\bf NMT}, {\bf NSL},
{\bf NVX} and the total number of Composite Lines Opacity wavelengths
(see Section 3, Notes) are for programming convenience; 
to enlarge them requires changing the source code.
\blankline
\blankline
\centerline{\underbar{$\qquad \qquad$ {\bf *30} $\qquad \qquad$}}
\space \noindent
When detailed PRD results are printed, the amount of output can be controlled
as follows: data for frequency value $XI = 0$ are printed, and for every
{\bf IPRDF}$^{th}$ one from there; data for the first Z value are printed,
and for every {\bf IPRDD}$^{th}$ one thereafter.
\ej
% \blankline
% \blankline
\centerline{\underbar{$\qquad \qquad$ {\bf *31} $\qquad \qquad$}}
\space \noindent
{\bf A}$^{u,\ell}$ can be greater than or equal to zero. If zero, then there
is no line transition between levels $u$ and $\ell$; if greater than zero,
then there is. The input parameter {\bf KTRANS}$^{u, \ell}$ (see Note 33)
further specifies the type of transition.
Line Source Functions are computed iteratively for radiative transitions,
and only once (from the final iterated results) for passive transitions.
\blankline
\blankline
\centerline{\underbar{$\qquad \qquad$ {\bf *32} $\qquad \qquad$}}
\space \noindent
The default values of
{\bf ND}$^j_i, \; 1 \leq i \leq$ N, $1 \leq j \leq$ NL, are zero, unless this
is a run with the POPUP switch on, in which case the input population data,
or the LTE population data computed before the first overall iteration, will
be used as the defaults for {\bf ND}$^j_i$.
\blankline
\blankline
\centerline{\underbar{$\qquad \qquad$ {\bf *33} $\qquad \qquad$}}
\space \noindent
{\bf KTRANS}$^{u,\ell}$ is used only when {\bf A}$^{u,\ell} > 0$, and describes
that transition. It may take on the values: ``{\tt RADIATIVE}'', 
``{\tt PASSIVE}'', ``{\tt THICK}, ``{\tt THIN}'', or ``{\tt 2-PHOTON}''.
\blankline
\blankline
\centerline{\underbar{$\qquad \qquad$ {\bf *34} $\qquad \qquad$}}
\space \noindent
The ordinates controlled by the options OPAGRAF and EMIGRAF increase 
logarithmically (base 10) upwards from {\bf BLIMG} to {\bf TLIMG}.
\blankline
\blankline
\centerline{\underbar{$\qquad \qquad$ {\bf *35} $\qquad \qquad$}}
\space \noindent
{\bf METSE}$^{u,\ell}$ selects the method of computing the statistical 
equilibrium equations ({\it i.e.} the values of PE$^{u,\ell}$ and 
FE$^{u,\ell}$) for transition $(u,\ell)$. \np
{\bf METSE}$^{u, \ell}$ = 0 means: NOVA; \np
{\bf METSE}$^{u, \ell}$ = 1 means: COMPLEX/upper; \np
{\bf METSE}$^{u, \ell}$ = 2 means: COMPLEX/lower; \np
{\bf METSE}$^{u, \ell}$ = 3 means: CHAIN; \np
{\bf METSE}$^{u, \ell}$ = 4 means: VAMOS. \np
(Except for VAMOS, the values of {\bf METSE} are analogous to those
of {\bf METEP}; see Note 21.) 
Suggestions for choosing {\bf METSE} are given in the printout
section ``LINE (U/L).'' 
The default value of {\bf METSE}$^{u, \ell}$ = {\bf METSEDW} if
$\ell = 1$, and = {\bf METSEDG} otherwise.
\ej
% \blankline
% \blankline
\centerline{\underbar{$\qquad \qquad$ {\bf *36} $\qquad \qquad$}}
\space \noindent
{\bf LSFFDB}$^{u,\ell} = 0$ means: use `constant background' for the Line
Source Function calculation ({\it i.e.} use line-core background opacity
and source function for all line integration frequencies).
{\bf LSFFDB}$^{u,\ell} = 1$ means: use `varying background' for the Line
Source Function calculation ({\it i.e.} compute background opacity and
source function explicitly at all line integration frequencies, as in a
PRD solution). FDB solutions can only be calculated for radiative
transitions ({\it i.e.} {\bf KTRANS}$^{u,\ell} =$ ``{\tt RADIATIVE}'')
using the `full' solution ({\it i.e.} {\bf LSFTYP}$^{u,\ell}$ = 0). \np
{\it Note:} {\bf LSFFDB} is set $= 1$ automatically when PRD is 
used (see Note 47).
\blankline
\blankline
\centerline{\underbar{$\qquad \qquad$ {\bf *37} $\qquad \qquad$}}
\space \noindent
{\bf MODLAB} (up to 8 characters long) is an `atmospheric model name'
used for the MODEL DATA section of the normal output, and for the
`performance data archive record' (in file 28; set Section 7).
\blankline
\blankline
\centerline{\underbar{$\qquad \qquad$ {\bf *38} $\qquad \qquad$}}
\space \noindent
The PANDORA output includes many printed graphs as functions of depth. \break
The abscissae of the graphs are established under control of input
parameters \break {\bf IZOPT}, {\bf NGRL} and {\bf NGRR},
as follows: \np
{\bf IZOPT} = 1 means: the abscissa is depth index $i, \; LG \leq i \leq MG$;
\np
{\bf IZOPT} = 2 means: the abscissa is $Z_i, \; LG \leq i \leq MG$; \np
{\bf IZOPT} = 3 means: the abscissa is $\log ( |Z_i| ), \; LG \leq i
\leq MG$, but excluding \break $Z=0$, (moreover, if $Z_i$ changes sign in the
range of interest, proceed as if \break {\bf IZOPT} = 1); \np
{\bf IZOPT} = 4 means: the abscissa is $\log(TAU_i), \; LG \leq i \leq MG$.

Here $LG$ and $MG$ are determined as follows: 
if {\bf NGRL} $\leq 0$, then $LG = 1$; if {\bf NGRL} $> 0$, then 
$LG = $ {\bf NGRL}, but the limit $LG \leq (N-1)$ is enforced;
if {\bf NGRR} $\leq 0$, then $MG = N$; if {\bf NGRR} $> 0$, then
$MG = $ {\bf NGRR}, but the limit $MG \leq N$ is enforced.

For some graphs, $LG$ and $MG$ are also controlled by the value of the
input parameter {\bf JZOPT}, as follows: \np
{\bf JZOPT} = 1 means: use the above procedure controlled by {\bf IZOPT}; \np
{\bf JZOPT} = 0 means: $LG$ is set to the greatest value of $i$ such that
$TAU_i \leq 10^{-4}$ ($LG > 1$ if {\bf IZOPT} = 4), and $MG$ is set to the
smallest value of $i$ such that $TAU_i \geq 10^3$.
\ej
% \blankline
% \blankline
\centerline{\underbar{$\qquad \qquad$ {\bf *39} $\qquad \qquad$}}
\space \noindent
The selector {\bf IRLS1} is used to select RLA ({\bf IRLS1} = 1) or RLB
({\bf IRLS1} = 2) for $RL^j, j =$ {\bf KOLEV}; {\bf IRLSN} is used similarly
for $RL^j, \; j \neq$ {\bf KOLEV}.
\blankline
\blankline
\centerline{\underbar{$\qquad \qquad$ {\bf *40} $\qquad \qquad$}}
\space \noindent
When the option JSTIN is on ({\i.e.} if {\bf JSTIN} $ > 0$),
then PANDORA will just read all the input, print it,
and stop; this is called an `input-only' run. See also Note 84.
\blankline
\blankline
\centerline{\underbar{$\qquad \qquad$ {\bf *41} $\qquad \qquad$}}
\space \noindent
{\bf KHFFS} controls whether H free-free is part of the subtotal labelled
`Total Hydrogen' in the cooling rate calculation of a Hydrogen run.
${\bf KHFFS} = 1$ means that it is, ${\bf KHFFS} = 0$ means that it is
not, part of that subtotal. 
\spice \noindent
{\bf KOOLSUM} controls which components are added to `Total Hydrogen'
to compute the grand total labelled `Total Cooling Rate' in a
Hydrogen run. \np
{\bf KOOLSUM}
should be set equal to $K1 + 2 \times K2 + 4 \times K3 + 8 \times K4
+ 16 \times K5$, where\np
$K1 = 1$ means: add H-minus, $K1 = 0$ means: do not add H-minus; \np
$K2 = 1$ means: add conduction, $K2 = 0$ means: do not add conduction; \np
$K3 = 1$ means: add composite lines, $K3 = 0$ means: do not add composite
lines; \np
$K4 = 1$ means: add X-ray, $K4 = 0$ means: do not add X-ray; \np
$K5 = 1$ means: add CO-lines, $K5 = 0$ means: do not add CO-lines.
\blankline
\blankline
\centerline{\underbar{$\qquad \qquad$ {\bf *42} $\qquad \qquad$}}
\space \noindent
See Section 15, Partial Redistribution.
\blankline
\blankline
\centerline{\underbar{$\qquad \qquad$ {\bf *43} $\qquad \qquad$}}
\space \noindent
{\bf BDOPT} specifies the BD option, and may take on one of the following
values: ``{\tt BDJ}'', ``{\tt BDR}'', or ``{\tt BDQ}''
(see explanation in RHO AND RBD section of normal printout).
\ej
% \blankline
% \blankline
\centerline{\underbar{$\qquad \qquad$ {\bf *44} $\qquad \qquad$}}
\space \noindent
PANDORA can read input from several files. The parameter {\bf USE} specifies
which one of these files to use; it can take on any one of the following
values: ``{\tt ATOM}'', ``{\tt INPUT}'', ``{\tt MODEL}'', ``{\tt RESTART}''
or ``{\tt GENERAL}''. \np
After a {\bf USE} statement has been read, the next input statement(s) will
be read from the file designated by the last {\bf USE} statement.
Specifications to read from any file may be arbitrarily intermixed among the
input statements. The first occurrence of {\bf USE} {\it must}
be in {\tt INPUT} ({\it i.e.} file 03, see Section 7). \np
When ``{\tt USE ( GENERAL )} '' occurs, reading will continue from 
the file defined by the last preceding {\bf FILE} statement. 
Thus, at least one {\bf FILE} statement {\it must} \break {\it precede} the 
first occurrence of  ``{\tt USE ( GENERAL )} ''. \np
The {\bf FILE} statement specifies two parameters, $k$ and $f$, in that order;
for example, ``{\tt FILE ( 0 FILESPEC ) }''. $f$ is \alfa\ 
($ \leq 60$ characters, embedded blanks not permitted); 
it is the ``complete'' file specification of a file to be read,
\ie whatever the operating system requires under the current circumstances
to identify the file uniquely, in other words, $f$ is used as the file-name
specification with the FILE keyword in a Fortran `{\tt open}' statement.
$k$ (\intg) is a switch: when $k = 1$, a message containing the value of
$f$ will be printed on in the `message printout file' (\ie file 16, see
Section 7) whenever ``{\tt USE ( GENERAL )} ''occurs; when $k = 0$,
no message will be printed.
\blankline
\blankline
\centerline{\underbar{$\qquad \qquad$ {\bf *45} $\qquad \qquad$}}
\space \noindent
{\bf IRKCOMP} is an array of switches specifying for which levels $RK$ should
be computed in the Rates calculation of the first iteration. 
{\bf IRKCOMP}$^j = 1$ means: compute $RK^j$; {\bf IRKCOMP}$^j = 0$ means:
do not compute $RK^j$ (in this case, presumably, $RK^j$ was supplied in
the input of the run). {\bf IRLCOMP}, analogously, concerns $RL$.
\blankline
\blankline
\centerline{\underbar{$\qquad \qquad$ {\bf *46} $\qquad \qquad$}}
\space \noindent
If a set of $JNU$ values for partial redistribution (PRD) calculations
has been provided in a file on unit 09, then PANDORA can be made to read them
by setting {\bf JNUNC} = 1. These $JNU$ values pertain to particular values of
$Z$ and $XI$; if necessary, these data will be interpolated
to the values of $Z_i, \; 1 \leq i \leq N$, and of $XI_k, \; 1 \leq k \leq K$,
currently in use.
\ej
% \blankline
% \blankline
\centerline{\underbar{$\qquad \qquad$ {\bf *47} $\qquad \qquad$}}
\space \noindent
For general background, see Section 15. \np
{\bf SCH}$^{u,\ell} = 0$ means: do not include partial redistribution (PRD)
calculations for line $(u,\ell)$; \np
{\bf SCH}$^{u,\ell} = 1$ means: do include PRD calculations for line $(u,\ell)$. \np
{\it Note:} Only {\bf SCH}$^{u,\ell} = 0$ is allowed when {\bf DIRECT}$^{u,\ell} = 1$. \np
{\it Note:} PRD solutions are only calculated for radiative transitions, {\it i.e.}
those with \break {\bf KTRANS}$^{u,\ell} = $ ``{\tt RADIATIVE}''. \np
{\it Note:} When PRD is requested for line $(u,\ell)$, then {\bf LSFFDB}$^{u,\ell}$
is set $= 1$ automatically. \np
{\it Note:} In a stationary plane-parallel atmosphere, a single ray 
({\it i.e.} the normal) is traced; in a spherical and/or expanding atmosphere,
many rays are traced. The number of frequency values used must be less 
than 1000. Moreover, the product of the number of frequency values used
(see also Note 66) times the number of depth points probably should not be
larger than 10000, except for good reason.
\blankline
\blankline
\centerline{\underbar{$\qquad \qquad$ {\bf *48} $\qquad \qquad$}}
\space \noindent
{\bf LSFTYP}$^{u,\ell}$ is the Line Source Function solution method selection
selector, used as follows: \np
{\bf LSFTYP}$^{u,\ell} = 0$ means: do a `full Line Source Function solution',
computing the line source function using frequency/angle sums and a final
grand matrix, and then $RHO$ and $JBAR$ from that $S$; \np
{\bf LSFTYP}$^{u,\ell} = 1$ means: compute the 
line source function `directly' from the number
densities of levels $u$ and $\ell$, do frequency/angle sums, then compute
$RHO$ from those sums and $S$, and then $JBAR$; \np
{\bf LSFTYP}$^{u,\ell} = 2$ means: compute the line source function from the
number densities of levels $u$ and $\ell$, compute $RHO$ with the
`escape probability approximation', and then $JBAR$ from them. 
(See also Note 51.) \np
`Direct' solutions are only calculated for radiative transitions,
{\it i.e.} {\bf KTRANS}$^{u,\ell}$ = ``{\tt RADIATIVE}''. \np
{\it Note:} 
If transition $(u,\ell)$ is a passive transition, or if the
Line Flux Distribution calculation has been requested for it, then the
value of {\bf LSFTYP}$^{u,\ell}$ will automatically be forced $= 1$! \np
{\it Note:} 
If $TAU^{u,\ell}_2 > $ {\bf ESCTAU}, then the `static'
escape probability approximation is used for transition $(u,\ell)$,
regardless of the value of {\bf LSFTYP}$^{u,\ell}$.
\ej
% \blankline
% \blankline
\centerline{\underbar{$\qquad \qquad$ {\bf *49} $\qquad \qquad$}}
\space \noindent
The statement {\bf SMOOTH} $u$ $\ell$ {\bf (} {\it q} {\bf )} 
({\it e.g.} ``{\tt SMOOTH 3 1 ( ILS 17 WSM 0.5 ) }'')
is used for the parameters governing smoothing of the $RHO$ values of
transition $(u,\ell)$. {\it q} consists of pairs of input fields of the
form ``{\it A} {\it v} '', where {\it A} is an alphanumeric identification
field whose value may be ``{\tt WSM}'', ``{\tt IFS}'' or ``{\tt ILS}'', and
{\it v} is a numerical field, which must be floating point if it follows
{\tt WSM}, and integer otherwise. WSM is the smoothing weight ( = 0.0 means:
no smoothing; 0.0 is the default). IFS and ILS are limiting indices
such that smoothing is done only for RHO$^{u,\ell}_i, \; 1 \leq IFS \leq i
\leq ILS \leq N$; (defaults are: IFS = 1 and ILS = N).
\blankline
\blankline
\centerline{\underbar{$\qquad \qquad$ {\bf *50} $\qquad \qquad$}}
\space \noindent
The Voigt profile may be computed by one of three subroutines, as specified
by the input parameter {\bf IVOIT}. \np
{\bf IVOIT} = 1 selects George Rybicki's method; this is the most precise
routine but also the slowest, as it takes about 12 times longer than the
high-speed routine. \np
{\bf IVOIT} = 2 selects S. R. Drayson's method, which takes about 2 times
longer than the high-speed routine; its relative difference from Rybicki's
routine is generally below the seventh significant figure, but sometimes in
the fourth significant figure. \np
{\bf IVOIT} = 3 selects Eric Peytremann's method, which is the high-speed
routine; it has a relative difference from Rybicki's routine that is generally
less than 0.001, but sometimes as large as 0.05. \np
If the value $V = Voigt(x,a)$ must be computed immediately after the 
value \break $V^{\prime} = Voigt(x^{\prime},a^{\prime})$ has been computed,
then, if both $x = x^{\prime}$ and $a = a^{\prime}$ to relative accuracy
{\bf VOITC}, $V = V^{\prime}$ will be used (to save time). \np
{\it Note:} If {\bf IHSSW} $= 1$ and any value of {\bf CSTARK}$^{u,\ell} = 1$
(indicating that a convolved Stark profile should be computed) in a Hydrogen
run, then PANDORA will set ${\bf IVOIT} = 3$ and ${\bf NVOIT} = 1$.
\ej
% \blankline
% \blankline
\centerline{\underbar{$\qquad \qquad$ {\bf *51} $\qquad \qquad$}}
\space \noindent
The {\bf SOBOLEV} statement specifies a particular transition $(u, \ell)$,
and provides an array of two indices: the first of these is ISB1$^{u, \ell}$,
and the second of these is ISB2$^{u, \ell}$. \np
The `escape probability' solution for transition $(u,\ell)$ ({\it i.e.}
{\bf LSFTYP}$^{u,\ell} = 2$) comes in two flavors: `static' and
`Sobolev'. The static solution is used by default. If ISB1$^{u, \ell} 
> 1$, the Sobolev solution will be used for depths $i, \; 1 \leq i
\leq $ ISB1$^{u, \ell}$; the static solution will be used for the 
depths $i$, ISB2$^{u, \ell} \leq i \leq N$; and a linear
transition from one to the other for depths $i$, ISB1$^{u, \ell}
 < i < $ ISB2$^{u, \ell}$. \np
A table of velocity values, {\bf VSB}, is needed for the Sobolev calculation
(see Section 16, Velocities). \np
The integral needed for the Sobolev solution is computed by trapezoidal rule
using an interval refinement procedure which seeks to achieve a
piecewise-linear approximation to the run of the integrand, with a specified
tolerance. Three parameters control this interval refinement process:
{\bf SOBDMN} is the smallest interval size needed ({\it i.e.} intervals need
not be reduced further than {\bf SOBDMN}); {\bf SOBDMX} is the maximum 
acceptable interval size ({\it i.e.} intervals must be reduced to a size smaller
than {\bf SOBDMX}); {\bf SOBFEQ} is the tolerance to which interpolated and
actual values of the integrand must agree ({\it i.e.} these two quantities
are computed at the midpoint of the current interval, and if they do not
agree to this tolerance, then the interval must be reduced further).
\blankline
\blankline
\centerline{\underbar{$\qquad \qquad$ {\bf *52} $\qquad \qquad$}}
\space \noindent
{\bf IPPOD} controls absorption/emission calculation debug dumps at all
wavelengths for all `population update ions'; {\bf IPPOD} = 0 for nothing,
{\bf IPPOD} = 1 for absorption, {\bf IPPOD} = 2 for emission, {\bf IPPOD} = 3
for both. Printout occurs for every $Z_i$ such that {\tt mod}[$i$,{\bf LDINT}]
= 0.
\blankline
\blankline
\centerline{\underbar{$\qquad \qquad$ {\bf *53} $\qquad \qquad$}}
\space \noindent
`Level designations' are {\alfa} labels consisting of up to 8 (non-blank)
characters, which will be printed with the levels data as part of the
ATOM printout. \np
The {\bf LEVDES} statement DOES NOT conform to the standard
PANDORA input conventions. It has the form: \np
``{\tt LEVDES ( label1 label2 . . . . ) }''. \np
Exactly {\bf NSL} labels {\it must} be provided.
\ej
% \blankline
% \blankline
\centerline{\underbar{$\qquad \qquad$ {\bf *54} $\qquad \qquad$}}
\space \noindent
Detailed printout of transition terms ({\it i.e.} `A*RHO' and `Term added
to upward C') is controlled by {\bf LDINT} and {\bf LDTYP}.
(Of course, option ARHODMP must be on.)

{\bf LDTYP}
specifies the context from which printout is wanted: \np
{\bf LDTYP} = 1 means: statistical equilibrium calculation; \np
{\bf LDTYP} = 2 means: b-ratio calculation; \np
{\bf LDTYP} = 3 means: both.

{\bf LDINT}
specifies the depth points for which printout is wanted: \np
${\bf LDINT} > 0$ means: print every {\bf LDINTH}'th depth point, beginning
with the first point;\np
${\bf LDINT} < 0$ means: print for depth \# $\vert {\bf LDINT} \vert$ only; \np
${\bf LDINT} = 0$ means: no printout.

In the same way, {\bf LDINT} specifies the interval between depth points for
the printouts controlled by the options EPDMP and SEBUG, and by {\bf IPPOD}
(see Note 52). 
\blankline
\blankline
\centerline{\underbar{$\qquad \qquad$ {\bf *55} $\qquad \qquad$}}
\space \noindent
A simple simulation of a shock wave can be obtained by specifying the shock's
velocity and temperature perturbation. The shock velocity is further treated
in Section 16. The temperature perturbation is computed from input parameters
and then added to the input values of {\bf TE} to obtain the $TE_i$ actually
used in the run:
$$ TE^{used}_i = {\bf TE}^{input}_i + DTE_i \, , $$
where $DTE_i = 0$ for $i < {\bf JSSV}$ but
$$ DTE_i = {\bf SCTA} \exp [({\bf Z}_{\bf JSSV} - {\bf Z}_i) / {\bf SCTA}] \, . $$
Here {\bf JSSV} is the depth index of the shock's location, {\bf SCTA}
is the amplitude of the temperature perturbation, and {\bf SCTS}
is the temperature scale height.
\blankline
\blankline
\centerline{\underbar{$\qquad \qquad$ {\bf *56} $\qquad \qquad$}}
\space \noindent
{\bf CP}$^{NSL+1} > 0.0$ is the signal that the K-shell photoionization effect
is to be computed. If {\bf MR}$^{NSL+1}$ is then set $\neq 0$ but, say, equal
to $IKS$, then values of {\bf RRCP}$^{NSL+1}_m, \; 1 \leq m \leq IKS$ must be 
provided (as for the other levels), as well as $IKS+1$ values of {\bf WRAT}
(see also Note 22). Note that 
{\bf WRAT} input for the K-shell differs from that of the
other levels in that the wavelength of the head of that continuum must be
specified explicitly.
\ej
% \blankline
% \blankline
\centerline{\underbar{$\qquad \qquad$ {\bf *57} $\qquad \qquad$}}
\space \noindent
{\bf VX} values must be specified as functions of {\bf Z}
(the main depth-table of the run).
Extra(inter)polation involving {\bf ZAUX} tables has not been provided
for {\bf VX}.
\blankline
\blankline
\centerline{\underbar{$\qquad \qquad$ {\bf *58} $\qquad \qquad$}}
\space \noindent
If the value of {\bf ECLI}$^{u,\ell}$ is $> 0$, then Eclipse emergent line 
intensity and flux profiles will be computed for transition $(u,\ell)$,
but only if this is a transition for which a full-profile integration
is done.
\blankline
\blankline
\centerline{\underbar{$\qquad \qquad$ {\bf *59} $\qquad \qquad$}}
\space \noindent
Detailed dump printouts from the Line Source Function calculations
performed in subroutine {\tt PERSEUS} can be
obtained using the options PERDMP0, PERDMP1, \break PERDMP2 and 
PERDMP3. \np
The `frequency sums'
(subroutine {\tt DIANA}) or the `frequency/angle sums' (subroutine {\tt 
ORION)} accumulate the contributions from the `frequency data blocks' or the
`frequency/angle data blocks'. The contents of these data blocks will be printed
when the option PERDMP0 is on. \np
WN matrices of each data block, and the results
of computing the above sums, will be printed when the option PERDMP1 is on. \np
All terms and intermediates for these sums at selected depths will be printed
when the option PERDMP2 is on. \np
PRD data arrays will be printed when the option PERDMP3 is on. \np
The data blocks are identified by the value of the index $IND$, whose values
range from 1 to $K$, or from 1 to $NR \times K$, where $NR$ is the total number
of rays, and $K$ is the number of entries in the $XI$ table, as explained
in Note 66, below. Only data for {\bf IPR01} $ \leq IND \leq $ {\bf IPR02} 
will be dumped by PERDMP0, PERDMP1, PERDMP2 and/or PERDMP3. Moreover,
only the details for the depths \break
{\bf IPR03} $ \leq i \leq $ {\bf IPR04} will be dumped by PERDMP2. \np
These dump printouts will be provided only for transition ({\bf MS},{\bf NS}).
\blankline
\blankline
\centerline{\underbar{$\qquad \qquad$ {\bf *60} $\qquad \qquad$}}
\space \noindent
PANDORA can be instructed to compute an expanding atmosphere by turning
the option EXPAND on. When EXPAND is on, then the velocity table {\bf VXS}
is used; see Section 16, Velocities.
\ej
% \blankline
% \blankline
\centerline{\underbar{$\qquad \qquad$ {\bf *61} $\qquad \qquad$}}
\space \noindent
{\bf PART} is the partition function of the next higher stage of ionization
of the ion of the run, and is a constant. \np
If {\bf PART} $> 0.0$, then all values
of the depth-dependent partition function table will be set equal to it. \np
If {\bf PART} $ \leq 0.0$, then values of the depth-dependent partition function
of the ion of the run can be obtained as follows: \np
If the option PARTVAR is off, then, if the appropriate value of U2 can be
located in the element data tables (see Section 10), that will be used; if not,
the run will stopped. \np
If the option PARTVAR is on, then PANDORA attempts to compute the values
of the depth-dependent partition function with the subroutines incorporating
the `Hamburg' data. If this is unsuccessful, then the run will be stopped. \np
{\it Note:} 
Therefore, to obtain truly depth-varying entries in the
partition function table, the option PARTVAR must be on, and the value of
{\bf PART} must not be greater than 0.0.
\blankline
\blankline
\centerline{\underbar{$\qquad \qquad$ {\bf *62} $\qquad \qquad$}}
\space \noindent
The set of shell rays, and the associated weight matrices, for a spherical
atmosphere, are regulated by {\bf NTAN} and {\bf MSKIP}. The value of {\bf NTAN}
must be $> 0$, that of {\bf MSKIP} must be 0, 1, or 3. \np
Rays will be sent tangent to every {\bf NTAN}$^{th}$ shell (= depth), 
beginning with the innermost one ({\it i.e.} $Z_N$), 
and going out as far as possible. (The
option TANG also plays a role here.) \np
The value of {\bf MSKIP} matters only
when {\bf NTAN} = 1. In that case, \np
when {\bf MSKIP} = 0, then the weight matrices
for every shell ray will be computed directly; \np
when {\bf MSKIP} = 1, then
only weight matrices for every $2^{nd}$ shell ray will be computed directly,
while the others will be obtained by interpolation; \np
when {\bf MSKIP} = 3,
then only weight matrices for every $4^{th}$ shell ray will be computed
directly, while the others will be obtained by interpolation.
\blankline
\blankline
\centerline{\underbar{$\qquad \qquad$ {\bf *63} $\qquad \qquad$}}
\space \noindent
{\bf IRTIS} controls extra(inter)polation of the input table
of incident radiation. \np
{\bf IRTIS} = 1 means: linear in XINK, linear in FINK; \np
{\bf IRTIS} = 2 means: linear in XINK, linear in log(FINK); \np
{\bf IRTIS} = 3 means: linear in log(XINK), linear in log(FINK).
\ej
% \blankline
% \blankline
\centerline{\underbar{$\qquad \qquad$ {\bf *64} $\qquad \qquad$}}
\space \noindent
{\bf NHTSW} controls the calculation of the correction to the abundance of
atomic Hydrogen due to the possible presence of $H_2$ molecules. \np
{\bf NHTSW} = 0 means: do not compute the H2 abundance correction; \np
{\bf NHTSW} = 1 means: compute the H2 abundance according to Kurucz 
(1970); \np
{\bf NHTSW} = 2 means: compute the H2 abundance according to Tsuji; \np
{\bf NHTSW} = 3 means: compute the H2 abundance according to Kurucz (1985).
\blankline
\blankline
\centerline{\underbar{$\qquad \qquad$ {\bf *65} $\qquad \qquad$}}
\space \noindent
Detailed dump printouts from the Continuum Calculations (intended to help
with handchecks) can be obtained by means of various options
and input switches. Such printouts will only be provided for wavelengths listed
in the input table {\bf DWAVE}, of length {\bf NDV}, {\it except} when
${\bf DWAVE}_1 = 0.0$, in which case such printouts will be provided for all
wavelengths.  When values of {\bf DWAVE} are provided they must match to
8 figures or more the values of the Continuum Calculation wavelengths for
which dumps are wanted.
\blankline
\blankline
\centerline{\underbar{$\qquad \qquad$ {\bf *66} $\qquad \qquad$}}
\space \noindent
See Section 18, Frequency Tables, for more information about this input parameter.
{\bf See also} the {\it Note} at the end of Note 47.
\blankline
\blankline
\centerline{\underbar{$\qquad \qquad$ {\bf *67} $\qquad \qquad$}}
\space \noindent
The ``Lyman'' source function can be calculated in three ways: \np
1) using the coupled ``Lyman'' continuum equations, Case A; \np
2) using the large-depth, saturation, or generalized on-the-spot approximation,
Case B (see: Avrett and Loeser 1988, Ap.J., {\bf 331}, 221, Appendix B); or \np
3) using Case A for depths 1 through $\eta$ (if any), and Case B for depths 
$\eta + 1$ to N (if any); this is Case C, the `normal' procedure. \np
({\it Note} that $\eta =$ N means: Case A throughout;
$\eta = 0$ means: Case B throughout.) \np
PANDORA computes the value of $\eta$ as follows: it finds
smallest index $k$ for which both $TAUK_k > $ {\bf TGLYM} and 
$(EP1_k)^2 \times TAUK_k \ge $ {\bf EXLYM}. (If it finds no such index,
then $k =$ N.) It then sets $\eta = k$
unless: (a) if {\bf LN} $\le 1$, it sets $\eta = 0$; 
(b) if $k \le$ {\bf LN}, it sets $\eta = 0$; or 
(c) if TAUK$_2 \ge 10$, it sets $\eta = 0$. \np
{\bf TGLYM}, {\bf EXLYM} and {\bf LN} are input parameters. \np
{\it Special cases:} \np
1) When the option SPHERE is on, $\eta$ is set = N; \np
2) when the  option SPHERE is off but the option FINITE is on, $\eta$ is set = N. 
\ej
% \blankline
% \blankline
\centerline{\underbar{$\qquad \qquad$ {\bf *68} $\qquad \qquad$}}
\space \noindent
{\bf LSFGC} controls the format of all the Line Source Function graphs. \np
When {\bf LSFGC}$ = 1$, then the x-axis is {\bf Z}-index; \np
when {\bf LSFGC}$ = 2$, then the x-axis is {\bf Z}$_i$; \np
when {\bf LSFGC}$ = 3$, then two graphs will be provided, one of each format.
\blankline
\blankline
\centerline{\underbar{$\qquad \qquad$ {\bf *69} $\qquad \qquad$}}
\space \noindent
Use of broadening velocity (= microturbulent velocity) depends on the option
VSWITCH. When the option VSWITCH is off, then the broadening velocity is
isotropic and is specified by the input table {\bf V}. 
When the option VSWITCH is on, then the broadening is anisotropic: the input
values of {\bf V} are the tangential component, and the input values of
{\bf VR} are the radial component. See Section 16, Velocities.
\blankline
\blankline
\centerline{\underbar{$\qquad \qquad$ {\bf *70} $\qquad \qquad$}}
\space \noindent
The input value of {\bf IFXDS} should $= 8 \times ICF + 4 \times ICM +
2 \times IFD + IFX$. \np
To print flux tables, use $IFX = 1$, to omit them,
use $IFX = 0$. \np
To print flux derivative tables use $IFD = 1$, to omit
them, use $IFD = 0$. \np
To print cumulative derivative tables, use $ICM = 1$,
to omit them, use $ICM = 0$. \np
To print quadratic coefficients, use $ICF = 1$,
to omit them, use $ICF = 0$. \np
The default is {\bf IFXDS} = 0 ({\it i.e.} no details at all are printed.)
\blankline
\blankline
\centerline{\underbar{$\qquad \qquad$ {\bf *71} $\qquad \qquad$}}
\space \noindent
Many quotient calculations are done in subroutine {\tt DIVIDE},
which checks for denominator = 0.0. {\bf VSMLL} is used in place of
such vanished divisors. \np
{\tt DIVIDE} will print various error messages,
depending on the value of the control switch {\bf IPZER}. When {\bf IPZER}
= 0, no messages are printed; when {\bf IPZER} = 1, a message will be
printed every time `$A/0$' occurs; when {\bf IPZER} = 2, a message will
be printed every time `$0/0$' occurs; {\bf IPZER} = 3 enables both messages.
\blankline
\blankline
\centerline{\underbar{$\qquad \qquad$ {\bf *72} $\qquad \qquad$}}
\space \noindent
When {\bf LFLUX}$^{u,\ell} = 1$, then the Line Flux Distribution and the
Radiative Force for transition $(u,\ell)$ will be computed. The option
LFDPRNT affects the amount of printout. Dump output (for transition
[{\bf MS},{\bf NS}]) is controlled by {\bf IHDMP}.
\ej
% \blankline
% \blankline
\centerline{\underbar{$\qquad \qquad$ {\bf *73} $\qquad \qquad$}}
\space \noindent
The `HI/BYE' system, and the control parameters for it, are intended as
debugging aids for me as program developer. `HI/BYE' is not intended for
use in regular production runs.
\blankline
\blankline
\centerline{\underbar{$\qquad \qquad$ {\bf *74} $\qquad \qquad$}}
\space \noindent
When the option IRUNT is on ({\it i.e.} when {\bf IRUNT} = 1),
then more extensive execution performance data and
program version description data will be printed than when 
the option IRUNT is off ({\it i.e.} when {\bf IRUNT} = 0
(the default)). It should not be necessary to use {\bf IRUNT} = 1
in regular production runs.
\blankline
\blankline
\centerline{\underbar{$\qquad \qquad$ {\bf *75} $\qquad \qquad$}}
\space \noindent
The `banner' pages at the start of the printout file (see Section 3) display
`giant' characters that are themselves composed (in `dot-matrix' fashion)
of individual print characters. The value of {\bf NARB} tells how many such
pages to print, the value of {\bf KARB} selects the print
character(s) used for this purpose. \np
{\bf KARB} = 1 means: use ``\$'' to represent the giant character. \np
{\bf KARB} = -1 means: use ``O'' over ``X'' over ``+''
(this of course works only with processing systems 
which fully implement traditional `Fortran carriage control'). \np
{\bf NARB} may assume the values 0, 1, or 2.
\blankline
\blankline
\centerline{\underbar{$\qquad \qquad$ {\bf *76} $\qquad \qquad$}}
\space \noindent
In many runs, the sets of `Composite Lines Opacity' wavelengths and
`CO-lines Opacity' wavelengths are added automatically to the list of
wavelengths for which continuum calculations are done. If any value of
{\bf DELWAVE}$_i, 1 \leq i \leq $ {\bf NWS}, equals (to one part in $10^{10}$)
a wavelength in one of these two sets, then that wavelength value will be
deleted from the list of wavelengths for which continuum calculations are
done. (These counterparts to the `additional' wavelengths are therefore
called `subtractional' wavelengths.)
\blankline
\blankline
\centerline{\underbar{$\qquad \qquad$ {\bf *77} $\qquad \qquad$}}
\space \noindent
See also the note regarding maximum counter values, and order of occurrence
among the input, at the end of Section 3. See also Note 29.
\ej
% \blankline
% \blankline
\centerline{\underbar{$\qquad \qquad$ {\bf *78} $\qquad \qquad$}}
\space \noindent
{\bf DOPROF} is an alternate form of the {\bf PROF} statement; it contains
pairs of transition indices, like the {\bf INPAIR} statement (see Note 2).
The occurrence of the index pair $u,\ell$ in {\bf DOPROF} causes the program
to set {\bf PROF}$^{u,\ell} = 1$.
\blankline
\blankline
\centerline{\underbar{$\qquad \qquad$ {\bf *79} $\qquad \qquad$}}
\space \noindent
{\bf DOFLUX} is an alternate form of the {\bf LFLUX} statement; it contains
pairs of transition indices, like the {\bf INPAIR} statement (see Note 2).
The occurrence of the index pair $u,\ell$ in {\bf DOFLUX} causes the program
to set {\bf LFLUX}$^{u,\ell} = 1$.
\blankline
\blankline
\centerline{\underbar{$\qquad \qquad$ {\bf *80} $\qquad \qquad$}}
\space \noindent
{\bf DOFDB} is an alternate form of the {\bf LSFFDB} statement; it contains
pairs of transition indices, like the {\bf INPAIR} statement (see Note 2).
The occurrence of the index pair $u,\ell$ in {\bf DOFDB} causes the program
to set {\bf LSFFDB}$^{u,\ell} = 1$.
\blankline
\blankline
\centerline{\underbar{$\qquad \qquad$ {\bf *81} $\qquad \qquad$}}
\space \noindent
{\bf DOSFPRNT} is an alternate form of the {\bf LSFPRINT} statement; it contains
pairs of transition indices, like the {\bf INPAIR} statement (see Note 2).
The \break occurrence of the index pair $u,\ell$ in {\bf DOSFPRNT} causes the
program to set \break {\bf LSFPRINT}$^{u,\ell} = 1$. \np
{\it Note:} {\bf LSFPRINT}$^{u,\ell}$ and option LSFPRNT are different
beasties, but they work together.
\blankline
\blankline
\centerline{\underbar{$\qquad \qquad$ {\bf *82} $\qquad \qquad$}}
\space \noindent
See Section 16, Velocities, for more information about this input parameter.
\blankline
\blankline
\centerline{\underbar{$\qquad \qquad$ {\bf *83} $\qquad \qquad$}}
\space \noindent
When {\bf MTHEI} = 0, then exponential integrals are computed using Cooley's
routine, which gives about 8 figures for the first exponential integral;
higher orders are computed with the recursion relation which rapidly loses
precision for increasing orders. See Avrett \& Loeser, SAO Special Report
No. {\bf 303}, 1969. \np
When {\bf MTHEI} = 1, then exponential integrals are computed using a routine
published by Press and Teukolsky. This routine has been set up to deliver
about 14 figures for all orders; it is slower than Cooley's routine.
See Press \& Teukolsky, Computers in Physics, Sep/Oct 1988.
\ej
% \blankline
% \blankline
\centerline{\underbar{$\qquad \qquad$ {\bf *84} $\qquad \qquad$}}
\space \noindent
The switches {\bf ISCRS}, {\bf NOION}, {\bf NVOIT}, {\bf IXSTA},
{\bf JBDNC}, {\bf JSTIN}, \break {\bf IRPUN}, {\bf IRUNT} and {\bf TOPE}
are alternate forms of the options \break ISCRS, DOION, NVOIT, IXSTA,
JBDNC, JSTIN, RABDAT, IRUNT \break and TOPE, respectively. The
program will set their {\it default} values to agree with the status
of the corresponding options.
\blankline
\blankline
\centerline{\underbar{$\qquad \qquad$ {\bf *85} $\qquad \qquad$}}
\space \noindent
Subroutine {\tt EDITH} is used in various context to edit values $\le 0$
out of various computed tables; it is set up to print error messages to
report what it did. For each editing context $k$, a separate count of
error messages, $KERM_k$, is kept. The value of the input parameter
{\bf NERM} controls the printing of error messages: \np
when {\bf NERM} $= 0$, then no error messages are printed; \np
when {\bf NERM} $> 0$, then error messages from context $k$ are printed as
long as $KERM_k \le$ {\bf NERM}; \np
when {\bf NERM} $< 0$, then all error messages are printed, but greatly
abbreviated.
\blankline
\blankline
\centerline{\underbar{$\qquad \qquad$ {\bf *86} $\qquad \qquad$}}
\space \noindent
If {\bf ZNDW}$ \neq 0$, then {\bf NDW} will be set equal to $i$, where
$\min(|Z_i - ZNDW|)$, \break $ 1 \leq i \leq N$. If {\bf ZNDW} $=0$, then,
if {\bf NDW} $<1$ or {\bf NDW} $> N$, {\bf NDW} will be set
equal to $N/2$.
\blankline
\blankline
\centerline{\underbar{$\qquad \qquad$ {\bf *87} $\qquad \qquad$}}
\space \noindent
As the final step in constructing WN-matrices (which are used in source
function calculations), the computed matrix is scanned and every
element whose absolute value is less than {\bf WNJUNK} is set = 0.
\blankline
\blankline
\centerline{\underbar{$\qquad \qquad$ {\bf *88} $\qquad \qquad$}}
\space \noindent
{\bf LODCG} and {\bf NODCG} are parameters controlling the depth range
for diffusion graphs (no graphs appear when {\bf NODCG} = 0).
Various parameters are plotted as functions of {\bf Z},
for the range {\bf Z}$_I$ to {\bf Z}$_J$. \np
When {\bf LODCG}$ < 0$, then $I = 1$; when {\bf LODCG}$ > 0$, then
$I = {\bf LODCG}$. \np
When {\bf NODCG}$ < 0$, then $J = K$; when {\bf NODCG}$ > 0$, then
$J = {\bf NODCG}$ or $K$, whichever is less.
$K$ is the smallest depth index such that ${\bf TE}_K < 9000$.
\ej
% \blankline
% \blankline
\centerline{\underbar{$\qquad \qquad$ {\bf *89} $\qquad \qquad$}}
\space \noindent
d-coefficients are computed for the diffusion calculations. \np
When {\bf IDFDM} = 0, the ``original'' method is used; \np
when {\bf IDFDM} = 1, an ``improved,'' more complete method is used. \np
When options AMDDMP and/or GNVDMP are on, then the d-coefficients
will be printed for depth $i$; when {\bf IDFDI} $> 0$, then
$i =$ {\bf IDFDI}, but when no value of {\bf IDFDI} is input,
then $i =$ {\bf N}$/4$.
\blankline
\blankline
\centerline{\underbar{$\qquad \qquad$ {\bf *90} $\qquad \qquad$}}
\space \noindent
If the input values of {\bf V}$_i = 0, 1 \leq i \leq {\bf N}$, then, 
if ${\bf NVH} > 0$, ``default'' values of ${\bf V}_i$ will be
computed from ${\bf VNH}_j, 1 \leq j \leq {\bf NVH}$. Here
{\bf VNH} is a velocity table appropriate for the quiet Sun, and is
specified as a function of Hydrogen density, {\bf HNDV}. Values of
{\bf V}$_i$ corresponding to the input values of {\bf NH}$_i$ are
obtained by interpolation from the tables {\bf HNDV}$_j$ and {\bf VNH}$_j$;
the logarithms of {\bf NH}$_i$ and {\bf HNDV}$_j$ are used for this 
purpose. \np
\blankline
{\bf IMPORTANT}: if {\bf V}$_i = 0$ is intended, then {\bf NVH} must be set
$= 0$ explicitly in the input, since the default value of ${\bf NVH} > 0$.
\blankline
\blankline
\centerline{\underbar{$\qquad \qquad$ {\bf *91} $\qquad \qquad$}}
\space \noindent
{\bf CIJADD} uses special cases of Statement Forms 5* and 5, in that
$u > \ell$ is not required -- rather, all combinations $u \neq \ell$ are 
accepted.
\blankline
\blankline
\centerline{\underbar{$\qquad \qquad$ {\bf *92} $\qquad \qquad$}}
\space \noindent
{\bf ASMCR} and {\bf NIASM} are parameters used in sequential smoothing; 
see writeups [89 Dec 22] and [92 Jan 29]. The smoothing
used here is based on the following principle. Consider the sequence
of graphical points representing the values of a function. If the function
varies smoothly, a given point usually lies within the triangle 
defined by the two lines through the pair of points on either side
and the line through the point on either side, provided the two lines
intersect within the given interval. If the given point falls outside,
it is moved to the boundary of this triangle; if the given point lies
within this triangle, it is left unchanged. \np
Smoothing consists of a repeated search for the most deviant point,
and then changing it if necessary. Then this edited sequence is searched
for the most deviant point; and so on. The search is repeated at most
{${\bf NIASM} \times n$} times, where $n$ is the number of points
in the sequence. However, the process stops after the first time that
the relative change applied to the most deviant point is less than
{\bf ASMCR}.
\ej
% \blankline
% \blankline
\centerline{\underbar{$\qquad \qquad$ {\bf *93} $\qquad \qquad$}}
\space \noindent
See Section 19, Atomic Models, for more information about this input
parameter.
\blankline
\blankline
\centerline{\underbar{$\qquad \qquad$ {\bf *94} $\qquad \qquad$}}
\space \noindent
{\bf NLPAIR} is a list of integers $(n_j, \ell_j), \; 1 \leq j \leq 
{\bf NL}$, such that $n_j$ is the principal quantum number of level $j$
and $\ell_j$ is the rotational quantum number of level $j$ of the
model of the ion-of-the-run. For example, $(n_{22}, \ell_{22}) = (5, 3)$
means the for level 22 the principal quantum number $n = 5$ and the
rotational quantum number $\ell = 3$. \np
The value $\ell_j = -1$ is meaningful; this means that level $j$ is a
synthetic level obtained by combining all of the sublevels characterized
by the same value of $n$ but different values of $\ell$. \np
When $\ell_j = -1$, then $n_j$ must be $> 0$. For
Hydrogen, default values of $(n_j, \ell_j)$ are provided: $n_j = j$
and $\ell_j = -1$. \np
In the {\bf NLPAIR} statement, the $(n, \ell)$ pairs should be specified
in level sequence, beginning with level 1.
\blankline
\blankline
\centerline{\underbar{$\qquad \qquad$ {\bf *95} $\qquad \qquad$}}
\space \noindent
The ``\underbar{upper-level charge-exchange}'' calculation 
is enabled when option CHEXUP is on and this is a run with Hydrogen, or
with a ``charge-exchange'' ion. The list of eligible ``charge-exchange'' ions
is built into PANDORA; it consists of: He-I, C-I, N-I, O-I, Na-I, Mg-I,
Al-I, Si-I, S-I, and \break Ca-I, in that order. A particular ``charge-exchange''
ion is internally known to PANDORA by the value of $MCX, \; 1 \leq MCX \leq 10$,
{\it i.e.} by the index of that ion in the above list. \np
In a run other than Hydrogen in which ``upper-level charge-exchange'' is
enabled, only a subset of levels, namely ``levels affected by charge-exchange''
a.k.a. ``CX-levels,'' is affected. Any level $j$ of the model of the 
ion-of-the-run whose rotational quantum number $\ell_j \neq 0, 1, 2$ is a
CX-level, but only if that level's principal quantum number $n_j > \ell_j$.
Such runs produce the output tables {\bf XRKH} and {\bf XRLH} for
use in Hydrogen upper-level charge-exchange runs. \np
In a Hydrogen run with ``upper-level charge-exchange,''
all levels $j, \; j \geq 4$, are CX-levels. Such runs make use of various
input sets of {\bf XRKH} and {\bf XRLH} produced by upper-level charge-exchange
calculations with other elements.
\ej
% \blankline
% \blankline
\centerline{\underbar{$\qquad \qquad$ {\bf *96} $\qquad \qquad$}}
\space \noindent
The arrays {\bf XRKH}$^{k,n}_i$ and {\bf XRLH}$^{k,n}_i$ are input for the
upper-level charge exchange calculation (option CHEXUP on) in Hydrogen runs. \np
The superscript $k, \; 1 \leq k \leq 10$, designates a particular
charge-exchange element (see also Note 95); the superscript $n, \; 1 \leq n 
\leq {\bf NL}$, designates the affected level of the Hydrogen ion model.
(In the case of Hydrogen, the level number is numerically equal to the
principal quantum number of that level.) The indices $k$ and $n$ appear
in that order in an {\bf XRKH} or {\bf XRLH} input statements; for example: \np
``{\tt XRKH 4 6 ( . . . . . ) }'' \np
which are data from a Mg-I run for level 4 of the Hydrogen ion model; or: \np
``{\tt XRLH 14 2 ( . . . . . ) }'' \np
which are data from a C-I run for level 14 of the Hydrogen ion model. \np
{\it Note:} values of $n$ must not be greater than {\bf NL}.
\blankline
\blankline
\centerline{\underbar{$\qquad \qquad$ {\bf *97} $\qquad \qquad$}}
\space \noindent
{\bf RCHX}$^{n,\ell}$ is a parameter used for the upper-level charge-exchange
calculation (option CHEXUP on). The value of {\bf RCHX}$^{n,\ell}$ is used
for all those levels of the model of the ion-of-the-run for which the
principal quantum number is $n$ and the rotational quantum number is $\ell$.
The indices $n$ and $\ell$ appear in that order in the {\bf RCHX} input
statement; for example: \np
``{\tt RCHX 4 3 ( 1.23 ) }'' \np
which sets {\bf RCHX}$^{4,3} = 1.23$.
\blankline
\blankline
\centerline{\underbar{$\qquad \qquad$ {\bf *98} $\qquad \qquad$}}
\space \noindent
The emergent line profile graphs for transition $(u,\ell)$ (option
INTGRAF) can be controlled with {\bf PROGLI}$^{u,\ell}$, which
controls the limits of the wavelength axis, and with {\bf SGRAF}$^{u,\ell}$,
which applies to `blended' lines. \np
When ${\bf PROGLI}^{u,\ell} = 0$ (the default), then the x-axis spans
the entire $\Delta\lambda$ range. \np
When ${\bf PROGLI}^{u,\ell} > 0$, then the x-axis extends from the 
core out to {\bf PROGLI}. \np
When ${\bf PROGLI}^{u,\ell} < 0$, then the x-axis extends from the core out
to that point where successive intensity values differ relatively
by less than $|{\bf PROGLI}|$. \break (If there are several such points,
the point with the largest $\Delta\lambda$ value is used.) \np
If transition $(u,\ell)$ is a blended line then, \np
when ${\bf SGRAF}^{u,\ell} = 1$, separate graphs are provided for each
component; \np
when ${\bf SGRAF}^{u,\ell} = 0$, only a single, composite graph is provided.
\ej
% \blankline
% \blankline
\centerline{\underbar{$\qquad \qquad$ {\bf *99} $\qquad \qquad$}}
\space \noindent
{\bf ISMSW} provides additional control over the format of Iterative
Summaries. \np
When ${\bf ISMSW} = 0$, then the format is controlled by option SUMGRAF. \np
When ${\bf ISMSW} > 0$, then the summaries will be provided in both formats.
\blankline
\blankline
\centerline{\underbar{$\qquad \qquad$ {\bf *100} $\qquad \qquad$}}
\space \noindent
Regardless of the setting of option RATEPRNT, a `minimal' RATES
calculation printout is provided when the input value of {\bf IRATE}
is a valid depth index. In that case, values of QU, QS, GM, RK, RL
and CK for all levels, and of CIJ and PIJ for all transitions,
will be printed, for that depth only.
\blankline
\blankline
\centerline{\underbar{$\qquad \qquad$ {\bf *101} $\qquad \qquad$}}
\space \noindent
{\bf ICDIT} specifies the classes of continuum wavelengths for which
dI/dh analyses will be printed when option DIDHC is on. \np
{\bf ICDIT}$ = 1$ means: for all `additional' wavelengths; \np
{\bf ICDIT}$ = 2$ means: for the line center wavelengths of those
transitions for which a Line Source Function calculation printout
was provided; \np
{\bf ICDIT}$ = 3$ means: 1 and 2.
\blankline
\blankline
\centerline{\underbar{$\qquad \qquad$ {\bf *102} $\qquad \qquad$}}
\space \noindent
The photorecombination rates {\bf RL} and photoionization rates
{\bf RK} can be computed either by detailed integration of the
continuum radiation field, or from specified runs of radiation
temperatures (input tables of {\bf TRN}$_i$). These alternatives
are controlled by option USETRIN. \np
When USETRIN is on, then the computed values of ${\bf RK}^\ell_i,
1 \leq i \leq {\bf N}, 1 \leq \ell \leq {\bf NSL}$, can be manipulated
by means of the factor $HJ_i$, the photoionization rates multiplier.
Values of $HJ_i = 1$ by default; however, if appropriate input values
of {\bf JH1} and {\bf JH2} are provided, then various specific runs of
$HJ_i$ are constructed such that $HJ = 0.5$ near the surface and
$HL = 1.0$ at depth, with an intermediate transition region defined
by depth indices {\bf JH1} and {\bf JH2}. This is also explained in
a note following the ATMOSPHERE printout. \np
When full continuum integrations are done (option USETRIN off),
PANDORA also computes ``effective radiation temperatures''---\ie 
those values of radiation temperature which yield the same values of
${\bf RK}^\ell_i$ as the full integrations did. These computed
${\bf TRN}^\ell_i$ values are included in output file {\tt FOR020}
(see Section 8), and can be used as input for a subsequent 
restart run with option USETRIN on.
\ej
% \blankline
% \blankline
\centerline{\underbar{$\qquad \qquad$ {\bf *103} $\qquad \qquad$}}
\space \noindent
PANDORA can be requested to adjust the values 
of {\bf Z} so that certain constraints are satisfied: 1) that each value
of the computed mass table be equal to the corresponding value of 
an input mass table (for this case option HSE must be on),
or 2) that each value of a computed TAUK table
(continuum optical depth at a specific wavelength) be equal to the
corresponding value of an input TAUK table. \np
What happens in such runs is that a particular relationship between
{\bf TE} and mass, or between {\bf TE} and TAUK, is intended, and
PANDORA tries to make sure ---by adjusting the {\bf Z} values---that its
computed mass or continuum optical depth values come as close as
possible to the specified ones. Like so much else in PANDORA, these
`Z-recalculation' procedures require several iterations to give 
good results. \np
(In principle, atmospheric parameters can be specified not only on
a grid of Z-values, but also on grids of other quantities, such as
mass or optical depth. We did not think to allow for such flexibility 
during the early years of PANDORA development, and internally PANDORA
is solidly based on a grid of geometric depth values. Thus---to
allow the user the convenience of working with fixed sets of
mass or optical depth values---the program must establish the
corresponding geometrical depth values in order to function at
all.) \np
When such `Z-recalculation' is requested, this will be done whenever
appropriate as the calculations proceed. \np
The final recomputed {\bf Z} table is included in output file {\tt fort.20}
(see Section 8), and can then be used to replace the previous {\bf Z}
table of that atmospheric model. \np
\spice
1) To specify an input mass table, use the input parameters {\bf ZMASS}
and \break {\bf RFMAS}; these define the `input' mass table:
$MASS_i = {\bf ZMASS}_i + {\bf RFMAS}$. This Z-recalculation procedure
will be done whenever the input values of \break {\bf ZMASS}$_i$ do not
all $= 0$. It is possible that overcorrections
computed during early iterations of this process can cause things
to go awry, and a weighting step has been included to keep things
under control. Thus, in the current iteration, the final new {\bf Z}
values = {\bf WZM} $\x$ (newly-computed Z) + (1 - {\bf WZM}) $\x$
(final Z from previous iteration).  \np
\spice
2) The other Z-recalculation procedure will be done whenever the
input values of {\bf TAUKIN}$_i$ do not all $= 0$; a value of {\bf PZERO}
(surface pressure) may need to be specified; the value of
{\bf REFLM} (the specific wavelength) can also be specified.
\ej
% \blankline
% \blankline
\centerline{\underbar{$\qquad \qquad$ {\bf *104} $\qquad \qquad$}}
\space \noindent
The meanings of {\bf LMA}, {\bf LMB}, {\bf LME}, {\bf LMF}, {\bf LMR},
{\bf LMT}, {\bf WEP}, and of options ENL and ENL2, are 
explained in detail in the text included in the \break
LEVEL 1 TO K (\ie ``Lyman'') printout.
\blankline
\blankline
\centerline{\underbar{$\qquad \qquad$ {\bf *105} $\qquad \qquad$}}
\space \noindent
When transition $(u,\ell)$ is designated as a blend of component lines,
then ${\bf LDL}^{u,\ell}$ input values of the displacements ${\bf DDL}^{u,\ell}$
and of the relative line strengths \break
${\bf CDL}^{u,\ell}$ must be provided for
that line. The units of {\bf DDL} are Angstroms; however, if it is more
convenient to specify displacements in units of wavenumbers (/cm), 
then the alternate input parameter ${\bf DWN}^{u,\ell}$ may be used. 
The sum of the values of ${\bf CDL}^{u,\ell}$ should be 1. 
Furthermore, ${\bf LDL}^{u,\ell}$ input values
of ${\bf CRD}^{u,\ell}$, of ${\bf CVW}^{u,\ell}$, and of ${\bf CSK}^{u,\ell}$
must also be provided (zeroes will be used otherwise). \np
For blended line transition $(u,\ell)$, the line profile function
is calculated as
%
$$ PHI_i = \sum_{L=1}^{LDL} {\bf CDL}^L PHI^L_i $$
%
where each set $PHI^L$ is displaced (in wavelength or in wavenumber)
as specified.
\blankline
\blankline
\centerline{\underbar{$\qquad \qquad$ {\bf *106} $\qquad \qquad$}}
\space \noindent
Details of a Hydrogen Stark broadening convolved profile calculation
for the \break ({\bf MS},{\bf NS}) transition are printed for the
{\bf IHSDP}'th depth point and for the \break {\bf IHSDD}'th frequency
value, when option ANALYSIS is on.
\blankline
\blankline
\centerline{\underbar{$\qquad \qquad$ {\bf *107} $\qquad \qquad$}}
\space \noindent
Input parameter {\bf IHSSW} overrides all ${\bf CSTARK}^{u,\ell}$.
\blankline
\blankline
\centerline{\underbar{$\qquad \qquad$ {\bf *108} $\qquad \qquad$}}
\space \noindent
When option AMDIFF is on and option VELS is off, then PANDORA will
automatically set ${\bf NVX} = 1$. See Section 16, Velocities.
\ej
% \blankline
% \blankline
\centerline{\underbar{$\qquad \qquad$ {\bf *109} $\qquad \qquad$}}
\space \noindent
When ${\bf NCB} > 0$ ({\it i.e.} when CO-lines wavelength bands are 
specified): \np
1) if ${\bf NCL} = 0$, then no CO-related wavelengths will be added to 
the list of wavelengths for which continuum calculations are done; \np
2) if ${\bf NCL} = 1$, then ${\bf XCOL}_1 = 0$ is assumed, and only
CO-line core wavelengths and the specified band limits will be added to
the list of wavelengths for which continuum calculations are done.
\blankline
\blankline
\centerline{\underbar{$\qquad \qquad$ {\bf *110} $\qquad \qquad$}}
\space \noindent
If ${\bf WAVEMN} > 0$ and ${\bf WAVEMX} > {\bf WAVEMN}$, then PANDORA attempts
to use built-in procedures to add more additional wavelengths to the table
${\bf WAVES}_i, 1 \leq i \leq {\bf NWV}$, as follows. \np 
If {${\bf NWV} > 0$} and there is an index I, {${\rm I} < {\bf NWV}$}, such that
{${\bf WAVES}_i = 0$} \break for all ${\rm I} < i \leq {\bf NWV}$, then: \np
1) all standard rates integration wavelengths $\lambda^c$ falling in the range
\break ${\bf WAVEMN} \leq \lambda^c \leq {\bf WAVEMX}$, will replace such zeroes
(working systematically upwards from $i = {\rm I}+1$). \np
If thereafter it is still the case that there is an
index I, ${\rm I} < {\bf NWV}$, such that \break
${\bf WAVES}_i = 0$ for all ${\rm I} < i \leq {\bf NWV}$, then: \np
2) those remaining zeroes will be replaced by values $\lambda^a$ such that \break
${\bf WAVEMN} \leq \lambda^a \leq {\bf WAVEMX}$; these $\lambda^a$ will be
equi-spaced in the log. \np
\blankline
\blankline
\centerline{\underbar{$\qquad \qquad$ {\bf *111} $\qquad \qquad$}}
\space \noindent
The input switch {\bf IXNCS} (default =1) allows reversion to the ``old'' method of
calculating default values of {\bf CE} and {\bf CI} in Hydrogen runs
(see Section 19). \np
``Old'' method (${\bf IXNCS} = 0$): if no input values of {\bf CE} or {\bf CI}
(as tabulated functions of {\bf TER}) are given, then default tables are computed
(using the method specified) and listed in the ATOM printout. These calculations
do not take the ``lowering of the ionization potential'' into account. Whenever
particular values of ${\bf CE}^{u,\ell}_i$ or ${\bf CI}^j_i$ as functions of
temperature are needed during the calculations, then they will be obtained from
these tables by interpolation. \np
``New'' method (${\bf IXNCS} = 1$): if no input values of {\bf CE} or {\bf CI}
(as tabulated functions of {\bf TER}) are given, then particular values of
${\bf CE}^{u,\ell}_i$ or ${\bf CI}^j_i$ as functions of temperature and charged
particle density will be computed as needed during the calculations; values
pertaining to {\bf TER} and a reference value of charged particle density (note
input parameter {\bf IRFNC}) appear in the ATOM printout as sample values only.
(This only makes sense if the method of Johnson, or of Vriens and Smeets, is
specified; or if no set of {\bf TER} values yields acceptable interpolated
results.)
\ej
% \blankline
% \blankline
\centerline{\underbar{$\qquad \qquad$ {\bf *112} $\qquad \qquad$}}
\space \noindent
The statement {\bf COLINES ( {\it q} )} 
({\it e.g.} ``{\tt COLINES ( KROT 15 RC1213 17. ) }'') \break is used for
parameters controlling the CO-lines calculations. {\it q} consists of
pairs of input fields of the form ``{\it A v}'', where {\it A} is an
alphanumeric input field whose value may be ``{\tt JFUND}'', ``{\tt KFUND}'',
``{\tt JOVER}'', ``{\tt KOVER}'',``{\tt JSECN}'', ``{\tt KSECN}'',
``{\tt JROT}'', ``{\tt KROT}'', ``{\tt ISOSLCT}'', ``{\tt METHCOF}'',
``{\tt METHCOW}'', or ``{\tt RC1213}'', and {\it v} is a numeric field,
which must be {\flpt} if it follows {\tt RC1213}, and {\intg} otherwise.
Here: \np
JFUND $\ldots$ KROT specify maximum values of the quantum numbers j and
$\nu$, as follows: \np
$\quad$ JFUND is the maximum value of j for fundamental lines (default = 111); \np
$\quad$ KFUND is the maximum value of $\nu$ for fundamental lines (default = 20); \np
$\quad$ JOVER is the maximum value of j for first overtone lines (default = 111); \np
$\quad$ KOVER is the maximum value of $\nu$ for first overtone lines (default = 13); \np
$\quad$ JSECN is the maximum value of j for second overtone lines (default = 111); \np
$\quad$ KSECN is the maximum value of $\nu$ for second overtone lines (default = 12); \np
$\quad$ JROT is the maximum value of j for rotational lines (default = 53); \np
$\quad$ KROT is the maximum value of $\nu$ for rotational lines (default = 21); \np
ISOSLCT selects the isotope; = 1 means: use $^{12}$CO,
= 2 means: use $^{13}$CO, \break = 3 means: use both (default = 3); \np
METHCOF selects the method for computing f-values; \break = 1 means: use
Chackerian and Tipping, = 2 means: use new Chackerian data, \break = 3 means:
use Goorvitch, 1994 (default = 3); \np
METHCOW selects the method for computing energies (wavelengths); \break
= 1 means: use Farrenq \ea, = 2 means: use Coxon and Hajigeorgiou,
\break = 3 means: use Goorvitch, 1994 (default = 3); \np
RC1213 is the isotopic abundance ratio ${^{12}{\rm CO}}/{^{13}{\rm CO}}$
(default = 90). \np
{\it Notes:} \np
The input values of the maximum quantum numbers JFUND $\ldots$ KROT
cannot be set larger than their respective default values. \np
${\rm JFUND} = 0$ and/or ${\rm KFUND} = 0$ means: do not include any
fundamental lines. \np
${\rm JOVER} = 0$ and/or ${\rm KOVER} = 0$ means: do not include any
first overtone lines. \np
${\rm JSECN} = 0$ and/or ${\rm KSECN} = 0$ means: do not include any
second overtone lines. \np
${\rm JROT} = 0$ and/or ${\rm KROT} = 0$ means: do not include any
rotational lines. \np
\ej
% \blankline
% \blankline
\centerline{\underbar{$\qquad \qquad$ {\bf *113} $\qquad \qquad$}}
\space \noindent
{\bf IPEX} controls debug printout intended to be useful for program
development only. ${\bf IPEX} = -1$ will generate a flood of printout;
${\bf IPEX} = k$, where $k$ is one of several integers $> 0$, will generate
selected printouts only.
\blankline
\blankline
\centerline{\underbar{$\qquad \qquad$ {\bf *114} $\qquad \qquad$}}
\space \noindent
${\bf LOGAS} > 0$ turns on the ``location analysis graph'' of the emergent
line profile calculation, and selects the type of abscissa: = 1 for
$\Delta\lambda$-index or wavenumber-index, =2 for $\Delta\lambda$-value or
wavenumber-value. (Option WAVENUMB controls whether wavelength or wavenumber 
is used.)
\blankline
\blankline
\centerline{\underbar{$\qquad \qquad$ {\bf *115} $\qquad \qquad$}}
\space \noindent
Debug output from the calculation of CI (in Hydrogen runs using {\tt AR}, 
{\tt VORONOV}, {\tt VS}, or {\tt JOHNSON})
can be obtained by setting ${\bf JDMCI} = [(1000 \times i) + j], \;
1 \leq i \leq {\bf N}, \break 1 \leq j \leq {\bf NL}$, where $i$ is a depth index and
$j$ is a level index. When $i = 0$, then output is provided for level $j$ at all
depths; when $j = 0$, then output is provided at depth $i$ for all levels. \np
Debug output from the calculation of CE (in Hydrogen runs using {\tt VS}
or {\tt JOHNSON}) can be obtained by setting ${\bf JDMCE} = [1000 \times 
(100 \times u + \ell) + i]$, \break $1 \leq i \leq {\bf N}$, where $i$ is a depth
index and $(u,\ell)$ are transition indices, $u > \ell$. When $i = 0$, then output
is provided for transition $(u,\ell)$ at all depths; when the transition indices
are $(0,0)$, then output is provided at depth $i$ for all transitions.
\blankline
\blankline
\centerline{\underbar{$\qquad \qquad$ {\bf *116} $\qquad \qquad$}}
\space \noindent
When ${\bf NGNV} > 0$, then ${GNV}^\ell, \; {\bf NGNV} \leq \ell \leq {\bf NL}$ will
be suppressed (i.e. set = 0) in the diffusion calculations.
\ej
% \blankline
% \blankline
\centerline{\underbar{$\qquad \qquad$ {\bf *117} $\qquad \qquad$}}
\space \noindent
CQM$_i$ occurs in the equation of the scattering albedo of the `Line \break
Background' opacities (see Section 9). \np
If input parameter ${\bf CQM} > 0$, then PANDORA uses CQM$_i$ = {\bf CQM},
$1 \leq i \leq {\bf N}$. \np
If input parameter ${\bf CQM} \leq 0$, then PANDORA expects to find the tables \break
{\bf CQT}$_k$, {\bf CQA}$_k$, $1 \leq k \leq {\bf NCQ}$, which specify
CQM as a tabulated function of temperature. PANDORA interpolates in $\log[{\bf CQA}
({\bf CQT})]$ to obtain \break CQM$_i$ = {\bf CQA}({\bf TE}$_i), 1 \leq i \leq {\bf N}$.
\blankline
\blankline
\centerline{\underbar{$\qquad \qquad$ {\bf *118} $\qquad \qquad$}}
\space \noindent
${\bf KRATE}^{u,\ell}$ controls the formulation of certain terms in the statistical
equilibrium equations: \np
${\bf KRATE}^{u,\ell} = 1$ means: use net-rate (computed from
$\rho$); \np
${\bf KRATE}^{u,\ell} = 2$ means: use single-rate (computed from \Jbar).
\blankline
\blankline
\centerline{\underbar{$\qquad \qquad$ {\bf *119} $\qquad \qquad$}}
\space \noindent
First derivatives of various quantities need to be computed for the diffusion calculations
(option AMDIFF). The input parameter {\bf KDIFD1} selects one of several methods: \np
{\bf KDIFD1} = 1 means: compute the slope at the given point as the slope of the straight
line through the two bracketing points. \np
{\bf KDIFD1} = 2 means: first apply ``sequential smoothing'' to the table of values; then
compute the slope at the given point as the average of two slopes: the slope of the straight
line through the given point and the adjacent point on the left, and the slope of the
straight line through the given point and the adjacent point on the right. \np
{\bf KDIFD1} = 3 means: first apply ``improved sequential smoothing with irregular point
spacing'' to the table of values; then compute the slope at the given point from the
cubic spline fitted to all the points. \np
{\bf KDIFD1} = 4 means: compute the slope at the given point as the average of two slopes:
the slope of the straight line through the given point and the adjacent point on the left,
and the slope of the straight line through the given point and the adjacent point on the
right (i.e., like {\bf KDIFD1} = 2 but without smoothing). \np
The default value is ${\bf KDIFD1} = 1$.
\ej
% \blankline
% \blankline
\centerline{\underbar{$\qquad \qquad$ {\bf *120} $\qquad \qquad$}}
\space \noindent
If any nonzero values of {\bf RABDL}$_i$ were input, then {\bf RABD}$_i$ 
will be set equal to antilog ({\bf RABDL}$_i$), all $i$.
Otherwise, if any nonzero values of {\bf RABD}$_i$ were input,
then those input values  will be used.
Otherwise, {\bf RABD}$_i$ will be set equal to 1.0, all $i$.
\blankline
\blankline
\centerline{\underbar{$\qquad \qquad$ {\bf *121} $\qquad \qquad$}}
\space \noindent
The detailed dump printout for the calculation of S-from-Number-Densities is controlled
by {\bf ISNDD}. When ${\bf ISNDD} = 0$, then no dump ever appears; when ${\bf ISNDD} = 1$,
then a dump appears for every such calculation; when ${\bf ISNDD} = 2$, then a dump
appears only if an error occurred during the calculation.
\blankline
\blankline
\centerline{\underbar{$\qquad \qquad$ {\bf *122} $\qquad \qquad$}}
\space \noindent
The `Special N1' calculation (part of the ``Diffusion Calculation'' when
option AMDN1 is on) can be done in various ways; the choice of method is controlled by
input parameters {\bf N1MET}, {\bf KDIAG}, {\bf I4DFM}, {\bf I4DEQ}, {\bf I4DIO},
{\bf KBNDS} and {\bf KDAMP}. (A more detailed explanation of these parameters appears
in sections INPUT and DIFFUSION of the main output file.)

When ${\bf N1MET} = 1$, use the exponential method (if possible, otherwise, a diagonal
method); when ${\bf N1MET} = 1$, use the diagonal method specified by {\bf KDIAG};
when ${\bf N1MET} = 3$, use the simultaneous method (for He-I or He-II only).

{\bf KDIAG} = 3, 4, or 5 selects the 3-diagonal, 4-diagonal, or 5-diagonal \break
method, respectively. Additional controls are needed for the 4-diagonal method:
${\bf I4DIO} = 1$ specifies the inward version, and ${\bf I4DIO} = 2$ the outward
version (in the staionary case); {\bf I4DEQ} = 0, 1, or 2 specifies the ``original,''
``method-1,'' or ``method-2'' equations, respectively; when ${\bf I4DEQ} = 1$, use
the results from the Z-formulation, when ${\bf I4DEQ} = 2$, use the results from
the $\zeta$-formulation.

${\bf KDAMP} = 0$ means: use the raw matrix solution, whereas ${\bf KDAMP} = 1$ means:
use the damped matrix solution (obtained by a weighted moving 3-point average), for
the three-diagonal, five-diagonal, or simultaneous method.

{\bf KBNDS} = 1 means: use equations incorporating relevant boundary conditions, = 0
means: do not.
\ej
% \blankline
% \blankline
\centerline{\underbar{$\qquad \qquad$ {\bf *123} $\qquad \qquad$}}
\space \noindent
If not input values of the ``Lyman'' calculation tables ${\bf XK}_i$,
${\bf GK}_i, 1 \leq i \leq {\bf KK}$ are specified, then defaults are provided from the
rates integration data tables \break $RRNU^{\bf KOLEV}_i$, ${\bf RRCP}^{\bf KOLEV}_i,
1 \leq i \leq {\bf MR}^{\bf KOLEV}$. If ${\bf MR}^{\bf KOLEV} > {\bf KK}$, then only
the first {\bf KK} values of $RRNU$ and {\bf RRCP} are copied into {\bf XK} and
{\bf GK}, respectively; if ${\bf KK} > {\bf MR}^{\bf KOLEV}$, then $RRNU$ and
{\bf RRCP} are copied into the first {\bf KK} elements of {\bf XK} and {\bf GK},
respectively, while the rest of those tables will be left undisturbed.
\blankline
\blankline
\centerline{\underbar{$\qquad \qquad$ {\bf *124} $\qquad \qquad$}}
\space \noindent
The ``Diffusion Calculations'' (options AMDIFF and VELGRAD) 
compute tables of $GNVL^\ell_i, \, 1 \leq i \leq {\bf N}, \,
1 \leq \ell \leq {\bf NL}$. At times the raw computed values of these tables should not all
be used; instead, they should be set $= 0$ at some depths. A table of $GRF_i$ values (`$GNVL$
reduction factor') has been introduced for this purpose; $GRF_i$ multiplies the raw values
to produce the final values of $GNVL^\ell_i$. The values, and the use, of $GRF_i$,
are controlled by input parameters {\bf KDIFGS}, {\bf KDIFGA}, and {\bf KDIFGB}. \np
The $GRF_i$ table is established as follows: $GRF_i = 1, \, 1 \leq i \leq {\bf KDIFGA}$; \break
$GRF_i = ({\bf KDIFGB} - i)/({\bf KDIFGB} - {\bf KDIFGA}), \, {\bf KDIFGA} < i < {\bf KDIFGB}$;
$GRF_i = 0, \, {\bf KDIFGB} \leq i \leq {\bf N}$. \np
When {\bf KDIFGS} = 0, then $GRF_i$ is not used (and the raw values become the final values);
when {\bf KDIFGS} = 2, then all values of $GNVL_i^\ell, \, 1 \leq \ell \leq {\bf NL}$
are multiplied by $GRF_i$;
when {\bf KDIFGS} = 1, then only values of $GNVL_i^1$ are multiplied by $GRF_i$.
\blankline
\blankline
\centerline{\underbar{$\qquad \qquad$ {\bf *125} $\qquad \qquad$}}
\space \noindent
{\bf IPDEE} and {\bf NEFDF} are related to the calculation of the d-coefficients that are
used in the diffusion calculation (the relavant routines are minor adaptations of code
written and provided by Juan Fontenla). The d-coefficient-calculation needs values of
the electron density; when ${\bf NEFDF} = 1$, then internally-computed values of electron
density are used, when ${\bf NEFDF} = 2$, then the program-wide {\bf NE}-table is used. \np
A schematic graph of the computed d-coefficients is always printed; the complete set of
values will be printed, in addition, when ${\bf IPDEE} = 1$.
\ej
% \blankline
% \blankline
\centerline{\underbar{$\qquad \qquad$ {\bf *126} $\qquad \qquad$}}
\space \noindent
The table $beta_i$ (the He II number density) used in the diffusion calculations can be
computed in different ways, controlled by {\bf IBETSW}. When ${\bf IBETSW} = 0$, then
$beta_i = { 1 \over 2} ( HEK_i + HE21_i )$; when ${\bf IBETSW} = 1$, then
$beta_i = HEK_i$; when ${\bf IBETSW} = 2$, then $beta_i = HE21_i$.
\blankline
\blankline
\centerline{\underbar{$\qquad \qquad$ {\bf *127} $\qquad \qquad$}}
\space \noindent
The array $DIJ$ is one of the aids that are calculated, printed, and plotted for analyzing
the effects of diffusion. When ${\bf IPDIJ} = 0$, then the entire array will be printed;
when ${\bf IPDIJ} = 1$, then only an abbreviated analysis of $DIJ$ will be printed.
\blankline
\blankline
\centerline{\underbar{$\qquad \qquad$ {\bf *128} $\qquad \qquad$}}
\space \noindent
{\bf KB1WA} and {\bf KB1WB} are depth indices that control the generation of B1-weights
in the same way that {\bf KDIFGA} and {\bf KDIFGB} control the generation of values of
$GRF_i$ (see Note 121).
\blankline
\blankline
\centerline{\underbar{$\qquad \qquad$ {\bf *129} $\qquad \qquad$}}
\space \noindent
The input switch {\bf KANTNU} controls the summary printout of all {TNU} tables of
a radiative transition. (The default is {\bf KANTNU} = 0.) \np
{\bf KANTNU} = 0 means: none; \np
{\bf KANTNU} = 1 means: for all transitions with {\bf LSFPRINT} = 1; \np
{\bf KANTNU} = 2 means: for all transitions when option LSFPRNT is on; \np
{\bf KANTNU} = 3 means: for all transitions when option LSFGRAF is on; \np
{\bf KANTNU} = 4 means: for all transitions.
\blankline
\blankline
\centerline{\underbar{$\qquad \qquad$ {\bf *130} $\qquad \qquad$}}
\space \noindent
The ORIGINS printout, and/or the CONTRIBUTORS printout, part of the spectrum
analysis aids, can be restricted to a specified range of wavelengths (or of
wavenumbers when option WAVENUMB is on). {\bf CORMIN}, when $> 0$, is the
minimum value, and {\bf CORMAX}, when $> 0$, is the maximum value of wavelength
(or of wavenumber, as the case may be).
\ej
% \blankline
% \blankline
\centerline{\underbar{$\qquad \qquad$ {\bf *131} $\qquad \qquad$}}
\space \noindent
$X^{s+1}_i = W_i \times X^{new}_i + {(1 - W_i)} \times X^s_i, \; 
1 \leq i \leq N$, where $X$ stands for RHO$^{u,\ell}$ and $W$ syands
for RHWT$^{u,\ell}$; or \np
$X^{s+1}_i = [ (X^{new}_i)^{W_i} ] \times [ (X^s_i)^{(1 - W_i)} ], \; 1 \leq i 
\leq N$, where $X$ stands for RK$^{KOLEV}$ and W stands for RKWT, as
used in the Lyman calculation. \np
The values of $X_i$ for iteration $s+1$ will be obtained, in the manner
shown, from the values of $X_i$ used during iteration $s$ and the $new$
values of $X_i$ calculated at the end of iteration $s$.

The first equation above illustrates {\it linear} weighting, while the
second illustrates {\it logarithmic} weighting. Logarithmic weighting
is always used for RK$^{KOLEV}$; logarithmic weighting can also be used
for RHO$^{u,\ell}$, depending on option \break WATESTR.

The values of {\bf RHWT}$_i$ and {\bf RKWT}$_i$
may vary with depth index $i$, or may be constants.
They are updated in every iteration as described in the writeup
\break [74 Oct 23]. The procedure for updating these weights 
uses input parameters {\bf INCH}, {\bf WRMN}, and {\bf WRMX}.

\blankline
{\it Note: }
RHO and RK weighting used to be done differently, using different input
parameters and input tables (see Note 13).
The former input quantities NTW, TAW, WTW,
NWRHO, WRLY, and WRHO are no longer recognized by the program and must be
removed from old files. The former input quantities WMN, WMX, RHOWT, and
RKW are no longer used, however, the program continues to recognize them
and uses their values to compute related values of {\bf WRMX}, {\bf WRMN},
{\bf RHWT}, and {\bf RKWT}, according to: \np
${\bf WRMX} = 1 - WMN$, $\qquad {\bf WRMN} = 1 - WMX$, \np
${\bf RHWT} = 1 - RHOWT$, $\qquad$ and $\qquad {\bf RKWT} = 1- RKW$. \np
Messages are written to the output file whenever these conversions are done.
It is best to remove {\bf WMN} and {\bf WMX} from the input (using the
above conversions). {\bf RHOWT} and {\bf RKW} tend to occur in the
restart files written by old versions of the program; the program now
writes {\bf RHWT} and {\bf RKWT} to these files. Thus, the old quantities
will normally be encountered only when an old run is restarted with the
current program, so that the automatic conversion described here needs to
be done only once.
\blankline
\blankline
\centerline{\underbar{$\qquad \qquad$ {\bf *132} $\qquad \qquad$}}
\space \noindent
The switch {\bf WORLDLY} can take on the values {\tt SCREEN} or {\tt DISK};
it controls the storage-management-system dump (which uses file
{\tt fort.97} when {\bf WORLDLY} \break = {\tt DISK}). I use this to check
the source code; it is not intended for general use.
\ej
% \blankline
% \blankline
\centerline{\underbar{$\qquad \qquad$ {\bf *133} $\qquad \qquad$}}
\space \noindent
In the ATMOSPHERE and ATOM printout sections, parameter values = 0 are nomally
printed as blanks. If the user prefers to see printed 0's, then the input
parameters {\bf JZATMO} and/or {\bf JZATOM} may be set = 1.
\blankline
\blankline
\centerline{\underbar{$\qquad \qquad$ {\bf *134} $\qquad \qquad$}}
\space \noindent
As explained in Section 11, output from the Continuum Calculations at various
wavelengths is controlled by various OPTIONS pertaining to the various
wavelength types (for example, output will be produced for all Additional
Continuum Wavelengths, {\bf WAVES}, when option ADDCOPR is on). However,
even if no wavelength-type OPTIONS are on, output will be produced for
every wavelength specified in the table {\bf SCOW}, of length {\bf NSW}.
For this purpose, a value of {\bf SCOW} must match the value of an actual
Continuum Calculation wavelength to at least 8 figures (note that a list
of all Continuum Calculation wavelengths appears in the WAVELENGTHS section
near the end of the PANDORA printout).
\blankline
\blankline
\centerline{\underbar{$\qquad \qquad$ {\bf *135} $\qquad \qquad$}}
\space \noindent
When option USEWTAB is on, the ``standard rates integrations wavelengths''
table is added to the list of wavelengths for which continuum calculations
are done. (This ``standard'' table contains values bracketing all the
absorption edges contained in the built-in ion models and tries to
capture, for integration purposes, all the lines included among the 
background opacity contributors.) Only that portion of the ``standard''
table falling between the limits {\bf WRATMN} and
\break {\bf WRATMX} will be used.
\ej
% \blankline
% \blankline
\centerline{\underbar{$\qquad \qquad$ {\bf *136} $\qquad \qquad$}}
\space \noindent
Ionization and excitation rates always include the effects of collisions
with electrons; the rates due to collisions with neutral Hydrogen are
always computed (and printed if the options RATEPRNT and COLHPRNT are
both on), but are used only if the input switch ${\bf ICHSW} = 1$.
Rates due to collisions with Hydrogen, however, are computed only for
those levels of the ion-of-the-run whose corresponding value of the
input array ${\bf LCH}^j > 0, \; 1 \leq j \leq {\bf NSL}$.
${\bf LCH}^1 = 0$ by definition; but a negative value of ${\bf LCH}^1$
can be specified and will be used as a special code, as described below.

Ionization rates due to collisions with Hydrogen are computed using the
formulation of B. Kaulakys, 1985, J.Phys.B, {\bf 18}, L167.

Kaulakys' formulation is also used to compute the effects of collisions
with Hydrogen on the collisonial excitation rates for all those transitions
$(u,\ell)$ for which both ${\bf LCH}^u$ and ${\bf LCH}^\ell > 0$. A
second set of collisional ionization rates is then
computed from H. W. Drawin, 1969, Z.Physik, {\bf 225}, 483
for all transitions to and from those lower levels whose
index $\leq |{\bf LCH}^1|$. (For an $A = 0$ transition the Drawin rate
is zero, and the Drawin rates for transitions whose lower level
index $= 1$ are zero if ${\bf LCH}^1 = 0$. Such Drawin rates then replace the 
corresponding Kaulakys transitional rates that had already been computed.
\blankline
\blankline
\centerline{\underbar{$\qquad \qquad$ {\bf *137} $\qquad \qquad$}}
\space \noindent
The Hydrogen Lyman lines background opacity parameters are explained in
Section 21, and in section INPUT of PANDORA's regular output file.
\blankline
\blankline
\centerline{\underbar{$\qquad \qquad$ {\bf *138} $\qquad \qquad$}}
\space \noindent
If no input values of {\bf DGM} are specified, and if {\bf NGM} = 0
has been specified explicitly, then {\bf DGM}$_i = 1, \; 1 \leq i \leq {\bf N}$
is set up internally (this is the "general" default, as specified above).
However, if no input vales of {\bf DGM} are specified, {\it and} if
{\bf NGM} = 0 has {\it not} been specified (so that {\bf NGM} retains its
built-in default value which is $> 0$), then ``solar default'' values of
{\bf DGM}$_i$ will be computed from {\bf DGMZ}$_j, \; 1 \leq j \leq {\bf NGM}$.
Here {\bf DGMZ} is a table appropriate for the quiet Sun, and is
specified as a function of Z, {\bf ZGM}. Values of {\bf DGM}$_i$
corresponding to the input values of {\bf Z}$_i$ are obtained by
interpolation from the tables {\bf ZGM}$_j$ and {\bf DGMZ}$_j$. \np
{\it Note}: {\bf DGM}$_i = 0$ is not acceptable.
\ej
% \blankline
% \blankline
\centerline{\underbar{$\qquad \qquad$ {\bf *139} $\qquad \qquad$}}
\space \noindent
The {\bf CIMETHOD} and {\bf CEMETHOD} statements each may contain an arbitrary
number of nonnumeric codewords for specifying the set of methods for
calculating automatic values of $CI^j$ and/or $CE^{u,\ell}$.
The {\bf CIMETHOD} statement recognizes: {\tt CLARK, AR, VORONOV, JOHNSON,
VS, SHAH, ONTHEFLY}; while the {\bf CEMETHOD} statement recognizes: {\tt SEATON,
VREGE, SCHOLZ, PB, VS, JOHNSON, AGGRWL, ONTHEFLY}. To enable a method, list
its name in the pertinent statement; to turn off a method already enabled
(say, because it is enabled by default) preface it with a minus sign (without
intervening blank).

With these statements the user can specify method
sets different from the default sets. It is necessary to turn off explicitly
any unwanted methods that are already enabled! For example, to use only
{\tt VS} for all Hydrogen $CI^j$-values, use: 
``{\tt CIMETHOD ( VS -SHAH -CLARK ) }''. 

{\tt ONTHEFLY} is a calculation mode, not a method.

See Section 19 for more details.
\blankline
\blankline
\centerline{\underbar{$\qquad \qquad$ {\bf *140} $\qquad \qquad$}}
\space \noindent
The {\bf MATRIX} statement accepts control parameters for matrix manipulation.
Their names are {\bf DRPSW} (integer, default = 0), {\bf EDJSW} (integer,
default = 0), and {\bf CRITJ} (floating point, default = $10^{-50}$). When
one or more of these names appear in a {\bf MATRIX} statement, each name
should be followed immediately by its desired input value. For example:

$\qquad$ {\tt MATRIX ( edjsw 1  critj 1.e-30 )  }
\blankline
\noindent When {\bf DRPSW} = 1: two lines of descriptive data will be written
to the printout file for every matrix to be inverted.
\blankline
\noindent When {\bf EDJSW} = 1 ``junk'' is edited out of every matrix prior
to inversion or determinant calculation: find $Y$, the matrix element having
the largest absolute value; compute $Z = {\bf CRITJ} \times |Y|$; and then
set = 0 every matrix element whose absolute value is less than $Z$.
%\par
%\vfill \vfill \vfill \vfill
%\vfill \vfill \vfill \vfill
%\vfill \vfill \vfill \vfill
%\vfill \vfill \vfill \vfill
%\vfill \vfill \vfill \vfill
%\end
