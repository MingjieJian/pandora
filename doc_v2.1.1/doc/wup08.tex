%\magnification=1200
%\input wupstuff.tex
\newtoks\footline \footline={\hss\tenrm 8.\folio\hss}
\pageno=1
\top
\vskip 1.5 true in
\centerline{Section 8: {\bf Output Files}}
\blankline
\blankline
\centerline{\bf ***}
\blankline
\blankline
Besides the printout files (described in Sections 7 and 11),
PANDORA provides output in several
additional files. Some of these contain data that can be used to
restart the run (and may also contain data that is neither needed
nor permissible for restarting -- such files must be edited appropriately).
Other files provide collections of data and results that are useful
for various special purposes. See also Section 7.

Each of the four basic restart data files begins with the set of four
``Run ID'' lines (in the form of PANDORA comments), containing the run
{\bf HEADING}, program version number, two time stamps (one expressed in
Smithsonian Days), and a summary of basic run data. Copies of the 
{\bf HEADING} line also appear at various places in several files.
\blankline
\blankline
\underbar{ {\bf File {\tt fort.19} contains} } general restart data,
as follows: 
\blankline
\bull {\it Always} --- \par \noindent
Run ID
\spice
\bull {\it Always} --- \par \noindent
for all radiative transitions $u,\ell$:
{\bf METSE}$^{u,\ell}$ ({\it if} METSW {\it is on}), {\bf RHO}$^{u,\ell}_i$,
\break {\bf RHWT}$^{u,\ell}_i$, {\bf JBAR}$^{u,\ell}_i$, {\bf CHI}$^{u,\ell}_i$,
{\bf AW}$^{u,\ell}_i$
\spice
\bull {\it If option} LYMAN {\it is on} --- \par \noindent
for level $j =$ {\bf KOLEV}:
{\bf IRKCOMP}$^j$ = 0, {\bf RK}$^j_i$, {\bf RKWT}$_i$, {\bf IRLCOMP}$^j$ = 0, 
{\bf RL}$^j_i$, {\bf EP1}$_i$, {\bf EP2}$_i$
\spice
\bull {\it Always} --- \par \noindent
{\bf NK}$_i$, 
{\bf ND}$^j_i$ for all levels $j$; {\bf BD}$^j_i$ for all levels $j$
\spice
\bull {\it Always} --- \par \noindent
``{\bf USE ( INPUT ) }''
\ej
\underbar{ {\bf File {\tt fort.20} contains} } miscellaneous restart
and other data, as follows:
\blankline
\bull {\it Always} --- \par \noindent
Run ID
\spice
\bull {\it Always} --- \par \noindent
{\bf N}, {\bf Z}$_i$, {\bf TE}$_i$
\spice
\bull {\it If } {\bf POPUP} {\it is on but} $\neq$ {\tt HYDROGEN} --- 
\par \noindent
{\bf NE}$_i$, {\bf ZME}$_i$, {\bf NC}$_i$
\spice
\bull {\it If option } HSE {\it is on} --- \par \noindent
{\bf MASS}$_i$, {\bf PGS}$_i$, {\bf PTO}$_i$, {\bf TE}$_i$,
{\bf VT}$_i$, {\bf VM}$_i$, {\bf GD}$_i$, {\bf T5000}$_i$
\spice
\bull {\it If option } ATOMSAV {\it is on and default values of atomic
parameters were computed} --- \np
atomic model parameter values
\spice
\bull {\it If options } DUSTEMP {\it and} DUSTYPE
{\it are both on} --- \par \noindent
{\bf TDST}$_i$
\spice
\bull {\it If } {\bf MH2N} = 1 --- \par \noindent
{\bf H2N}$_i$
\spice
\bull {\it If } {\bf MCON} = 1 --- \par \noindent
{\bf CON}$_i$
\spice
\bull {\it If} {\bf POPUP} = {\tt HYDROGEN} {\it and} {\bf MFONT} = 1 --- \np
atmosphere model data tables for Juan Fontenla
\spice
\bull {\it If} {\bf NAME} = {\tt HELIUM2} --- \par \noindent
{\bf N}, {\bf Z}$_i$, {\bf HE304}$_i$
\spice
\bull {\it If this is a K-shell run} --- \par \noindent
{\bf N}, {\bf Z}$_i$, {\bf QIN}$_i$
\spice
\bull {\it If } {\bf IRPUN} = 1 --- \par \noindent
two sets of data for a separate program (CENSUS) which calculates abundance
ratios of ions of the same element \par
\spice
\vbox{\parindent=0pt \hangindent=10pt \hangafter=0
(these data include {\bf Z}, and FLVSL and FIONL, as they
appear in the \break `POPULATIONS' printout; if AMDIFF is on
and this is a Helium run, then a second set of these data
will appear)}
\spice
\bull {\it If option } JLYSAV {\it is on} --- \par \noindent
Level-$\cal N$-to-Continuum (Lyman) J-bar values
\spice
\bull {\it If option {\rm USETRIN} is off and} {\bf MTREF} = 1 --- \np
Effective Radiation Temperature values
\ej
% \spice
\bull {\it If } {\bf MDFV} = 1 {\it and} AMDIFF
{\it is on} --- \par \noindent
{\bf N}, {\bf Z}$_i$, {\bf TE}$_i$, {\bf VM}$_i$,
{\bf VAMB}$_i$, {\bf VBMB}$_i$, {\bf VCMB}$_i$, {\bf VDMB}$_i$, 
{\bf VH}$_i$, {\bf VP}$_i$, {\bf VE}$_i$, {\bf V1}$_i$, 
{\bf V2}$_i$, {\bf V3}$_i$
\spice
\bull {\it If } {\bf MDFG} = 1 {\it and} AMDIFF
{\it is on} --- \par \noindent
{\bf GVL}$_i^j, \; 1 \leq j \leq {\bf NL}$
\spice
\bull {\it If {\bf RHEAB} was recomputed} --- \np
{\bf N}, {\bf Z}$_i$, {\bf RHEAB}$_i$
\spice
\bull {\it If} {\bf NAME} = {\tt HELIUM} {\it or} {\tt HELIUM2}
{\it and} AMDIFF {\it is on} --- \np
{\bf N}, {\bf Z}$_i$, and either: {\bf PALBET}$_i$,
{\bf PBETAL}$_i$ if {\tt HELIUM}; or {\bf PGMBET}$_i$,
\break {\bf PBETGM}$_i$ if {\tt HELIUM2}
\spice
\bull {\it If option } CHEXUP {\it is on and this is one of the 10
``charge exchange'' ions} --- \par \noindent
{\bf N}, {\bf Z}$_i$, {\bf XRKH}$^j_i$, {\bf XRLH}$^j_i,
\; 4 \leq j \leq $ NPQMX
\spice
\bull {\it If } {\bf MKURU} = 1 --- \par \noindent
then data for Kurucz's spectrum programs \par
\spice
\vbox{\parindent=0pt \hangindent=10pt \hangafter = 0
(these data comprise the computed departure coefficients of the
ion-of-the-run plus, if this is a Hydrogen run, some atmosphere
data tables)}
\spice
\bull {\it If option } CALCOOL {\it is on} --- \par \noindent
data for separate program to calculate new {\bf TE} values \par
\spice
\vbox{\parindent=0pt \hangindent=10pt \hangafter=0
(these data include {\bf NE}, {\bf NH}, {\bf TE}, SUM [which
is called `Total Rate' in the `NET RADIATIVE COOLING RATES' printout],
and, in a Hydrogen run, SUMSM [\ie smoothed values of SUM])}
\spice
\bull {\it If option } FLUXSAV {\it is on} --- \par \noindent
data for separate programs to calculate new {\bf TE} values \par
\spice
\vbox{\parindent=0pt \hangindent=10pt \hangafter=0
(these data include {\bf Z} and {\bf TE}, and `Integrated Flux', 
`Integrated Derivatives' and `Effective Temperature', as they appear
in the `CONTINUUM FLUX INTEGRALS' printout)}
\spice
\bull {\it If option } ECLISAV {\it is on} --- \par \noindent
results of eclipse emergent profile calculations \par
\spice
\vbox{\parindent=0pt \hangindent=10pt \hangafter=0
(these data include {\bf R1N}, {\bf Z} and {\bf FRR}, and `DL', `ID'
and `IS', as they appear in the `ECLIPSE INTENSITY AND FLUX' printout)}
\spice
\bull {\it If option } CLNORM {\it is on and this run calculates H
Lyman $\alpha$ and/or $\beta$} --- \par \noindent
{\bf FNRMLA} and {\bf FNRMLB} \par
\spice
\vbox{\parindent=0pt \hangindent=10pt \hangafter=0
(used by PANDORA to adjust the simulated H Lyman $\alpha$
and $\beta$ lines in the background)}
\ej
% \blankline
% \blankline
\underbar{ {\bf File {\tt fort.21} contains} } populations restart data
as follows:
\blankline
\bull {\it Always} --- \par \noindent
Run ID
\spice
\bull {\it If option } HSE {\it is on, and } {\bf POPUP}
{\it is set, and } {\bf NAME} = {\tt HYDROGEN} --- \par \noindent
{\bf NE}$_i$, {\bf ZME}$_i$, {\bf NC}$_i$
\spice
\bull {\it If option } HMS {\it is on} --- \par \noindent
{\bf BDHM}$_i$
\spice
\bull {\it If option } AMDIFF {\it or option } VELGRAD
{\it is on, and {\rm VBMB} $\neq 0$} --- \np
{\bf VBMB}$_i$
\spice
\bull {\it If option } HSE {\it is on, or if option }
HSE {\it is off and option } CPSW {\it is on} --- \par \noindent
{\bf NH}$_i$
\spice
\bull {\it If } {\bf POPUP} = {\tt HYDROGEN} --- \par \noindent
for all levels $j$ such that $1 \leq j \leq 15$ :
{\bf NP}$_i$, {\bf HN}$^j_i$, {\bf BDH}$^j_i$, \par \noindent
plus all those `population ions' populations tables (in the order shown
just below) for which there was at least some input
\spice
\bull {\it If } {\bf POPUP} = {\tt CARBON} --- \par \noindent
for all levels $j$ such that $1 \leq j \leq 8$ :
{\bf CK}$_i$, {\bf CN}$^j_i$, {\bf BDC}$^j_i$
\spice
\bull {\it If } {\bf POPUP} = {\tt SILICON} --- \par \noindent
for all levels $j$ such that $1 \leq j \leq 8$ :
{\bf SIK}$_i$, {\bf SIN}$^j_i$, {\bf BDSI}$^j_i$
\spice
\bull {\it If } {\bf POPUP} = {\tt HELIUM} --- \par \noindent
for all levels $j$ such that $1 \leq j \leq 13$ :
{\bf HEK}$_i$, {\bf HEN}$^j_i$, {\bf BDHE}$^j_i$
\spice
\bull {\it If } {\bf POPUP} = {\tt HELIUM2} --- \par \noindent
for all levels $j$ such that $1 \leq j \leq 8$ :
{\bf HE2K}$_i$, {\bf HE2N}$^j_i$, {\bf BDHE2}$^j_i$
\spice
\bull {\it If } {\bf POPUP} = {\tt ALUMINUM} --- \par \noindent
for all levels $j$ such that $1 \leq j \leq 8$ :
{\bf ALK}$_i$, {\bf ALN}$^j_i$, {\bf BDAL}$^j_i$
\spice
\bull {\it If } {\bf POPUP} = {\tt MAGNESIUM} --- \par \noindent
for all levels $j$ such that $1 \leq j \leq 8$ :
{\bf MGK}$_i$, {\bf MGN}$^j_i$, {\bf BDMG}$^j_i$
\spice
\bull {\it If } {\bf POPUP} = {\tt IRON} --- \par \noindent
for all levels $j$ such that $1 \leq j \leq 8$ :
{\bf FEK}$_i$, {\bf FEN}$^j_i$, {\bf BDFE}$^j_i$
\ej
% \spice
\bull {\it If } {\bf POPUP} = {\tt SODIUM} --- \par \noindent
for all levels $j$ such that $1 \leq j \leq 8$ :
{\bf NAK}$_i$, {\bf NAN}$^j_i$, {\bf BDNA}$^j_i$
\spice
\bull {\it If } {\bf POPUP} = {\tt CALCIUM} --- \par \noindent
for all levels $j$ such that $1 \leq j \leq 8$ :
{\bf CAK}$_i$, {\bf CAN}$^j_i$, {\bf BDCA}$^j_i$
\spice
\bull {\it If } {\bf POPUP} = {\tt OXYGEN} --- \par \noindent
for all levels $j$ such that $1 \leq j \leq 14$ :
{\bf OK}$_i$, {\bf ON}$^j_i$, {\bf BDO}$^j_i$
\spice
\bull {\it If } {\bf POPUP} = {\tt OXYGEN2} --- \par \noindent
for all levels $j$ such that $1 \leq j \leq 8$ :
{\bf O2K}$_i$, {\bf O2N}$^j_i$, {\bf BDO2}$^j_i$
\spice
\bull {\it If } {\bf POPUP} = {\tt OXYGEN3} --- \par \noindent
for all levels $j$ such that $1 \leq j \leq 8$ :
{\bf O3K}$_i$, {\bf O3N}$^j_i$, {\bf BDO3}$^j_i$
\spice
\bull {\it If } {\bf POPUP} = {\tt SULPHUR} --- \par \noindent
for all levels $j$ such that $1 \leq j \leq 8$ :
{\bf SK}$_i$, {\bf SN}$^j_i$, {\bf BDS}$^j_i$
\ej
% \blankline
% \blankline
\underbar{ {\bf File {\tt fort.22} contains} } special restart data
as follows:
\blankline
\bull {\it Always} --- \par \noindent
Run ID
\spice
\bull {\bf JNU}$^{u,\ell}_{i,k}$ for all PRD transitions
\blankline
\blankline
\underbar{ {\bf File {\tt fort.23} contains} } 
data from all emergent spectrum calculations, to make them available
to various auxiliary programs; this file is written
only if options SPECSAV, PROSAV and/or CONSAV are on. \par
\spice
\vbox{\parindent=0pt \hangindent=10pt \hangafter=0
(These data include much of the input for the continuous intensity
and flux calculations, and for the line intensity and flux calculations --
mostly labelled in the file. The various quatities in the file can be
identified by comparing them with the detailed normal printout. Note
that, for Continuum Flux, the value printed is $4 \pi H$, while the
value saved is $H$. Both continuous intensity and line intensity are
given for each value of $\mu$, as is obvious. Extra data are saved when
there is incident radiation -- again, this should be clear.
{\it Note}: flow-broadened profiles are indicated by setting the value
of the velocity table index ${\bf NVY} = 0$.
{\it Note} the effect of options WAVENUMB and INCIFRNT)}
\blankline
\blankline
\underbar{ {\bf File {\tt fort.24} contains} } 
results of the cooling and heating rates calculations, for separate
plotting programs; this file is written only when options
CALCOOL and COOLSAV are both on. \par
\spice
\vbox{\parindent=0pt \hangindent=10pt \hangafter=0
(These data include almost all the quantities that appear in the two
COOLING RATES and the two HEATING RATES printouts)}
\blankline
\blankline
\underbar{ {\bf File {\tt fort.25} contains} } 
data for separate programs to produce `continuum data' plots;
this file is written only when the option TOPE is on. \par
\spice
\vbox{\parindent=0pt \hangindent=10pt \hangafter=0
(These data include, for each wavelength, the complete `Continuum
Data \break Blocks' (which contain all the information that appears in
complete `CONTINUUM DATA' printouts), followed by, again for each
wavelength, the computed continuous emergent intensities and intermediate
results from those computations. Since these data are not normally 
accessible, they are provided in this special file.}
\ej
% \blankline
% \blankline
\underbar{ {\bf File {\tt fort.26} contains} } 
contains data for separate programs to
analyze the performance of PANDORA's matrix inversion algorithms;
this file is written only when the switch {\bf SMATC} $ > 0$ and
qualifying matrices were encountered; (this file is only of use to me,
as the program developer)
\blankline
\blankline
\underbar{ {\bf File {\tt fort.27} contains} } 
final values of data from the line source function calculations for
all radiative transitions, as they appear in the printout file; this
file has them in a format intended to be more convenient for 
user-written analysis and plotting programs. This file is written
when the option TRANSAV is on. \par
\spice
\vbox{\parindent=0pt \hangindent=10pt \hangafter=0
(These data include {\bf Z} and {\bf TE}, and then `TAU', `S', `RHO',
`JBAR' and ST, the total source function as plotted)}
\blankline
\blankline
\underbar{ {\bf File {\tt fort.30} contains} } 
`iterative studies' data for a separate program to
analyze the performance of PANDORA's iterative
procedures, and provides various graphical displays of the data. \par
\spice
\vbox{\parindent=0pt \hangindent=10pt \hangafter=0
(These data include {\bf Z} and {\bf TE}, and `EPSILON', `S',
`TAU' and `JBAR', as they appear in the `LINE SOURCE FUNCTION' printout,
for transition \break ({\bf MS}/{\bf NS}) {\it only}, for every iteration)}
\blankline
\blankline
\underbar{ {\bf File {\tt fort.32} contains} } 
debug checksums, for separate program {\tt RELAX} that compares checksum
data files from different runs. \par
\spice
\vbox{\parindent=0pt \hangindent=10pt \hangafter=0
(This is not intended for routine use; it is a debugging aid for
ongoing program development)}
\blankline
%\blankline
%\vfill \vfill
%\vfill \vfill
%\vfill \vfill
%\vfill \vfill
%\vfill \vfill
%\vfill \vfill
%\vfill \vfill
%\vfill 
\vfill
\noindent (Section 8 -- last revised: 2007 Feb 05) \par
\message{Section 8 ends at page 8.\the\pageno}
\ej
%\end
