%\magnification=1200
%\input wupstuff.tex
\newtoks\footline \footline={\hss\tenrm 9.\folio\hss}
\pageno=1
\top
\vskip 1.5 true in
\centerline{Section 9: {\bf Background Line Opacities}}
\blankline
\blankline
\centerline{\bf ***}
\blankline
\blankline
PANDORA's calculation of line source functions requires knowledge of
the background opacity; this opacity is referred to casually as the
``background continuum.'' Opacity from other lines may contribute
to this background continuum for any given line. When such other
lines are lines of the ion-of-the-run, then PANDORA's ``blended
line'' mechanism is available to handle it. Some background lines
due to H, He-II, and CO can be included explicitly (see the table printed
by PANDORA as part of the ATMOSPHERE printout). In addition, general
background line opacity can be included in three other ways:
Statistical Line opacity, Composite Line opacity, and/or Averaged
Line opacity; these are all added into the total background continuum.
\blankline
This section describes the {\it control}, the {\it input}
and the {\it data} requirements of these background absorbers.
Note that the necessary input files are also mentioned in Section 7.
\blankline
These three Background Line opacities are mutually exclusive: only one
of them can be used at any given wavelength. Averaged Line opacity has
highest priority: if it is used, then neither of the other two is used.
Composite Line opacity has the next highest priority: if it is used, then
Statistical Line opacity is not used; thus the latter is used only if
neither of the others is used.
\ej
% \blankline
% \blankline
\centerline{\bf A) Scattering albedo}
\blankline
Each of these Background Line
opacities is apportioned between an absorption part and a
scattering part by means of an albedo table that is a function of depth.
This table is obtained by interpolation from the input tables
{\bf ZALBK}$_\ell$ and {\bf ALBK}$_\ell, 1 \leq \ell \leq$ {\bf NKA}.
\blankline
When ${\bf NKA} = 0$, then the scattering albedo $= 1 / (1+Q_i)$, where
$Q_i$ is a function of depth and wavelength
%
$$ Q_i = X_i \left( 1 + {\bf CLM} { 10^4 \over LM } \right) + PNH
         \left( {{\bf NH}_i \over {\bf NH}_N} \right) ,
$$
%
and $X_i$ is either
%
$$ X_i = {{\bf XNE}_i \over 1.07 \x 10^{14} }
\left( {LM \over 5000} \right)^3 { 1 \over {\bf CQM} }, 
\qquad {\bf CQM} > 0 ,
$$
%
or
%
$$ X_i = {{\bf XNE}_i \over 1.07 \x 10^{14} }
\left( {LM \over 5000} \right)^3 { 1 \over {\bf CQA}({\bf TE}_i) }, 
\qquad {\bf CQM} \leq 0 .
$$
%
Here $LM$ is wavelength in Angstroms, and {\bf CQM} and {\bf CQA}
(a tabulated function of ${\bf CQT}_k, \; 1 \leq k \leq {\bf NCQ}$)
are input parameters (see Section 5, Note *117).

The default values of this tabulated parameter are:
\spice
\settabs 6 \columns \bf
\+ & & CQT & CQA \cr \rm
\spice
\+ & & 4000 & 0.0001 \cr
\+ & & 5000 & 0.001 \cr
\+ & & 6000 & 0.01 \cr
\+ & & 7000 & 0.1 \cr
\+ & & 8000 & 1.0 \cr
\ej
% \blankline
% \blankline
\centerline{\bf B) Statistical Line opacities}
\blankline
PANDORA can use opacity distribution functions computed by Kurucz, as
described in VAL II. Ten component opacity distributions, identified by
$k = 1$ through $k = 10$, are given for 11 $\x$ 9 grids of
temperature and electron density, at each of 53 wavelength grid points
in the range 1269.80 -- 8352.95 \AA.
\spice
Temperature grid: log(TE) =
\spice
\settabs 12 \columns
\+ & 3.60 & 3.64 & 3.68 & 3.72 & 3.76 & 3.80 & 3.84 & 3.88 & 3.92 &
3.96 & 4.00 \cr
\spice
Density grid: log(NE) =
\spice
\+ & 9.0 & 10.0 & 11.0 & 12.0 & 13.0 & 14.0 & 15.0 & 16.0 & 17.0 \cr
\spice
The necessary tabulated numerical data from Kurucz are in a special data
file, ``{\tt statistical},'' which must be connected
to logical unit 10 (see Section 7). 
Using the actual input tables of {\bf TE}$_i$ and {\bf NE}$_i$, 
1 $\leq i \leq$ {\bf N}, at the start of
a run PANDORA constructs an array of 53 $\x$ {\bf N} values of 
Statistical opacity, using the specified component distribution function.
For ultimate control over the computed values of this array, a table
of {\it ad hoc} multipliers, {\bf FKUR}$_\ell$, 1 $\leq \ell \leq$ 53,
\break ({\it i.e.} one for each of the given wavelength values), can be
specified in the input; normally all values of {\bf FKUR} = 1.

The input parameter {\bf KURIN} is used to specify which component
distribution should be used: $k = {\bf KURIN}$. (When the value of
${\bf KURIN} = 0$, then no Statistical opacities are computed, and the
data file need not be provided.) A subset of the available wavelength
range can be selected by specifying a lower limit, {\bf KURMI}
(printed as LLMIN), and an upper limit, {\bf KURMA} (printed as LLMAX).

When option KURPRNT is on then a printout and a graph appear. Values of
Statistical opacity are printed for all {\bf N} depths, at wavelength
points \# 1 and every {\bf LWNT}'th wavelength point thereafter. Values
of Statistical opacity are plotted for all 53 wavelengths, at five
depths whose indices are: \break {\bf KINMAX}$ \; + \; ${\bf KININT},
{\bf KINMAX}, {\bf KINMAX}$ \; - \; 1 \times${\bf KININT}, \break
{\bf KINMAX}$ \; - \; 2 \times${\bf KININT},
and {\bf KINMAX}$ \; - \; 3 \times${\bf KININT}.

A dump of raw data (as read from {\tt statistical}), at wavelength points
\# 1 and every {\bf KUDNT}'th wavelength point thereafter, appears if
${\bf KUDNT} > 0$.
\ej
% \blankline
% \blankline
\centerline{\bf C) Composite Line opacities}
\blankline
Kurucz has calculated the composite opacity due to all known atomic and
molecular lines ({\it circa} mid-1989) as a function of temperature,
pressure, velocity and wavelength, assuming local thermodynamic equilibrium.
His opacities are given for 56 values of temperature, 
21 values of pressure, 5 values of velocity, and
35093 wavelength grid points in the range 8.97666 -- 9998.56236 nm.
\spice
Temperature grid: log(TE) =
\spice
\+ & 3.32 & 3.34 & 3.36 & 3.38 & 3.40 & 3.42 & 3.44 & 3.46 & 3.48 & 3.50 \cr
\+ & 3.52 & 3.54 & 3.56 & 3.58 & 3.60 & 3.62 & 3.64 & 3.66 & 3.68 & 3.70 \cr
\+ & 3.73 & 3.76 & 3.79 & 3.82 & 3.85 & 3.88 & 3.91 & 3.94 & 3.97 & 4.00 \cr
\+ & 4.05 & 4.10 & 4.15 & 4.20 & 4.25 & 4.30 & 4.35 & 4.40 & 4.45 & 4.50 \cr
\+ & 4.55 & 4.60 & 4.65 & 4.70 & 4.75 & 4.80 & 4.85 & 4.90 & 4.95 & 5.00 \cr
\+ & 5.05 & 5.10 & 5.15 & 5.20 & 5.25 & 5.30 \cr
\spice
Pressure grid: log(P) =
\spice
\+ & -2.0 & -1.5 & -1.0 & -0.5 & +0.0 & +0.5 & +1.0 & +1.5 & +2.0 & +2.5 \cr
\+ & +3.0 & +3.5 & +4.0 & +4.5 & +5.0 & +5.5 & +6.0 & +6.5 & +7.0 & +7.5 \cr
\+ & +8.0 \cr
\spice
Velocity grid: V (km/s) =
\spice
\+ & 0.0 & 1.0 & 2.0 & 4.0 & 8.0 \cr
\spice
These $35,093 \x (56 \x 21 \x 5) = 35,093 \x 5,880 = 206,346,840$
opacity \break values have been written to six binary files, as follows:
\spice
\settabs 5 \columns \bf
\+ & File name & \# points & Range (nm) \cr \rm
\spice
\+ & BSAMHE & 4659 & 8.97666 -- 22.78209 \cr
\+ & BSAMLY & 6936 & 22.78377 -- 91.16094 \cr
\+ & BSAMBA & 6934 & 91.17535 -- 364.62911 \cr
\+ & BSAMVI & 6163 & 364.70183 -- 1249.82029 \cr
\+ & BSAMIR1 & 6934 & 1250.0 -- 4999.28118 \cr
\+ & BSAMIR2 & 3467 & 5000.0 -- 9998.56236 \cr
\spice
PANDORA works with a reduced set of $1,750 \x 5,880 = 10,290,000$ opacity
values at 1,750 wavelength grid points covering almost the same range,
at the same temperatures, pressures and velocities. 
This reduction is done by {\tt PROCURE}, a separate program; its output
file, ``{\tt composite},'' must be connected to logical unit 11 
(see Section 7). ({\tt CONCUR}, a second separate
program, selects a specified, unreduced subset of Kurucz's data;
its output file has the same format as that produced by {\tt PROCURE}.)
PANDORA cannot read Kurucz's original files---it only
reads files produced by {\tt PROCURE} (or {\tt CONCUR}).
\ej
\hrule height 1pt \vskip 1pt \hrule
\blankline
\centerline{Reduction Procedure}
\blankline
1) The original set of 35,093 wavelengths is cut into 350 blocks of 100
consecutive ``original'' wavelengths each (the last 93 original wavelengths
are ignored). \np
\spice
2) Such a block pertains to ``original'' wavelengths $W_j, 1 \leq j \leq 100$,
and contains $100 \x 5,880$ opacity values. \np
\spice
3) ``Nominal'' wavelength values, $N_i, 1 \leq i \leq 5$, are computed such that
\break $N_i = W_1 + ({2 \x i} - 1) \x (W_{100} - W_1) / 10$. \np
\spice
4) In each original block the opacity values pertaining to a given
(TE,P,V)-triple are treated as a separate table of opacity {\it vs.} $W_j$. \np
\spice
5) Such a table of 100 opacity values is sorted into increasing order. The
$10^{th}$, $30^{th}$, $50^{th}$, $70^{th}$ and $90^{th}$ entries of this
ordered opacity table, together with their associated (original) values
of $W_j$, are pulled out. These opacity values constitute the ``reduced''
opacity table. \np
\spice
6) The 5 selected opacity values are then ordered so that their associated
\break (original) wavelengths $W_j$ are in increasing order. \np
\spice
7) The  5 ``nominal'' wavelengths $N_i$ are then assigned to these 5
(ordered) opacity values of the ``reduced'' table. \np
\spice
8) This is done for all 5,880 (TE,P,V)-triples, so that 5,880 ``reduced''
tables of 5 entries each are obtained. \np
\spice
9) In this manner, each original block of data pertaining to 100
``original'' wavelengths is replaced by a reduced block of data pertaining
to 5 ``nominal'' wavelengths, for all values of temperature, pressure and
velocity. \np
\blankline
\hrule \vskip 1pt \hrule height 1pt
\blankline
\blankline
\blankline
A typical PANDORA run need not use all 1750 wavelengths of the reduced data
set, since it uses only those that fall within the wavelength bands specified
in the input. There are {\bf NAB} such band(s), and Composite Line opacity
will be computed at all those wavelengths W (in \AA), arising in the context of
a given PANDORA run, such that {\bf BANDL}$_\ell \leq$ W $\leq$ 
{\bf BANDU}$_\ell$, 1 $\leq \ell \leq$ {\bf NAB}. (When {\bf NAB} = 0,
then no Composite Line opacities will be computed, and the data file need
not be provided.) Using the input tables of {\bf TE}$_i$ and {\bf V}$_i$,
and a computed table of P$_i$, 1 $\leq i \leq$ {\bf N}, PANDORA constructs an 
array of KOMP $\x$ {\bf N} values of Composite opacity at the start of
a run; here KOMP is the count of all the wavelength points of
the input `reduced' data set that are contained within the specified
band(s).

PANDORA automatically computes `continuum data' for these KOMP \break
wavelength points. Values of the `method control parameters'
for the relevant Continuum Source Function calculations can be specified,
for each band, using the input values of {\bf BANDY}. Printout from these
continuum calculations is controlled by option COMPCOPR.

When option KOMPRNT is on then a printout and a graph appear. Values of
Composite opacity are printed for all {\bf N} depths, at wavelength
points \# 1 and every {\bf LWNT}'th wavelength point thereafter. Values
of Composite opacity are plotted for all KOMP wavelengths, at five
depths selected by {\bf KINMAX} and {\bf KININT} as with Statistical
opacity, above.

A dump of raw data (as read from {\tt composite}), at wavelength points
\# 1 and every {\bf KODNT}'th wavelength point thereafter, appears if
${\bf KODNT} > 0$.
\blankline
\blankline
{\it Note}: Kurucz's original data files (BSAMHE, etc) no
longer exist; thus the reduction and extraction procedures described here are
of historical interest only. We have used our ``standard'' reduced file
{\tt composite} for many years; it is the only data input file now available.
\ej
\blankline
\blankline
\centerline{\bf D) Averaged Line opacities}
\blankline
PANDORA can use arbitrary opacity distributions specified as functions of
arbitrary grids of depth (in km, like the {\bf Z} table) and of wavelength
(in \AA). These opacity sets are interpolated to the values of depth
and wavelength that are needed for the calculation. If option AVELOP
is on, then PANDORA expects a special data file, connected to logical
unit 12 (see Section 7). If there is data in this
file (\ie if a wavelength grid is specified),
then Averaged Line opacity will be used at all wavelengths
within the range of the wavelength grid.

This special data file is a binary file that PANDORA attempts to read
as in the following quasi-Fortran code fragment:
\blankline
\settabs 12 \columns {\tt
\+ & & - - - - \cr
\+ & & integer KWA, NT, NP, NZ \cr
\+ & & real*8  WAVES(KWA), ZGRID(NZ), AVOP(NZ) \cr
\+ & & - - - - \cr
\+ & & read (12) KWA, NT, NP, NZ \cr
\+ & & - - - - \cr
\+ & & if (KWA .gt. 0) then \cr
\+ & & & read (12) (WAVES(I), I = 1,KWA) \cr
\+ & & & read (12) (ZGRID(I), I = 1,NZ) \cr
\+ & & & - - - - \cr
\+ & & & do 999 J = 1,KWA \cr
\+ & & & & read (12) (AVOP(I), I = 1,NZ) \cr
\+ & & & & - - - - \cr
\+ & 999 & & continue \cr
\+ & & end if \cr
\+ & & - - - - \cr
}
\blankline
\noindent (The parameters {\tt NT} and {\tt NP} provide for possible
future program additions; currently they should both be set $= 0$.)

% \ej
As each set of {\tt NZ} opacity values pertaining to a given value of the
wavelength grid is read, the data are interpolated to the {\bf Z} table
of the run; this process produces a final data array of size {\tt KWA}
$ \times $ {\bf N}.

When option AVOPRNT is on then a printout and a graph appear. Values of
Averaged opacity are printed for all {\bf N} depths, at wavelength
points \# 1 and every {\bf LWNT}'th wavelength point thereafter. Values
of Averaged opacity are plotted for all KWA wavelengths, at five
depths selected by {\bf KINMAX} and {\bf KININT} as with Statistical
opacity, above.
%\blankline
%\blankline
%\vfill \vfill
%\vfill \vfill
%\vfill \vfill
%\vfill \vfill
%\vfill \vfill
%\vfill \vfill
%\vfill \vfill
%\vfill 
\vfill
\noindent (Section 9 -- last revised: 2006 Nov 02) \par
\message{Section 9 ends at page 9.\the\pageno}
\ej
%\end
