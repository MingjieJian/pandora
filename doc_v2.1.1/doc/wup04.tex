%\magnification=1200
%\input wupstuff.tex
\newtoks\footline \footline={\hss\tenrm 4.\folio\hss}
\pageno=1
\top
\vskip 1.5 true in
\centerline{Section 4: {\bf Remarks on the Input Process}}
\blankline
\blankline
\centerline{\bf ***}
\blankline
\blankline
PANDORA's input phase must establish all the input data for the subsequent
computations. The major activities of the input phase comprise: initialization,
setting defaults, reading all input statements, expanding shorthand input
notations ({\it e.g.} for {\bf KPC} or {\bf TR}), establishing defaults that
cannot be pre-set ({\it e.g.} for {\bf NP} or {\bf ABD}), extra(inter)polating
to {\bf Z} if necessary (see Section 13), and printing out most of the input
values.
\blankline
There are two kinds of defaults: pre-read defaults, which can be set before
any input statements are read ({\it e.g.} {\bf M} or {\bf TS} or {\bf TBAR}),
and post-read defaults, which can only be established by referring to other
input, and thus cannot be set until after all the input statements (of a
given Part, {\it i.e.} B, D, F or H) have been read.
\blankline
The processing in the input phase proceeds as follows:
\blankline
\bull Read, print and write the {\bf HEADING}.
\space
\bull Set pre-read defaults for Part B, as specified in Section 5 of this
writeup.
\space
\bull Read Part B of the input.
\space
\bull Read Part C of the input. After C (= {\bf GO}) has been read, only those
of the input parameters in Part B that were explicitly mentioned in input
statements will have had new values given them. (These explicit values need
not necessarily be different from the defaults.) If any of these parameters
were mentioned more than once, they will have the values given them by the
respective input statements that were read last.
\space
\bull Digest the Part B input.
\ej
\bull Set up storage for all the parameters in Parts D and F of the input,
and set all this storage to zero ({\it i.e.} zero is the basic default).
\space
\bull Set pre-read defaults for Parts D and H, as specified in Section 5.
Pre-read defaults for tables are potential defaults
only; they will actually be set according to the current
values of their respective lengths.
(This makes it easy to shorten or lengthen tables {\it vis-a-vis}
their default states. For example, consider the table {\bf MUF} of length
{\bf LF}. The pre-read defaults are {\bf LF} = 2 and
{\bf MUF} = 1.0, 0.3. If {\bf LF} = 1 resulted from Part B of the input,
then the pre-read default is {\bf MUF} = 1.0. If {\bf LF} = 4 resulted from
part B of the input, then the pre-read default is {\bf MUF} = 1.0, 0.3, 0.0, 
0.0. To replace the two zeroes in this {\bf MUF}-table, a statement in
Part F of the form: ``{\bf muf} ( {\bf I} 3 0.2 0.1 ) '' would be
sufficient.
\space
\bull Read Part D of the input.
\space
\bull Read Part E of the input. After E (= {\bf go}) has been read, only those
input parameters of Part D that were explicitly mentioned in input
statements will have had new values given them. (These explicit values need
not necessarily be different from the defaults.) If any of these input
parameters were mentioned more than once, they will have the values given
them by the respective input statements that were read last. Provisional
input tables of depth-dependent variables will have been inter(extra)polated
to {\bf Z} (see Section 13); the actual provisional values will have been
discarded.
\space
\bull Digest the Part D input.
\space
\bull Read Part F of the input.
\space
\bull Read Part G of the input. After G (= {\bf go}) has been read, only those
input parameters of Part F that were explicitly mentioned in input
statements will have had new values given them. (These explicit values need
not necessarily be different from the defaults.) If any of these input
parameters were mentioned more than once, they will have the values given
them by the respective input statements that were read last. Provisional
input tables of depth-dependent variables will have been inter(extra)polated
to {\bf Z} (see Section 13); the actual provisional values will have been
discarded.
\space
\bull Digest the Part F input.
\ej
\bull Set up storage, in memory and in the random-access scratch file, for the
population data of Part H of the input.
\space
\bull Set pre-read defaults for Part H, as specified in Section 5.
\space
\bull Read Part H of the input.
\space
\bull Read Part I of the input. After I (= {\bf GO}) has been read, only those
input parameters of Part H that were explicitly mentioned in input statements
will have had new values given them. (These explicit values need not
necessarily be different from the defaults.) If any of these input parameters
were mentioned more than once, they will have the values given them by the
respective input statements that were read last. Provisional input tables of
depth-dependent variables will have been inter(extra)polated to {\bf Z}
(see Section 13); the actual provisional values, and the auxiliary Z tables
as well, will have been discarded.
\space
\bull Print the input (see Section 11 for more details about this).
\blankline
\blankline
\blankline
\noindent {\it Note}: 
default values for unspecified populations and departure coefficients
of Part H will be calculated later as part of the various precalculations 
done before the first iteration. Tables of data for the various population
ions are printed, also later, if requested with the corresponding options.
\blankline
%\blankline
%\vfill \vfill
%\vfill \vfill
%\vfill \vfill
%\vfill \vfill
%\vfill \vfill
%\vfill \vfill
%\vfill \vfill
%\vfill 
\vfill
\noindent (Section 4 -- last revised: 2003 Jul 31) \par
\message{Section 4 ends at page 4.\the\pageno}
\ej
%\end
