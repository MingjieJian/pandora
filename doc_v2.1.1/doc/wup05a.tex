%\magnification=1200
%\input wupstuff.tex
\newtoks\footline \footline={\hss\tenrm 5.\folio\hss}
\pageno=1
\def\pa{$\times 10^{4}$}
\def\pb{$\times 10^{5}$}
\def\pc{$\times 10^{6}$}
\def\pd{$\times 10^{7}$}
\def\pe{$\times 10^{8}$}
\def\pf{$\times 10^{9}$}
\def\pg{$\times 10^{10}$}
\def\ph{$\times 10^{11}$}
\def\pj{$\times 10^{12}$}
\def\pk{$\times 10^{13}$}
\def\pl{$\times 10^{14}$}
\def\pm{$\times 10^{15}$}
\def\pn{$\times 10^{16}$}
\def\po{$\times 10^{17}$}
\def\z{[@Z]}
\def\Na{\space \vbox}
\def\bang{\cr \par \hangindent=10pt \hangafter=0}
\def\Df{\par \hangindent=10pt \hangafter=0 {\it Default: }}
\top
\vskip 1.5 true in
\centerline{Section 5: {\bf Input Parameters}}
\blankline
\blankline
\centerline{\bf ***}
\blankline
\blankline
This section contains the complete list of input parameter specifications,
in alphabetical order by name, followed by a set of explanatory notes,
and finally by an alphabetized listing of keywords and descriptive
phrases for each parameter --- this last list should be used
when a parameter's significance or function are vaguely known and
its name is sought. After the parameter name has been located
in the keywords list, the complete specification can then be
consulted in the first part of this section.

The list of input parameter specifications has two or more lines
of information for each parameter.

Line 1 has the following format:
first the parameter {\bf NAME}, in boldface type; then an optional
reference, {\bf *note}, to one of the notes collected at the end
of the spefications list; then, if the parameter is a table, 
the [length] of the table, in square brackets, (if no length is
given, then the parameter is a single item; if {\z} appears, this
signifies that the parameter must have as many elements as the
associated depth table ({\bf Z} or {\bf ZAUX}) -- {\it see Section 13}); 
and then a final group of codes: 
\bull first the letter designating the Part(s) of the input
file the parameter may appear in ({\it see Section 3}),
\bull then the number specifying the Statement Form to be used with this
parameter ({\it see Section 2}), and
\bull finally the required mode of the parameter.

Line 2 contains a brief description of the function or significance 
of the input parameter. More information can be found in the {\bf *note}
specified (if any).

Beginning on Line 3 there appears an optional
specification of the default value(s) of the parameter. 
\blankline
\centerline{\bf
If no default(s) are specified, then the default(s) equal(s) zero.}
\ej
\parindent=0pt
\settabs 6 \columns
\Na { 
\+{\bf A}&{\bf *31, *93}&&D, 4, \flpt \bang
Einstein A value}
\Na { 
\+{\bf ABD}&{\bf *1}&&D, 1, \flpt \bang
abundance of the ion of the run}
\Na { 
\+{\bf ACE}&&[NSL]&D, 2, \flpt \bang
CE (default) addend}
\Na { 
\+{\bf ACI}&&[NSL]&D, 2, \flpt \bang
CI (default) addend}
\Na { 
\+{\bf ADMAS}&&&D, 1, \flpt \bang
angular diameter (milliarcseconds)
\Df as implied by ADS}
\Na { 
\+{\bf ADS}&&&D, 1, \flpt \bang
star/Sun angular diameter ratio
\Df 1.0}
\Na { 
\+{\bf ADT}&&[NDT]&D, 2, \flpt \bang
Type-2 dust opacity function
\Df (5.1, 4.7, 4.3, 3.8, 3.4, 3.0, 2.65, 2.4, 2.2, 2.1,
2.1, 2.0, 2.1, 2.3, 2.3, 2.2, 1.9, 1.8, 1.3, 0.93, 0.34, 0.3, 0.3, 0.3, 0.3,
0.3, 0.3, 0.3, 0.3)}
\Na { 
\+{\bf AEL}&&\z &D, 2*,2, \flpt \bang
added Helium electrons}
\Na { 
\+{\bf AHM}&&[MHM]&D, 2, \flpt \bang
$H^-$ bound-free absorption coefficient
\Df (0.01989, 0.04974, 0.1302, 0.4052, 0.7407, 1.107, 1.485,
1.862, 2.226, 2.571, 2.887, 3.172, 3.419, 3.625, 3.789, 3.906, 3.977,
4.001, 3.977, 3.907, 3.791, 3.632, 3.432, 3.194, 2.923, 2.624, 2.302,
1.965, 1.619, 1.275, 0.9453, 0.7918, \break 0.6512, 0.5431)}
\Na { 
\+{\bf AL}&&[NL]&D, 2, \flpt \bang
added recombination fraction
\Df AL$_1$ = 1.0, $\quad$ AL$_i$ = 0.0, $\quad i \leq 2 \leq $NL}
\Na { 
\+{\bf ALBDT}&&[NDT]&D, 2, \flpt \bang
Type-2 dust albedo
\Df 0.9}
\Na { 
\+{\bf ALBDUST}&&[LDU]&D, 2, \flpt \bang
Type-1 dust albedo}
\Na { 
\+{\bf ALBK}&&[NKA]&D, 1, \flpt \bang
scattering albedo parameter for Background Line Opacities
(see Section 9)
\Df (1.0, 0.0)}
\Na { 
\+{\bf ALK}&&\z &H, 2*,2, \flpt \bang
singly-ionized Aluminum number density
\Df computed in LTE}
\Na { 
\+{\bf ALN}&&\z &H, 3*,3, \flpt \bang
Aluminum-I level populations
\Df ALN$_{ij}$ computed in LTE, for all levels $j$ 
such that $j >$ NAL}
\Na { 
\+{\bf AOWXP}&&&D, 1, \flpt \bang
alpha-old weight exponent for Special He-II (Special N-1 calculation, Diffusion)}
\Na { 
\+{\bf APARAD}&&&D, 1, \flpt \bang
dielectronic recombination parameter}
\Na { 
\+{\bf APCDP}&&&D, 1, \flpt \bang
dielectronic recombination parameter}
\Na { 
\+{\bf APCI}&&[NAPKNT]&D, 1, \flpt \bang
dielectronic recombination parameter}
\Na { 
\+{\bf APDDIFC}&&&D, 1, \flpt \bang
ambipolar diffusion velocity calculation parameter
\Df 90.7}
\Na { 
\+{\bf APDDTFC}&&&D, 1, \flpt \bang
ambipolar diffusion velocity calculation parameter
\Df 36.6}
\Na { 
\+{\bf APDTEXP}&&&D, 1, \flpt \bang
ambipolar diffusion velocity calculation parameter
\Df 1.76}
\Na { 
\+{\bf APDXICA}&&&D, 1, \flpt \bang
ambipolar diffusion velocity calculation parameter
\Df 1.75}
\Na { 
\+{\bf APDXICB}&&&D, 1, \flpt \bang
ambipolar diffusion velocity calculation parameter
\Df 4.5}
\Na { 
\+{\bf APDXICC}&&&D, 1, \flpt \bang
ambipolar diffusion velocity calculation parameter
\Df 0.02}
\Na { 
\+{\bf APDXICD}&&&D, 1, \flpt \bang
ambipolar diffusion velocity calculation parameter
\Df -3.5 \pb}
\Na { 
\+{\bf APEI}&&[NAPKNT]&D, 1, \flpt \bang
dielectronic recombination parameter}
\Na { 
\+{\bf APETA}&&&D, 1, \flpt \bang
dielectronic recombination parameter}
\Na { 
\+{\bf APWRA}&&&D, 1, \flpt \bang
dielectronic recombination parameter}
\Na { 
\+{\bf APWRB}&&&D, 1, \flpt \bang
dielectronic recombination parameter}
\Na { 
\+{\bf ASMCR}& {\bf *92}&&D, 1, \flpt \bang
sequential smoothing parameter
\Df 0.001}
\Na { 
\+{\bf ATOLAB}&&&D, 2, \alfa \bang
``name'' of ion model data file
\Df ``!NONAME!''}
\Na { 
\+{\bf AW}&&\z &D, 5*,5, \flpt \bang
integrated diagonal of WN-matrix}
\Na { 
\+{\bf BANDE}& {\bf *77}&[NAB]&B, 2, \intg \bang
Composite Line Opacities wavelengths bands continuum eclipse calculation switch}
\Na { 
\+{\bf BANDL}& {\bf *77}&[NAB]&B, 2, \flpt \bang
lower limits of Composite Line Opacity wavelengths bands (see Section 9)}
\Na { 
\+{\bf BANDU}& {\bf *77}&[NAB]&B, 2, \flpt \bang
upper limits of Composite Line Opacity wavelengths bands (see Section 9)}
\Na { 
\+{\bf BANDY}&&[NAB]&D, 2, \flpt \bang
method control parameter for Composite Line Opacity (see Section 9)
\Df all = -1.0}
\Na { 
\+{\bf BD}&&\z &D, 3*,3, \flpt \bang
departure coefficient of the levels of the ion of the run
\Df computed, using input number densities}
\Na { 
\+{\bf BDAL}&&\z &H, 3*,3, \flpt \bang
departure coefficients of the levels of Aluminum-I
\Df computed, using input Aluminum-I level populations}
\Na { 
\+{\bf BDC}&&\z &H, 3*,3, \flpt \bang
departure coefficients of the levels of Carbon-I
\Df computed, using input Carbon-I level populations}
\Na { 
\+{\bf BDCA}&&\z &H, 3*,3, \flpt \bang
departure coefficients of the levels of Calcium-I
\Df computed, using input Calcium-I level populations}
\Na { 
\+{\bf BDFE}&&\z &H, 3*,3, \flpt \bang
departure coefficients of the levels of Iron-I
\Df computed, using input Iron-I level populations}
\Na { 
\+{\bf BDH}&&\z &H, 3*,3, \flpt \bang
departure coefficients of the levels of Hydrogen
\Df computed, using input Hydrogen level populations}
\Na { 
\+{\bf BDHE}&&\z &H, 3*,3, \flpt \bang
departure coefficients of the levels of Helium-I
\Df computed, using input Helium-I level populations}
\Na { 
\+{\bf BDHE2}&&\z &H, 3*,3, \flpt \bang
departure coefficients of the levels of Helium-II
\Df computed, using input Helium-II level populations}
\Na { 
\+{\bf BDHM}&&\z &D, 2*,2, \flpt \bang
departure coefficient of H-minus
\Df BDHM$_i$ = 1.0, all $i$}
\Na { 
\+{\bf BDMG}&&\z &H, 3*,3, \flpt \bang
departure coefficients of the levels of Magnesium-I
\Df computed, using input Magnesium-I level populations}
\Na { 
\+{\bf BDNA}&&\z &H, 3*,3, \flpt \bang
departure coefficients of the levels of Sodium-I
\Df computed, using input Sodium-I level populations}
\Na { 
\+{\bf BDO}&&\z &H, 3*,3, \flpt \bang
departure coefficients of the levels of Oxygen-I
\Df computed, using input Oxygen-I level populations}
\Na { 
\+{\bf BDOPT}& {\bf *43}&&D, 1, \alfa \bang
b-ratios selection parameter
\Df ``BDJ''}
\Na { 
\+{\bf BDO2}&&\z &H, 3*,3, \flpt \bang
departure coefficients of the levels of Oxygen-II
\Df computed, using input Oxygen-II level populations}
\Na { 
\+{\bf BDO3}&&\z &H, 3*,3, \flpt \bang
departure coefficients of the levels of Oxygen-III
\Df computed, using input Oxygen-III level populations}
\Na { 
\+{\bf BDS}&&\z &H, 3*,3 \flpt \bang
departure coefficients of the levels of Sulphur-I
\Df computed, using input Sulphur-I level populations}
\Na { 
\+{\bf BDSI}&&\z &H, 3*,3 \flpt \bang
departure coefficients of the levels of Silicon-I
\Df computed, using input Silicon-I level populations}
\Na { 
\+{\bf BHORIZ}&&\z &D, 2*,2, \flpt \bang
magnetic field strength}
\Na { 
\+{\bf BLCSW}& {\bf *14}&&D, 4, \intg \bang
broadening components switch ({\it a.k.a.} damping components selector)
\Df 31}
\Na { 
\+{\bf BLIMG}& {\bf *34}&&D, 1, \flpt \bang
absorption contributors graph axis limit
\Df -1.301 [ = log(0.05)]}
\Na {
\+{\bf BMWAC}&&&D, 1, \flpt \bang
beam width parameter for continuum eclipse calculation
\Df 0.1}
\Na { 
\+{\bf BXI}& {\bf *66}&[KBX]&D, 2, \flpt \bang
background lines frequency table (half profile)
\Df (0., .03, .06, .1, .15, .22, .3, .4, .6, .9, 1.2, 1.5, 2., 3.,
5., 10., 20., 40., 70., 110., 200., 500., 1000., 2000., 5000.)}
\Na { 
\+{\bf CAK}&&\z &H, 2*,2, \flpt \bang
singly ionized Calcium number density
\Df computed in LTE}
\Na { 
\+{\bf CAN}&&\z &H, 3*,3, \flpt \bang
Calcium-I level populations
\Df CAN$_{ij}$ computed in LTE, for all levels $j$ 
such that $j >$ NCA}
\Na {
\+{\bf CCHX}& {\bf *95}&&D, 1, \flpt \bang
upper-level charge-exchange cross-section multiplier
\Df 1.0}
\Na { 
\+{\bf CDL}& {\bf *77, *105}&[LDL]&D, 4, \flpt \bang
weights for blended line components
\Df 1.0}
\Na {
\+{\bf CDZ}& {\bf *82}&&D, 1, \flpt \bang
fluid velocity parameter
\Df 100.}
\Na { 
\+{\bf CE}&{\bf *25, *93}&[NTE]&D, 5, \flpt \bang
collisional excitation coefficient
\Df computed (see Section 19)}
\Na { 
\+{\bf CEDMN}&&&D, 1, \flpt \bang
impact-parameter CE-value calculation parameter (for integration)
\Df $10^{-6}$}
\Na { 
\+{\bf CEDMX}&&&D, 1, \flpt \bang
impact-parameter CE-value calculation parameter (for integration)
\Df $10^3$}
\Na { 
\+{\bf CEFEQ}&&&D, 1, \flpt \bang
impact-parameter CE-value calculation parameter (for integration)
\Df $10^{-2}$}
\Na {
\+{\bf CEMETHOD}& $\qquad$ {\bf *93, *139}&&D \bang
CE-method selectors
\Df for Hydrogen: ({\tt SCHOLZ}, {\tt GIOVAN}, {\tt JOHNSON});
otherwise: ({\tt SEATON}, {\tt VREGE})}
\Na { 
\+{\bf CEQMX}&&&D, 1, \flpt \bang
H2 number density control parameter
\Df $10^6$}
\Na { 
\+{\bf CGR}&&&D, 1, \flpt \bang
gravity ratio, with respect to Sun
\Df 1.0}
\Na {
\+{\bf CHEFLOW}&&&D, 1, \flpt \bang
Helium flow constant for RHEAB calculation}
\Na { 
\+{\bf CHI}&&\z &D, 5*,5, \flpt \bang
RHO-like line transfer quantity}
\Na { 
\+{\bf CHLIM}& {\bf *19}&&D, 1, \flpt \bang
RHOW parameter
\Df 0.5}
\Na { 
\+{\bf CHOP}&{\bf *19}&&D, 1, \flpt \bang
a RHO selection parameter
\Df 1.0}
\Na { 
\+{\bf CI}&{\bf *93}&[NTE]&D, 3, \flpt \bang
collisional ionization coefficient
\Df computed (see Section 19)}
\Na {
\+{\bf CIJADD}&{ \bf *91}&\z &D, 5*,5, \flpt \bang
term to be added to CIJ}
\Na {
\+{\bf CIMETHOD}& $\qquad$ {\bf *93, *139}&&D \bang
CE-method selectors
\Df for Hydrogen: ({\tt SHAH}, {\tt CLARK});
otherwise: ({\tt AR}, {\tt CLARK})}
\Na { 
\+{\bf CK}&&\z &H, 2*,2, \flpt \bang
singly-ionized Carbon number density
\Df computed in LTE}
\Na {
\+{\bf CKADD}&&\z &D, 3*,3, \flpt \bang
term to be added to CK}
\Na { 
\+{\bf CLEVELS}&&&D, 1, \flpt \bang
diffusion calculation parameter
\Df 2.0}
\Na { 
\+{\bf CLM}&&&D, 1, \flpt \bang
scattering albedo parameter for Background Line Opacities
(see Section 9)
\Df 1.0}
\Na { 
\+{\bf CLOGG}&&&D, 1, \flpt \bang
log(surface gravity)
\Df as implied by CGR}
\Na { 
\+{\bf CLNH}&&&D, 1, \flpt \bang
HSE calculation parameter
\Df 2.0}
\Na { 
\+{\bf CN}&&\z &H, 3*,3, \flpt \bang
Carbon-I level populations
\Df CN$_{ij}$ computed in LTE, for all levels $j$ 
such that $j >$ NLC}
\Na {
\+{\bf CN1S}&&&D, 1, \flpt \bang
rcheck-criterion for Special-N1 (diffusion)
\Df 0.01}
\Na { 
\+{\bf COLINES}& {\bf *112}&&D \bang
CO lines control parameters
\Df see Note *112}
\Na {
\+{\bf COMU}&&&D, 1, \flpt \bang
mu-value for CO lines opacity calculation}
\Na { 
\+{\bf CORMAX}& {\bf *130}&&D, 1, \flpt \bang
limit parameter for ORIGINS and CONTRIBUTORS printouts
\Df -1.0}
\Na { 
\+{\bf CORMIN}& {\bf *130}&&D, 1, \flpt \bang
limit parameter for ORIGINS and CONTRIBUTORS printouts
\Df -1.0}
\Na { 
\+{\bf CP}& {\bf *56, *93}&[NSL]&D, 2, \flpt \bang
photoionization cross-section
\Df computed (see Section 19)}
\Na { 
\+{\bf CPRESS}&&&D, 1, \flpt \bang
specified constant pressure}
\Na { 
\+{\bf CQA}& {\bf *117}&[NCQ]&D, 2, \flpt \bang
`Line Opacity' scattering albedo parameter
\Df $(10^{-4}, 10^{-3}, 10^{-2}, 10^{-1}, 1.0)$}
\Na { 
\+{\bf CQM}&&&D, 1, \flpt \bang
scattering albedo parameter for Background Line Opacities
(see Section 9)}
\Na { 
\+{\bf CQT}& {\bf *117}&[NCQ]&D, 2, \flpt \bang
`Line Opacity' scattering albedo parameter
\Df (4000, 5000, 6000, 7000, 8000)}
\Na { 
\+{\bf CRD}& {\bf *77, *93,* 105}&$\qquad$[LDL]&D, 4, \flpt \bang
radiative broadening halfwidth}
\Na { 
\+{\bf CRS}& {\bf *93}&&D, 4, \flpt \bang
resonance broadening halfwidth}
\Na { 
\+{\bf CSDW}&&&D, 1, \flpt \bang
number of Doppler widths from line center at which the Hydrogen Stark
components strengths are reduced by the factor 1/e
\Df 1.0}
\Na { 
\+{\bf CSFCRIT}&&&D, 2, \flpt \bang
convergence criterion for CSF iteration
\Df $10^{-5}$}
\Na { 
\+{\bf CSK}& {\bf *93, *105}&&D, 4, \flpt \bang
Stark broadening halfwidth}
\Na { 
\+{\bf CSTARK}& {\bf *107}&&D, 4, \flpt \bang
Hydrogen Stark broadening (convolution) switch}
\Na { 
\+{\bf CUTFE}&&&D, 1, \flpt \bang
cut-off criterion for injection function integration (fast electrons)
\Df $10^{-8}$}
\Na { 
\+{\bf CTCO}&&&D, 1, \flpt \bang
NCO calculation temperature enhancement factor}
\Na { 
\+{\bf CTMX}&&&D, 1, \flpt \bang
maximum NCO temperature enhancement
\Df 0.2}
\Na { 
\+{\bf CVSB}& {\bf *82}&&D, 1, \flpt \bang
fluid velocity parameter for VSB}
\Na { 
\+{\bf CVW}& {\bf *93, *105}&&D, 4, \flpt \bang
van der Waals broadening halfwidth}
\Na { 
\+{\bf CVX}& {\bf *82}&[NVX]&F, 2, \flpt \bang
fluid velocity parameters for VX}
\Na {
\+{\bf CVXF}& {\bf *82}&&F, 1, \flpt \bang
fluid velocity parameter for flow broadening velocities
\Df 25.0}
\Na {
\+{\bf CVXM}& {\bf *82}&&F, 1, \flpt \bang
fluid velocity parameter for flow broadening velocities}
\Na { 
\+{\bf CVXS}& {\bf *82}&&D, 1, \flpt \bang
fluid velocity parameter for VXS}
\Na { 
\+{\bf CVZ}& {\bf *82}&&D, 1, \flpt \bang
fluid velocity parameter
\Df smallest Z$_i$ such that NH$_i > 10^{11}$}
\Na { 
\+{\bf CWJ}& {\bf *19}&&D, 1, \flpt \bang
a RHOJ calculation parameter
\Df 0.5}
\Na { 
\+{\bf CWR}& {\bf *19}&&D, 1, \flpt \bang
a RHO selection parameter
\Df 0.1}
\Na { 
\+{\bf DDL}& {\bf *77, *105}&[LDL]&D, 4, \flpt \bang
displacements from reference wavelength of blended line components,
\break in Angstroms (see also DWN)}
\Na { 
\+{\bf DDR}& {\bf *42}&[NDR]&D, 2, \flpt \bang
DR parameter, PRD transitions
\Df (1.0, 0.9, 0.65, 0.4, 0.1, 0.05, 0.0)}
\Na { 
\+{\bf DDT}&&&D, 1, \flpt \bang
Type-2 dust opacity calculation convergence criterion
\Df 0.01}
\Na {
\+{\bf DELLIM}&&&D, 1, \flpt \bang
DEL-criterion for using DIRECT instead of FULL solution
\Df $10^{-6}$}
\Na { 
\+{\bf DELTB}&&&D, 1, \flpt \bang
departure coefficients editing parameter
\Df 0.01}
\Na { 
\+{\bf DELWAVE}& {\bf *76}&[NWS]&D, 2, \flpt \bang
`subtractional' wavelengths for continuum calculations}
\Na { 
\+{\bf DFDUST}&&[LDU]&D, 2, \flpt \bang
Type-1 dust factor}
\Na { 
\+{\bf DGM}& {\bf *138}&\z &D, 2*,2, \flpt \bang
depth-dependent G multiplier (HSE)
\Df all = 1.0}
\Na {
\+{\bf DGMZ}& {\bf *138}&[NGM]&D, 2, \flpt \bang
standard table of DGM (as a function of ZGM) for the quiet sun
\Df (.64, .645, .66, .68, .71, .74, .77, .81, .86, .89, .92,
.95, .97, .985, .99, .995, .995, .995, .99, .98, .97,
.965, .96)}
\Na { 
\+{\bf DLU}&&&D, 1, \flpt \bang
dilution factor
\Df 1.0}
\Na { 
\+{\bf DO}&&[variable]&B, 2, \alfa \bang
enable program options (see Section 6 for further details)}
\Na { 
\+{\bf DOFDB}& {\bf *80}&[$ \leq 2 \times$NT]&D, 2, \intg \bang
alternate form of LSFFDB}
\Na { 
\+{\bf DOFLUX}& {\bf *79}&[$ \leq 2 \times$NT]&D, 2, \intg \bang
alternate form of LFLUX}
\Na { 
\+{\bf DOPROF}& {\bf *78}&[$ \leq 2 \times$NT]&D, 2, \intg \bang
alternate form of PROF}
\Na { 
\+{\bf DOSFPRNT}& $\qquad$ {\bf *81}&[$ \leq 2 \times$NT]&D, 2, \intg \bang
alternate form of LSFPRINT}
\Na { 
\+{\bf DPMULT}&&&D, 4, \flpt \bang
damping multiplier
\Df 1.0}
\Na { 
\+{\bf DQMAX}&&&D, 1, \flpt \bang
parameter for injection function integration (fast electrons)
\Df 2.0}
\Na { 
\+{\bf DQMIN}&&&D, 1, \flpt \bang
parameter for injection function integration (fast electrons)
\Df 0.01}
\Na { 
\+{\bf DRHO}&&&D, 4, \flpt \bang
RHO editing parameter
\Df 0.05}
\Na { 
\+{\bf DRLIM}& {\bf *42}&&D, 1, \flpt \bang
DR parameter, PRD transitions
\Df 0.01}
\Na { 
\+{\bf DWAVE}& {\bf *65}&[NDV]&D, 2, \flpt \bang
Continuum Source Function dump wavelengths table}
\Na { 
\+{\bf DWN}& {\bf *77}&[LDL]&D, 4, \flpt \bang
= DDL, but in wavenumbers}
\Na { 
\+{\bf DZMSS}&&&D, 1, \flpt \bang
Z-from-Mass calculation parameter
\Df 0.01}
\Na { 
\+{\bf ECLI}& {\bf *58}&&D, 4, \intg \bang
Eclipse line profiles computation switch}
\Na { 
\+{\bf EIDIF}&&&D, 1, \flpt \bang
NE-iterations convergence criterion
\Df $10^{-4}$}
\Na { 
\+{\bf ELEMENT}&&&D  \bang
element data (see Section 10)}
\Na { 
\+{\bf ELLED}&&&D, 1, \flpt \bang
particle energy dissipation calculation parameter `L' (fast electrons)
\Df $2.4 \times 10^{-11}$}
\Na { 
\+{\bf ELSYM}& {\bf *93}&&D, 1, \alfa \bang
chemical symbol of the ion of the run
\Df ``ZZ''}
\Na { 
\+{\bf EMXED}&&&D, 1, \flpt \bang
particle energy dissipation calculation parameter `EMAX' (fast electrons)
\Df $10^{-6}$}
\Na { 
\+{\bf EPCBR}&&&D, 1, \flpt \bang
branching ratio for supplementary levels in Lyman EPSILON-1}
\Na { 
\+{\bf EPDUST}&&[LDU]&D, 2, \flpt \bang
Type-1 dust dilution factor}
\Na { 
\+{\bf EP1}&&\z &D, 2*,2, \flpt \bang
Lyman EPSILON-1}
\Na { 
\+{\bf EP2}&&\z &D, 2*,2, \flpt \bang
Lyman EPSILON-2}
\Na { 
\+{\bf ESCTAU}& {\bf *48}&&D, 1, \flpt \bang
TAU criterion for automatic use of escape probability solution
\Df 5.0}
\Na { 
\+{\bf EXLYM}& {\bf *67}&&D, 1, \flpt \bang
Lyman change-over TAU parameter
\Df 10.}
\Na { 
\+{\bf FABD}&&&D, 1, \flpt \bang
multiplier for element abundances
\Df 1.0}
\Na {
\+{\bf FBVMX}&&&F, 1, \flpt \bang
maximum velocity value for flow broadening
\Df 100.}
\Na { 
\+{\bf FCE}&&\z &D, 5*,5, \flpt \bang
CE-enhancement factors}
\Na { 
\+{\bf FEK}&&\z &H, 2*,2, \flpt \bang
singly-ionized Iron number density
\Df computed in LTE}
\Na { 
\+{\bf FEN}&&\z &H, 3*,3, \flpt \bang
Iron-I level populations
\Df FEN$_{ij}$ computed in LTE, for all levels $j$ 
such that $j >$ NFE}
\Na { 
\+{\bf FILE}& {\bf *44}&&B,D,F,H \alfa \bang
input file designation}
\Na { 
\+{\bf FINK}&&[INK]&D, 2, \flpt \bang
incident radiation input values
\Df (0., $3.0 \times 10^{-12}$, $3.0 \times 10^{-12}$,
     $1.5 \times 10^{-12}$, $1.5 \times 10^{-12}$, 0.)}
\Na { 
\+{\bf FKUR}&&[KURNWV]&D, 2, \flpt \bang
multiplier for Statistical Line Opacity (see Section 9)
\Df all = 1.0 (KURNWV = 53, built in)}
\Na { 
\+{\bf FMCDL}&&&D, 1, \flpt \bang
Hydrogen Stark splitting components elimination criterion
\Df 0.1}
\Na { 
\+{\bf FMVLIM}& {\bf *82}&&D, 1, \flpt \bang
fluid velocity multiplier limit
\Df $10^{-4}$}
\Na {
\+{\bf FNH}&&[NFH]&F, 2, \flpt \bang
standard tabel of flow velocity for flow broadening
\Df (10., 9., 7., 5., 3., 2., 1., 0.)}
\Na {
\+{\bf FNRMLA}&&65&D, 2, \flpt \bang
normalizing factor for simulated background H Ly $\alpha$ profile
\Df all = 1.0}
\Na {
\+{\bf FNRMLB}&&65&D, 2, \flpt \bang
normalizing factor for simulated background H Ly $\beta$ profile
\Df all = 1.0}
\Na { 
\+{\bf FRCDL}&&&D, 1, \flpt \bang
Hydrogen Stark splitting components elimination criterion
\Df 0.01}
\Na { 
\+{\bf FROSCE}& {\bf *93}&&D, 1, \flpt \bang
fraction-of-classical-oscillator-strength used in the calculation of
collision rates for forbidden transitions
\Df 0.01}
\Na { 
\+{\bf FRR}&&[MRR]&D, 2, \flpt \bang
radius fraction
\Df (0.0, 0.5, 0.8, 0.9, 0.95, 1.0)}
\Na { 
\+{\bf FSTKM}&&&D, 1, \flpt \bang
Hydrogen Stark splitting reduction factor
\Df 1.0}
\Na { 
\+{\bf FZION}&&&D, 1, \flpt \bang
ZION-multiplier for diffusion calculation
\Df 1.0}
\Na { 
\+{\bf FZLIM}&&&D, 1, \flpt \bang
Z-from-Mass calculation parameter
\Df 1.5}
\Na { 
\+{\bf GK}& {\bf *123}&[KK]&D, 2, \flpt \bang
Gaunt factors for Level-KOLEV-to-Continuum calculation
\Df RRCP(KOLEV)}
\Na { 
\+{\bf GMMA}& {\bf *42}&&D, 4, \flpt \bang
gamma-parameter, PRD transitions
\Df -1.0}
\Na {
\+{\bf HEABL}&&&D, 1, \flpt \bang
Helium abundance limit factor for RHEAB calculation
\Df 3.0}
\Na { 
\+{\bf HEK}&&\z &H, 2*,2, \flpt \bang
singly-ionized Helium number density
\Df computed in LTE}
\Na { 
\+{\bf HEL}&&&D, 1, \flpt \bang
weight for HSE calculation
\Df 1.0}
\Na { 
\+{\bf HEN}&&\z &H, 3*,3, \flpt \bang
Helium-I level populations
\Df HEN$_{ij}$ computed in LTE, for all levels $j$ 
such that $j >$ NLZ}
\Na { 
\+{\bf HE2K}&&\z &H, 2*,2, \flpt \bang
doubly-ionized Helium number density
\Df computed in LTE}
\Na { 
\+{\bf HE2N}&&\z &H, 3*,3, \flpt \bang
Helium-II level populations
\Df HE2N$_{ij}$ computed in LTE, for all levels $j$ 
such that $j >$ NZ2}
\Na { 
\+{\bf HE304}&&\z &D, 2*,2, \flpt \bang
He-II $\lambda 304$ line mean intensity}
\Na { 
\+{\bf HN}&&\z &H, 3*,3, \flpt \bang
Hydrogen level populations
\Df HN$_{ij}$ computed in LTE, for all levels $j$ 
such that $j >$ NLH}
\Na {
\+{\bf HNAJL}&&&D, 1, \flpt \bang
limit for NH-adjustment factor in HSE calculation
\Df $10^{20}$}
\Na {
\+{\bf HNDF}&&[NFH]&F, 2, \flpt \bang
Hydrogen density table for FNH
\Df (1.0\ph, 3.0\ph, 1.0\pj, 3.0\pj, 1.0\pk, 3.0\pk, \break
1.0\pl, 3.0\pl)}
\Na {
\+{\bf HNDV}& {\bf *90}&[NVH]&D, 2, \flpt \bang
Hydrogen density table for VNH
\Df (1.0\pf, 2.0\pf, 5.0\pf, 1.0\pg, 2.0\pg, 5.0\pg, 8.38\pg,
1.07\ph, 1.61\ph, 3.17\ph, 7.73\ph, 2.71\pj, 9.32\pj,
2.04\pk, 6.69\pk, 9.82\pk, 2.25\pl, 3.55\pl, 6.01\pl, 
9.87\pl, 1.64\pm, 2.09\pm, 3.37\pm, 4.22\pm, 6.58\pm, 
1.02\pn, 2.33\pn, 4.24\pn, 6.05\pn, 8.33\pn, 1.03\po,
1.15\po, 1.22\po, 1.27\po, 1.30\po, 1.32\po, 1.34\po,
1.35\po)}
\Na { 
\+{\bf HSBDMN}&&&D, 1, \flpt \bang
Hydrogen Stark broadening parameter, for convolution calculation
\Df $10^{-5}$}
\Na { 
\+{\bf HSBDMX}&&&D, 1, \flpt \bang
Hydrogen Stark broadening parameter, for convolution calculation
\Df $10^3$}
\Na { 
\+{\bf HSBFEQ}&&&D, 1, \flpt \bang
Hydrogen Stark broadening parameter, for convolution calculation
\Df 0.1}
\Na { 
\+{\bf HSBM}&&&D, 1, \flpt \bang
Hydrogen Stark broadening parameter, for convolution calculation
\Df 20.0}
\Na { 
\+{\bf HSEC}& {\bf *15}&&D, 1, \flpt \bang
weight for HSE calculation
\Df 1.0}
\Na { 
\+{\bf HSLITER}&&&D, 1, \flpt \bang
number of HSL iterations
\Df 1}
\Na { 
\+{\bf HTAU}&&&D, 1, \flpt \bang
HSE calculation parameter
\Df 1.0}
\Na {
\+{\bf IBETSW}& {\bf *126}&&D, 1, \intg \bang
beta-equation selection switch, diffusion}
\Na {
\+{\bf IBNVIEW}&&&D, 1, \intg \bang
depth index for illustration of BD- and ND-calculations trace
\Df (JEDIT+1)}
\Na {
\+{\bf IBRDP}&&&D, 1, \intg \bang
diffusion d-coefficients debug dump switch}
\Na { 
\+{\bf ICDIT}& {\bf *101}&&D, 1, \intg \bang
dI/dh continuum wavelengths selector
\Df 1}
\Na { 
\+{\bf ICHDP}&&&D, 1, \intg \bang
hydrogen collision rates calculation dump depth index}
\Na { 
\+{\bf ICHSW}& { \bf *136}&&D, 1, \intg \bang
collision-with-Hydrogen switch}
\Na { 
\+{\bf ICR}&&[NCR]&D, 2, \flpt \bang
values of incident coronal radiation}
\Na {
\+{\bf ICXDP}& {\bf *95}&&D, 1, \intg \bang
upper-level charge-exchange dump depth index}
\Na {
\+{\bf IDEDP}&&&D, 1, \intg \bang
ion broadening (Hydrogen) dump switch}
\Na { 
\+{\bf IDEX}&&&D, 1, \intg \bang
extra information switch for standard-output (or log file)
\Df 10}
\Na {
\+{\bf IDFDI}& {\bf *89}&&D, 1, \intg \bang
d-coefficients dump index
\Df N/4}
\Na {
\+{\bf IDFDM}& {\bf *89}&&D, 1, \intg \bang
d-coefficients method selection switch
\Df 1}
\Na {
\+{\bf IDFDS}&&&D, 1, \intg \bang
d-coefficients smoothing
\Df 1}
\Na {
\+{\bf IDFSW}&&&D, 1, \intg \bang
dI/dh details print switch}
\Na { 
\+{\bf IDNRT}&&&D, 1, \intg \bang
switch for calculation of DNRT, DNRTC in Lyman
\Df 1}
\Na { 
\+{\bf IDRCD}&&&D, 1, \intg \bang
index of disk ray for CSF debug printout
\Df 1}
\Na { 
\+{\bf IDRDP}&&&D, 1, \intg \bang
depth index for option DRDMP 
\Df N/2}
\Na { 
\+{\bf IDWIN}&&&D, 1, \intg \bang
DW-dump index increment}
\Na { 
\+{\bf IFXDS}& {\bf *70}&&D, 1, \intg \bang
continuum flux detail output control}
\Na {
\+{\bf IGII}& {\bf *42}&&D, 1, \intg \bang
RII-approximation selector (PRD) 
\Df 1}
\Na {
\+{\bf IGMSW}& {\bf *42}&&D, 1, \intg \bang
alternate GMMA for H Lyman alpha and beta (PRD)}
\Na { 
\+{\bf IHDMP}&&&D, 1, \intg \bang
dump output control switch for Line Flux Distribution calculation}
\Na {
\+{\bf IHEAB}&&&D, 1, \intg \bang
reference depth index for RHEAB calculation}
\Na { 
\+{\bf IHSDD}& {\bf *106}&&D, 1, \intg \bang
Hydrogen Stark broadening dump switch}
\Na { 
\+{\bf IHSDP}& {\bf *106}&&D, 1, \intg \bang
Hydrogen Stark broadening dump switch}
\Na { 
\+{\bf IHSKM}&&&D, 1, \intg \bang
Hydrogen Stark broadening calculation table limit
\Df 100}
\Na { 
\+{\bf IHSSM}&&&D, 1, \intg \bang
Hydrogen Stark broadening calculation table limit
\Df 2000}
\Na { 
\+{\bf IHSSP}&&&D, 1, \intg \bang
Hydrogen Stark splitting control switch}
\Na { 
\+{\bf IHSSW}&&&D, 1, \intg \bang
Hydrogen Stark broadening (convolution) switch}
\Na { 
\+{\bf ILI}& {\bf *19}&&D, 1, \intg \bang
a RHO selection parameter}
\Na { 
\+{\bf IMUCD}&&&D, 1, \intg \bang
index of XMU for Continuum Source Function debug printout
\Df 1}
\Na {
\+{\bf INCEI }& {\bf *93}&&D, 1, \intg \bang
depth index for CI and CE comparison calculations
\Df index of TE-value closest to 8000 K, going in}
\Na { 
\+{\bf INCH}&&&D, 1, \flpt \bang
RHO weight adjustment parameter
\Df 0.1}
\Na {
\+{\bf INDRN}&&&D, 1, \intg \bang
input number densities renormalization switch
\Df 1}
\Na { 
\+{\bf INFSM}&&&D, 1, \intg \bang
Lyman RK-Kolev smoothing delimiter
\Df 1}
\Na { 
\+{\bf INK}&&&B, 1, \intg \bang
length of XINK
\Df 6}
\Na { 
\+{\bf INLSM}&&&D, 1, \intg \bang
Lyman RK-Kolev smoothing delimiter
\Df N}
\Na { 
\+{\bf INPAIR}& {\bf *2, *77}&[2$\times$NT]&B, 2, \intg \bang
list of transition indices}
\Na { 
\+{\bf INRHO}&&&D, 4, \intg \bang
input-RHO use switch}
\Na { 
\+{\bf IOMX}&&&D, 1, \intg \bang
number of overall iterations
\Df 1}
\Na { 
\+{\bf IONSTAGE}& $\qquad$ {\bf *93}&&D, 1, \intg \bang
stage of ionization of the ion of the run
\Df 1}
\Na {
\+{\bf IORIC}&&&D, 1, \intg \bang
line-center depths-of-formation print switch
\Df 1}
\Na { 
\+{\bf IPDEE}& {\bf *125}&&D, 1, \intg \bang
d-coefficients printout switch, diffusion calculation}
\Na { 
\+{\bf IPDIJ}& {\bf *127}&&D, 1, \intg \bang
DIJ printout switch, diffusion analysis}
\Na {
\+{\bf IPERFA}&&&D, 1, \intg \bang
performance data archive record switch
\Df 1}
\Na { 
\+{\bf IPEX}& {\bf *113}&&D, 1, \intg \bang
switch for extra debug output}
\Na {
\+{\bf IPIJG}&&&D, 1, \intg \bang
fudge GNV in the equation for PIJ
\Df 1}
\Na { 
\+{\bf IPPOD}& {\bf *52}&&D, 1, \intg \bang
`population ion' absorption/emission calculation dump switch}
\Na { 
\+{\bf IPR01}& {\bf *59}&&D, 1, \intg \bang
Line Source Function debug printout limiting index
\Df 1}
\Na { 
\+{\bf IPR02}& {\bf *59}&&D, 1, \intg \bang
Line Source Function debug printout limiting index
\Df 5}
\Na { 
\+{\bf IPR03}& {\bf *59}&&D, 1, \intg \bang
Line Source Function debug printout limiting index
\Df 10}
\Na { 
\+{\bf IPR04}& {\bf *59}&&D, 1, \intg \bang
Line Source Function debug printout limiting index
\Df 15}
\Na { 
\+{\bf IPRDD}& {\bf *30}&&D, 1, \intg \bang
depth interval for PRD printout
\Df 1}
\Na { 
\+{\bf IPRDF}& {\bf *30}&&D, 1, \intg \bang
frequency printout for PRD printout
\Df 1}
\Na { 
\+{\bf IPZER}& {\bf *71}&&D, 1, \intg \bang
dump printout selector for subroutine {\tt DIVIDE}}
\Na { 
\+{\bf IRATE}& {\bf *100}&&D, 1, \intg \bang
depth index for `minimal' rates printout}
\Na { 
\+{\bf IRFNC}& {\bf *111}&&D, 1, \intg \bang
depth index of reference value of charged particle number density
for printed sample values of Hydrogen CE and CI
\Df $i$ such that $TE_i$ ``equals" the middle value of TER, where
$i$ is the smallest such value}
\Na { 
\+{\bf IRKCOMP}& {\bf *45}&[NSL]&D, 2, \intg \bang
RK-components compute switches
\Df all =1}
\Na { 
\+{\bf IRLCOMP}& {\bf *45}&[NSL]&D, 2, \intg \bang
RL-components compute switches
\Df all =1}
\Na { 
\+{\bf IRLSN}& {\bf *39}&&D, 1, \intg \bang
RL integration methods switch
\Df 1}
\Na { 
\+{\bf IRLS1}& {\bf *39}&&D, 1, \intg \bang
RL integration methods switch
\Df 1}
\Na { 
\+{\bf IRPUN}& {\bf *84}&&D, 1, \intg \bang
RABD calculation data output switch
\Df 1}
\Na { 
\+{\bf IRTIS}& {\bf *63}&&D, 1, \intg \bang
incident radiation table interpolation selector
\Df 2}
\Na { 
\+{\bf IRUNT}& {\bf *74, *84}&&D, 1, \intg \bang
run type switch}
\Na { 
\+{\bf ISCOMP}&&&D, 1, \intg \bang
line source functions comparison printout details switch}
\Na { 
\+{\bf ISCRS}& {\bf *20, *84}&&B, 1, \intg \bang
scratch I/O mode switch}
\Na { 
\+{\bf ISRCD}&&&D, 1, \intg \bang
index of shell ray for Continuum Source Function debug printout
\Df 1}
\Na { 
\+{\bf ISMBD}&&&D, 1, \intg \bang
intensity integration ({\tt SIMBA}) dump interval}
\Na { 
\+{\bf ISMSW}& {\bf *99}&&D, 1, \intg \bang
iteration summaries format switch}
\Na {
\+{\bf ISMVE}&&&D, 1, \intg \bang
small-values editing switch
\Df 1}
\Na {
\+{\bf ISNDD}& {\bf *121}&&D, 1, \intg \bang
S(n) calculation dump switch}
\Na { 
\+{\bf ISNUD}&&&D, 1, \intg \bang
PRD SNU-shift debug dump switch}
\Na { 
\+{\bf ISOD}&&&D, 1, \intg \bang
depth index for Sobolev integration dump}
\Na {
\+{\bf ISSV}& {\bf *55}&[NVX]&F, 2, \intg \bang
shock velocity depth indices}
\Na { 
\+{\bf ISTARK}&&&D, 1, \intg \bang
default value of NE-index for Stark splitting of Hydrogen lines
\Df largest index where NE $\approx 10^{12}$}
\Na { 
\+{\bf ISUB}&&&D, 1, \intg \bang
number of sub-iterations (= RHO-iterations)
\Df 1}
\Na {
\+{\bf ITKZA}&&&D, 1, \intg \bang
Z-augmentation (diffusion) iteration limit
\Df 1}
\Na { 
\+{\bf ITN1R}&&&D, 1, \intg \bang
``Special N1'' iterations limit (diffusion calculations)
\Df 10}
\Na {
\+{\bf ITPRD}& {\bf *42}&&D, 1, \intg, \bang
PRD-iterations limit
\Df 4}
\Na { 
\+{\bf ITRFI}&&&D, 1, \intg \bang
TR-iteration debug output control}
\Na { 
\+{\bf IVOIT}& {\bf *50}&&D, 1, \intg \bang
Voigt profile methods selector
\Df 1}
\Na { 
\+{\bf IWEIT}&&&D, 1, \intg \bang
weighting details print switch}
\Na { 
\+{\bf IWSMD}&&&D, 1, \intg \bang
WAVELENGTHS summary Part-2 switch}
\Na {
\+{\bf IXASM}&&&D, 1, \intg \bang
smoothing dump (IPEX=25) detail control index}
\Na { 
\+{\bf IXNCS}& {\bf *111}&&D, 1, \intg \bang
switch controlling calculation of on-the-fly Hydrogen CE and CI values}
\Na { 
\+{\bf IXSTA}& {\bf *84}&&D, 1, \intg \bang
performance statistics printout control
\Df 1}
\Na { 
\+{\bf IZOPT}& {\bf *38}&&D, 1, \intg \bang
graph Z-scale (axis) option
\Df 1}
\Na { 
\+{\bf I4DEQ}& {\bf *122}&&D, 1, \intg \bang
four-diagonal method (``Special N1''), equation selector}
\Na { 
\+{\bf I4DFM}& {\bf *122}&&D, 1, \intg \bang
four-diagonal method (``Special N1''), version selector
\Df 1}
\Na { 
\+{\bf I4DIO}& {\bf *122}&&D, 1, \intg \bang
four-diagonal method (``Special N1''), flow direction specifier
\Df 1}
\Na {
\+{\bf JATAW}&&&D, 1, \intg \bang
write values of WRAT and RRCP as part of atomic data defaults output}
\Na { 
\+{\bf JBAR}&&\z &D, 5*,5, \flpt \bang
mean intensity}
\Na { 
\+{\bf JBDNC}& {\bf *84}&&D, 1, \intg \bang
Rho and b-ratio calculation bypass switch}
\Na { 
\+{\bf JBFSW}&&&D, 1, \intg \bang
b calculation method selector for supplementary levels
\Df 1}
\Na { 
\+{\bf JDMCE}& {\bf *115}&&D, 1, \intg \bang
debug dump switch for default calculation of Hydrogen CE values}
\Na { 
\+{\bf JDMCI}& {\bf *115}&&D, 1, \intg \bang
debug dump switch for default calculation of Hydrogen CI values}
\Na {
\+{\bf JEDIT}&&&D, 1, \intg \bang
depth index for N-editing
\Df N/2}
\Na {
\+{\bf JHBFD}&&&D, 1, \intg \bang
debug dump switch for H-bf background absorption and emission}
\Na {
\+{\bf JHEAS}&&&D, 1, \intg \bang
secret HEABD switch}
\Na {
\+{\bf JHLSK}& {\bf *137}&&D, 1, \intg \bang
Stark broadening in H Lyman lines background opacity
\Df 1}
\Na { 
\+{\bf JH1}& {\bf *102}&&D, 1, \intg \bang
photoionization rates multiplier index}
\Na { 
\+{\bf JH2}& {\bf *102}&&D, 1, \intg \bang
photoionization rates multiplier index}
\Na { 
\+{\bf JM}&&&B, 1, \intg \bang
lenght of LMM
\Df 1}
\Na {
\+{\bf JNEDP}&&&D, 1, \intg \bang
N-editing dump switch}
\Na { 
\+{\bf JNUNC}& {\bf *46}&&D, 1, \intg \bang
JNU input switch}
\Na {
\+{\bf JSFEX}&&&D, 1, \intg \bang
LSF-solution-explanation print switch
\Df 1}
\Na {
\+{\bf JSSV}& {\bf * 55}&&D, 1, \intg \bang
shock temperature depth index}
\Na { 
\+{\bf JSTCN}& {\bf *5, *6, *11}&&D, 1, \intg \bang
Continuum-only run type selector}
\Na { 
\+{\bf JSTIN}& {\bf *6, *40, *84}&&D, 1, \intg \bang
input-check only switch}
\Na {
\+{\bf JZATMO}& {\bf *133}&&D, 1, \intg \bang
zero-print mode switch for ATMOSPHERE}
\Na {
\+{\bf JZATOM}& {\bf *133}&&D, 1, \intg \bang
zero-print mode switch for ATOM}
\Na { 
\+{\bf JZOPT}& {\bf *38}&&D, 1, \intg \bang
graph Z-scale (axis) option}
\Na { 
\+{\bf K}&&&B, 1, \intg \bang
= KS}
\Na {
\+{\bf KALHD}& {\bf *73}&&B, 1, \alfa \bang
{\tt Hi/Bye/Abort}-system control parameter
\Df `` '' (\ie blank)}
\Na {
\+{\bf KALOR}& {\bf *73}&&B, 1, \intg \bang
{\tt Hi/Bye/Abort}-system control parameter
\Df 1}
\Na {
\+{\bf KANTNU}& {\bf *129}&&D, 1, \intg \bang
TNU-analysis switch}
\Na { 
\+{\bf KAPDB}&&&D, 1, \intg \bang
continuum contributors control debug switch}
\Na { 
\+{\bf KARB}& {\bf *75}&&B, 1, \intg \bang
`print character' selector for `banner' page
\Df 1}
\Na { 
\+{\bf KB}& {\bf *66}&&B, 1, \intg \bang
length of XIBLU
\Df KS}
\Na {
\+{\bf KBNDS}& {\bf *122}&&D, 1, \intg \bang
boundary condition switch for diffusion (``Special N1'')
\Df 1}
\Na { 
\+{\bf KBT}& {\bf *66}&&D, 4, \intg \bang
length of XIBLUT
\Df KST}
\Na { 
\+{\bf KBTMAX}&&&B, 1, \intg \bang
maximum of the various values of KBT occurring in {\bf Part D}}
\Na { 
\+{\bf KBX}& {\bf *66}&&B, 1, \intg \bang
length of BXI
\Df 25}
\Na {
\+{\bf KB1WA}& {\bf *128}&&D, 1, \intg \bang
B1-weights depth index}
\Na {
\+{\bf KB1WB}& {\bf *128}&&D, 1, \intg \bang
B1-weights depth index}
\Na {
\+{\bf KB1WS}& {\bf *128}&&D, 1, \intg \bang
B1-weights type selection switch
\Df 2}
\Na {
\+{\bf KCOAA}&&&D, 1, \intg \bang
switch of short form of Composite Line Analysis output}
\Na {
\+{\bf KDAMP}& {\bf *122}&&D, 1, \intg \bang
matrix solution damping switch for ``Special N1'' (diffusion)}
\Na { 
\+{\bf KDIAG}& {\bf *122}&&D, 1, \intg \bang
diagonal method selector for diffusion (``Special N1'')
\Df 3}
\Na {
\+{\bf KDIFD1}& {\bf *119}&&D, 1, \intg \bang
method switch for derivatives in diffusion calculations
\Df -1}
\Na {
\+{\bf KDIFGA}& {\bf *124}&&D, 1, \intg \bang
GNV-fudging depth index
\Df -1}
\Na {
\+{\bf KDIFGB}& {\bf *124}&&D, 1, \intg \bang
GNV-fudging depth index
\Df -1}
\Na {
\+{\bf KDIFGS}& {\bf *124}&&D, 1, \intg \bang
GNV-fudging switch}
\Na { 
\+{\bf KDRDP}&&&D, 1, \intg \bang
frequency index for option DRDMP
\Df K/2}
\Na { 
\+{\bf KDUST}&&&D, 1, \flpt \bang
dust constant}
\Na { 
\+{\bf KHFFS}& $\quad$ {\bf *41}&&D, 1, \intg \bang
H free-free contribution to Total Hydrogen cooling
\Df 1}
\Na { 
\+{\bf KININT}&&&D, 1, \intg \bang
plot index selection increment for Line Background opacities (see Section 9)
\Df 5}
\Na { 
\+{\bf KINMAX}&&&D, 1, \intg \bang
plot index for Line Background opacities (see Section 9)
\Df index of depth near the minimum of TE}
\Na { 
\+{\bf KK}&&&B, 1, \intg \bang
length of XK
\Df MR(KOLEV)}
\Na { 
\+{\bf KKPR}&&&D, 1, \intg  \bang
frequency index for detailed Lyman printout
\Df KK}
\Na {
\+{\bf KLDIN}&&&D, 1, \intg \bang
Lyman dump depth interval}
\Na {
\+{\bf KLFIN}&&&D, 1, \intg \bang
Lyman dump frequency interval}
\Na { 
\+{\bf KMMAX}& {\bf *66}&&D, 1, \intg \bang
maximum XIFUL length}
\Na { 
\+{\bf KODNT}&&&D, 1, \intg \bang
Composite Line Opacity raw data dump interval (see Section 9)}
\Na { 
\+{\bf KOELS}&&&D, 1, \intg \bang
every-line switch for ORIGIN printout
\Df 1}
\Na { 
\+{\bf KOLEV}&&&B, 1, \intg \bang
level index ({\it i.e.} $\cal N$) for `Level-$\cal N$-to-Continuum' transfer
calculation
\Df 1}
\Na { 
\+{\bf KONFORM}& $\quad$ {\bf *12}&&D, 1, \intg \bang
detail contributions printout format selector
\Df 2}
\Na { 
\+{\bf KOOLSUM}& $\quad$ {\bf *41}&&D, 1, \intg \bang
control for components added into Total Cooling Rate for Hydrogen runs}
\Na { 
\+{\bf KPC}& {\bf *10}&\z &D, 5*,5, \flpt \bang
continuous opacity
\Df KPC$^{u,\ell}$ = KPC$^{MS,NS} \times$ KPCR$^{u,\ell}, \quad$
for $u \neq$ MS, $\ell \neq$ NS}
\Na { 
\+{\bf KPCR}&&&D, 4, \flpt \bang
ratio of the continuous opacity with respect to that of transition (MS,NS)}
\Na { 
\+{\bf KR}& {\bf *66}&&B, 1, \intg \bang
length of XIRED
\Df KS}
\Na { 
\+{\bf KRATE}& {\bf *118}&&D, 4, \intg \bang
single-vs.-net rate switch for transition terms
\Df 1}
\Na { 
\+{\bf KRT}& {\bf *66}&&D, 4, \intg \bang
length of XIREDT
\Df KST}
\Na { 
\+{\bf KRTMAX}&&&B, 1, \intg \bang
maximum of the various values of KRT occurring in {\bf Part D}}
\Na { 
\+{\bf KS}& {\bf *66}&&B, 1, \intg \bang
length of XISYM
\Df 24}
\Na { 
\+{\bf KST}& {\bf *66}&&D, 4, \intg \bang
length of XISYMT
\Df KS}
\Na { 
\+{\bf KSTMAX}&&&B, 1, \intg \bang
maximum of the various values of KST occurring in {\bf Part D}}
\Na { 
\+{\bf KTRANS}& {\bf *33}&&D, 4, \alfa \bang
transition descriptor
\Df ``RADIATIVE''}
\Na { 
\+{\bf KUDNT}&&&D, 1, \intg \bang
Statistical Line Opacity raw data dump interval (see Section 9)}
\Na { 
\+{\bf KURIN}&&&D, 1, \intg \bang
step selection index for Statistical Line Opacity data
(see Section 9)}
\Na { 
\+{\bf KURMA}&&&D, 1, \intg \bang
long-wavelength cutoff for Statistical Line Opacity data
(see Section 9)
\Df 9000.0}
\Na { 
\+{\bf KURMI}&&&D, 1, \intg \bang
short-wavelength cutoff for Statistical Line Opacity data
(see \break Section 9)
\Df 1682.0}
\Na {
\+{\bf KXLYM}&&&D, 1, \intg \bang
XK-table augmentation switch}
\Na { 
\+{\bf L}& {\bf *24}&&B, 1, \intg \bang
length of MU
\Df LF}
\Na { 
\+{\bf LCEX}&&&D, 1, \intg \bang
charge exchange index
\Df 1}
\Na { 
\+{\bf LCH}& {\bf *136}&[NSL]&D, 2, \intg \bang
collisions-with-Hydrogen codes}
\Na { 
\+{\bf LCOA}&&[NCB]&D, 2, \flpt \bang
CO-lines opacity wavelength band lower limit}
\Na { 
\+{\bf LCOB}&&[NCB]&D, 2, \flpt \bang
CO-lines opacity wavelength band upper limit}
\Na { 
\+{\bf LCOD}&&&D, 1, \flpt \bang
CO-lines opacity dump printout wavelength}
\Na { 
\+{\bf LCR}&&[LCR]&D, 2, \flpt \bang
wavelengths at which incident coronal radiation is specified}
\Na {
\+{\bf LDFD1}&&&D, 1, \intg \bang
smoothing control switch for computed derivatives}
\Na { 
\+{\bf LDINT}& {\bf *54}&&D, 1, \intg \bang
depth increment for detailed printout of transition terms
\Df 5}
\Na { 
\+{\bf LDL}& {\bf *77, *105}&&D, 3, \intg \bang
length of DDL
\Df 1}
\Na {
\+{\bf LDLMAX}&&&B, 1, \intg \bang
maximum of the various values of LDL occurring in {\bf Part D}
\Df 1}
\Na { 
\+{\bf LDT}&&[NDT]&D, 2, \flpt \bang
wavelengths tables for Type-2 dust opacity calculation
\Df (910.0, 952.0, 1000.0, 1050.0, 1110.0, 1180.0, 1250.0, 1330.0,
1430.0, \hfill\break 1540.0, 1670.0, 1820.0, 2000.0, 2080.0, 2170.0,
2270.0, 2380.0, 2500.0, 3330.0, \hfill\break 5000.0, 
1.0\pa, 3.0\pa, 1.0\pb, 3.0\pb, 1.0\pc, 3.0\pc, 1.0\pd, \break
3.0\pd, 1.0\pe)}
\Na { 
\+{\bf LDTYP}& {\bf *54}&&D, 1, \intg \bang
type control for detailed printout of transition terms
\Df 1}
\Na { 
\+{\bf LDU}&&&B, 1, \intg \bang
length of LMDUST
\Df 1}
\Na { 
\+{\bf LEEDS}&&&D, 1, \intg \bang
He-I background lines opacity calculation debug switch}
\Na { 
\+{\bf LEVDES}& {\bf *53}&[NSL]&D \bang
level designation (term designation)}
\Na { 
\+{\bf LF}& {\bf *17, *24}&&B, 1, \intg \bang
length of MUF
\Df 2}
\Na { 
\+{\bf LFLUX}& {\bf *72}&&D, 1, \intg \bang
Line Flux Distribution calculation control switch}
\Na { 
\+{\bf LG}&&&B, 1, \intg \bang
length of XMU
\Df 8}
\Na { 
\+{\bf LHEDS}&&&D, 1, \intg \bang
He-II background lines opacity calculation debug switch}
\Na {
\+{\bf LHHSE}&&&D, 1, \intg \bang
reference depth index for H and M in HSE}
\Na { 
\+{\bf LHM}&&[MHM]&D, 2, \flpt \bang
wavelengths for H-minus continuum calculations
\Df (16300.0, 16200.0, 16000.0, 15500.0, 15000.0, 14500.0, 14000.0,
\hfill\break 13500.0, 13000.0, 12500.0, 12000.0, 11500.0, 11000.0,
10500.0, 10000.0, 9500.0, \hfill\break 9000.0, 8500.0, 8000.0, 7500.0,
7000.0, 6500.0, 6000.0, 5500.0, 5000.0, 4500.0, \hfill\break 4000.0,
3500.0, 3000.0, 2500.0, 2000.0, 1750.0, 1500.0, 1250.0)}
\Na { 
\+{\bf LLY}&&&B, 1, \intg \bang
length of LMXX}
\Na { 
\+{\bf LMA}& {\bf *104}&&D, 1, \flpt \bang
Lyman EP-1 edit parameter
\Df 0.3}
\Na { 
\+{\bf LMB}& {\bf *104}&&D, 1, \flpt \bang
Lyman EP-1 edit parameter
\Df $10^4$}
\Na { 
\+{\bf LMCR}& {\bf *137}&&D, 1, \flpt \bang
Hydrogen Lyman lines background opacity parameter
\Df 85.0}
\Na { 
\+{\bf LMDL2}& {\bf *137}&&D, 1, \flpt \bang
DR parameter, Hydrogen Lyman lines background opacity
\Df 0.01}
\Na { 
\+{\bf LMDL3}& {\bf *137}&&D, 1, \flpt \bang
DR parameter, Hydrogen Lyman lines background opacity
\Df 0.01}
\Na { 
\+{\bf LMDR}& {\bf *137}&[LLY]&D, 2, \flpt \bang
DR parameter, Hydrogen Lyman lines background opacity}
\Na { 
\+{\bf LMDUST}&&[LDU]&D, 2, \flpt \bang
wavelengths table for Type-1 dust opacity data
\Df 5000.0}
\Na { 
\+{\bf LME}& {\bf *104}&&D, 1, \flpt \bang
Lyman EP-1 edit parameter
\Df $10^{-4}$}
\Na { 
\+{\bf LMF}& {\bf *104}&&D, 1, \flpt \bang
Lyman EP-1 edit parameter
\Df $10^{-5}$}
\Na { 
\+{\bf LMH}& {\bf *137}&&D, 1, \flpt \bang
wavelength cutoff for highest H Lyman lines background opacity
\Df 950.}
\Na { 
\+{\bf LMM}&&[JM]&D, 2, \flpt \bang
wavelengths table for opacity multiplier
\Df 1682.0}
\Na { 
\+{\bf LMR}& {\bf *104}&&D, 1, \flpt \bang
Lyman EP-1 edit parameter
\Df $10^4$}
\Na { 
\+{\bf LMT}& {\bf *104}&&D, 1, \flpt \bang
Lyman EP-1 edit parameter
\Df 0.3}
\Na { 
\+{\bf LMXC}& {\bf *137}&&D, 1, \flpt \bang
DR parameter, Hydrogen Lyman lines background opacity
\Df 2.0}
\Na { 
\+{\bf LMXP}& {\bf *137}&&D, 1, \flpt \bang
DR parameter, Hydrogen Lyman lines background opacity
\Df 3.0}
\Na { 
\+{\bf LMXX}& {\bf *137}&[LLY]&D, 2, \flpt \bang
DR parameter, Hydrogen Lyman lines background opacity}
\Na { 
\+{\bf LMZ}&&&D, 1, \flpt \bang
Lyman alpha wing background opacity cut-off wavelength
\Df 2500.0}
\Na { 
\+{\bf LN}& {\bf *67}&&D, 1, \intg \bang
depth index limit for saturation approximation in ``Lyman'' calculation
\Df 8}
\Na { 
\+{\bf LODCG}& {\bf *88}&&D, 1, \intg \bang
depth index for diffusion calculation graphs
\Df -1}
\Na { 
\+{\bf LOGAS}& {\bf *114}&&D, 1, \intg \bang
location analysis graph switch}
\Na { 
\+{\bf LOXDS}&&&D, 1, \intg \bang
O-I background lines opacity calculation debug switch}
\Na {
\+{\bf LPMLR}&&&F, 1, \intg \bang
mass-loss-rates print switch (used with LPVEL)
\Df 1}
\Na {
\+{\bf LPVEL}&&&F, 1, \intg \bang
profile-velocities print switch
\Df 1}
\Na { 
\+{\bf LR}& {\bf *77}&[NL]&B, 2, \intg \bang
number of RKC values (one LR for each level)}
\Na { 
\+{\bf LSFBOC}& {\bf *16}&&D, 4, \intg \bang
Line Source Function background opacity control}
\Na { 
\+{\bf LSFFDB}& {\bf *36}&&D, 4, \intg \bang
Line Source Function background type selector}
\Na {
\+{\bf LSFGC}& {\bf *68}&&D, 1, \intg \bang
Line Source Function graph control code
\Df 1}
\Na { 
\+{\bf LSFPRINT}& {\bf *81}&&D, 4, \intg \bang
Line Source Function printout switch}
\Na {
\+{\bf LSTMP}&&&D, 1. \intg \bang
STIM-for-GTN details print switch}
\Na { 
\+{\bf LSFTYP}& {\bf *48}&&D, 4, \intg \bang
Line Source Function solution method selector}
\Na { 
\+{\bf LWNT}&&&D, 1, \intg \bang
`Line Opacity' printout wavelengths interval (see Section 9)
\Df 1}
\Na { 
\+{\bf LX2DS}&&&D, 1, \intg \bang
O-II background lines opacity calculation debug switch}
\Na { 
\+{\bf LX3DS}&&&D, 1, \intg \bang
O-III background lines opacity calculation debug switch}
\Na { 
\+{\bf LYMITER}&&&D, 1, \intg \bang
number of Lyman iterations
\Df 1}
\Na { 
\+{\bf LYODS}&&&D, 1, \intg \bang
H Lyman background lines opacity calculation debug switch}
\Na { 
\+{\bf LZA}&&&B, 1*, \intg \bang
length of ZAUX}
\Na { 
\+{\bf M}&&&B, 1, \intg \bang
length of TS
\Df 33}
\Na { 
\+{\bf MAMAS}&&&D, 1, \intg \bang
matrix elements magnitude scan switch
\Df 1}
\Na { 
\+{\bf MASS}&&&D, 1, \flpt \bang
atomic mass}
\Na { 
\+{\bf MATRIX}& {\bf *140}&&D  \bang
matrix manipulation control data}
\Na { 
\+{\bf MAUX}&&&D,F,H, 1, \intg \bang
index specifying a ZAUX table}
\Na { 
\+{\bf MCE}&&[NSL]&D, 2, \flpt \bang
CE (default) multiplier
\Df 1.0}
\Na { 
\+{\bf MCI}&&[NSL]&D, 2, \flpt \bang
CI (default) multiplier
\Df 1.0}
\Na {
\+{\bf MCOA}&&&D, 1, \flpt \bang
mulitplier of van der Waals damping for CO-lines profiles
\Df all = 1.0}
\Na { 
\+{\bf MCON}&&&D, 1, \intg \bang
CO number density output switch}
\Na {
\+{\bf MDFG}&&&D, 1, \intg \bang
diffusion terms (GVL) output switch
\Df 1}
\Na { 
\+{\bf MDFV}&&&D, 1, \intg \bang
diffusion velocities output switch
\Df 1}
\Na { 
\+{\bf MDTR1}&&&D, 1, \intg \bang
Type-2 dust opacity calculation iteration limit
\Df 10}
\Na { 
\+{\bf MDTR2}&&&D, 1, \intg \bang
Type-2 dust opacity calculation iteration limit
\Df 20}
\Na { 
\+{\bf METEP}& {\bf *21}&&D, 1, \intg \bang
Lyman EP-1 and EP-2 calculation methods selector
\Df 3}
\Na { 
\+{\bf METSE}& {\bf *35}&&D, 4, \intg \bang
Statistical Equilibrium equations calculation methods selector
\Df {\bf METSEDG} or {\bf METSEDW}}
\Na {
\+{\bf METSEDG}& {\bf \quad *35}&&D, 1, \intg \bang
general default value of METSE
\Df 1}
\Na {
\+{\bf METSEDW}& {\bf \quad *35}&&D, 1, \intg \bang
default value of METSE for transitions down to Level 1
\Df 3}
\Na {
\+{\bf MFONT}&&&D, 1, \intg \bang
Fontenla atmosphere data output switch
\Df 1}
\Na { 
\+{\bf MGK}&&\z &H, 2*,2, \flpt \bang
singly-ionized Magnesium number density
\Df computed in LTE}
\Na { 
\+{\bf MGN}&&\z &H, 3*,3, \flpt \bang
Magnesium-I level populations
\Df MGN$_{ij}$ computed in LTE, for all levels $j$ 
such that $j >$ NMG}
\Na { 
\+{\bf MHM}&&&B, 1, \intg \bang
length of LHM
\Df 34}
\Na { 
\+{\bf MH2N}&&&D, 1, \intg \bang
H2 number density output switch}
\Na {
\+{\bf MKURU}&&&D, 1, \intg \bang
Kurucz spectrum data output switch
\Df 1}
\Na { 
\+{\bf MLC}&&[JM]&D, 2, \flpt \bang
opacity multiplier
\Df 1.0}
\Na { 
\+{\bf MN1}&&&D, 1, \intg \bang
depth limit for N1 recalculation in the ambipolar diffusion calculation
\Df N}
\Na { 
\+{\bf MNG1}&&&D, 1, \intg \bang
depth limit for GNV-1 replacement in the ambipolar diffusion calculation
\Df -MN1}
\Na { 
\+{\bf MODLAB}& {\bf *37}&&D, 1, \alfa \bang
name of atmospheric model
\Df ``!NONAME!''}
\Na { 
\+{\bf MOPRNT}&&&D, 1, \intg \bang
switch to print built-in population-ion models
\Df 1}
\Na { 
\+{\bf MQT}&&&B, 1, \intg \bang
length of QTAIL
\Df 3}
\Na { 
\+{\bf MR}&{\bf *29, *77}&[NSL]&B, 2, \intg \bang
number of WRAT values (one MR for each level)}
\Na { 
\+{\bf MRR}&&&B, 1, \intg \bang
length of FRR
\Df 6}
\Na { 
\+{\bf MS}& {\bf *2, *59}&&D, 1, \intg \bang
index of upper level of `reference transition'
\Df from INPAIR; see Note *2}
\Na { 
\+{\bf MSKIP}& {\bf *62}&&D, 1, \intg \bang
ray selection parameter for computing weight matrices in spherical coordinates}
\Na { 
\+{\bf MSSPR}&&&D, 1, \intg \bang
print switch for matrix of simultaneous ``Special N1'' solution (diffusion)
\Df 1}
\Na { 
\+{\bf MTHEI}& {\bf *83}&&D, 1, \intg \bang
exponential integral method selector
\Df 1}
\Na {
\+{\bf MTREF}&&&D, 1, \intg \bang
TR-effective output switch}
\Na { 
\+{\bf MU}& {\bf *24}&[L]&F, 2, \flpt \bang
cosine-of-lookangle values for emergent intensity calculation
\Df MUF}
\Na { 
\+{\bf MUF}& {\bf *24}&[LF]&F, 2, \flpt \bang
cosine-of-lookangle values for emergent flux calculation
\Df (1.0, 0.3)}
\Na {
\+{\bf MXPPI}&&&D, 1, \intg \bang
limit for individual KZAUG values (Z-augmentation, diffusion)
\Df 5}
\Na {
\+{\bf MXTAP}&&&D, 1, \intg \bang
limit for sum of KZAUG values (Z-augmentation, diffusion)
\Df 100}
\Na { 
\+{\bf M304}&&&D, 1, \intg \bang
index of reference value of He-II $\lambda 304$ line mean intensity
\Df 10}
\Na { 
\+{\bf N}&&&B, 1, \intg \bang
length of Z}
\Na { 
\+{\bf NAB}& {\bf *77}&&B, 1, \intg \bang
length of BANDL (see Section 9)}
\Na { 
\+{\bf NABS}& {\bf *7}&[37]&D, 2, \intg \bang
absorber/emitter switches
\Df NABS$_i =$ on, for all $i$}
\Na { 
\+{\bf NAK}&&\z &H, 2*,2, \flpt \bang
singly-ionized Sodium number density
\Df computed in LTE}
\Na { 
\+{\bf NAL}&&&B, 1, \intg \bang
number of levels for which Aluminum populations are specified}
\Na { 
\+{\bf NAME}& {\bf *3}&&D, 1, \alfa \bang
name of ion of run}
\Na { 
\+{\bf NAN}&&\z &H, 3*,3, \flpt \bang
Sodium-I level populations
\Df NAN$_{ij}$ computed in LTE, for all levels $j$ 
such that $j >$ NNA}
\Na { 
\+{\bf NANAL1}&&&D, 1, \intg \bang
profile ANALYSIS depth selection parameter
\Df 1}
\Na { 
\+{\bf NANAL2}&&&D, 1, \intg \bang
profile ANALYSIS depth selection parameter
\Df 5}
\Na { 
\+{\bf NAPKNT}&&&D, 1, \intg \bang
recombination parameter}
\Na { 
\+{\bf NAPWRA}&&&D, 1, \intg \bang
recombination parameter
\Df 1}
\Na { 
\+{\bf NAPWRB}&&&D, 1, \intg \bang
recombination parameter
\Df 2}
\Na {
\+{\bf NARB}& {\bf *75}&&B, 1, \intg \bang
number of `banner' pages}
\Na { 
\+{\bf NBS}&&&D, 1, \intg \bang
b-smoothing control (level) index
\Df 2}
\Na { 
\+{\bf NC}&&\z &H, 2*,2, \flpt \bang
charged particle number density}
\Na { 
\+{\bf NCA}&&&B, 1, \intg \bang
number of levels for which Calcium populations are specified}
\Na { 
\+{\bf NCB}&&&B, 1, \intg \bang
length of LCOA, LCOB}
\Na { 
\+{\bf NCL}& {\bf *29, *109}&&B, 1, \intg \bang
length of XCOL
\Df 5}
\Na { 
\+{\bf NCOI}&&\z &D, 2*,2, \flpt \bang
CO number density, input (to replace computed values)}
\Na {
\+{\bf NCOPT}&&&D, 1, \intg \bang
CO-lines opacity calculations statistics-keeping switch}
\Na { 
\+{\bf NCOSW}&&&D, 1, \intg \bang
Carbon Monoxide abundance correction computation method selector
\Df 1}
\Na { 
\+{\bf NCQ}&&&B, 1, \intg \bang
length of CQT
\Df 5}
\Na { 
\+{\bf NCR}&&&B, 1, \intg \bang
length of LCR}
\Na { 
\+{\bf ND}& {\bf *32}&\z &D, 3*,3, \flpt \bang
number densities of the levels of the ion of the run
\Df computed in LTE}
\Na { 
\+{\bf NDR}&&&B, 1, \intg \bang
length of XDR
\Df 7}
\Na {
\+{\bf NDSN1}&&&D, 1, \intg \bang
skip Special N-1 recalculation in first overall calculation}
\Na { 
\+{\bf NDT}&&&B, 1, \intg \bang
length of LDT
\Df 29}
\Na { 
\+{\bf NDV}&&&B, 1, \intg \bang
length of DWAVE}
\Na { 
\+{\bf NDW}& {\bf *86}&&D, 1, \intg \bang
depth index for reference value of DW (Doppler width)
\Df either index of Z-value closest to ZNDW, or N/2}
\Na {
\+{\bf NDWM}&&&D, 1, \intg \bang
depth index for reference value of DW (Doppler width), for
atmospheric model
\Df NDW}
\Na { 
\+{\bf NE}&&\z &D, 2*,2, \flpt \bang
electron number density}
\Na {
\+{\bf NECLIP}&&&D, 1, \intg \bang
continuum eclipse printout quantity selector}
\Na { 
\+{\bf NED}&&&D, 4, \intg \bang
RHO editing index
\Df N}
\Na { 
\+{\bf NEFDF}& {\bf *125}&&D, 4, \intg \bang
switch for NE for d-coefficients in diffusion calculation
\Df 1}
\Na { 
\+{\bf NERM}& {\bf *85}&&D, 1, \intg \bang
limit for some error messages from {\tt EDITH}
\Df 10}
\Na { 
\+{\bf NEWELE}&&&D  \bang
element data (see Section 10)}
\Na {
\+{\bf NFB}&&&B, 1, \intg \bang
number of isotropic flow broadening velocities
\Df 6}
\Na { 
\+{\bf NFE}&&&B, 1, \intg \bang
number of levels for which Iron populations are specified}
\Na {
\+{\bf NFH}&&&B, 1, \intg \bang
length of HNDF
\Df 8}
\Na {
\+{\bf NGM}& {\bf *138}&&B, 1, \intg \bang
length of DGMZ
\Df 23}
\Na {
\+{\bf NGNV}& {\bf *116}&&D, 1, \intg \bang
GNVL-suppression level limit}
\Na { 
\+{\bf NGRL}& {\bf *38}&&D, 1, \intg \bang
graphs Z-scale (axis) limit}
\Na { 
\+{\bf NGRR}& {\bf *38}&&D, 1, \intg \bang
graphs Z-scale (axis) limit}
\Na { 
\+{\bf NH}&&\z &D, 2*,2, \flpt \bang
total Hydrogen number density}
\Na { 
\+{\bf NHN}  \bang
= ``HN 1 ''}
\Na { 
\+{\bf NHTSW}& {\bf *64}&&D, 1, \intg \bang
H2 abundance correction method selector
\Df 2}
\Na { 
\+{\bf NIASM}&&&D, 1, \intg \bang
sequential smoothing parameter
\Df 20}
\Na { 
\+{\bf NIL}& {\bf *19}&&D, 1, \intg \bang
a RHO selection parameter
\Df 2}
\Na { 
\+{\bf NK}&&\z &D, 3*,3, \flpt \bang
ionized number density of the ion of the run
\Df computed in LTE}
\Na { 
\+{\bf NKA}&&&B, 1, \intg \bang
length of ZALBK (see Section 9)
\Df 2}
\Na { 
\+{\bf NL}&{\bf *29, *77}&&B, 1, \intg \bang
number of levels of the ion of the run
\Df 2}
\Na { 
\+{\bf NLC}&&&B, 1, \intg \bang
number of levels for which Carbon populations are specified}
\Na { 
\+{\bf NLH}&&&B, 1, \intg \bang
number of levels for which Hydrogen populations are specified}
\Na { 
\+{\bf NLO}&&&B, 1, \intg \bang
number of levels for which Oxygen populations are specified}
\Na {
\+{\bf NLPAIR}& {\bf *94, *95}&[2$\times$NL]&B, 2, \intg \bang
list of quantum numbers}
\Na { 
\+{\bf NLS}&&&B, 1, \intg \bang
number of levels for which Silicon populations are specified}
\Na { 
\+{\bf NLU}&&&B, 1, \intg \bang
number of levels for which Sulphur populations are specified}
\Na { 
\+{\bf NLY}&&&D, 1, \intg \bang
H Ly lines background opacity limit
\Df 15}
\Na { 
\+{\bf NLZ}&&&B, 1, \intg \bang
number of levels for which Helium populations are specified}
\Na { 
\+{\bf NMG}&&&B, 1, \intg \bang
number of levels for which Magnesium populations are specified}
\Na {
\+{\bf NMLR}&&&D, 1, \intg \bang
mass-loss-rate index
\Df NDW}
\Na { 
\+{\bf NMT}& {\bf *16, *77}&&B, 1, \intg \bang
number of rows in the table ELE (see Section 10)
\Df 38}
\Na { 
\+{\bf NNA}&&&B, 1, \intg \bang
number of levels for which Sodium populations are specified}
\Na { 
\+{\bf NNDFE}&&&D, 1, \intg \bang
dump control for injection function FJIN (fast electrons)
\Df -1}
\Na { 
\+{\bf NODCG}& {\bf *88}&&D, 1, \intg \bang
depth index for diffusion calculation graphs
\Df -1}
\Na { 
\+{\bf NOION}& {\bf *11, *84}&&B, 1, \intg \bang
`no ion' switch}
\Na { 
\+{\bf NO2}&&&B, 1, \intg \bang
number of levels for which Oxygen-II populations are specified}
\Na { 
\+{\bf NO3}&&&B, 1, \intg \bang
number of levels for which Oxygen-III populations are specified}
\Na { 
\+{\bf NP}&&\z &H, 2*,2, \flpt \bang
proton number density
\Df computed in LTE}
\Na { 
\+{\bf NQLYM}& {\bf *137}&&D, 1, \intg \bang
weight limit for highest H Lyman lines background opacity}
\Na { 
\+{\bf NS}& {\bf *2, *59}&&D, 1, \intg \bang
index of lower level of `reference transition'
\Df from INPAIR; see Note *2}
\Na { 
\+{\bf NSL}&{\bf *29, *77}&&B, 1, \intg \bang
total number of levels of the ion of the run (including supplementary levels)
\Df NL}
\Na { 
\+{\bf NSPED}&&&D, 1, \intg \bang
particle energy dissipation calculation parameter ``NS'' (fast electrons)
\Df 1}
\Na {
\+{\bf NSPRD}&&&D, 1, \intg \bang
secret PRD switch}
\Na {
\+{\bf NSW}&{\bf *134}&&B, 1, \intg \bang
length of SCOW}
\Na { 
\+{\bf NT}&{\bf *29, *77}&&B, 1, \intg \bang
number of transitions specified in INPAIR}
\Na { 
\+{\bf NTAN}& {\bf *62}&&B, 1, \intg \bang
ray selection parameter for computing weight matrices in spherical coordinates
\Df 4}
\Na { 
\+{\bf NTE}&&&B, 1, \intg \bang
length of TER
\Df 1}
\Na { 
\+{\bf NU}& {\bf *93}&[NSL]&D, 2, \flpt \bang
frequency intervals between levels of the ion of the run,
in frequency units (see also WNU)}
\Na { 
\+{\bf NUC}& {\bf *93}&[NSL]&D, 2, \flpt \bang
auxiliary continuum frequency intervals of the ion of the run,
in frequency units (see also WNUC)
\Df NUK}
\Na { 
\+{\bf NUK}& {\bf *93}&&D, 1, \flpt \bang
continuum frequency interval of the ion of the run,
in frequency units (see also WNUK)}
\Na { 
\+{\bf NVDFE}&&&D, 1, \intg \bang
dump control for injection function FINJ (fast electrons)
\Df -1}
\Na { 
\+{\bf NVF}&&&B, 1, \intg \bang
number of velocity values (fast electrons)
\Df 30}
\Na {
\+{\bf NVH}& {\bf *90}&&B, 1, \intg \bang
length of HNDV
\Df 38}
\Na { 
\+{\bf NVOIT}& {\bf *50, *84}&&D, 1, \intg \bang
Voigt profiles subroutine execution statistics printout switch
\Df 1}
\Na { 
\+{\bf NVX}& {\bf *29, *82, *108}&&B, 1, \intg \bang
number of VX tables}
\Na { 
\+{\bf NWS}&&&B, 1, \intg \bang
length of DELWAVE}
\Na { 
\+{\bf NWV}&&&B, 1, \intg \bang
length of WAVES}
\Na { 
\+{\bf NXF}&&&B, 1, \intg \bang
maximum number of integrand values for injection functions (fast electrons)
\Df 1000}
\Na { 
\+{\bf NZDFE}&&&D, 1, \intg \bang
dump control (fast electrons)
\Df -1}
\Na { 
\+{\bf NZE}&&&B, 1, \intg \bang
length of ZECL}
\Na { 
\+{\bf NZ2}&&&B, 1, \intg \bang
number of levels for which Helium-II populations are specified}
\Na { 
\+{\bf N1MET}& {\bf *122}&&D, 1, \intg \bang
``Special N1'' (diffusion calculation), method selector
\Df 2}
\Na {
\+{\bf N1NUP}&&&D, 1, \intg \bang
populations-of-the-run update switch, ``Special N1'' calculation
\Df 1}
\Na { 
\+{\bf OK}&&\z &H, 2*,2, \flpt \bang
singly-ionized Oxygen number density
\Df computed in LTE}
\Na { 
\+{\bf OLL}& {\bf *18}&&D, 4, \flpt \bang
line opacity multiplier
\Df 1.0}
\Na { 
\+{\bf OMIT}&&[variable]&B, 2, \alfa \bang
disable program options (see Section 6 for further details)}
\Na { 
\+{\bf OML}& {\bf *16}&&D, 4, \flpt \bang
line-background opacity multiplier
\Df 1.0}
\Na { 
\+{\bf ON}&&\z &H, 3*,3, \flpt \bang
Oxygen-I level populations
\Df ON$_{ij}$ computed in LTE, for all levels $j$ 
such that $j >$ NLO}
\Na { 
\+{\bf OPF}&&&D, 1, \flpt \bang
incident radiation extinction factor
\Df 1.0}
\Na { 
\+{\bf OUTPUT}& {\bf *23}&&B, 1, \alfa \bang
`general printout' file scope switch
\Df ``MERGE''}
\Na { 
\+{\bf O2K}&&\z &H, 2*,2, \flpt \bang
doubly-ionized Oxygen number density
\Df computed in LTE}
\Na { 
\+{\bf O2N}&&\z &H, 3*,3, \flpt \bang
Oxygen-II level populations
\Df O2N$_{ij}$ computed in LTE, for all levels $j$ 
such that $j >$ NO2}
\Na { 
\+{\bf O3K}&&\z &H, 2*,2, \flpt \bang
triply-ionized Oxygen number density
\Df computed in LTE}
\Na { 
\+{\bf O3N}&&\z &H, 3*,3, \flpt \bang
Oxygen-I level populations
\Df O3N$_{ij}$ computed in LTE, for all levels $j$ 
such that $j >$ NO3}
\Na { 
\+{\bf P}& {\bf *93}&[NSL]&D, 2, \flpt \bang
statistical weight}
\Na { 
\+{\bf PALBET}&&\z &D, 2*,2, \flpt \bang
Helium diffusion parameter}
\Na { 
\+{\bf PART}& {\bf *61}&&D, 1, \flpt \bang
partition function for the ion of the run}
\Na { 
\+{\bf PARTLIM}&&&D, 1, \flpt \bang
partition functions component limit (for ions in table ELE)
\Df 3.0}
\Na { 
\+{\bf PBETAL}&&\z &D, 2*,2, \flpt \bang
Helium diffusion parameter}
\Na { 
\+{\bf PBETGM}&&\z &D, 2*,2, \flpt \bang
Helium diffusion parameter}
\Na { 
\+{\bf PCE}&&&D, 5, \flpt \bang
FCE adjustment factors}
\Na { 
\+{\bf PGMBET}&&\z &D, 2*,2, \flpt \bang
Helium diffusion parameter}
\Na { 
\+{\bf PMSK}&&&D, 1, \flpt \bang
multiplier of default Stark half-width
\Df 1.0}
\Na { 
\+{\bf PNH}&&&D, 1, \flpt \bang
scattering albedo parameter for Background Line Opacities
(see Section 9)}
\Na { 
\+{\bf POPION}& {\bf *28}&&H  \bang
`population update' ion data}
\Na { 
\+{\bf POPRCP}& {\bf *28}&&H  \bang
`population update' ion data}
\Na { 
\+{\bf POPUP}& {\bf *8}&&D, 1, \alfa \bang
populations data update switch}
\Na { 
\+{\bf POPXLM}& {\bf *28}&&H  \bang
`population update' ion data}
\Na {
\+{\bf PRDCV}& {\bf *42}&&D, 1, \flpt \bang
PRD-iterations convergence criterion
\Df 0.1}
\Na { 
\+{\bf PROF}& {\bf *17}&&D, 4, \intg \bang
emergent line profiles calculations switch}
\Na {
\+{\bf PROGLI}& {\bf *98}&&D, 4, \flpt \bang
profile graphs control parameter}
\Na { 
\+{\bf PW}&&&D, 1, \flpt \bang
exponent for Stark broadening term
\Df 1.0}
\Na { 
\+{\bf PZERO}& {\bf *103}&&D, 1, \flpt \bang
Z-from-TAUKIN recalculation parameter}
\Na { 
\+{\bf QIN}&&\z &D, 2*,2, \flpt \bang
K-shell ionization calculation data}
\Na { 
\+{\bf QNL}&&[NSL]&D, 2, \intg \bang
number of ``$n\ell$'' electrons
\Df all = 1}
\Na { 
\+{\bf QTAIL}&&[MQT]&D, 2, \flpt \bang
Lyman EP-1 Q-smoothing tail
\Df (0.5, 0.1, 0.01)}
\Na { 
\+{\bf RABD}& {\bf *120}&\z &D, 2*,2, \flpt \bang
depth variation of abundance ratio
\Df antilog(RABDL$_i$), or RABD$_i$ = 1.0, for all $i$}
\Na { 
\+{\bf RABDL}& {\bf *120}&\z &D, 2*,2, \flpt \bang
depth variation of log of abundance ratio}
\Na { 
\+{\bf RCCFE}&&&D, 1, \flpt \bang
accuracy criterion for injection function integrations (fast electrons)
\Df 0.1} 
\Na {
\+{\bf RCHX}& {\bf *95, *97}&&D, 4, \flpt \bang
upper-level charge-exchange parameter}
\Na { 
\+{\bf RCOMIN}&&&D, 1, \flpt \bang
CO abundance lower limit
\Df $10^{-10}$} 
\Na { 
\+{\bf REFLM}& {\bf *103}&&D, 1, \flpt \bang
wavelength to which TAUKIN values correspond
\Df 911.1236}
\Na {
\+{\bf RFAC }&&&D, 1, \flpt \bang
reduction factor for all collision rates
\Df 1.0}
\Na {
\+{\bf RFHEAB}&&&D, 1, \flpt \bang
Helium abundance coefficient reduction factor for RHEAB calculation
\Df 1.0}
\Na { 
\+{\bf RFMAS}& {\bf *103}&&D, 1, \flpt \bang
reference mass}
\Na { 
\+{\bf RHEAB}&&\z &D, 2*,2, \flpt \bang
depth dependence of total Helium abundance
\Df RHEAB$_i$ = 1.0, for all $i$}
\Na { 
\+{\bf RHO}&&\z &D, 5*,5, \flpt \bang
net radiative bracket}
\Na { 
\+{\bf RHOPT}& {\bf *4}, {\bf *19}&&D, 1, \alfa \bang
a RHO selection parameter
\Df ``RHOJ''}
\Na { 
\+{\bf RHOWT}& {\bf *13, *131}&\z &D, 5*,5, \flpt \bang
RHO weights}
\Na { 
\+{\bf RHWT}& {\bf *13, *131}&\z &D, 5*,5, \flpt \bang
RHO weights
\Df 1.0}
\Na { 
\+{\bf RK}& {\bf *45}&\z &D, 3*,3, \flpt \bang
photoionization rate (for level KOLEV)}
\Na { 
\+{\bf RKC}&&[LR]&D, 3, \flpt \bang
additional photoionization parameter}
\Na {
\+{\bf RKMULT}&&[NSL]&D, 2, \flpt \bang
RK enhancement factor}
\Na { 
\+{\bf RKW}& {\bf *13, *131}&\z &D, 3*,3, \flpt \bang
RK-KOLEV weights}
\Na { 
\+{\bf RKWT}& {\bf *13, *131}&\z &D, 3*,3, \flpt \bang
RK-KOLEV weights
\Df 1.0}
\Na { 
\+{\bf RL}& {\bf *39}, {\bf *45}&\z &D, 3*,3, \flpt \bang
photorecombination rate (for level KOLEV)}
\Na { 
\+{\bf RQCP}& {\bf *22, *93}&[MR+1]&D, 3, \flpt \bang
= RRCP}
\Na { 
\+{\bf RRCP}& {\bf *22}, {\bf *56}&[MR]&D, 3, \flpt \bang
ratios of photoionization cross-sections
\Df computed for Level 1 for some non-Hydrogen runs}
\Na { 
\+{\bf RUNTOPOP}& {\bf \quad *8}&[8]&B, 2, \intg \bang
ion-of-the-run vs. built-in population-ion-model level correspondences,
for \break `population update' runs}
\Na { 
\+{\bf RZM}&&\z &D, 2*,2, \flpt \bang
metal electrons multiplier
\Df RZM$_i$ = 1.0, for all $i$}
\Na { 
\+{\bf R1N}&&&D, 1, \flpt \bang
distance from illuminating source}
\Na { 
\+{\bf SCH}& {\bf *47}&&D, 4, \intg \bang
partial redistribution calculation selector for Line Source Function
calculations}
\Na {
\+{\bf SCOW}&{\bf *134}&[NSW]&D, 2, \flpt \bang
selected Continuum output wavelengths}
\Na {
\+{\bf SCTA}&{ \bf *55}&&D, 1, \flpt \bang
shock temperature amplitude}
\Na {
\+{\bf SCTS}&{ \bf *55}&&D, 1, \flpt \bang
shock temperature scale height}
\Na {
\+{\bf SCVA}&{ \bf *55}&&F, 1, \flpt \bang
shock velocity amplitude}
\Na {
\+{\bf SCVB}&{ \bf *55}&&F, 1, \flpt \bang
shock velocity parameter}
\Na {
\+{\bf SCVS}&{ \bf *55}&&F, 1, \flpt \bang
shock velocity scale height}
\Na { 
\+{\bf SGRAF}& {\bf *98}&&D, 4, \intg \bang
profile graphs control parameter}
\Na { 
\+{\bf SHCOC}&&&D, 1, \flpt \bang
CO chromospheric scale height
\Df 100.}
\Na { 
\+{\bf SHCOP}&&&D, 1, \flpt \bang
CO photospheric scale height
\Df 400.}
\Na { 
\+{\bf SIK}&&\z &H, 2*,2, \flpt \bang
singly-ionized Silicon number density
\Df computed in LTE}
\Na { 
\+{\bf SIN}&&\z &H, 3*,3, \flpt \bang
Silicon-I level populations
\Df SIN$_{ij}$ computed in LTE, for all levels $j$ 
such that $j >$ NLS}
\Na { 
\+{\bf SK}&&\z &H, 2*,2, \flpt \bang
singly-ionized Sulphur number density
\Df computed in LTE}
\Na { 
\+{\bf SMATC}&&&D, 1, \flpt \bang
matrix samples output selection criterion}
\Na { 
\+{\bf SMOOTH}& {\bf *49}&&D  \bang
RHO smoothing control parameters
\Df see Note *49}
\Na { 
\+{\bf SMP}&&&D, 1, \flpt \bang
RHO weights adjustment parameter
\Df 0.3}
\Na { 
\+{\bf SN}&&\z &H, 3*,3, \flpt \bang
Sulphur-I level populations
\Df SIN$_{ij}$ computed in LTE, for all levels $j$ 
such that $j >$ NLU}
\Na { 
\+{\bf SN1CC}&&&D, 1, \flpt \bang
convergence criterion for ``Special N1'' calculation (diffusion)
\Df $10^{-8}$}
\Na { 
\+{\bf SOBDMN}& {\bf *51}&&D, 1, \flpt \bang
Sobolev escape probability calculation integration control parameter
\Df 0.001}
\Na { 
\+{\bf SOBDMX}& {\bf *51}&&D, 1, \flpt \bang
Sobolev escape probability calculation integration control parameter
\Df 0.01}
\Na { 
\+{\bf SOBFEQ}& {\bf *51}&&D, 1, \flpt \bang
Sobolev escape probability calculation integration control parameter
\Df 0.001}
\Na { 
\+{\bf SOBOLEV}& {\bf *51, *82}&[2]&D, 5, \intg \bang
Sobolev solution limit indices
\Df (1, 1)}
\Na {
\+{\bf SRCO}&&&D, 1, \flpt \bang
scattering ratio for CO lines
\Df 0.1}
\Na { 
\+{\bf STARKI}&&&D, 4, \flpt \bang
value of NE-index for Stark splitting of Hydrogen lines
\Df ISTARK}
\Na { 
\+{\bf TAUCL}& {\bf *42}&&D, 1, \flpt \bang
DR parameter, PRD transitions
\Df $10^4$}
\Na { 
\+{\bf TAUKIN}& {\bf *103}&[N]&D, 2*,2, \flpt \bang
input TAUK values}
\Na { 
\+{\bf TB}&&\z &D, 2*,2, \flpt \bang
blanketing temperature
\Df TE}
\Na { 
\+{\bf TBAR}&&&D, 1, \flpt \bang
change-over TAU-value for weight matrix calculation
\Df 0.5}
\Na { 
\+{\bf TDST}&&\z &D, 2*,2, \flpt \bang
Type-2 dust opacity calculation temperature values}
\Na { 
\+{\bf TDUST}&&&D, 1, \flpt \bang
Type-1 dust temperature
\Df 200.0}
\Na { 
\+{\bf TE}&&\z &D, 2*,2, \flpt \bang
kinetic temperature}
\Na { 
\+{\bf TER}& {\bf *93}&[NTE]&D, 2, \flpt \bang
temperature table for input values of CE and CI
\Df 4000.0}
\Na { 
\+{\bf TEX}&&\z &D, 2*,2, \flpt \bang
excitation temperature
\Df TE}
\Na { 
\+{\bf TGLYM}& {\bf *67}&&D, 1, \flpt \bang
Lyman change-over TAU parameter
\Df 100.0}
\Na { 
\+{\bf TKR}&&[LR]&D, 3, \flpt \bang
wavelengths for which additional photoionization values are specified}
\Na { 
\+{\bf TLARGE}&&&D, 1, \flpt \bang
TNP-from-TNU selection parameter
\Df 200.0}
\Na { 
\+{\bf TLIMG}& {\bf *34}&&D, 1, \flpt \bang
absorption contributors graph axis limit
\Df 0.0 [ = log(1.0)]}
\Na { 
\+{\bf TLTR}&&&D, 1, \flpt \bang
limiting multiplier for TDST recalculation
\Df 1.3}
\Na { 
\+{\bf TML}&&&D, 1, \flpt \bang
large-TAU cut-off for intensity integrals
\Df 30.0}
\Na { 
\+{\bf TMS}&&&D, 1, \flpt \bang
small-TAU change-over for RT weight matrices
\Df 5.0}
\Na { 
\+{\bf TOPE}& {\bf *84}&&D, 1, \intg \bang
Continuum Plots save switch}
\Na { 
\+{\bf TR}  \bang
= ``TRN 1 ''}
\Na {
\+{\bf TRFLI}&&&D, 1, \flpt \bang
limit-interval for TR-effective calculation
\Df 1.1}
\Na { 
\+{\bf TRN}& {\bf *9, *102}&\z &D, 3*,3, \flpt \bang
radiation temperature (for a given level of the ion of the run)
\Df TRN$_{i,1}$ = TE$_i, \quad$ TRN$_{i,j}$ = TRN$_{i,1}, \,$ all $i, \,$
all $j >$ 1}
\Na { 
\+{\bf TS}&&[M]&D, 2, \flpt \bang
standard TAU table
\Df (0.0, 0.0001, 0.0002, 0.0003, 0.0006, 0.001, 0.002, 0.003, 0.006,
0.01, 0.02, 0.03, 0.06, 0.1, 0.2, 0.3, 0.6, 1.0, 2.0, 3.0, 5.0, 7.0,
10.0, 15.0, 20.0, 30.0, 50.0, 75.0, 100.0, 150.0, 200.0, 300.0, 500.0)}
\Na { 
\+{\bf TSM}&&&D, 1, \flpt \bang
change-over TAU for mean intensity and emergent intensity calculations
\Df $10^{-4}$}
\Na { 
\+{\bf TSMALL}&&&D, 1, \flpt \bang
TNP-from-TNU selection parameter
\Df $10^{-10}$}
\Na { 
\+{\bf TX}&&&D, 1, \flpt \bang
brightness temperature of illuminating source}
\Na { 
\+{\bf USE}& {\bf *44}&&B,D,F,H, 1, \alfa \bang
input file designation
\Df ``INPUT''}
\Na { 
\+{\bf V}& {\bf *69, *82, *90}& $\qquad$ \z &D, 2*,2, \flpt \bang
broadening velocity}
\Na { 
\+{\bf VM}& {\bf *82}&\z &D, 2*,2, \flpt \bang
mass motion velocity}
\Na { 
\+{\bf VMNFE}&&&D, 1, \flpt \bang
minimum velocity (fast electrons)
\Df 0.003}
\Na {
\+{\bf VNH}& {\bf *90}&[NVH]&D, 1, \flpt \bang
standard table of V (as a function of HNDV) for the quiet sun
\Df (15.18, 13.68, 11.92, 10.75, 9.68, 8.44, 7.81, 7.52, 6.95,
6.28, 5.52, 4.60, 3.59, 2.98, 2.20, 2.00, 1.54, 1.38, 1.18,
1.00, 0.86, 0.80, 0.68, 0.65, 0.55, 0.52, 0.63, 0.90, 1.10,
1.30, 1.46, 1.56, 1.64, 1.71, 1.76, 1.80, 1.82, 1.83)}
\Na { 
\+{\bf VOITC}& {\bf *50}&&D, 1, \flpt \bang
Voigt function calculation cut-off
\Df $10^{-6}$}
\Na { 
\+{\bf VR}& {\bf *69, *82}&\z &D, 2*,2, \flpt \bang
broadening velocity}
\Na { 
\+{\bf VSB}& {\bf *51, *82}&\z &D, 2*,2, \flpt \bang
Sobolev velocity
\Df VXS or VM}
\Na { 
\+{\bf VSMLL}& {\bf *71}&&D, 1, \flpt \bang
replacement value for divisors that equal zero (subroutine
{\tt DIVIDE})
\Df $10^{-100}$}
\Na { 
\+{\bf VT}& {\bf *82}&\z &D, 2*,2, \flpt \bang
turbulent pressure velocity}
\Na { 
\+{\bf VX}& {\bf *57, *82}&[N]&F, 3, \flpt \bang
additional expansion velocity for emergent profile calculation}
\Na { 
\+{\bf VXS}& {\bf *60, *82}&\z &D, 2*,2, \flpt \bang
basic expansion velocity for Line Source Function calculations
\Df VM}
\Na { 
\+{\bf WAVEMN}& {\bf *110}&&D, 1, \flpt \bang
automatic additional wavelengths limit}
\Na { 
\+{\bf WAVEMX}& {\bf *110}&&D, 1, \flpt \bang
automatic additional wavelengths limit}
\Na { 
\+{\bf WAVES}& {\bf *27}&[NWV]&D, 2, \flpt \bang
additional wavelengths for continuum calculations}
\Na { 
\+{\bf WBD}& {\bf *26}&&D, 1, \flpt \bang
weight for departure coefficient updating
\Df WPOP}
\Na { 
\+{\bf WBDIR}&&&D, 1, \flpt \bang
weight for results of ``direct'' departure coefficient calculation
\Df 1.0}
\Na { 
\+{\bf WEIGHT}&&&D, 6, \flpt \bang
weight for Statistical Equilibrium equations calculation}
\Na { 
\+{\bf WEP}& {\bf *104}&&D, 1, \flpt \bang
Lyman EP-1 weighting parameter
\Df 1.0}
\Na {
\+{\bf WFB}&&&F, 1, \flpt \bang
weight for flow broadening component velocities
\Df 0.6}
\Na { 
\+{\bf WMN}& {\bf *131}&&D, 1, \flpt \bang
RHO weight adjustment parameter
\Df 0.3}
\Na { 
\+{\bf WMX}& {\bf *131}&&D, 1, \flpt \bang
RHO weight adjustment parameter
\Df 0.9}
\Na { 
\+{\bf WNJUNK}& {\bf *87}&&D, 1, \flpt \bang
WN-matrix ``cleanup'' parameter}
\Na { 
\+{\bf WNU}&&[NSL]&D, 2, \flpt \bang
= NU, but in wavenumbers}
\Na { 
\+{\bf WNUC}&&[NSL]&D, 2, \flpt \bang
= NUC, but in wavenumbers}
\Na { 
\+{\bf WNUK}&&&D, 1, \flpt \bang
= NUK, but in wavenumbers}
\Na {
\+{\bf WORLDLY}& {\bf *132}&&B, 1, \alfa \bang
storage management dump control switch}
\Na { 
\+{\bf WPOP}& {\bf *26}&&D, 1, \flpt \bang
weight for number densities updating
\Df WBD, or 1.0}
\Na { 
\+{\bf WPRESS}&&&D, 1, \flpt \bang
weight for adjusting NH to achieve constant pressure
\Df 1.0}
\Na { 
\+{\bf WR}& {\bf *13}&&D, 4, \flpt \bang
RHO weighting parameter
\Df 1.0}
\Na { 
\+{\bf WRAT}& {\bf *22}, {\bf *56}&[MR]&D, 3, \flpt \bang
wavelengths for rates integrations}
\Na { 
\+{\bf WRATMN}& {\bf *135}&&D, 1, \flpt \bang
standard rates integrations wavelengths table limit
\Df 100.}
\Na { 
\+{\bf WRATMX}& {\bf *135}&&D, 1, \flpt \bang
standard rates integrations wavelengths table limit
\Df 20000.}
\Na { 
\+{\bf WRMN}& {\bf *131}&&D, 1, \flpt \bang
RHO weight adjustment parameter
\Df 0.1}
\Na { 
\+{\bf WRMX}& {\bf *131}&&D, 1, \flpt \bang
RHO weight adjustment parameter
\Df 0.7}
\Na { 
\+{\bf WSM}&&&D, 1, \flpt \bang
Lyman RK-KOLEV smoothing parameter}
\Na { 
\+{\bf WSN1D}&&&D, 1, \flpt \bang
weight for Special N1 and Special NK, Diffusion
\Df 1.0}
\Na { 
\+{\bf WTD}&&&D, 1, \flpt \bang
weight for TDST recalculation
\Df 1.0}
\Na { 
\+{\bf WZ}&&&D, 1, \flpt \bang
Z-from-TAUKIN weight
\Df 0.5}
\Na { 
\+{\bf WZM}& {\bf *103}&&D, 1, \flpt \bang
Z weighting parameter
\Df 0.8}
\Na { 
\+{\bf XC}& {\bf *42}&&D, 4, \flpt \bang
DR parameter, PRD transitions
\Df 2.0}
\Na { 
\+{\bf XCL}& {\bf *42}&&D, 1, \flpt \bang
DR parameter, PRD transitions
\Df 3.5}
\Na { 
\+{\bf XCOL}&&[NCL]&D, 2, \flpt \bang
table of wavelengths for {\it each} CO line, in doppler widths,
for the CO-lines opacity
\Df (0.0, 0.5, 1.0, 1.5, 2.0)}
\Na { 
\+{\bf XCOMX}&&&D, 1, \flpt \bang
width limit for {\it each} CO line, in \AA, for the CO-lines
opacity
\Df 3.0}
\Na { 
\+{\bf XDR}& {\bf *42}&[NDR]&D, 2, \flpt \bang
DR parameter, PRD transitions
\Df (5.5, 6.0, 7.0, 8.0, 10.0, 12.0, 15.0)}
\Na { 
\+{\bf XI}&&&D, 2, \flpt \bang
= XISYM}
\Na { 
\+{\bf XIBLU}& {\bf *66}&[KB]&D, 2, \flpt \bang
Line Transitions frequency table, for blue side
\Df XISYM}
\Na { 
\+{\bf XIBLUT}& {\bf *66}&[KBT]&D, 5, \flpt \bang
Line Transition frequency table, for blue side
\Df XISYMT}
\Na { 
\+{\bf XINK}&&[INK]&D, 2, \flpt \bang
table of frequencies for which incident radiation is specified
\Df (5.948, 5.949, 7.48, 7.51, 13.16, 13.17)}
\Na { 
\+{\bf XIRED}& {\bf *66}&[KR]&D, 2, \flpt \bang
Line Transitions frequency table, for red side
\Df XISYM}
\Na { 
\+{\bf XIREDT}& {\bf *66}&[KRT]&D, 5, \flpt \bang
Line Transition frequency table, for red side
\Df XISYMT}
\Na { 
\+{\bf XISYM}& {\bf *66}&[KS]&D, 2, \flpt \bang
Line Transitions frequency table, for half of a symmetric profile
\Df (0.0, 0.1, 0.2, 0.3, 0.4, 0.5, 0.6, 0.7, 0.8, 0.9, 1.1, 1.3, 1.5,
1.8, 2.1, 2.4, 2.8, 3.4, 3.9, 4.5, 6.0, 8.0, 15.0, 50.0)}
\Na { 
\+{\bf XISYMT}& {\bf *66}&[KST]&D, 5, \flpt \bang
Line Transition frequency table, for half of a symmetric profile
\Df XISYM}
\Na { 
\+{\bf XJFE}&&&D, 1, \flpt \bang
electron flux (fast electrons)
\Df $2.0\times10^{-10}$}
\Na { 
\+{\bf XK}& {\bf *123}&[KK]&D, 2, \flpt \bang
Level-$\cal N$-to-Continuum calculation (Lyman) frequency table
\Df RRNU(KOLEV), where RRNU is computed from WRAT}
\Na { 
\+{\bf XMU}&&[LG]&D, 1, \flpt \bang
MU table for GR-method weight matrix calculation
\Df (1.0, 0.8, 0.6, 0.5, 0.4, 0.3, 0.2, 0.1)}
\Na { 
\+{\bf XP}& {\bf *42}&&D, 4, \flpt \bang
DR parameter, PRD transitions
\Df 2.0}
\Na { 
\+{\bf XR}& {\bf *42}&&D, 4, \flpt \bang
DR parameter, PRD transitions
\Df -1.0}
\Na { 
\+{\bf XQMAX}&&&D, 1, \flpt \bang
parameter for injection function integrations (fast electrons)
\Df 200.0}
\Na {
\+{\bf XRKH}& {\bf *95, *96}& \z &D, 5,5*, \flpt \bang
upper-level charge-exchange data for Hydrogen}
\Na {
\+{\bf XRLH}& {\bf *95, *96}& \z &D, 5,5*, \flpt \bang
upper-level charge-exchange data for Hydrogen}
\Na { 
\+{\bf Y}&&&D, 1, \flpt \bang
damping parameter for frequency integrations weights (see Section 12)
\Df 0.5}
\Na { 
\+{\bf YCOL}&&&D, 1, \flpt \bang
weight matrix method control parameter for the continuum calculations
\break required for the CO-lines opacity (see Section 12)
\Df -1.0}
\Na { 
\+{\bf YCONT}&&&D, 4, \flpt \bang
weight matrix method parameter, for Continuum calculations (see Section 12)
\Df -1.0}
\Na { 
\+{\bf YCR}&&[NCR]&D, 2, \flpt \bang
weight matrix method control parameter, for incident coronal radiation (see
Section 12)
\Df -1.0}
\Na { 
\+{\bf YFLUX}&&&D, 1, \flpt \bang
damping parameter for emergent continuum flux
\Df 0.5}
\Na { 
\+{\bf YH}&&&D, 1, \flpt \bang
Helium-to-Hydrogen ratio
\Df Helium abundance (from ELE table)}
\Na { 
\+{\bf YHM}&&[MHM]&D, 2, \flpt \bang
weight matrix method parameter, for H-minus calculation (see Section 12)
\Df -1.0}
\Na { 
\+{\bf YK}&&[NSL]&D, 2, \flpt \bang
Hydrogen recombination parameter}
\Na { 
\+{\bf YKR}&&[LR]&D, 3, \flpt \bang
weight matrix method parameter, for additional photoionization (see \break
Section 12)
\Df -1.0}
\Na { 
\+{\bf YL}&&&D, 1, \flpt \bang
weight matrix parameter, for Level-$\cal N$-to-Continuum source function
calculation (see Section 12)
\Df -1.0}
\Na { 
\+{\bf YLDT}&&[NDT]&D, 2, \flpt \bang
weight matrix method parameter, for Type-2 dust opacity calculation (see
Section 12)
\Df -1.0}
\Na { 
\+{\bf YLINE}&&&D, 4, \flpt \bang
weight matrix method parameter, for Line Source Function (see Section 12)
\Df -1.0}
\Na { 
\+{\bf YLYM}&&[KK]&D, 2, \flpt \bang
weight matrix parameter, for Level-$\cal N$-to-Continuum continuum source
function calculations (see Section 12)
\Df -1.0}
\Na { 
\+{\bf YPRE}&&&D, 1, \flpt \bang
damping parameter for standard weight matrix (see Section 12)}
\Na { 
\+{\bf YRATE}& {\bf *22}&[MR+1]&D, 3, \flpt \bang
weight matrix method parameter, for rates calculations (see Section 12)
\Df -1.0}
\Na { 
\+{\bf YRATS}&&&D, 1, \flpt \bang
damping parameter for standard rates integrations wavelengths (see Section 12)
\Df -1.0}
\Na { 
\+{\bf YWAVE}&&[NWV]&D, 2, \flpt \bang
weight matrix method parameter, for additional calculations (see Section 12)
\Df -1.0}
\Na { 
\+{\bf Z}&&[N]&D, 2, \flpt \bang
grid of geometrical depths (main Z-table of the run)}
\Na { 
\+{\bf ZALBK}&&[NKA]&D, 1, \flpt \bang
scattering albedo parameter for Background Line Opacities (see Section 9)
\Df (Z$_1$, Z$_N$)}
\Na { 
\+{\bf ZAUX}&&[LZA]&D,F,H, 3, \flpt \bang
auxiliary Z-table}
\Na { 
\+{\bf ZECL}&&[NZE]&F, 2, \flpt \bang
selected Z-values for eclipse continuum calculation}
\Na {
\+{\bf ZGM}& {\bf *138}&[NGM]&D, 2, \flpt \bang
Z-table for DGMZ
\Df (-2000, -1900, -1800, -1700, -1600, -1500, -1400, -1300,
-1200, -1100, -1000, -900, -800, -700, -600, -500, -400, -300,
-200, -100, -50, 0, 50)}
\Na { 
\+{\bf ZMASS}& {\bf *103}&[N]&D, 2, \flpt \bang
gas column mass}
\Na { 
\+{\bf ZME}&&\z &D, 2*,2, \flpt \bang
non-H electron ratio}
\Na { 
\+{\bf ZNDW}& {\bf *86}&&D, 1, \flpt \bang
Z-value for optional NDW-default calculation}
\Na { 
\+{\bf ZRCO}&&&D, 1, \flpt \bang
CO reference height
\Df -500.}
\Na { 
\+{\bf ZXMIN}&&&D, 1, \flpt \bang
diffusion calculation parameter for ZION
\Df 0.1}
\par\vfill
%\end
