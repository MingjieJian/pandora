%\magnification=1200
%\input wupstuff.tex
\newtoks\footline \footline={\hss\tenrm 6.\folio\hss}
\pageno=1
\def\bang{\par \hangindent=10pt \hangafter=0}
\top
\vskip 1.5 true in
\centerline{Section 6: {\bf Program Options}}
\blankline
\blankline
\centerline{\bf ***}
\blankline
\blankline
Many features of PANDORA may be enabled or disabled, as desired. Some of
these optional features are controlled by assigning particular values to
certain input parameters; all others are controlled through the following
two input statements: \par
\centerline{``{\bf DO ( } OPTION$_1$ OPTION$_2$ OPTION$_3$ 
$\ldots$ OPTION$_n$ {\bf ) }'',}
\noindent {\it i.e.} {\tt DO ( ADDCOPR EPSW ) }, and \par
\centerline{``{\bf OMIT ( } OPTION$_1$ OPTION$_2$ OPTION$_3$ 
$\ldots$ OPTION$_n$ {\bf ) }'',}
\noindent {\it i.e.} {\tt OMIT ( FELE SCOPRNT QSFEDIT ) }, \par
\noindent where OPTION$_i$ is the {\alfa} code name, to be described below,
of a particular processing or output feature of PANDORA. Options appearing
in {\bf DO} statements will be enabled, those appearing in {\bf OMIT}
statements will be disabled. Only the OPTIONs appearing in the following
table are valid. There may be any number of {\bf DO} and/or {\bf OMIT}
statements, and a particular OPTION may appear several times --
however, only the status resulting from its last appearance will prevail.
Every OPTION not mentioned in {\bf DO} and/or {\bf OMIT} statements
will retain its default setting.

All OPTIONS are listed and defined on the following table. This list is
in alphabetical order by option name; the default settings are given
in parentheses. After this table of definitions there follows an
alphabetized listing of keywords and descriptive phrases for each
OPTION -- this last list should be consulted when an OPTION's significance
or function are only vaguely known, and its name is sought. After
the OPTION name has been located in the keywords list, the complete
definition can then be found in the first part of this section.
\blankline
\blankline
\centerline{\bf See next page for important information about}
\centerline{\bf automatic options adjustments.}
\ej
\top
\vskip 2 in
\centerline{\bf Note:}
\blankline
After all {\bf DO} and/or {\bf OMIT} statements have been read, and before
the \break printout OPTIONS is produced, some options are reset automatically
as follows, in the sequence shown:
\bull If PHASE2 is off: LIGHT, ECLIPSE and EMERINT are turned off;
\bull If FINITE is off: REFLECT is turned off;
\bull If FINITE and INCRFRNT are both off: INCIDNT is turned off;
\bull If SPHERE is on: SPHOUT, GDS, ENHANCE, HSE, INCIDNT and
\break INCIFRNT are turned off;
\bull If USETRIN is off: CSWITCH is turned off;
\bull If ORT is on: GDS is turned off;
\bull If SPECSAV is on: CONSAV and PROSAV are turned on.
\bull If SPHERE and PRODMP are both on: ECLIDMP and LINTDMP are turned on.
\ej
\parindent=0pt
\space \vbox {\noindent {\bf  ACSFPRNT} (off): \bang
print abbreviated results of continuum source function calculation
(used only if CSFPRNT is on).}
\space \vbox {\noindent {\bf  ADDCOPR} (off): \bang
print continuum data for additional wavelengths.}
\space \vbox {\noindent {\bf  ADN1DMP} (off): \bang
print details of ``sepcial N1'' calculation (used only when ADN1PRNT
is on).}
\space \vbox {\noindent {\bf  ADN1PRNT} (on): \bang
print ``special N1'' calculation results (used only if AMBPRNT is on).}
\space \vbox {\noindent {\bf  AEDIT} (off): \bang
replace negative final A-values (frequency integration weights) with zero.}
\space \vbox {\noindent {\bf  AHSEPRNT} (off): \bang
print abbreviated results of HSE calculation.}
\space \vbox {\noindent {\bf  AINDPRNT} (off): \bang
print abbreviated version of INPUT listing (used only when INDAPRNT
is on).}
\space \vbox {\noindent {\bf  AINTPRNT} (off): \bang
print abbreviated frequency integrations data.}
\space \vbox {\noindent {\bf  ALL} (off): \bang 
provide complete printout for all overall iterations,
not just the last one.}
\space \vbox {\noindent {\bf  ALLY} (off): \bang 
provide complete Lyman and HSE printouts for every
iteration, not just the last one.}
\space \vbox {\noindent {\bf  ALSFPRNT} (off): \bang
print abbreviated results of Line Source Function calculation.}
\space \vbox {\noindent {\bf  ALUPRNT} (off): \bang 
print Aluminum populations and departure coefficients.}
\space \vbox {\noindent {\bf  ALYMPRNT} (off): \bang
print abbreviated results of Level-${\cal N}$-to-Continuum transfer
calculation.}
\space \vbox {\noindent {\bf  AMBPRNT} (on): \bang
print data from ambipolar diffusion calculations, (used only
when AMDIFF is on).}
\space \vbox {\noindent {\bf  AMDDMP} (off): \bang
print details of ambipolar diffusion calculations, (used only when
AMBPRNT is on).}
\space \vbox {\noindent {\bf  AMDIFF} (off): \bang
include ambipolar diffusion (only in Hydrogen or Helium runs).}
\space \vbox {\noindent {\bf  AMDN1} (on): \bang
use ``special N1'' calculation for ambipolar diffsuion, (used only
when AMDIFF is on). Caution: AMDN1 should {\bf not} be turned off
(this switch is provided for testing only).}
\space \vbox {\noindent {\bf  ANALYSIS} (off): \bang 
print details of line absorption profile calculations (note input parameters
\break NANAL1, NANAL2).}
\space \vbox {\noindent {\bf  AOPTPRNT} (off): \bang
print abbreviated version of OPTIONS listing (used only when OPTPRNT is on).}
\space \vbox {\noindent {\bf  APHICOPR} (on): \bang
print continuum data for additional photoionization wavelengths.}
\space \vbox {\noindent {\bf  APHIPRNT} (off): \bang
print results and details of generalized additional photoionization
calculation.}
\space \vbox {\noindent {\bf  APOPPRNT} (off): \bang
print abbreviated populations of the ion-of-the-run.}
\space \vbox {\noindent {\bf  APRFPRNT} (off): \bang
print abbreviated line profile calculation results.}
\space \vbox {\noindent {\bf  ARHODMP} (off): \bang
print detailed sets of `transition terms' (uses input parameter LDINT).}
\space \vbox {\noindent {\bf  ATMOPRNT} (on): \bang
print ATMOSPHERE input data listing.}
\space \vbox {\noindent {\bf  ATOMPRNT} (on): \bang
print ATOM input data listing.}
\space \vbox {\noindent {\bf  ATOMSAV} (off): \bang
save computed default atomic model parameters in output file.}
\space \vbox {\noindent {\bf AVCON } (off): \bang
calculate the average of the continuum intensity.}
\space \vbox {\noindent {\bf AVELOP } (off): \bang
use Averaged Line opacities.}
\space \vbox {\noindent {\bf AVOPRNT } (on): \bang
print Averaged Line opacities data table as used in this run (also uses
the value of LWNT).}
\space \vbox {\noindent {\bf  BDCALC} (off): \bang
departure coefficients calculated from all level equations, rather than
from the continuum equation.}
\space \vbox {\noindent {\bf  BDGRAF} (on): \bang
print graphs of departure coefficients.}
\space \vbox {\noindent {\bf  BDMP} (off): \bang
print details of departure coefficients calculation.}
\space \vbox {\noindent {\bf  BDPRNT} (off): \bang
print complete sets of BDR, BDJ, BDS, and S* from `RHO + RBD' calculation 
(only if RHBPRNT is on).}
\space \vbox {\noindent {\bf  BEDIT} (off): \bang 
edit departure coefficients.}
\space \vbox {\noindent {\bf  BLENDMP} (off): \bang
print Voigt function details for each blended line component,
(used only when ANALYSIS is on).}
\space \vbox {\noindent {\bf  BRATDMP} (off): \bang
print details of b-ratios calculation (uses input parameter LDINT).}
\space \vbox {\noindent {\bf  BSMOOTH} (off): \bang
do ``alternate'' sequential smoothing for departure coefficients
(note input parameter ASMCR).}
\space \vbox {\noindent {\bf  CALCOOL} (off): \bang
calculate net cooling rates.}
\space \vbox {\noindent {\bf  CALHEAT} (off): \bang
calculate net heating rates.}
\space \vbox {\noindent {\bf  CALPRNT} (off): \bang
print Calcium populations and departure coefficients.}
\space \vbox {\noindent {\bf  CARPRNT} (off): \bang 
print Carbon populations and departure coefficients.}
\space \vbox {\noindent {\bf  CEFACTS} (off): \bang
update CE-enhancement factors where possible and needed.}
\space \vbox {\noindent {\bf  CHEXLO} (off): \bang 
use lower-level charge exchange.}
\space \vbox {\noindent {\bf  CHEXLOL} (off): \bang
use LTE hydrogen number density for lower-level charge exchange.}
\space \vbox {\noindent {\bf  CHEXUP} (off): \bang 
use upper-level charge exchange.}
\space \vbox {\noindent {\bf  CHKGRAF} (on): \bang 
print graphs of CHECKs for all iterations.}
\space \vbox {\noindent {\bf  CHKPRNT} (on): \bang 
print Consistency CHECKs from `RHO + RBD' calculation.}
\space \vbox {\noindent {\bf  CHXDMP} (off): \bang 
print details of upper-level charge-exchange calculation.}
\space \vbox {\noindent {\bf  CHXPRNT} (on): \bang 
print results of upper-level charge-exchange calculation
(used only when CHXCNG is on).}
\space \vbox {\noindent {\bf  CIJPRNT} (off): \bang 
print the bound-bound collision rates CIJ (for `minimal' printout,
see input parameter IRATE).}
\space \vbox {\noindent {\bf  CLNORM} (on): \bang
compute H Lyman lines normalization factors.}
\space \vbox {\noindent {\bf  CNFLXDMP} (off): \bang
print debug dump output from continuum flux calculation.}
\space \vbox {\noindent {\bf COCLIPSE } (off): \bang
calculate emergent continuum eclipse intensities for CO lines wavelengths.}
\space \vbox {\noindent {\bf  COCOPR} (off): \bang
print continuum data for CO-lines opacity wavelengths.}
\space \vbox {\noindent {\bf  COCRID} (off): \bang
print details of CO-lines cooling rate wavelength integration.}
\space \vbox {\noindent {\bf  CODMP} (off): \bang
print details of CO-lines absorption calculation.}
\space \vbox {\noindent {\bf  COLHPRNT} (on): \bang
print calculated rates for collisions with Hydrogen atoms.}
\space \vbox {\noindent {\bf  COLTEMP} (off): \bang
calculate color temperatures.}
\space \vbox {\noindent {\bf  COMCRID} (off): \bang
print details of Composite Lines cooling rates wavelength integration.}
\space \vbox {\noindent {\bf  COMOPAN} (off): \bang
Composite Line Opacity analysis.}
\space \vbox {\noindent {\bf  COMPCOPR} (off): \bang
print continuum data for Composite Line Opacity wavelengths.}
\space \vbox {\noindent {\bf  COMPRK} (off): \bang
print comparison RKs.}
\space \vbox {\noindent {\bf  CONFLUX} (off): \bang
calculate continuum flux.}
\space \vbox {\noindent {\bf  CONSAV} (off): \bang 
write continuum spectrum data in Special Spectrum Save File,
(this option will be turned on automatically when SPECSAV is on).}
\space \vbox {\noindent {\bf  COOLCO} (off): \bang
calculate CO-lines cooling rate, (used only when CALCOOL is on).}
\space \vbox {\noindent {\bf  COOLCOM} (on): \bang
calculate Composite Lines cooling rate, (used only when CALCOOL is on).}
\space \vbox {\noindent {\bf  COOLINT} (off): \bang
calculate integrated net heating and cooling rates, (used only when
CALCOOL is on).}
\space \vbox {\noindent {\bf  COOLXRAY} (on): \bang
calculate X-rays cooling rate, (used only when CALCOOL is on).}
\space \vbox {\noindent {\bf  COOLSAV} (off): \bang
write cooling and heating rates in a save file.}
\space \vbox {\noindent {\bf  CPSW} (on): \bang
adjust total Hydrogen density to give constant pressure, (used only when
HSE is off).}
\space \vbox {\noindent {\bf  CSF} (on): \bang
use calculated background continuum source function when calculating line
source function.}
\space \vbox {\noindent {\bf  CSFB} (off): \bang
set BC in line source function calculation = min(CSF, B), instead of = CSF,
(used only when CSF is on).}
\space \vbox {\noindent {\bf  CSFDMP} (off): \bang
print debug dump output for background continuum source function calculation.}
\space \vbox {\noindent {\bf  CSFPRNT} (on): \bang
print results from background continuum source function calculation.}
\space \vbox {\noindent {\bf  CSFGRAF} (off): \bang
print graphs of background continuum opacity and background continuum source
function.}
\space \vbox {\noindent {\bf  CSWITCH} (off): \bang
use TR (level 1), instead of TE, in the stimulated emission factor (BETA) for BC in the
line source function calculation, (this option  will be turned off automatically
when USETRIN is off).}
\space \vbox {\noindent {\bf DELABORT} (on): \bang
try to read all the input before stopping because of error(s).}
\space \vbox {\noindent {\bf  DIDHC} (off): \bang
print dI/dh for emergent continuum intensities (see also input parameter 
\break ICDIT).}
\space \vbox {\noindent {\bf  DIDHL} (on): \bang
print and save dI/dh for emergent line intensity profiles.}
\space \vbox {\noindent {\bf DIFFANA} (on): \bang
analyze results of diffusion calculations (used only if AMDIFF is on and/or
VELGRAD is on) for radiative CRD transitions.}
\space \vbox {\noindent {\bf  DOION} (on): \bang
do all the normal calculations pertaining to the ion-of-the-run.}
\space \vbox {\noindent {\bf  DPDWPRNT} (off): \bang
print results of Doppler Width and Damping Paramater calculations.}
\space \vbox {\noindent {\bf  DRDMP} (off): \bang
print debug details of PRD DR calculation for transition (MS/NS) at depth
IDRDP and frequency KDRDP.}
\space \vbox {\noindent {\bf  DSMOOTH} (off): \bang
smooth the calculated diffusion terms with the sequential
smoothing procedure.}
\space \vbox {\noindent {\bf  DUSTCOPR} (on): \bang
print continuum data for Dust opacity wavelengths, (used only when 
\break DUSTEMP is on).}
\space \vbox {\noindent {\bf  DUSTDMP} (off): \bang
print debug dump output from Type-2 Dust temperature recalculation.}
\space \vbox {\noindent {\bf  DUSTEMP} (on): \bang
calculate new values of TDST, (used only when DUSTYPE is on).}
\space \vbox {\noindent {\bf  DUSTYPE} (off): \bang
use Type-2 version of Dust background continuum opacity.}
\space \vbox {\noindent {\bf  ECLIDMP} (off): \bang
print debug dump output from Eclipse calculation.}
\space \vbox {\noindent {\bf  ECLIGRAF} (on): \bang
print graphs of results of Line Eclipse calculation.}
\space \vbox {\noindent {\bf  ECLIPSE} (off): \bang
calculate emergent continuum Eclipse intensities, (pertains to
WAVES$_i < 0$ only).}
\space \vbox {\noindent {\bf  ECLISAV} (off): \bang
write Eclipse emission data to an output file.}
\space \vbox {\noindent {\bf  ELECPRNT} (off): \bang
print and plot results of electron density calculations.}
\space \vbox {\noindent {\bf  EMERBACK} (off): \bang
calculate intensity and flux emerging from back face (far face) of finite 
atmosphere.}
\space \vbox {\noindent {\bf  EMERINT} (on): \bang
calculate emergent continuum intensities for all wavelengths for which
background continuum calculations were done.}
\space \vbox {\noindent {\bf  EMIGRAF} (off): \bang
print graph of emitters.}
\space \vbox {\noindent {\bf  EMINDMP} (off): \bang
print debug dump output from emergent continuum intensity calculations.}
\space \vbox {\noindent {\bf  EMIPRNT} (off): \bang
print detailed analysis of the contributions to the background continuum
emission.}
\space \vbox {\noindent {\bf  EMISUM} (off): \bang
print an `Emitters Summary'.}
\space \vbox {\noindent {\bf  ENHANCE} (off): \bang
use R$^2$ source function enhancement factor in emergent intensity 
calculations, (used only when SPHERE is off).}
\space \vbox {\noindent {\bf  ENL} (off): \bang
edit negative values out of tables of EP1 (Lyman Epsilon-1) by 
interpolating from neighboring positive values.}
\space \vbox {\noindent {\bf  ENL2} (off): \bang
edit negative values out of tables of EP1 (Lyman Epsilon-1) by 
shifting the \break entire EP1 and EP2 tables.}
\space \vbox {\noindent {\bf  EPCOMP} (off): \bang
print comparison of the results obtained from the various methods for the
\break Lyman EP1, EP2 calculations.}
\space \vbox {\noindent {\bf  EPDMP} (on): \bang
print intermediate details of the Lyman EP1, EP2 calculations,
(also uses the value of LDINT).}
\space \vbox {\noindent {\bf  EPSNEDC} (off): \bang
eliminate negative values of EP1 (Lyman Epsilon-1) when using the CHAIN
method.}
\space \vbox {\noindent {\bf  EPSW} (off): \bang
replace any line-source-function-epsilons $< -0.9999$ by $-0.9999$.}
\space \vbox {\noindent {\bf  EVERY} (off): \bang
print complete printout for all sub-iterations, not just the last one.}
\space \vbox {\noindent {\bf  EXPAND} (off): \bang
the atmosphere is expanding.}
\space \vbox {\noindent {\bf  FDBCOPR} (off): \bang
print continuum data for FDB (frequency-dependent line source function
background) wavelengths.}
\space \vbox {\noindent {\bf  FDBDMP} (off): \bang
print frequency-dependent background terms for FDB transitions.}
\space \vbox {\noindent {\bf  FELE} (off): \bang
use results from fast electrons calculation.}
\space \vbox {\noindent {\bf  FELEC} (off): \bang
calculate fast electrons.}
\space \vbox {\noindent {\bf  FELEDMP} (off): \bang
print debug dump output from fast electrons calculation.}
\space \vbox {\noindent {\bf  FELEPRNT} (on): \bang
print results of fast electrons calculations.}
\space \vbox {\noindent {\bf  FEPRNT} (off): \bang
print Iron populations and departure coefficients.}
\space \vbox {\noindent {\bf  FINITE} (off): \bang
the atmosphere is finite, instead of semi-infinite.}
\space \vbox {\noindent {\bf  FLUXDMP} (off): \bang
print details of emergent continuum flux calculation at each wavelength, (also
uses the value of IFXDS).}
\space \vbox {\noindent {\bf  FLUXSAV} (off): \bang
write integrated flux quantities to an output file.}
\space \vbox {\noindent {\bf  FLWBPRNT} (off): \bang
print component profiles of flow broadening.}
\space \vbox {\noindent {\bf  FLWBROAD} (off): \bang
compute flow-broadened profiles, (used only when EXPAND is off).}
\space \vbox {\noindent {\bf  GDMP} (on): \bang
print debug dump consisting of all `geometric dilution matrices', (used
only when GDS is on).}
\space \vbox {\noindent {\bf  GDS} (off): \bang
calculate geometric dilution.}
\space \vbox {\noindent {\bf  GDSDMP} (off): \bang
print geometric dilution terms, (used only when GDS is on).}
\space \vbox {\noindent {\bf  GNVCALC} (off): \bang
compute the diffusion term GNV1 from the non-local N1 calculation, (used
only when AMDIFF is on).}
\space \vbox {\noindent {\bf  GTNSMTH} (on): \bang
smooth STIM when calculating GTN(u,l).}
\space \vbox {\noindent {\bf  GTNSTIM} (off): \bang
use departure coefficients, instead of number densities, in STIM
for GTN(u,l).}
\space \vbox {\noindent {\bf  HBROAD} (on): \bang
in a Hydrogen run, use ion collision broadening for transitions
above level 5.}
\space \vbox {\noindent {\bf  HEABD} (off): \bang 
calculate depth dependence of Helium abundance (used only when
AMDIFF is on).}
\space \vbox {\noindent {\bf  HELPRNT} (off): \bang
print Helium populations and departure coefficients.}
\space \vbox {\noindent {\bf  HEL2PRNT} (off): \bang
print Helium-II populations and departure coefficients.}
\space \vbox {\noindent {\bf  HENORM} (on): \bang
renormalize Helium number densities in the diffusion calculations.}
\space \vbox {\noindent {\bf  HFFCOOLD} (off): \bang
print details of H free-free and H-minus free-free net cooling rates
integrations.}
\space \vbox {\noindent {\bf  HMS} (off): \bang
calculate H-minus departure coefficient.}
\space \vbox {\noindent {\bf  HMSCOPR} (off): \bang
print continuum data for H-minus departure coefficient calculation
wavelengths.}
\space \vbox {\noindent {\bf  HMSJPRNT} (off): \bang
print values of mean intensity used in H-minus departure coefficient
calculation.}
\space \vbox {\noindent {\bf  HMSONLY} (off): \bang
use only continuum data pertaining to `H-minus wavelengths' in the
H-minus departure coefficient calculation.}
\space \vbox {\noindent {\bf  HNPRNT} (on): \bang
print Hydrogen populations and departure coefficients.}
\space \vbox {\noindent {\bf  HSE} (off): \bang
recalculate NE and NH via the hydrostatic equilibrium equation.}
\space \vbox {\noindent {\bf  HSEDMP} (off): \bang
print debug dump output from the hydrostatic equlibrium calculation,
(used only when HSE is on).}
\space \vbox {\noindent {\bf  HSEV} (on): \bang
include mass motion velocity in the hydrostatic equilibrium calculations, 
(used only when HSE is on).}
\space \vbox {\noindent {\bf  HSTSUMM} (on): \bang
print summary of calculation of Stark splitting of Hydrogen lines.}
\space \vbox {\noindent {\bf  ILR} (on): \bang
calculate incident line radiation terms, (used only when INCIDNT is on).}
\space \vbox {\noindent {\bf  INBED} (off): \bang
edit negative values out of tables of BDIJ values calculated from
unfudged \break input RHO values.}
\space \vbox {\noindent {\bf  INCIDNT} (off): \bang
there is external radiation shining upon the atmosphere, and it is
shining on the back face (far face).}
\space \vbox {\noindent {\bf  INCIFRNT} (off): \bang
the external incident radiation is shining upon the front face of the
atmosphere, (used only when INCIDNT is on).}
\space \vbox {\noindent {\bf  INDAPRNT} (on): \bang
print miscellaneous INPUT data listing.}
\space \vbox {\noindent {\bf  INDPRNT} (off): \bang
print the input values of number density and departure coefficient of the
ion of the run.}
\space \vbox {\noindent {\bf  INDXDMP} (off): \bang
print debug dumps of the contents of the random-access scratch file indices.}
\space \vbox {\noindent {\bf  INPEX} (on): \bang
extrapolate input tables to added depths, instead of just extending the end
values.}
\space \vbox {\noindent {\bf  INNBPRNT} (on): \bang
print input values of NK, ND and BD.}
\space \vbox {\noindent {\bf  INPEXW} (on): \bang
print input table extrapolation warning messages.}
\space \vbox {\noindent {\bf  INSCARD} (on): \bang
write the current iterates to the restart files for every iteration, not just
for the last iteration only. (Note: the results for iteration $i+1$ will
overwrite the results for iteration $i$.)}
\space \vbox {\noindent {\bf  INTAPRNT} (on): \bang
print frequency integrations data for all radiative transitions.}
\space \vbox {\noindent {\bf  INTEDIT} (off): \bang
use edited TE values instead of input.}
\space \vbox {\noindent {\bf  INTGRAF} (on): \bang
print graphs of the results of emergent line profile calculations, and graphs
of S and B {\it vs.} Z with the emergent line profile calculation.}
\space \vbox {\noindent {\bf  INTRPRNT} (on) \bang
print input values of RHO, JBAR, CHI and AW.}
\space \vbox {\noindent {\bf  IRHWED} (on): \bang
edit input values of RHOWT.}
\space \vbox {\noindent {\bf  IRUNT} (off): \bang
print the most detailed execution performance and version description data.}
\space \vbox {\noindent {\bf  ISCRS} (on): \bang
use `in-memory' scratch I/O to the extent possible.}
\space \vbox {\noindent {\bf  ITDMP} (off): \bang
print debug dump of `Iterative Summary File' contents summary.}
\space \vbox {\noindent {\bf  ITERB} (on): \bang
print Iterative Summary of values of departure coefficients.}
\space \vbox {\noindent {\bf  ITERCHI} (off): \bang
print Iterative Summary of CHI.}
\space \vbox {\noindent {\bf  ITERCHK} (on): \bang
print Iterative Summary of values of consistency checks.}
\space \vbox {\noindent {\bf  ITERN} (on): \bang
print Iterative Summary of values of number densities.}
\space \vbox {\noindent {\bf  ITERNE} (on): \bang
print Iterative Summary of values of electron density.}
\space \vbox {\noindent {\bf  ITERNH} (on): \bang
print Iterative Summary of values of total Hydrogen density.}
\space \vbox {\noindent {\bf  ITERRHO} (on): \bang
print Iterative Summary of values of net radiative bracket.}
\space \vbox {\noindent {\bf  ITERRK} (on): \bang
print Iterative Summary of values of Lyman RK-1.}
\space \vbox {\noindent {\bf  ITERRWT} (off): \bang
print Iterative Summary of values of RHO-weights.}
\space \vbox {\noindent {\bf  ITERS} (on): \bang
print Iterative Summary of values of line source functions.}
\space \vbox {\noindent {\bf  ITERTAU} (off): \bang
print Iterative Summary of values of line core optical depths.}
\space \vbox {\noindent {\bf  ITERTD} (on): \bang
print Iterative Summary of values of TDST (Type-2 Dust temperature).}
\space \vbox {\noindent {\bf  ITERZ} (on): \bang
print Iterative Summary of values of Z.}
\space \vbox {\noindent {\bf  IVALICK} (on): \bang
check `validity' of emergent intensity integrations.}
\space \vbox {\noindent {\bf  IXSTA} (on): \bang
print performance statistics.}
\space \vbox {\noindent {\bf  JBDNC} (off): \bang
bypass the calculation of unused Rhos and b-ratios.}
\space \vbox {\noindent {\bf  JLYSAV} (off): \bang
save Lyman-continuum JNU (=JB).}
\space \vbox {\noindent {\bf  JNTRPOL} (off): \bang
interpolate input values of PRD JNU to new Z-scale (instead of simply
assigning values to the new Z-scale), in runs for which TAUKIN is
specified.}
\space \vbox {\noindent {\bf  JNUPRNT} (off): \bang
print values of mean intensity for PRD transitions, for each iteration.}
\space \vbox {\noindent {\bf  JSTIN} (off): \bang
just read and check all the input, then stop.}
\space \vbox {\noindent {\bf  KOMPRNT} (off): \bang
print Composite Line Opacity (Kurucz) data table, as used in this run
(also uses the value of LWNT).}
\space \vbox {\noindent {\bf  KROSSID} (off): \bang
print details of Rosseland mean opacity wavelength integration.}
\space \vbox {\noindent {\bf  KSHLCOPR} (off): \bang 
print continuum data for K-shell wavelengths.}
\space \vbox {\noindent {\bf  KURPRNT} (off): \bang
print Statistical Line Opacity (Kurucz) data table, as used in this run
(also uses the value of LWNT).}
\space \vbox {\noindent {\bf  LBDPRNT} (on): \bang
print summary of line center background.}
\space \vbox {\noindent {\bf  LFDPRNT} (off): \bang
print full arrays of values of Line Flux Distribution.}
\space \vbox {\noindent {\bf  LIGHT} (on): \bang
calculate emergent line intensity and flux profiles.}
\space \vbox {\noindent {\bf  LINECDMP} (off): \bang
print details of scattering albedo analysis (used only when LINECOMP is on).}
\space \vbox {\noindent {\bf  LINECOMP} (off): \bang
compare results from Line Source Function and Composite Line calculations.}
\space \vbox {\noindent {\bf  LINECOPR} (off): \bang
print continuum data for line core wavelengths.}
\space \vbox {\noindent {\bf  LINTDMP} (off): \bang
print debug dump output from emergent line intensity calculation.}
\space \vbox {\noindent {\bf  LNUMDMP} (off): \bang
print dump of number density calculations.}
\space \vbox {\noindent {\bf  LONGNBM} (off): \bang
use the long, detailed version of the error message that is printed when
a calculated value of BDIJ $ < 0$.}
\space \vbox {\noindent {\bf  LSCALE} (off): \bang
print graph of logs of TAU scales.}
\space \vbox {\noindent {\bf  LSFFULL} (on): \bang
print full set of standard line source function calculation results
(used only if the basic set of results is printed).}
\space \vbox {\noindent {\bf  LSFGRAF} (on): \bang
print Line Source Function graphs for all transitions, not just
for those with LSFPRINT = 1.}
\space \vbox {\noindent {\bf  LSFPRNT} (on): \bang
print results of Line Source Function calculations for all transitions,
not just for those with LSFPRINT = 1.}
\space \vbox {\noindent {\bf  LTE} (off): \bang
calculate LTE versions of emergent line intensity and flux profiles.}
\space \vbox {\noindent {\bf  LTEDATA} (on): \bang
print LTE values of S and FR, (used only when LTE is on).}
\space \vbox {\noindent {\bf  LYMAN} (off): \bang
calculate Level-${\cal N}$-to-Continuum transfer in detail (the `Lyman'
calculation).}
\space \vbox {\noindent {\bf  LYMCOPR} (off): \bang
print continuum data for Level-${\cal N}$-to-Continuum transfer calculation
wavelengths, (used only when LYMAN is on).}
\space \vbox {\noindent {\bf  LYMDMP} (off): \bang
print debug dump output from Level-${\cal N}$-to-Continuum transfer
calculation, (used only when LYMAN is on).}
\space \vbox {\noindent {\bf  MAGPRNT} (off): \bang
print Magnesium populations and departure coefficients.}
\space \vbox {\noindent {\bf  MAKIX} (on): \bang
insert place-markers in printout file, and generate the corresponding
index file.}
\space \vbox {\noindent {\bf MCINPUT} (on): \bang
accept lower- and MiXed-case input statements, instead of UPPER-case only.}
\space \vbox {\noindent {\bf  METPRNT} (off): \bang
print details of automatic Statistical Equilibrium Equation Method selection,
(used only when METSW is on).}
\space \vbox {\noindent {\bf  METSW} (off): \bang
try the other Statistical Equlibrium Equation methods whenever a value of
line-source-function-epsilon $ < -0.9999$.}
\space \vbox {\noindent {\bf  MITPRNT} (off): \bang
provide only a bare minimum printout for each iteration, (used only when
ALL is off or EVERY is off).}
\space \vbox {\noindent {\bf  MONOTAU} (off): \bang
force all sets of calculated optical depths to increase monotonically.}
\space \vbox {\noindent {\bf  NBPRNT} (off): \bang
print final, weighted values of number density and departure coefficient.}
\space \vbox {\noindent {\bf  NEDIT} (off): \bang
edit number densities to insure positive line source functions.}
\space \vbox {\noindent {\bf  NESWICH} (on): \bang
recalculate NE in detail, instead of setting NE = NP.}
\space \vbox {\noindent {\bf  NHADJ} (on): \bang
adjust the calculated values of NH so that TAU5000 = 1 where Z = 0,
(used only when HSE is on; will be turned off automatically when HSE is off).}
\space \vbox {\noindent {\bf NRSMOOTH} (on): \bang
smooth the n$\ell$/n1 ratios in the diffusion calculations.}
\space \vbox {\noindent {\bf  NVOIT} (on): \bang
print Voigt function calculations execution statistics.}
\space \vbox {\noindent {\bf  OPAGRAF} (off): \bang
print graph of absorbers.}
\space \vbox {\noindent {\bf  OPANEG} (on): \bang
edit negative Line Opacity only to keep Total Opacity positive.}
\space \vbox {\noindent {\bf  OPAPRNT} (off): \bang
print detailed analysis of the contributions to the background opacity.}
\space \vbox {\noindent {\bf  OPASUM} (off): \bang 
print an `Absorber Summary'.}
\space \vbox {\noindent {\bf  OPTHINL} (off): \bang
use the optically-thin-limit approximation.}
\space \vbox {\noindent {\bf  OPTPRNT} (on): \bang
print OPTIONS listing.}
\space \vbox {\noindent {\bf  ORIGIN} (off): \bang
provide `long version' of analyses of regions of formation of values of 
emergent intensity.}
\space \vbox {\noindent {\bf  ORSHORT} (off): \bang
provide `short version' of analyses of regions of formation of values of
emergent intensity, (used only when ORIGIN is on).}
\space \vbox {\noindent {\bf  ORT} (off): \bang
radiation flows in the `outward' direction only, instead of isotropically.}
\space \vbox {\noindent {\bf  OXYPRNT} (off): \bang
print Oxygen populations and departure coefficients.}
\space \vbox {\noindent {\bf  OXY2PRNT} (off): \bang
print Oxygen-II populations and departure coefficients.}
\space \vbox {\noindent {\bf  OXY3PRNT} (off): \bang
print Oxygen-III populations and departure coefficients.}
\space \vbox {\noindent {\bf  PARTPRNT} (off): \bang
print the tables of Ionization Potentials, Partition Functions and/or 
Partition Function ratios.}
\space \vbox {\noindent {\bf  PARTVAR} (on): \bang
use depth-varying values of Partition Functions (instead of constant values).}
\space \vbox {\noindent {\bf  PASSPRNT} (on): \bang
print results and details of line source function calculations for passive
\break transitions.}
\space \vbox {\noindent {\bf  PDCHECK} (off): \bang
print debug checksums.}
\space \vbox {\noindent {\bf PDETPRNT} (on): \bang
print details of BD and ND calculations, illustrated at depth \# IBNVIEW
(used only when POPPRNT is on).}
\space \vbox {\noindent {\bf  PED} (off): \bang
print results and details of particle energy dissipation calculation.}
\space \vbox {\noindent {\bf  PEDDMP} (off): \bang
print debug dump output from particle energy dissipation calculation.}
\space \vbox {\noindent {\bf  PEGTNALL} (on): \bang
print GTN-editing messages for every iteration, not just the last one.}
\space \vbox {\noindent {\bf  PERDMP0} (off): \bang
print debug dump output from {\tt PERSEUS} for transition (MS/NS):
contents of data blocks, (see Note 59, Section 5).}
\space \vbox {\noindent {\bf  PERDMP1} (off): \bang
print debug dump output from {\tt PERSEUS} for transition (MS/NS):
but without details of frequency/angle summations, (see Note 59, Section 5).}
\space \vbox {\noindent {\bf  PERDMP2} (off): \bang
print debug dump output from {\tt PERSEUS} for transition (MS/NS):
details of frequency/angle summations, (see Note 59, Section 5).}
\space \vbox {\noindent {\bf  PERDMP3} (off): \bang
print debug dump output from {\tt PERSEUS} for transition (MS/NS):
PRD data arrays, (see Note 59, Section 5).}
\space \vbox {\noindent {\bf  PESRJALL} (off): \bang
print S, RHO and JBAR editing messages in every iteration,
not just the last one.}
\space \vbox {\noindent {\bf  PHASE2} (on): \bang
calculate emergent spectrum, and provide summary analyses.}
\space \vbox {\noindent {\bf  PIJPRNT} (off): \bang 
print the bound-free collision rates PIJ (for `minimal' printout,
see input parameter IRATE).}
\space \vbox {\noindent {\bf  POPBSW} (off): \bang
set the departure coefficients for higher levels equal to those for the
highest \break calculated level (instead of their LTE values), for `population
update ions' only.}
\space \vbox {\noindent {\bf  POPGRAF} (on): \bang
print graph of number densities.}
\space \vbox {\noindent {\bf  POPPRNT} (on): \bang
print results and details of the number densities calculations.}
\space \vbox {\noindent {\bf  PRDCOPR} (off): \bang
print continuum data for PRD (partial redistribution profile points) 
\break wavelengths (see also input parameter {\bf IPRDF}).}
\space \vbox {\noindent {\bf  PRDITER} (off): \bang
print results of all PRD-iterations, not just the last one.}
\space \vbox {\noindent {\bf  PRDMETH} (on): \bang
Use the Hubeny-Lites, instead of the Kneer-Heasley, formulation for PRD.}
\space \vbox {\noindent {\bf  PRDPRNT} (off): \bang
print results and details of PRD calculations.}
\space \vbox {\noindent {\bf  PROCPRNT} (on): \bang
print line profile-specific emergent continuum intensities.}
\space \vbox {\noindent {\bf  PRODMP} (off): \bang
print debug dump output from emergent intensity calculations.}
\space \vbox {\noindent {\bf  PROSAV} (off): \bang
write emergent line profile data in the Special Spectrum Save File,
(this option will be turned on automatically when SPECSAV is on).}
\space \vbox {\noindent {\bf  PTN} (on): \bang
edit negative values out of tables of TAU integrands.}
\space \vbox {\noindent {\bf  QSFEDIT} (on): \bang
edit negative values out of tables of QSF (PRD modified source function).}
\space \vbox {\noindent {\bf  RABDAT} (on): \bang
save data needed for separate RABD calculation.}
\space \vbox {\noindent {\bf  RATEALL} (off): \bang
print complete details of rates integrations.}
\space \vbox {\noindent {\bf  RATECOPR} (off): \bang 
print continuum data for rates integrations wavelengths.}
\space \vbox {\noindent {\bf  RATEFULL} (on): \bang
print details of the rates integrations.}
\space \vbox {\noindent {\bf  RATEGRAF} (on): \bang
print graphs of the TR sets and JNU sets used in rates calculations.}
\space \vbox {\noindent {\bf  RATEPRNT} (off): \bang
print results and details of the `SETTUP' calculation, \ie rates
calculation (for `minimal' printout, see input parameter IRATE).}
\space \vbox {\noindent {\bf  RATESUMM} (off): \bang
print rate integration summary for every level.}
\space \vbox {\noindent {\bf  RCOMPRNT} (on): \bang
print details of recombination calculation.}
\space \vbox {\noindent {\bf  REFLECT} (off): \bang
the plane-parallel atmosphere is symmetric about the lowest depth value.}
\space \vbox {\noindent {\bf  RHBPRDT} (on): \bang
print details of `RHO + RBD' calculation for each radiative transition 
(only if RHBPRNT is on).}
\space \vbox {\noindent {\bf  RHBPRNT} (on): \bang
print Explanation, and Results (optional), of the `RHO + RBD' calculations 
(see also options BDPRNT, RHBPRDT, RHBPRSM).}
\space \vbox {\noindent {\bf  RHBPRSM} (off): \bang
print final sets of Rho and b-ratios from `RHO + RBD' calculation 
(only if RHBPRNT is on).}
\space \vbox {\noindent {\bf  RHEDIT} (off): \bang
edit the calculated values of RHO (net radiative bracket).}
\space \vbox {\noindent {\bf  RHOFUDGE} (off): \bang
`fudge' RHO's as necessary for calculation of BDIJ, (to assure positive
values of BDIJ).}
\space \vbox {\noindent {\bf  RHOWOPT} (on): \bang
use the same `artificial' set of TAU values for all transitions when
computing RHO/W, instead of the line-core TAUs of each transition.}
\space \vbox {\noindent {\bf  RIJPRNT} (on): \bang
print ratio of collision rates RIJ.}
\space \vbox {\noindent {\bf  RKINCR} (off): \bang
artificial RK enhancement.}
\space \vbox {\noindent {\bf  RSMOOTH} (off): \bang
smooth the calculated RHO sets using the sequential smoothing procedure.}
\space \vbox {\noindent {\bf  RSQUARE} (off): \bang
use depth-varying (instead of constant) dilution factor, (used only when
INCIDNT is on).}
\space \vbox {\noindent {\bf  SCALE} (off): \bang
print collated TAU scales.}
\space \vbox {\noindent {\bf  SDIRECT} (off): \bang
replace negative Line Source Function values in ``Direct'' calculations
by interpolated positive ones.}
\space \vbox {\noindent {\bf  SEBUG} (off): \bang
print intermediate details of the statistical equlibrium calculations
for transition (MS/NS), (also uses the value of LDINT).}
\space \vbox {\noindent {\bf  SECOMP} (off): \bang
print comparative analysis of the results from the different Statistical
Equlibrium methods.}
\space \vbox {\noindent {\bf  SEDIT} (on): \bang
edit negative values out of calculated sets of Line Source Functions
obtained by the ``Full'' solution.}
\space \vbox {\noindent {\bf  SEDITIF} (off): \bang
edit negative values out of frequency-dependent Line Source Functions
used for intensity and flux profiles.}
\space \vbox {\noindent {\bf  SEPRNT} (off): \bang
print results and details of Statistical Equilibrium calculations.}
\space \vbox {\noindent {\bf  SETIME} (off): \bang
print timings for the different Statistical Equilibrium methods, (used only
when SECOMP is on).}
\space \vbox {\noindent {\bf  SILPRNT} (off): \bang
print Silicon populations and departure coefficients.}
\space \vbox {\noindent {\bf  SLFGRAF} (off): \bang
print graph of SLF for transition (MS/NS).}
\space \vbox {\noindent {\bf  SLFPRNT} (off): \bang
print SLF for transition (MS/NS).}
\space \vbox {\noindent {\bf  SLFSAV} (off): \bang
save SLF in the Special Spectrum Save File.}
\space \vbox {\noindent {\bf  SLYR} (on): \bang
smooth the calculated Lyman RK-${\cal N}$ set.}
\space \vbox {\noindent {\bf  SNUSHFT} (off): \bang
apply frequency-shift to PRD SNU in emergent profile calculations.}
\space \vbox {\noindent {\bf  SOBDMP} (off): \bang
print debug dump output from Sobolev escape probability calculation.}
\space \vbox {\noindent {\bf SOBINT} (off): \bang
print Sobolev integration details, (used only when SOBDMP is on).}
\space \vbox {\noindent {\bf  SODPRNT} (off): \bang
print Sodium populations and departure coefficients.}
\space \vbox {\noindent {\bf  SPECSAV} (off): \bang 
save all spectrum data in the Special Spectrum Save File, 
(this is equivalent to turning on both CONSAV and PROSAV).}
\space \vbox {\noindent {\bf  SPECSUM} (off): \bang
provide a `Spectrum Summary'.}
\space \vbox {\noindent {\bf  SPHEGEOM} (off): \bang
print complete sets of all Z-dependent geometrical quantities, (used only
when SPHERE is on).}
\space \vbox {\noindent {\bf  SPHERE} (off): \bang
the atmosphere is spherically symmetric, instead of being plane-parallel
and semi-infinite.}
\space \vbox {\noindent {\bf  SPHETAU} (off): \bang
calculate optical depths along rays by quadratic integration, instead of by
\break trapezoidal rule, (used only when SPHERE is on).}
\space \vbox {\noindent {\bf  SPHOUT} (off): \bang
use spherically-symmetric geometry in computing emergent line and
continuum flux, (used only in cases where the line source functions were
calculated with plane-parallel coordinates).}
\space \vbox {\noindent {\bf  SQSMDMP} (off): \bang
print details of sequential smoothing, (used only when SQSMPRNT is on).}
\space \vbox {\noindent {\bf  SQSMPRNT} (off): \bang
print brief messages for all occurrences of sequential smoothing.}
\space \vbox {\noindent {\bf  SSMOOTH} (off): \bang
do ``alternate'' sequential smoothing for S-from-number densities.}
\space \vbox {\noindent {\bf  STANDARD} (on): \bang
print various basic numerical data.}
\space \vbox {\noindent {\bf  STANCOPR} (off): \bang
print results of continuum calculations for standard background
wavelengths.}
\space \vbox {\noindent {\bf  STANPRNT} (off): \bang
print provisional depth-dependent input tables, before interpolation
to the standard Z table of this run.}
\space \vbox {\noindent {\bf  STAUREDM} (on): \bang
print short version of TAU-reduction error message from the WN-matrix
calculation.}
\space \vbox {\noindent {\bf  STIMPRNT} (off): \bang
print values of stimulated emission factors.}
\space \vbox {\noindent {\bf  STKWATT} (on): \bang
attenuate Hydrogen line Stark splitting components outside the Doppler
core at each depth.}
\space \vbox {\noindent {\bf  SULPRNT} (off): \bang
print Sulphur populations and departure coefficients.}
\space \vbox {\noindent {\bf  SUMGRAF} (on): \bang
print `iterative summaries' in graphical, rather than in tabular, form,
(used only when SUMMARY is on.}
\space \vbox {\noindent {\bf  SUMMARY} (on): \bang
print `iterative summaries,' and a fudging summary.}
\space \vbox {\noindent {\bf  SUMTREND} (on): \bang
print Iteration Trends summary.}
\space \vbox {\noindent {\bf  TANG} (on): \bang
include the ray tangent to the first depth ({\it i.e.} shell) in the angle
integrations for spherical geometry, (used only when NTAN = 1, and only
when SPHERE \break is on).}
\space \vbox {\noindent {\bf  TAUDMP} (off): \bang
print debug dump output from all optical depth integrations (TAU calculations).}
\space \vbox {\noindent {\bf  TAUPLOT} (on): \bang
print TAU scales for all transitions beneath the departure coefficient graphs.}
\space \vbox {\noindent {\bf  TAUPRNT} (off): \bang
print results and details of the line-core optical depth calculations.}
\space \vbox {\noindent {\bf  TAUSUM} (off): \bang
print a `TAU Summary'.}
\space \vbox {\noindent {\bf  TEGRAF} (on): \bang
print graph of temperature vs. optical depth.}
\space \vbox {\noindent {\bf  TOPE} (off): \bang
save data for separate `Continuum Plots' program.}
\space \vbox {\noindent {\bf  TRANSAV} (off): \bang
save transitions data in special file (.tsf).}
\space \vbox {\noindent {\bf  TRPRNT} (off): \bang
print computed effective radiation temperatures.}
\space \vbox {\noindent {\bf  TRUECONT} (off): \bang
compute ``true continuum'' as needed for residual line profiles.}
\space \vbox {\noindent {\bf  TRUECOPR} (off): \bang
print results of ``true continuum'' calculations needed for residual
line profiles.}
\space \vbox {\noindent {\bf  ULNORM} (on): \bang
use H Lyman lines normalization factors.}
\space \vbox {\noindent {\bf  USENCJ} (on): \bang
calculate background continuum JNU directly, instead of from the source
\break function.}
\space \vbox {\noindent {\bf  USETRIN} (off): \bang
calculate values of the rates RK and RL with the `old' method, which
requires input values of TR.}
\space \vbox {\noindent {\bf  USETSM} (off): \bang
eliminate TAU values smaller than TSM in emergent intensity and
continuum mean intensity calculations.}
\space \vbox {\noindent {\bf  USEWTAB} (on): \bang
use standard rates integrations wavelengths.}
\space \vbox {\noindent {\bf  VELGDMP} (off): \bang
print details of velocity gradient terms calculation, (used only when
VLGPRNT is on).}
\space \vbox {\noindent {\bf  VELGRAD} (off): \bang
include velocity gradient terms in statistical equilibrium equations
(only in Hydrogen or Helium runs).}
\space \vbox {\noindent {\bf  VELS} (off): \bang
use calculated diffusion velocity in source function calculations.}
\space \vbox {\noindent {\bf  VESCAPE} (off): \bang
use Voigt expression for the escape probability.}
\space \vbox {\noindent {\bf  VLGPRNT} (on): \bang
print data from velocity terms calculation (used only when VELGRAD
is on).}
\space \vbox {\noindent {\bf  VSWITCH} (off): \bang
use two broadening velocity sets, instead of just one.}
\space \vbox {\noindent {\bf  VTV} (on): \bang
set VT = V.}
\space \vbox {\noindent {\bf  WATESTE} (on): \bang
use logarithmic style of weighting for Lyman EP1, EP2.}
\space \vbox {\noindent {\bf  WATESTR} (on): \bang
use logarithmic style of weighting for all RHO sets.}
\space \vbox {\noindent {\bf  WAVENUMB} (off): \bang
print results in wavenumber instead of wavelength units.}
\space \vbox {\noindent {\bf  WAVEPRNT} (on): \bang
print Continuum Wavelengths summary table (also uses the value
of LWNT).}
\space \vbox {\noindent {\bf  WISFILE} (off): \bang
write an `iterative studies' save file for transition (MS/NS).}
\space \vbox {\noindent {\bf  WNDMP} (off): \bang
print debug dump output for all WN-matrix calculations.}
\space \vbox {\noindent {\bf  WTABPRNT} (off): \bang
print standard rates integrations wavelengths.}
\space \vbox {\noindent {\bf  XRAYCRID} (off): \bang
print details of X-ray cooling rates wavelength integration.}
\space \vbox {\noindent {\bf  ZCOMP} (off): \bang
when mass is prescribed, adjust Z to match the input mass scale.}
\space \vbox {\noindent {\bf  ZPRNT} (off): \bang
reprint input values of Z and TE to nine significant figures.}
\par\vfill
\ej
%\end
