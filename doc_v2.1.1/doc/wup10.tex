%\magnification=1200
%\input wupstuff.tex
\newtoks\footline \footline={\hss\tenrm 10.\folio\hss}
\pageno=1
\top
\vskip 1.5 true in
\centerline{Section 10: {\bf Element Data}}
\blankline
\blankline
\centerline{\bf ***}
\blankline
\blankline
Numerical data pertaining to electron-contributing elements are required
for the electron density ({\bf NE}) calculations. These data comprise an
`element table,' of {\bf NMT} rows and 9 columns, that is set up for the
run. Each row of the table pertains to a different ion; the default
value of {\bf NMT} = 38. The columns are for: element symbol (M), atomic
number (atno), abundance realtive to Hydrogen (AB), ionization potential
(CHI), partition function U-I and U-II (UI and UII), logarithmic abundance
on a scale such that Hydrogen = 12.0 (LAB), default value of logarithmic
abundance (DEF), and an integer (k) identifying the source of the
abundance value. The default element table is given at the end of this
section. {\it Note} option MCINPUT.
\blankline
Data other than the default values (listed below) can be input with 
\break {\bf NEWELE} and {\bf ELEMENT} statements, as follows:
\blankline \noindent
{\bf NEWELE} ( {\it j M atno AB CHI UI UII LAB DEF k} ) 
\space \noindent
where {\it j} (an {\intg} quantity) is the index of a group of entries
({\it i.e.} a row) in the element table; {\it M} (an {\alfa}
quantity) must be a valid chemical element symbol; and {\it n} and {\it k}
({\intg} quantities) and {\it AB, CHI, UI, UII, LAB} and {\it DEF} 
({\flpt} quantities) will constitute the $j^{th}$ row of the element table. 
These fields must be given in the order shown.
A {\bf NEWELE} statement provides all the data for element {\it M}, 
which will override the entire current contents of the $j^{th}$ row.
\blankline \noindent
{\bf ELEMENT} ( {\it M I$_1$ I$_2$ $\ldots$ I$_n$ } ) 
\space \noindent
where {\it M} must be a valid chemical element symbol, and {\it I}$_i$
is one or more phrases, each consisting of a pair of input fields:
{\it A V }. The field {\it A} must be {\alfa} and is an identifier; it can take
on the values: {\tt AB} (= abundance), {\tt LAB} (= logarithmic
abundance), {\tt UI} (= partition function), {\tt UII} (= partition function),
{\tt CHI} (= ionization potential), or {\tt ATNO} (= atomic number).
The field {\it V} is numeric, and must be {\intg} for {\it A} = {\tt ATNO},
but {\flpt} otherwise. Upon reading the phrase {\it I} = {\it A V},
PANDORA will set the value of {\it A} for element {\it M} equal to {\it V}.

Examples:
\blankline
1) $\quad$ {\tt ELEMENT  ( FE  AB 3.5E-5  CHI 7.87  UI 24. ) }
\spice \noindent
specifies Iron, and sets the abundance, ionization potential and the
partition function UI equal to 0.000035, 7.87 and 24.0, respectively.
\blankline
2) $\quad$ {\tt ELEMENT  ( FE  CHI 7.87 ) } \par $\phantom{2)} \; \quad$
{\tt ELEMENT  ( fe  ui 24.  ab 3.5e-5 ) } 
\spice \noindent
is one of many configurations equivalent to example (1).
\blankline
Note that an {\bf ELEMENT} statement is normally used to override
single existing entries in the current element table. The
{\bf NEWELE} statement is provided to allow emergency wholesale changes
in the built-in default element data; (the preferred way to make bulk
permanent changes in this table is to modify the PANDORA source code).
In general the {\bf NEWELE} statement should be avoided since,
during input reading, {\bf ELEMENT} statements are checked much more
carefully for correctness and consistency than {\bf NEWELE}
statements.
\ej
PANDORA sets the value of {\bf NMT} to its default before Part B of the input
is read, and sets up the default values for {\bf NMT} rows of the element
table before Part D of the input is read (see Section 3). This makes a
number of things possible; for instance:
\spice \bull If ``{\tt NMT ( 5 ) }'' occurs, then the element table of the 
run will contain only 5 rows, whose default data will be set equal to those
contained in the first five rows of the default element table (see below).
\spice \bull If ``{\tt NMT ( 39 ) }'' occurs, then the element table of the 
run will contain 39 rows; the default data for the first 38 rows will be
equal to those contained in the default element table (see below), while
the default entries of the 39$^{th}$ row will all be blanks or zero,
respectively. Note that the input value of {\bf NMT} may not exceed 50.
\spice \bull Any of the numeric entries in the element table may be changed
from their default values by mentioning them explicitly in
{\bf ELEMENT} statements. For example, if only AB(Na) ({\it i.e.} the
abundance of Sodium) needs to be changed, to 0.00001, then there need
not be an {\bf NMT} statement, and only \break
``{\tt Element ( NA AB 1.E-5 ) }''
would be needed in Part D of the input.
\spice \bull If {\bf NMT} is explicitly input as greater than 38, then 
additional ions, with their associated data, can be input with {\bf ELEMENT}
statements. When an {\bf ELEMENT} statement is processed, its element
symbol ({\it M}), is compared with those currently present in the
element table. If there is no match, and if there is at least one empty
row, then the new element symbol and its associated data will be
accepted. Conversely, if an attempt is made to introduce more than
{\bf NMT} entries, then the run will be stopped. When a {\bf NEWELE}
statement is processed, it will override the contents of the $j^{th}$
row, regardless.
\ej
\centerline{\bf Default Element Table}
\blankline
\blankline
\settabs 10 \columns
\it
\+ & \underbar{M} & \underbar{atno} & \underbar{AB} & \underbar{CHI} & 
\underbar{UI} & \underbar{UII} & \underbar{LAB} & \underbar{DEF} & 
\underbar{k} \cr \rm
\spice
\+ & H & 1 & 0 & 13.595 & 2.00 & 1.00 & 0 & 12.00  \cr
\+ & HE & 2 & 0 & 24.58 & 1.00 & 2.00 & 0 & 11.00  \cr
\+ & HE2 & 2 & 0 & 54.403 & 2.00 & 1.00 & 0 & 11.00 \cr
\+ & LI & 3 & 0 & 5.39 & 2.09 & 1.00 & 0 & 1.05 & 3 \cr
\+ & BE & 4 & 0 & 9.32 & 1.02 & 2.00 & 0 & 1.38 & 3 \cr
\+ & B & 5 & 0 & 8.296 & 6.03 & 1.00 & 0 & 2.70 & 1 \cr
\+ & C & 6 & 0 & 11.256 & 9.28 & 5.94 & 0 & 8.39 & 4 \cr
\+ & N & 7 & 0 & 14.529 & 4.07 & 8.91 & 0 & 7.85 & 5 \cr
\+ & O & 8 & 0 & 13.614 & 8.70 & 3.98 & 0 & 8.66 & 2 \cr
\+ & F & 9 & 0 & 17.418 & 5.62 & 8.32 & 0 & 4.56 & 1 \cr
\+ & NE & 10 & 0 & 21.558 & 1.00 & 5.37 & 0 & 7.84 & 3 \cr
\+ & NA & 11 & 0 & 5.138 & 2.02 & 1.00 & 0 & 6.17 & 3 \cr
\+ & MG & 12 & 0 & 7.644 & 1.01 & 2.01 & 0 & 7.53 & 3 \cr
\+ & AL & 13 & 0 & 5.984 & 5.83 & 1.03 & 0 & 6.37 & 3 \cr
\+ & SI & 14 & 0 & 8.149 & 9.26 & 5.82 & 0 & 7.51 & 3 \cr
\+ & P & 15 & 0 & 10.474 & 4.46 & 8.12 & 0 & 5.36 & 3 \cr
\+ & S & 16 & 0 & 10.357 & 8.12 & 4.16 & 0 & 7.14 & 3 \cr
\+ & CL & 17 & 0 & 13.014 & 5.25 & 7.76 & 0 & 5.50 & 1 \cr
\+ & AR & 18 & 0 & 15.755 & 1.00 & 4.89 & 0 & 6.18 & 3 \cr
\+ & K & 19 & 0 & 4.339 & 2.18 & 1.00 & 0 & 5.08 & 3 \cr
\+ & CA & 20 & 0 & 6.111 & 1.03 & 2.29 & 0 & 6.31 & 3 \cr
\+ & SC & 21 & 0 & 6.538 & 12.0 & 22.9 & 0 & 3.05 & 3 \cr
\+ & TI & 22 & 0 & 6.818 & 30.2 & 56.2 & 0 & 4.90 & 3 \cr
\+ & V & 23 & 0 & 6.738 & 41.7 & 43.7 & 0 & 4.00 & 1 \cr
\+ & CR & 24 & 0 & 6.763 & 10.3 & 7.24 & 0 & 5.64 & 3 \cr
\+ & MN & 25 & 0 & 7.432 & 6.45 & 7.76 & 0 & 5.39 & 1 \cr
\+ & FE & 26 & 0 & 7.896 & 24.5 & 39.2 & 0 & 7.45 & 3 \cr 
\+ & CO & 27 & 0 & 7.863 & 31.6 & 27.5 & 0 & 4.92 & 1 \cr
\+ & NI & 28 & 0 & 7.633 & 28.8 & 10.0 & 0 & 6.23 & 3 \cr
\+ & CU & 29 & 0 & 7.724 & 2.29 & 1.02 & 0 & 4.21 & 1 \cr
\+ & ZN & 30 & 0 & 9.391 & 1.00 & 2.00 & 0 & 4.60 & 1 \cr
\+ & GA & 31 & 0 & 5.997 & 5.37 & 1.00 & 0 & 2.88 & 1 \cr
\+ & GE & 32 & 0 & 7.899 & 8.13 & 4.37 & 0 & 3.58 & 3 \cr
\+ & RB & 37 & 0 & 4.177 & 2.29 & 1.00 & 0 & 2.60 & 1 \cr
\+ & SR & 38 & 0 & 5.693 & 1.26 & 2.19 & 0 & 2.92 & 3 \cr
\+ & Y & 39 & 0 & 6.377 & 12.0 & 15.1 & 0 & 2.21 & 3 \cr
\+ & ZR & 40 & 0 & 6.835 & 33.9 & 45.7 & 0 & 2.59 & 3 \cr
\+ & BA & 56 & 0 & 5.210 & 2.29 & 4.17 & 0 & 2.17 & 3 \cr
\blankline
\blankline
\noindent (The source citations identified by the value of the integer k
appear in the PANDORA printout; these are built into the source code, and
cannot be changed by {\bf ELEMENT} or {\bf NEWELE} statements.)
\blankline
\blankline
When input reading is finished, the `element table' exists only
in {\it provisional} form. It is converted into {\it final}
form by the following sequence of steps:

1) Look at every row. If any AB $>$ 0, then compute the corresponding LAB
from it; if AB $\leq$ 0, then set LAB = DEF.

2) Look at every row. If any AB = 0, then compute it from the corresponding
LAB; if AB $\leq$ 0, then set AB = 0 and LAB = 0.
(AB $<$ 0 is legitimate in {\bf ELEMENT} and {\bf NEWELE} statements.)

3) If {\bf FABD} $\neq$ 0, then look at every row {\it except}
those of Hydrogen or Helium. Set the final value of AB = {\bf FABD} $\times$ AB.
If this final AB $>$ 0, then compute LAB from it.

4) {\it Squeeze out} (eliminate) from the table every row whose AB = 0,
and adjust the value of {\bf NMT} accordingly.

5) If the Hamburg data (used to calculate depth-dependent partition functions
when option PARTVAR is on) has usable {\tt CHI} values for any of the ions in
the final table, then such Hamburg values will replace the corresponding default
or input values of {\tt CHI} (regardless of the status of option PARTVAR).
\blankline
%\blankline
%\vfill \vfill
%\vfill \vfill
%\vfill \vfill
%\vfill \vfill
%\vfill \vfill
%\vfill \vfill
%\vfill \vfill
%\vfill 
\vfill
\noindent (Section 10 -- last revised: 2007 Feb 05) %\par
\message{Section 10 ends at page 10.\the\pageno}
\ej
%\end
