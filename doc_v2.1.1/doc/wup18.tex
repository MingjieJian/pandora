%\magnification=1200
%\input wupstuff.tex
\newtoks\footline \footline={\hss\tenrm 18.\folio\hss}
%
\pageno=1
%
\top
\vskip 1.5 true in
\centerline{Section 18: {\bf Frequency Tables}}
\blankline
\blankline
\centerline{\bf ***}
\blankline
\blankline
Tables of frequency values, $\xi$ of length $k$,
are needed ---1) for the calculations performed for
every radiative transition $(u,\ell)$ as follows:
a) integration over frequency for the Line Source Function,
b) integration over frequency for PRD terms (if needed),
c) emergent line profile calculation;
and ---2) to capture simulated lines in the ``background''
(continuum). A frequency table can span either a 
half-profile (which then implies a symmetric whole profile), or a
whole profile (which need not be symmetric).

Internally, PANDORA establishes separate complete sets of preliminary frequency tables
for every transition $(u,\ell)$. Such a complete set consists of:
a symmetric half-profile table, a blue-side half-profile table,
and a red-side half-profile table. These tables will be set equal
to {\it unique} transition-specific half-profile input tables if
such were provided, otherwise to general-purpose {\it common} half-profile input tables.

Full-profile frequency tables are assembled from half-profile tables. However, for every
transition $(u,\ell)$ that is a blended line, a composite full-profile frequency
table is specially constructed by combining the full-profile tables of each
of the component lines. Full-profile tables are augmented automatically
to account properly for ``background lines.''
\blankline
The remainder of this section describes the input, the defaults provided
when no inputs are specified, how the full-profile table and the composite
full-profile table (if needed) are constructed, and which tables are selected
for the source function and emergent profile calculations.
\blankline
\vfill \vfill
\vfill \vfill
\vfill \vfill
\vfill \vfill
\centerline{\bf Important Note}
\blankline
The lengths of the various input and constructed frequency tables should
not exceed the limits explained below for {\bf KMMAX}, and also in
Section 5, Note 47.
\ej
\centerline{\bf Input}
\blankline
For historical reasons, {\it only half-profile frequency tables may be specified in the input.}
All these tables must be specified with their values in ascending order, and each table's
first value must = 0.
\blankline
One may specify {\it unique} half-profile frequency tables for some or all transitions
$(u,\ell)$, and/or one may specify {\it common} half-profile frequency tables to be
used for all those transitions for which no unique sets are specified.
\blankline
\noindent 1) {\it Common} half-profile frequency tables:

The basic {\it common} half-profile frequency table is {\bf XISYM} of length {\bf KS}.
(The defaults are given is Section 5, above.)
A {\it common} blue-side half-profile table, {\bf XIBLU} of length {\bf KB},
and a {\it common} red-side half-profile table, {\bf XIRED} of length {\bf KR},
may also be specified. After PANDORA has read Part D of the input,
defaults will be provided for unspecified tables: {\bf XIBLU} will be set
equal to {\bf XISYM} if no values were input, and {\bf XIRED} will be set
equal to {\bf XISYM} similarly.
\blankline
\noindent 2) {\it Unique} half-profile frequency tables:

[The transition-superscripts $(u,\ell)$ on names ending in `T'
are omitted in the next paragraph.]

The following half-profile frequency tables can be input for transition $(u,\ell)$:
symmetric {\bf XISYMT} of length {\bf KST},
blue-side {\bf XIBLUT} of length {\bf KBT}, and
red-side {\bf XIREDT} of length {\bf KRT}. After PANDORA has read Part D
of the input, defaults will be provided for unspecified tables:
The {\it unique} {\bf XISYMT} will be set equal to {\it common}
{\bf XISYM} if no values were input; thereafter, {\bf XIBLUT} will
be set equal to {\bf XISYMT} if no values were input, and {\bf XIREDT}
will be set equal to {\bf XISYMT} similarly.

{\it Note} also the relevant input parameters {\bf KSTMAX},
{\bf KBTMAX}, and \break {\bf KRTMAX}, which must be specified.
\blankline
\noindent 3) {\it Full-profile} frequency table:

[The transition-superscripts $(u,\ell)$ on names ending in `T'
are omitted in the next paragraph.]

After {\bf XIBLUT} and {\bf XIREDT} for transition $(u,\ell)$ have been
established, a whole-profile frequency table XIFULT of length KFT
is constructed by concatenating the values of {\bf XIREDT} and the negatives
of the values of {\bf XIBLUT}:
${\rm XIFULT}_1 = - {\bf XIBLU}_{\bf KBT}$, ${\rm XIFULT}_{\bf KBT} = 0.0$,
${\rm XIFULT}_{\rm KFT} = {\bf XIREDT}_{\bf KRT}$,
where ${\rm KFT} = {\bf KBT} + {\bf KRT} - 1$.
\ej
% \blankline
\noindent 4) {\it Blended} line transitions:

A special composite full-profile frequency table XICMPT$^{u,\ell}$
of length KCT$^{u,\ell}$ is constructed by merging {\bf LDL}$^{u,\ell}$
XIFULT$^{u,\ell}$ tables, each offset appropriately, with redundant
points eliminated as much as possible.

{\it Note:} For storage allocation purposes, PANDORA estimates the value
of the largest KCT$^{u,\ell}$ {\it before} any have been computed.
This estimate may be much too large (in cases when many
redundant points can be eliminated). Yet a run could be stopped if this
estimate is larger than can be accommodated. The input parameter
{\bf KMMAX} has been provided for such a case: if ${\bf KMMAX} > 0$ then,
if it is larger than the estimate, {\bf KMMAX} will be used for
storage allocation purposes instead of an internal estimate;
if ${\bf KMMAX} < 0$ then, if $-{\bf KMMAX}$ is less than the estimate,
$-{\bf KMMAX}$ will be used for storage allocation purposes instead.
A run will still be stopped, however, if ultimately any
${\rm KCT}^{u,\ell} > |{\bf KMMAX}|$.
\blankline
\blankline
\centerline{\bf Uses}
\blankline
After all these preliminaries, PANDORA is ready to set up the frequency table 
$\xi$ for each transition. The choice of table depends on whether the range of the
Line Source Function frequency integration extends over the whole profile,
or only over half a profile:
\spice
1) If option EXPAND is on, then full-profile integrations are done for all
transitions.
\spice
2) If ${\bf NVX} > 0$, then full-profile integrations are done for all
transitions.
\spice
3) If ${\bf LDL}^{u,\ell} > 1$ (\ie if this is a blended line transition), then
full-profile integration is used for transition $(u,\ell)$.
\spice
4) If ${\bf LSFFDB}^{u,\ell} = 1$ (\ie if frequency-dependent background is required
for this transition), then full-profile integration is used for transition
$(u,\ell)$.  {\it Note} that PANDORA sets ${\bf LSFFDB}^{u,\ell} = 1$
automatically for every PRD transition.
\spice
5) In {\bf all other} cases, half-profile integrations are used.
\ej
% \blankline
\noindent {\bf Then}, for every radiative transition $(u,\ell)$:
\spice
If half-profile integration is used for transition $(u,\ell)$, then
$k^{u,\ell}$ is set equal to ${\bf KST}^{u,\ell}$ and $\xi^{u,\ell}$
is set equal to ${\bf XISYMT}^{u,\ell}$.
\spice
If full-profile integration is used for transition $(u,\ell)$ and this
is a blended line, then $k^{u,\ell}$ is set equal to ${\rm KCT}^{u,\ell}$
and $\xi^{u,\ell}$ is set equal to ${\rm XICMPT}^{u,\ell}$.
\spice
If full-profile integration is used for transition $(u,\ell)$ and this
is not a blended line, then $k^{u,\ell}$ is set equal to ${\rm KFT}^{u,\ell}$
and $\xi^{u,\ell}$ is set equal to ${\rm XIFULT}^{u,\ell}$.
\blankline
\noindent {\bf Moreover}, if a full-profile $\xi$-table has been selected:
\spice
PANDORA checks whether the centers of any simulated ``background lines''
occur in the wavelength band corresponding to the range $\xi_1$ to $\xi_k$.
If yes, then the $\xi$-table is augmented with enough additional values to resolve
such lines properly. The Hydrogen Lyman lines are handled as a special
case. All other background lines cause all or part of a set of frequency
values---obtained by constructing a full-profile frequency table from
the input table {\bf BXI} of length {\bf KBX}---to be inserted in the
$\xi$-table.
({\it Note}: {\bf BXI} is also used for inserting ``background
line wavelengths'' in the standard Rates Integrations wavelengths table
constructed and used when option USEWTAB is on.)
%\vfill \vfill
%\vfill \vfill
%\vfill \vfill
%\vfill \vfill
%\vfill \vfill
%\vfill \vfill
%\vfill 
\vfill
\noindent (Section 18 -- last revised: 2004 Jul 07) %\par
\message{Section 18 ends at page 18.\the\pageno}
\ej
%\end
